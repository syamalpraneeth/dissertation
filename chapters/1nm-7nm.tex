\chapter{Cluster-size effects in Cluster-assembled Metallic Glasses} \label{c:cbmg}
In the last two chapters, a protocol was established to study the \gls{camg}, and contrast it with \gls{rq} \gls{mg} and \gls{ng}. We observed that the processing of \cz \gls{camg}s and \gls{ng} led to the formation of a core-interface network which influenced the structural characteristics of the materials. Consequently, the nature of the interfaces in terms of length scales are therefore expected to final states of the glassy systems. \par

For instance, it has recently been shown in various models of the \gls{ng}s that the diminution of the nanoparticle size or the ``grain size" improves the plasticity of the material drastically, characterized by a reduced flow stress \cite{Adibi2013,Adibi2014,Cheng2019a}. Adopting smaller particle sizes and \gls{pvd} have also lately been shown to correlate with thermal ultra-stability in small-chain polymer glasses experiments \cite{Raegen2020} and simulated \gls{lj} model-atoms \cite{Singh2013}. The current chapter discusses the attempts to explore such size effects in \gls{camg}s and \gls{ng}s. \par

\begin{figure}[!h] \centering
	\includegraphics[width=0.94\linewidth,trim={0 2cm 0 1.5cm},clip]{NG_1-7nm/9.pdf}
	\mycaption{Cluster-sizes and associated length scales}{\cz CAMG building blocks with diameters in the range of 1-7 nm. The 7 nm nanoparticle has $\sim$300 times as many atoms as the 1 nm cluster.}
	\label{f:clussizes}
\end{figure}

%\cz CAMG building blocks with diameters in the range of 1-7 nm.
%\cz CAMG building blocks ranging from 7 nm to 1 nm in diameter. 
%{\cz CAMG building blocks ranging from 7 nm nanoparticles to 1 nm clusters; the 7 nm nan in population range by $\sim$300 times.}

\newpage
\section{Size-effects in CAMGs}
Chapters~\ref{c:dev} and ~\ref{c:camg} describe in detail the \gls{camg}s and \gls{ng}s made from 3 nm sized clusters. In this section, CAMGs made from monodisperse nanoparticles of two sizes (\mbox{1 nm} and 7 nm in diameter) are studied as a first step towards understanding cluster-size influence. 

\subsection{7 nm \cz Single Nanoparticle Synthesis and Deposition} 
In previous simulation-based investigations of \gls{ng}s, nanoparticles of sizes 7-15 nm were chosen  \cite{Adjaoud2016,Adjaoud2018,Kalcher2017,Cheng2019}. Therefore, CAMGs were explored with this nanoparticle\footnote{It is worth noting that in this chapter, the clusters and nanoparticles, although the same in constitution and composition, refer to two different entities. The clusters refer to small aggregations of atoms---usually in the size range of 10-2000 atoms \cite{Kartouzian2013,Kartouzian2014,Benel2018,Benel2019,Gack2020}, whereas the nanoparticles are the large masses of atoms typically used in the context of NGs \cite{Jing1989,Nandam2017,Adjaoud2016,Adjaoud2018}.} size range. A large spherical nanoparticle, 7 nm in diameter, was cut from a \cz \qr{10} \gls{mg}\footnote{All simulated MGs, NGs and CAMGs discussed in this chapter are made from \cz \qr{10} MGs.}. The resulting nanoparticle contained $\sim$10,000 atoms, and had a \cz composition with 0.3\% deviation. After cutting the sphere, the nanoparticle was subject to a short heating above \glsdesc{tg} (\gls{tg}), and cooled to 50 K, similar to the 3 nm cluster (as described in Section \ref{s:clus}). \par

\begin{figure}[!h] \centering
	\includegraphics[width=0.9\textwidth,trim={0 1.5cm 0 1.2cm},clip]{NG_1-7nm/1.pdf}
	\mycaption{Compositional variation in the 7 nm \cz nanoparticle}{The evolution of chemical segregation of (a) Cu-atoms and (b) Zr-atoms, in the core and the shell regions of the \cz amorphous nanoparticle.}
	\label{f:7nm-clus}
\end{figure}

\clearpage

A core-shell structure was found to evolve in the 7 nm nanoparticle (See Figure~\ref{f:clus_rad-3nm} in \nameref{c:supple}), with a shell thickness of 3 nm evaluated using a radial composition analysis like in Figure~\ref{f:clus_rad-3nm}, Chapter~\ref{c:dev}. Figure~\ref{f:7nm-clus} describes the variation of Cu and Zr concentrations in the core and shell regions. Like in the \mbox{3 nm} cluster (Figure~\ref{f:clus_comp-3nm}), a sharp change in the compositions in the core and shell is observed with the heat treatment. The high temperature mobilizes the atoms to move to their preferred positions. This configuration is by no means the equilibrium concentration of the nanoparticle, as the atoms need a longer time above the \gls{tg} to diffuse across the sphere completely \cite{Adjaoud2016}. However, long duration of heat-treatment above \gls{tg} is known to produce crystalline nuclei in the \gls{rq} \gls{mg} experiments. Moreover, it was necessary to set the simulation parameters in the \mbox{7 nm} nanoparticle same as the 3 nm cluster for the sake of consistency. \par

\begin{figure}[!h]
	\begin{subfigure}{0.55\textwidth} \centering
		\begin{subfigure}{\textwidth} \centering
			\includegraphics[width=\textwidth,trim={2.8cm 1cm 2.8cm 1cm},clip]{NG_1-7nm/2.pdf}
		\end{subfigure}%
		%	\hspace{-0.5cm}
		\vfill
		\begin{subfigure}{\textwidth} \centering
			\includegraphics[width=\textwidth,trim={2.8cm 2cm 2.8cm 1cm},clip]{NG_1-7nm/3.pdf}
		\end{subfigure}
	\end{subfigure}%
	\hspace{0.3cm}
	\begin{subfigure}{0.45\textwidth} \centering
		\includegraphics[width=\textwidth,trim={8.8cm 1cm 0.5cm 1cm},clip]{NG_1-7nm/4.pdf}
	\end{subfigure}%
	\mycaption{7 nm \cz single nanoparticle deposition}{The cross sections of the as-deposited states of the nanoparticle being deposited at varying energies (per atom) colour coded by (a) the core and shell regions of undeposited nanoparticle, like in Figure \ref{f:clus_single3}. (b) The local atomic shear strains of the nanoparticle and substrate, as compared to their pre-deposition states. In (c), the as deposited states are characterized by \gls{rc} and \gls{rmsds} (See text for more details).}
	\label{f:7nm-cibds}
\end{figure}

Nevertheless, the Cu-concentration in the 7 nm nanoparticle is seen to increase in the nanoparticle shell, with a corresponding decrease in the core region. The opposite is observed with the Zr-concentration, which is seen to preferentially segregate to the core. A brief discussion on the driving force behind this segregation is mentioned in Section \ref{s:clus} and Reference \cite{Adjaoud2016}. \par

After the preparation of the nanoparticle, its deposition on a substrate is studied (similar to Chapter~\ref{c:dev}). A layered substrate thermal model is once again implemented for reasons mentioned in Chapters~\ref{c:dev}~and~\ref{c:camg}. The substrate chosen is 15 nm $\times$ 15 nm in the XY-plane, larger than the one used for the single 3 nm cluster deposition. The substrate thermal layers were flat, unlike for the 3 nm single deposition case. In Figure \ref{f:7nm-cibds}a, the cross sections of a single 7 nm nanoparticle deposition are depicted, 2 ns after impact\footnote{The singly deposited nanoparticles at various energies were found to attain a stable configuration after \mbox{2 ns} of deposition, see Figure~\ref{f:cibds_eval} in \nameref{c:supple} for more details.}. The impact energy of the deposition is varied from 6 meV/atom to 3000 meV/atom. The 7 nm nanoparticle, like the 3 nm cluster (see Figure~\ref{f:cibdsmod}) is seen to deform with increasing impact energy. Furthermore, the nanoparticle is also seen to be embedded more into the substrate at higher energies. \par

At 3000 meV/atom impact energy, some nanoparticle and substrate atoms can be seen to ejected from the substrate, not just from the sides of the impacted nanoparticle, but also along the impact-center, indicating a reflection of impact-shockwaves by the substrate. Such an explosive behaviour would not only be a consequence of the high momentum of deposited nanoparticle, but also a result of the inability of the modelled substrate to dispel sufficient heat from the system (See Section \ref{s:camgdev} for more details). \par

Although the energy per atom is half the maximum deposition energy of 6000 meV/atom in the 3 nm clusters case (See Sections \ref{s:camgdev} and \ref{c:cibd_single}), the deposition of the bigger 7 nm size nanoparticle introduces a larger momentum into the system. Since the high impact is not desirable for formation of interfaces, the deposition at 3000 meV/atom case will not be studied in detail. However, to simulate high impact and high momentum depositions, hybrid substrate models using \gls{md} and continuum mechanics would serve as better physical models, for instance, by reducing the shockwave reflections upon impact \cite{Insepov1997,Allen2002}. \par

When the atoms in the simulated cross-sections are colour-coded by their local von Mises strain as in Figure \ref{f:7nm-cibds}b, the deformation of the 7 nm nanoparticle is clearly visible. The deformed zones in the nanoparticle-substrate system are much wider than in the 3 nm cluster case, hinting that the cluster-cluster interfaces in CAMGs would increase in width with the size of the chosen clusters. \par

In Chapter~\ref{c:dev}, two quantities \gls{rc} and \gls{rmsds} were defined to quantify the deposited states of a single 3 nm cluster, 2 ns after deposition. Figure~\ref{f:7nm-cibds}c shows the \gls{rc} and \gls{rmsds} corresponding the depositions of the 7 nm nanoparticle. Unsurprisingly, the \gls{rmsds} is seen to increase drastically with impact energy. At an impact energy of 600 meV/atom, a $\sim$6\% increase in \gls{rmsds} relative to the 6 meV/atom soft-landed state. In contrast, for a 3 nm cluster at the same impact energy, the relative increase in \gls{rmsds} from the 6 meV/atom soft-landed state is only $\sim$1.9\%. Correspondingly, the convexity of the nanoparticle, characterized by \gls{rc}, is seen to change with deposition energy. In the energy ranges of 0-600 meV/atom, the nanoparticle remains largely convex in nature on the substrate. In comparison, the 3 nm cluster loses its convexity already at 600 meV/atom impact energy (Figure~\ref{f:clus_single3}). At 1000 meV/atom, the 7 nm nanoparticle is more concave, indicating that the nanoparticle begins to embed into the substrate beyond this energy. The CAMG of the 3000 meV/atom case---as mentioned before---was observed to be losing atoms likely due to shockwave reflections. If the snapshots were made later in time, more atoms would have been lost from the deposited nanoparticle at the 3000 meV/atom energy. The \gls{rc} and \gls{rmsds} values in this case are hence merely representative, and will not describe the end state of the deposited nanoparticle. To facilitate a comparison with the \mbox{3 nm} CAMGs studied in Chapter~\ref{c:camg}, CAMGs made from \mbox{7 nm} nanoparticles are simulated with 60, 300 and 600 meV/atom impact energies in the next section. \par

\subsection{7 nm \cz Multiple Nanoparticle Deposition}
The deposition of the CAMG film was made in an \gls{hcp} pattern (see Section~\ref{c:dev} for more details regarding this choice). To fit the deposited CAMG exactly onto the substrate, the substrate dimensions were chosen to be 14 nm $\times$ 13 nm in the XY-plane. This helped save simulation costs. In Figure~\ref{f:7nm-cibdpores}, the deposited film for the low energy case of 60 meV/atom is shown. Like the 3 nm clusters, all the nanoparticles retain most of their sphericity. In Figure~\ref{f:7nm-cibdpores}a, the film atoms are once again colour-coded with the scheme used in Chapters~\ref{c:dev}~and~\ref{c:camg}. While the substrate atoms are coloured-black, the atoms in the shell region of the undeposited nanoparticle are coloured yellow, the core atoms are coloured magneta. These core and shell atoms defined from the undeposited nanoparticle are once again used to define the core and interface atoms in the CAMGs films. \par

\begin{figure}[!h] \centering
	\includegraphics[width=\linewidth,trim={2cm 2.8cm 2cm 2cm},clip]{NG_1-7nm/5.pdf}
	\mycaption{A perspective view of the 60 meV/atom 7 nm \cz CAMG}{(a) The deposited film atoms are coloured by a core-shell colour scheme also used in Figures~\ref{f:clus_rad-3nm}~and~\ref{f:film_network}. (b) A sliced view of a constructed surface mesh in the same perspective view as Figure~\ref{f:7nm-cibdpores}a shows the presence of pores in the CAMG film. In (c) the atoms shown in Figure\ref{f:7nm-cibdpores}a are represented by their local atomic strains, corroborating the presence of interface regions (similar to Figure~\ref{f:film_network}).}
	\label{f:7nm-cibdpores}
\end{figure}

A surface mesh construction \cite{Stukowski2010a,Stukowski2014} with a probe sphere radius of 3 \r{A} revealed the formation of pores in the films (Figure~\ref{f:7nm-cibdpores}b). Even when the deposition of the nanoparticles were adjusted (deviating from the HCP pattern), the pores were seen to not fully close up. This can be attributed to the large size of the nanoparticles. In atomic clusters of smaller sizes, voids formed in cluster-cluster interfaces can easily be closed up in a \gls{hcp} deposition. \par

The local shear strain or von Mises strain of the CAMG atoms and the substrate are depicted by a colour coding scheme in Figure~\ref{f:7nm-cibdpores}c. In this manner, the deformed interfaces can be clearly visible, as observed in Figures~\ref{f:film_network}~and~\ref{f:7nm-cibds}. \par

\subsection{Effect of Cluster Size on CAMG SRO and Energetic States}
After setting up a deposition protocol for the 7 nm nanoparticle, it is now possible to proceed to the evaluation of size effects in the 7 nm CAMGs, which were prepared at three energies of 60, 300 and 600 meV/atom. With the exception of the 60 meV/atom CAMG, the other two simulated samples were observed to not show any pores. To avoid the inclusion of any surface artefacts in the non-porous films (See Section~\ref{s:corint} for more details), representative slabs were constructed for the CAMGs. \par

\begin{figure}[!ht] \centering
	\begin{subfigure}{\textwidth}
		\includegraphics[width=\textwidth,trim={0.65cm 3.65cm 1cm 2cm},clip]{NG_1-7nm/6.pdf}
	\end{subfigure}%
	\vfill
	\begin{subfigure}{\textwidth}
		\includegraphics[width=\textwidth,trim={0.65cm 3.65cm 1cm 2cm},clip]{NG_1-7nm/7.pdf}
	\end{subfigure}%
	\vfill
	\begin{subfigure}{\textwidth}
		\includegraphics[width=\textwidth,trim={0.65cm 2.8cm 1cm 2cm},clip]{NG_1-7nm/8.pdf}
	\end{subfigure}%
	\mycaption{Comparison of SRO and energetic states of 3 nm and 7 nm CAMGs}{(a)-(c) Full-icosahedral (FI) packing, (d)-(f) icosahedral-like ordering (ILO), and (g)-(i) average P.E/atom of the CAMGs.}
	\label{f:7nm-cibdeval}
\end{figure}

The 7 nm CAMGs are compared to the 3 nm CAMGs based on their energetic states and local \gls{sro} in Figure~\ref{f:7nm-cibdeval}. The \gls{rq} \gls{mg} and \gls{mght}, like in Sections~\ref{s:vorocamg}~and~\ref{s:camg-pe} are also represented as a reference. For reasons elucidated in Section~\ref{s:corint}, the CAMGs are compared with 7 nm NGs (compacted in an HCP layout), \gls{rq} \gls{mg}, and \gls{mght}. The porous 7 nm 60 meV/atom CAMG, which is present with surface artefacts is marked in all the sub-figures with a red cross (`\textcolor{red}{\textbf{\faClose}}'). The \gls{fi} order and \gls{ilo} are used to describe the SRO. Figures~\ref{f:7nm-cibdeval}(a)-(c) show the \gls{fi} variation in the entire CAMG representative slab, and also in the core and the interface regions. Likewise, the \gls{ilo} is depicted in Figures~\ref{f:7nm-cibdeval}(d)-(f). It is clearly observed for both the 3 nm and 7 nm CAMGs that the \gls{fi} order and \gls{ilo} both increase in the core and interfaces with increasing deposition energy. \par

In the core regions, a clear difference is seen in the \gls{sro} with cluster size. For the case of the CAMG with the larger nanoparticle, the \gls{fi} order and \gls{ilo} are both higher than for the 3 nm CAMG of the corresponding deposition energies. This same trend is also observed in the NGs. The average P.E/atom of the CAMGs of different cluster sizes show the opposite behaviour in Figure~\ref{f:7nm-cibdeval}(g)-(i). The change in the \gls{fi} order and \gls{ilo} in both the 3 nm and 7 nm CAMGs is a clear evidence of tailoring of local structure with the deposition energy. The variation in \gls{sro} and P.E/atom between the 3 nm and 7 nm CAMGs can be correlated to the difference in local compositions, as indicated in Table~\ref{t:chem_heter}. In the entire CAMG samples overall, the SRO and P.E/atom are seen to be invariant with cluster size. The observed change in the formation of interfaces and previous reports of scaling of elastic properties with the size of the cluster in nanoglasses \cite{Cheng2019a} motivates the exploration of CAMGs with atomic clusters of 10-30 atoms in size, wherein the scope of interface formation improves drastically. \par 

\begin{table} \centering
	\begin{tabular}{c c c c} 
		\hline \hline	
	    \multirow{2}{2cm}{\centering Cluster diameter} & \multicolumn{3}{c}{Composition}\\
%	    \cline{2-4}
		 & Total sample & Core & Interface \\
		\hline	
		3 nm	& Cu$_{50}$Zr$_{50}$ & Cu$_{46}$Zr$_{54}$ & Cu$_{55}$Zr$_{45}$ \\
		7 nm	& Cu$_{50}$Zr$_{50}$ & Cu$_{49}$Zr$_{51}$ & Cu$_{52}$Zr$_{48}$ \\
		\hline \hline	
	\end{tabular}
	\mycaption{Chemical heterogeneity in the 3 nm and 7 nm CAMGs}{The cores and interfaces are in the two CAMGs are found to exhibit distinct chemical concentrations, although the macroscopic chemical composition remains constant at \cz.}
	\label{t:chem_heter}
\end{table}

\subsection{Small-cluster CAMGs}
One of the objectives of this work was to explore the effects of CAMGs made from clusters of 1 nm diameter. Based on previous works on \gls{pvd} glasses \cite{Singh2013,Raegen2020}, these CAMGs can be expected to show thermodynamic ultrastability. Since the 1 nm clusters contain only $\sim$30 atoms, the stability of the cluster on the surface after deposition is of concern. Theoretically, after deposition, it is expected that a certain amount of the impact energy dissipated from the cluster is transferred back to it. However, the cluster will be stable on the substrate as long as the energy transferred back to it is lesser than the \glsdesc{ecoh} (\gls{ecoh}) of the material. Even in an extreme case where 100\% of the deposition energy is transferred back to the cluster, the depositions in the energy range of 60-600 meV/atom should result in stable clusters, as the energy is much lower than $\sim$-4.88 eV/atom---the \gls{ecoh} of a \cz \gls{rq} \gls{mg} \cite{Jekal2019}. The following paragraph briefly describes the initial simulations of CAMGs made from small clusters performed with the assistance of Ms. Veronika Stangier, who briefly worked as a student researcher in the group of Prof. Wolfgang Wenzel at the \gls{kit}.  \par

In conjunction with the work discussed in this dissertation, attempts to simulate \cz CAMGs from clusters of 20-30 atoms were also made to explore local chemical heterogeneity at the size scale of $\leq$ 1 nm, as envisioned by \textcite{Kartouzian2014}. In contrary to the expectations elucidated above, the clusters were found to spontaneously melt onto the surface, even with soft landing of 6 meV/atom energy. Alternatively, clusters which were first arranged on the substrate in a `frozen' state (with their thermal models and net computed forces set to zero)--were also found to dissolve onto the surface once they were unfrozen. This led to conclusion that the dissolution of clusters is likely due to their cohesion with the \cz substrate. \par

As a workaround to these modeling limitations with small clusters in \gls{camg}s, the grain size effects were explored in \gls{ng}s, as discussed in the following section.

\section{Size-effects in Nanoglasses}
In this section, the influence of grain size or cluster size on the \gls{ng}s is studied. Firstly, the clusters of diameters 1, 2, 3, 5, and 7 nm were once again prepared as discussed in Chapter~\ref{c:dev} and in the above sectin. It was also ensured that the clusters were made with an exact \cz composition, with no deviation in composition. The NGs were then made from monodisperse cluster distributions, inserted into a simulation box in a random fashion\footnote{In this section, all NGs are made from a random cluster-insertion model unlike the HCP packing used in previous sections and chapters. Without the constraint of a patterned arrangement, the total number of clusters in the box can be controlled to ensure that the simulated NG samples made from varying cluster sizes have approximately the same number of atoms, thereby ruling out any possible simulation-box size effects.} and compacted at 5 GPa pressure before unloading (See Figure~\ref{f:NGintersiz}a).   \par

\subsection{Variation of Interfaces with Cluster Size}

\begin{figure}[!h]
	\includegraphics[width=\linewidth,trim={1.2cm 0cm 0.6cm 0cm},clip]{NG_1-7nm/10.pdf}
	\mycaption{NG interfaces vary with cluster size}{(a) The cluster compaction model is illustrated. (b)-(f) A frontal sliced view of the NGs, with the atoms coloured by their von Mises shear strain made from clusters of sizes 1-7 nm shows the homogenization of interface regions with reducing cluster size.}
	\label{f:NGintersiz}
\end{figure}

Figures~\ref{f:NGintersiz}(b)-(f) show the cross-sectional views of the NGs. Similar to Figure~\ref{f:film_network}(d), here the atoms are colour-coded with their local \glsdesc{vmstr} (\gls{vmstr}) or atomic strain values evaluated using \gls{ovito} \cite{Stukowski2014} to distinguish the interfaces from the core by means of atomic-level deformation. As mentioned in Section~\ref{s:corint}, this is done because the interfacial atoms are expected to deviate more from their original positions near the surface of the uncompressed cluster/nanoparticle, than the core atoms \cite{Gleiter1991}. With the decreasing cluster size, the interfacial width is not only found to drop, but also simultaneously one can visually observe how the interfacial atoms begin to dominate the bulk of the \gls{ng}s with smaller clusters. This is in agreement with previous NG simulations \cite{Cheng2019a} and also the expectation that smaller clusters have a higher surface to volume ratio, thereby enabling the possibility for more interfaces to occur. In other words, the net number of atoms participating in interface formation increase with decreasing cluster size. \par

\subsection{Atomic Readjustment upon Compaction}
In the previous subsection, the formation of interfaces in NGs was observed using the local atomic strain or \glsdesc{vmstr} (\gls{vmstr}). The change in interfacial characteristics was visually noticeable when the NG building blocks were varied from large nanoparticles (5-7 nm) to small clusters (1-3 nm). The \glsdesc{vmstr}-information of every atom in the \glspl{ng} are now used to further explore the cluster-size effects on the atomic displacement which occurs during formation of the cluster-cluster interfaces. \par

\begin{figure}[!h]
	\includegraphics[width=\linewidth,trim={2cm 0.3cm 2cm 0.475cm},clip]{NG_1-7nm/11.pdf}
	\mycaption{NG atomic readjustment after compaction}{(a) The fraction of atoms with a local strain higher than a threshold of $\eta^{thres}$ (indicated in the legend). (b) Atoms with $\eta^{thres}=1$ are seen to readjust further with annealing, and with decrease in cluster size. (c) The average strain of NG atoms increases with annealing and with lowering cluster size.}
	\label{f:NGvonMises}
\end{figure}

In Figure~\ref{f:NGvonMises}a, the `Atomic readjustment' in the compacted NGs---defined as the percentage of atoms having \gls{vmstr} above a threshold of $\eta^{thres}$ is illustrated. The atomic readjustment is named so, as the von Mises strain essentially captures the change of the local environment of around an atom in a system of atoms, with respect to a reference configuration \cite{Adjaoud2018,Cheng2019,Zheng2021,Cheng2009}. In this manner, one can quantify the change in the environment of each atom in a cluster before and after compaction into a \gls{ng}. At low thresholds of $\eta^{thres} \leq 1$, the atomic readjustment in the NGs is seen to increase with decreasing cluster size. It is also noticed that above $\eta^{thres} = 1$, the atomic readjustment does not vary with cluster size. \par

Both these observations together indicate that the number of highly readjusting interfacial atoms in the \glspl{ng} do not vary significantly with cluster size. More number of atoms participate in the atomic-level deformation in the smaller cluster NGs, overall. This is also confirmed by Figures \mbox{\ref{f:NGvonMises}(b)-(c)}. Additionally, the NG samples were annealed below \gls{tg} at 600 K for 2 ns. The heating from 50 K to 600 K was done at a rate of \qr{12}. After equilibration at 600 K for 2 ns, the samples were cooled down back to 50 K at a rate of \qr{12} and equilibrated again for 2 ns. Upon annealing, the increase in atomic readjustment (at a constant threshold of $\eta^{thres} = 1$) is greater with the decrease of NG cluster size. Additionally, the average von Mises strain in the as-prepared NG system is higher with reduction of cluster size, and the increase in this value upon annealing is also higher at smaller cluster sizes. \par

\subsection{Short-range Order and Thermal Behaviour of the NGs}
The as-prepared and annealed NGs are next characterized by their local \gls{sro} and energetic states. In Figure~\ref{f:NG_str-PE}, the characteristics of the entire NG samples are depicted. The core and interface atoms are not individually investigated. The FI order, \gls{ilo}, and P.E/atom in the as-prepared samples are noted to vary slightly with the cluster size. However, in proportion to the effect of quench rates or energetic deposition as seen in Figures~\ref{f:camg_pote}~and~\ref{f:7nm-cibdeval}, respectively, the \gls{sro} and energetic states are fail to show the expected cluster size dependency. \par

\begin{figure}[!h] \centering
	\includegraphics[width=0.98\linewidth,trim={2cm 0.3cm 2cm 0.475cm},clip]{NG_1-7nm/12.pdf}
	\mycaption{Influence of cluster size on structural and energetic states of NGs}{(a) FI-ordering (b) ILO (c) P.E/atom states of the various as-prepared and annealed NGs.}
	\label{f:NG_str-PE}
\end{figure}

Upon thermal relaxation after annealing, the modelled NG samples are observed to have an increased SRO and a lowered energetic state in comparison to their as-prepared states. This appears to be a consequence of the atomic readjustment that was noted to be promoted by the annealing process in the previous subsection. The annealing possibly assists the simulated NGs to attain their final states, which may not be accessible by the as-prepared samples due the short \gls{md} simulation time scales. However, even after the thermal treatment, no discernible variation of the \gls{sro} and energetic features with cluster size can be noted. This study on \cz NGs is in agreement with previous works on \czsix NGs \cite{Cheng2019a}. \par

The preceding research activity on cluster-size influence in the NGs, although demonstrating an invariance of local order with cluster-size, reported some interesting mechanical properties in the Cu$_{36}$Zr$_{64}$, \cz and \czsix Voronoi-tesselated NGs \cite{Adibi2013,Adibi2014} and cluster-compacted NGs \cite{Cheng2019a}. Additionally, the mechanical and thermal properties of \gls{rq} \gls{mg}s have been linked in earlier works \cite{Su2016,Battezzati2009,Bian2021}. This motivates the need to explore the thermal properties of the \gls{ng}s, in which the reduced grain/cluster-size has notably demonstrated a reduction of flow stress and increased plasticity. \par

\begin{figure}[!h]
	\includegraphics[width=\linewidth,trim={0cm 1cm 0cm 1cm},clip]{NG_1-7nm/13.pdf}
	\mycaption{Thermal behaviour of MGs and NGs}{The average enthalpy per atom with increasing temperature in (a) the simulated reference samples of 8000-atom sized \cz RQ MGs of varying quench rates and (b) the NGs of varying cluster sizes. In the insets of the respective figures, the system temperature with time is indicated.}
	\label{f:NG_enth}
\end{figure}

The simulated \gls{ng}s of various cluster sizes were subjected to a heat-treatment at zero pressure from a low temperature to well above the \gls{tg} (~800 K for simulated \cz \gls{rq} \gls{mg}s), and compared with a reference system of \gls{rq} \gls{mg}s. The average atomic enthalpy\footnote{The enthalpy H = U + p$\cdot$V, where U, p, and V are the internal energy, pressure and volume of the system, respectively. In the heating simulations, as p=0, the enthalpy can be expressed as H = U.} as a function of temperature is represented for both the NGs and the reference \cz \gls{rq} \gls{mg}s\footnote{Further studies to establish the protocol for heating and cooling of \gls{rq} \gls{mg}s are discussed in Section \ref{s:mgsquench} of the \nameref{c:supple} chapter.} in Figure~\ref{f:NG_enth}. The NG samples, in their as-prepared states at 50 K are first equilibrated for 2 ns and then heated to 1200 K at a rate of 0.25 K/ps. For the reference \gls{rq} \gls{mg}s prepared at various quench rates, the initial average enthalpies at 50 K are found to be ordered according to quench rates associated with their formation: the lower the quench rate, the lower the enthalpy. This trend continues in the entire temperature regime below \gls{tg}, after which the enthalpies of the glasses of various quench rates are equal. These observations are in agreement with known knowledge from literature \cite{Berthier2016,Ediger1996}. In contrast, the thermal behaviour of the \gls{ng}s is found to be independent of cluster-size. The curves depicting average enthalpy versus temperature for various cluster-sizes overlap with each other. \par

\section{Discussion and Summary}
The ability to design a network of distinct amorphous cluster-core and cluster-cluster interface phases into the CAMGs and NGs begets the emergence of cluster-size/grain size effects in these novel amorphous materials. In the current chapter, simulated \cz CAMGs and NGs prepared from monodisperse clusters of varying sizes were studied. \par

First, the CAMGs made from a 3 nm cluster in Chapter~\ref{c:camg} was compared to those prepared from a 7 nm nanoparticle at corresponding per-atom deposition energies. The \gls{sro} of the 7 nm CAMGs were also observed to be tailorable with deposition energy in the samples. However, the \gls{sro} and the average \gls{pe}/atom of CAMGs (for a given deposition energy) were found to be invariant with the cluster-size. Upon further inspection, it was revealed that within the core and interfacial regions, the \gls{sro} and the average energetic states in the CAMGs varied considerably with cluster size. These effects were attributed to the difference in the local chemical heterogeneity introduced into the simulated CAMGs from the precursor clusters. An attempt to prepare CAMGs from clusters of 1 nm diameter ($\sim$30 atoms) proved to be non-viable, owing to spontaneous melting of deposited clusters onto the substrate even at per-atom deposition energies below the cohesive energy of \cz . Replacing the \cz substrate with a Si substrate may reduce cluster-substrate cohesion, promoting retention of the structure of the smaller clusters after deposition. \par

The influence of grain size studies in smaller clusters (diameter $\leq$ 2 nm) were explored in the \cz NGs made using a monodisperse clusters inserted randomly before compaction. This approach offers the advantage of exploring size effects without considering additional substrate interactions unlike in the CAMGs. With the reduction of the cluster size in the NGs, the interfacial width was seen to correspondingly reduce, while the cluster atoms participating in readjustment upon compaction and annealing increase drastically. Despite the readjustment of the interfacial atoms, and the opportunity for interfaces to relax better, no significant change is seen in the NGs with the cluster size. Additionally, the smaller 1-2 nm clusters present a challenge of defining interfaces in the NGs, in terms of distinguishing the shell atoms from the core. In the future, using the von Mises strain or the \gls{qna} approach \cite{Feng2020} may serve as better parameters to identify glass-glass interfaces by making no \textit{a priori} inferences about the clusters' core and shell regions. Further studies on the evolution of density and free volume with the composition and size of the precursor clusters are in progress. \par