\chapter{Scientific Background} \label{c:theory}
This chapter discusses the relevant historical overview and scientific knowledge that forms the basis of this thesis. First and foremost, the traditional \gls{mg}, which is made by rapid quenching is discussed. The fundamentals of MGs synthesis, their properties and the computation-aided structural models are provided. Next, the \gls{ng}---which is a recent class of \glspl{mg}--is introduced. The details on how to prepare the NGs, and their relevant structural models are elaborated. Equipped with the fundamentals of MGs and NGs, the reader is then acquainted with the concept of the novel \gls{camg}. The details of the initial \gls{camg} experiments, and the current advances and limits of the understanding of \gls{camg} is covered.  \par 

\section{Metallic Glasses} \label{s:mg}
In their natural state, metallic solids exhibit crystalline order. However, when an alloy is cooled from a liquid to solid state fast enough i.e., supercooled, typically at high cooling rates of $10^5-10^6$ K/s, it gives rise to a metallic solid with an amorphous structure \cite{Debenedetti2001}. This is because the atomic mobility of the liquid decreases drastically before any crystallisation nucleation events can occur during the cooling, and the so-obtained solid is trapped in a state with no \gls{lro}. These materials, exhibiting a lack of \gls{lro}, are broadly termed as glasses. They are already known to occur in nature (as obsdian and amber), and have also been artificially produced from silicates, polymers and even dextrose \cite{Doremus1994,Berthier2016}. Primarily, gls{mg}s are supercooled metallic materials exhibiting amorphous atomic arrangement. In 1960, Duwez et al. \cite{Klement1960,Duwez1960} devised an apparatus to rapidly quench molten materials. Their successful synthesis of an entirely amorphous \mbox{Au$_{75}$Si$_{25}$} flake led the foray into \gls{mg} research \cite{Klement1960}. The interesting physics in the \gls{mg}s will be discussed in the following sections. \par

\subsection{Glass Transition and Free Volume} \label{s:gt-fv}
The liquid to solid transition of glasses, known as glass transition, is very different from that exhibited in crystalline materials---as shown in Figure~\ref{f:rq-mg-sch}a. For the crystalline material, the liquid upon solidification undergoes a sharp phase transition in the specific volume or enthalpy at the \glsdesc{tm} (\gls{tm}). The glass transition is much more gradual, occurring at the \glsdesc{tg} (\gls{tg}).  A slower cooled glass (glass 2 in Figure~\ref{f:rq-mg-sch}a) demonstrates a lower glass transition temperature ($T_{g2} < T_{g1}$). \par

In general, physical changes in materials can be classified based on the change in the phases, typically described by an order parameter. By the Ehrenfest classification, the order of the lowest derivative of the \glsdesc{gibbs} (defined as \gls{gibbs} = H - TS, where H, T, and S are enthalpy, temperature, and entropy respectively) exhibiting a discontinuity upon crossing the phase boundary is the order of a phase transition \cite{Jaeger1998}. By this definition, \gls{gibbs}(T,p) is continuous in a first order phase transition but its first derivatives are discontinuous:
\begin{equation}
	S = - \left( \frac{\partial G}{\partial T} \right)_p \text{, and } V = \left( \frac{\partial G}{\partial p} \right)_T = \frac{\partial H}{\partial p} =  \frac{\partial (pV)}{\partial p}
\end{equation}
where S, p and V are the entropy, pressure and volume of the system, and H = pV. In the second order phase transition, the first derivatives are continuous but the second-order derivatives of \gls{gibbs}(T,p), and also the following response functions are discontinuous:
\begin{equation}
\begin{gathered}
C_p = T\left(\frac{\partial S}{\partial T} \right)_p = -T \left( \frac{\partial^2 G}{\partial T^2} \right)_p \text{, } \alpha = \frac{1}{V} \left(\frac{\partial V}{\partial T} \right) = \frac{1}{V} \left(\frac{\partial^2 G}{\partial T \partial p} \right), \\
\text{and }\kappa_T = -\frac{1}{V} \left(\frac{\partial V}{\partial p} \right)_T = -\frac{1}{V} \left( \frac{\partial^2 G}{\partial p^2} \right)_T\text{,}
\end{gathered}
\end{equation}
where $C_p$, $\alpha$, and $\kappa _T$ are the isobaric heat capacity, coefficient of thermal expansion, and isothermal compressibility, respectively. The first- and second-order phase transitions are illustrated in Figure~\ref{f:rq-mg-sch}b. The first order transition is discontinuous in entropy (S) and volume (V) (See Figure~\ref{f:rq-mg-sch}bi-bii). The second order transition, while continuous in S, is discontinuous in heat capacity ($C_p$) (See Figure~\ref{f:rq-mg-sch}biii-biv). It can now be clearly noticed that the transition from the liquid to the crystalline material depicted in Figure~\ref{f:rq-mg-sch}a is a first-order phase transition. The glasses, on the other hand, exhibit the glass transition, which is continuous in the volume or enthalpy order parameters, but discontinuous in viscosity and specific heat. The glass transition is a second order phase-transition \cite{Cohen1959,Berthier2016} and has been experimentally demonstrated in many materials \cite{Kauzmann1948}. \par

\begin{figure}[!h] \centering
	\begin{subfigure}{0.4\textwidth} \vfill
		\includegraphics[width=0.9\linewidth]{glasstrans_mod}
		\subcaption{}
	\end{subfigure}%
	\begin{subfigure}{0.6\textwidth}
		\includegraphics[width=\linewidth]{phasetrans.png}
		\subcaption{}
	\end{subfigure}
	\mycaption{Glass transition and phase transitions}{(a) Specific volume of a glass upon cooling demonstrates a glass transition at \gls{tg}, different from the liquid-to-solid transition in its crystalline counterpart \adaptfig{Ediger1996}{1996}{American Chemical Society}. (b) First-order (i-ii) and second-order (iii-iv) phase transitions. \adaptfig{Nishimori2010}{2010}{Oxford University Press}.}
	\label{f:rq-mg-sch}
\end{figure}

Contrary to the liquid-to-crystalline solid transition, the glass transition is not an equilibrium transition, rather is an effect of failure of kinetic readjustments due to change in temperatures \cite{Fox1950,Ramachandrarao1977,Ediger1996}. The observation that the product of viscosity and total volume is constant with temperature \cite{Batschinski1913} leads to the concept of free volume. One of the popular definitions of free volume of a solid system at a given temperature is the excess measured specific volume compared to its specific volume at 0 K temperature \cite{Ramachandrarao1977}. The glass transition in supercooled liquids is said to originate from the reduction of relative free volume in the bulk \cite{Fox1951}. \textcite{Cohen1959} proposed a model of free volume: the atoms were assumed to be transported into the voids that appeared, when the void volume was greater than a critical volume, and no energy is required to free volume redistribution. The free volume would be negligible in low temperature regimes, but with increasing temperature, the volume gained upon expansion is ``free'' to be redistributed into the entire bulk of the solid. After the volume is distributed, the system would attain a minimum free energy configuration \cite{Turnbull1961}. The glass transition is a change in the viscosity, it is not really a physical change. It is for this reason that a glass transition is not considered to be a thermodynamic phase change like melting. \par

\subsection{Glass Forming Ability and Energy Landscape} \label{s:gfa}
Some popular routes today to synthesise \gls{mg}s are \gls{rq} by melt-spinning \cite{Greer1995}, ball-milling \cite{Suryanarayana2001,Weeber1988} and solid-state reactions \cite{Schwarz1983}. 
The \gls{mg} synthesis involves accessing a metastable amorphous state, and avoiding nucleation of crystals is not without difficulties. There is hence a need to search for good glass formers, and enhancing the \gls{gfa}. \textcite{Turnbull1969} proposed criteria to bypass crystallisation when supercooling, one of the most important being the reduced glass transition temperature $T_r = T_g/T_m$, where \gls{tg} and \gls{tm} are the \glsdesc{tg} and \glsdesc{tm} at the composition respectively \cite{Turnbull1969,Wang2004a}. It was predicted that glasses around the deep eutectic compositions would have good \gls{gfa}. For $T_r=0.5-0.67$, the system becomes sluggish in crystallisation at experimental timescales \cite{Turnbull1969}. \par 

While Turnbull's criterion is a good rule of thumb for \gls{gfa}, the experimentally realisable size of metallic glass samples is another challenge. The larger the dimensions of the glass sample, the greater the chance of non-uniform cooling rates, promoting crystallisation within the bulk of the undercooled sample. This greatly limits the size of the \gls{mg}s. For instance, the first AuSi glass could be made in very small dimensions in the order of 0.1 mm$^2$ area and 10 \gls{um} thickness\footnote{Today, industrial production of melt-spun amorphous ribbons of even 150 mm in width is possible \cite{Wu2014}.}. The interest to produce glasses of larger critical sizes in the 1990s led to research on \gls{bmg}, which are up to several millimetres in thickness \cite{Johnson1999,Greer2007,Inoue2000}. In 1999, \textcite{Johnson1999} developed the first \glspl{bmg}. Shortly after, it was empirically determined in 2000 by \textcite{Inoue2000} that the crystallisation in larger glasses could be avoided using the following criteria:
\begin{enumerate}[noitemsep]
	\item \textit{Confusion principle}: Choosing multi-component alloys ($\ge$ 3 components) can cause frustration in the undercooled liquid, delaying crystallisation
	\item The difference in radii of atoms should be more than 12-15\% 
	\item Negative heat of mixing amongst the main constituent elements, 
\end{enumerate}
These conditions increasing solid/liquid interfacial energy, and make it difficult for atomic rearrangement on a long-range scale. Thereby, the visosity and the \gls{tg} increase, and the driving force for crystallisation is reduced. By choosing multi-component systems and along with these empirical rules, \textcite{Inoue2000} successfully prepared \gls{bmg} of 72 mm in size. The good \gls{gfa} of \gls{bmg}s is attributed to their Arrhenius-like behaviour, demonstrating strong-liquid-like viscosities as classified by the empirical Vogel–Fulcher–Tammann (VFT) relation \cite{Debenedetti2001,Wang2004a}. In the quest to produce better glasses, which are also thermally stable, a better understanding of the underlying thermodynamic mechanisms is warranted. \par

One of the first observations about glass stability was its dependence on its quench rate. Figure~\ref{f:rq-mg-sch}b shows the specific volume behaviour with temperature for a liquid forming a glass. At lower quench rates, glasses can reach a state with lower specific volume, which translates to lower enthalpy and entropy. Furthermore, the \glsdesc{tg} (\gls{tg}) of the glasses reduces when lowering the cooling rates. However, there is a limit on the slowest cooling rate possible in the glasses. It was pointed out that theoretically, at an infinitely slow quench rate, the entropy of the system can be lowered below the entropy of the solid, forming an \textit{ideal glass} \cite{Kauzmann1948}. Visually, this can be understood from Figure~\ref{f:rq-mg-sch}a, where the liquid transition is extrapolated to low temperatures, and where the extended (dotted) line meets the crystal entropy is the \glsdesc{tk} (\gls{tk}). However, this renders the entropy of the liquid to attain a value below that of the crystalline solid at absolute zero temperature. This is called the Kauzmann paradox \cite{Kauzmann1948,Debenedetti2001,Berthier2016}. A resolution to this entropy paradox was postulated by Kauzmann himself, suggesting that all supercooled liquids must crystallise before the \gls{tk} is reached. In actual experiments there is a limit to quench rates possible \cite{Kauzmann1948}, and at very slow quench rates, the system will crystallise. \par

\begin{figure}[!h]
	\begin{subfigure}{0.5\textwidth}
		\includegraphics[width=\linewidth]{pelglass2}
		\subcaption{}
	\end{subfigure}%
	\begin{subfigure}{0.5\textwidth}
		\includegraphics[width=\linewidth]{alphabeta}
		\subcaption{}
	\end{subfigure}
	\mycaption{Energy landscape of metallic glasses}{(a) The potential-energy landscape of MGs in configurational space, depicted with basins and sub-basins \adaptfig{Debenedetti2001}{2001}{Springer Nature} (b) $\alpha$ and $\beta$ transitions in configurational space, $\alpha$ relaxations constitute a change of the all the particle coordinates. \reprintfig{Stillinger1995}{1995}{AAAS}.}
	\label{f:mgpel}
\end{figure}

The synthesis of \gls{mg}s is evidently complicated because of their inherent metastability, as the preferred states of the systems are either crystalline (at low temperatures) or liquid (at high temperatures). To better understand the process, the system of atoms that constitute the glass can be imagined to have many accessible states, which are potential wells (also called basins) in configurational space---often referred to as a \gls{pel}. In Figure~\ref{f:mgpel}a, an illustration of a \gls{pel} is depicted. The various amorphous states are local minima in the \gls{pel}, with the crystalline structure at the global minimum of the configurational coordinates. While various processing treatments bring the system to a certain state, it is possible for the system to hop from one state to another by transitions or relaxation processes. This is visualised in Figure~\ref{f:mgpel}b. $\beta$-relaxations are the hopping of a system amongst subbasins, and involves a local rearrangement of atoms. The larger and relatively slower hopping events, which constitute a rearrangement of the entire system, are referred to as $\alpha$-relaxations, transporting a system to an entirely different basin in the \gls{pel}. \par

In order to stabilise the glasses at the nearest local minimum, sometimes heat-treatment processes are implemented. A low temperature thermal cycling such as annealing below \gls{tg} for instance, lowers the enthalpy and potential energy of the glassy system. This is referred to as aging. In contrast, when a treatment of the system increases the total energy of the system, the process is called rejuvenation. The rejuvenation and aging processes are depicted in Figure~\ref{f:mgpel}a. 

\subsection{Structural Models for Amorphous Metals}\label{s:sro-mgs}
As mentioned in the previous section, a system can attain multiple glassy confiurations (see Figure~\ref{f:mgpel}); the thermodynamic state of a glass depends upon the formation routes---and hence affects its properties. This leads to the age-old problem of understanding the structure of \gls{mg}s, to explore structure-property relationships. At a first glance, glasses appear to have no particular structural order, only to be characterised by their general lack of \gls{lro}. A unifying structural definition of the amorphous structures, requires understanding them beyond simply defining the lack of \gls{lro}. Recent studies discuss models in which metallic glasses possess ordered substructures \cite{Sheng2006,Fukunaga2006,Greer2007}. The first kind, termed as \gls{sro} is defined over the length scales of $\leq$ 0.5 nm. \gls{sro} is the structure at the level of an atom and its first and second nearest neighbours. Beyond the \gls{sro}, the \gls{mro} is defined at length scales of $\sim$1 nm, and describe how the SRO-motifs pack together. An experimentalist may already be familiar with these terms upon calculating \gls{rdf} (described in Chapter~\ref{c:methods}) from electron diffraction, \gls{xrd} or \gls{exafs}, however the \gls{rdf} provides only an average 
picture of the atomic structure. A description of the local structure of atoms provides essential clues to understand the amorphous \gls{mg}s, and finds value particularly in Chapters~\ref{c:camg}~and~\ref{c:cbmg} of this thesis.  \par

The local atomic arrangements and their contribution to the packing and, eventually, to the stability of glassy structures have been discussed in previous studies. A first theory of \gls{sro} in random packing in solids was from empirical evidence of compressing densely packed plasticine (modelling clay) balls by \textcite{Bernal1959}, in connection with an investigation of the structure of liquids. It was noticed that the hard spheres upon compression deformed into a wide variety of polyhedra, which were mostly made up of pentagonal faces. The convex polyhedra so formed in the plasticine models could also be constructed in atomic structures, as shapes enclosed by planes are drawn to bisect distances between every geometrical neighbour of all the atoms \cite{Bernal1959,Finney1970a}. Such a construction has been known in the mathematics and physics communities by various names, such as the Voronoi tessellation \cite{Voronoi1908,Coxeter1973}, Dirichlet region \cite{Dirichlet1850}, Delaunay triangulation \cite{Delaunay1934} or as a Wigner-Seitz cell \cite{Kittel2004}. The Bernal plasticine model for irregular packing in liquids, fails to describe multicomponent glasses with large size differences between the atoms, and also the realistic atomic systems, which interact differently than hard spheres. Another model was proposed by \textcite{Gaskell1978} for metal-metalloid systems, arguing that the nearest neighbour units resemble the structure of the crystalline phases at the same compositions. The structure factors of the glasses obtained from this model were in agreement with those obtained from neutron diffraction experiments. However, this model did not support metal-metal glass systems, and failed to explain the stability of \gls{bmg}s in their supercooled states \cite{Chen2011a}. \par

In recent times, \textcite{Miracle2004} proposed an alternative theory for metallic glasses. He supposed that the structure of glasses could be constructed by efficient packing of solute-centered clusters (or solute clusters) in \gls{fcc} or \gls{hcp} layouts, and called this the \gls{ecp} 
model. By this model, the solute atoms ($\alpha$ type) exist inside the solvent atom clusters ($\Omega$ type), and also in the cluster-octahedral interstices ($\beta$ type) and cluster-tetrahedral interstices ($\gamma$ type) depending upon the ratio of atomic radii of the solute and solvent atoms. This makes up four topologically distinct classes of atoms in the glass. The solute clusters were deemed to be predominantly face-sharing, but also exhibit edge- and 
vertex-sharing to accommodate for internal strains. The model theorised that the solute clusters in turn were packed in \gls{fcc} and \gls{hcp} configurations, giving rise to the \gls{mro}. This \gls{ecp} model was successful in predicting the compositions and also \gls{mro} up to 1 nm length 
scale. However, this model has some major deficiencies. The model fails to explain the evolution of the local structure of \gls{mg}s, which are known to vary with the quench rate. Furthermore, the \gls{ecp} is a static model, so it also can not explain the dynamical behaviour of glasses. \par

A more recent method to describe simulated local amorphous structure and \gls{sro} is the evaluation of the local topological order of the atoms. Unlike the \textcite{Miracle2004,Miracle2013} approach, first an amorphous structure is prepared either by \gls{md} or \gls{rmc} simulations. Then, a Voronoi tessellation is performed to study the topology \cite{Sheng2006,Fukunaga2006}. This results in a broad distribution of polyhedra of various shapes and sizes, reminiscent of the polyhedra obtained by \textcite{Bernal1959}. Such a method has various advantages over the Miracle model, i.e., the dynamic evolution of \gls{sro} can be studied. \par

Initially, \textcite{Honeycutt1987} calculated the stability of free-standing agglomerations of 13 atoms in size—arranged in an icosahedral packing with five-fold symmetry. The stability in supercooled liquids was suggested to be a result of icosahedral clusters by \textcite{Frank1952}. It was later reported that the occurrence of icosahedral packing–or, \gls{fi} order---can be correlated with increased packing fraction in model metallic glasses \cite{Clarke1993}. Recently, performing Voronoi tessellation on simulated glasses \cite{Sheng2006,Fukunaga2006} showed that the partitioned \gls{3d} space assigned to atoms were equivalent to Kasper polyhedra \cite{Frank1958,Doye1996}, which constitute the local SRO in MGs. \textcite{Sheng2006} showed that the glass structure need not be made from a single \gls{fi} motif, but that every alloy system has distribution of coordinations, dominated by a certain coordination polyhedra. For the case of \czsix MGs, icosahedral atomic packing was observed to be the highest occurring structural motif \cite{Ding2014,Ding2014a}. Furthermore, for \cz MGs, it was observed that these FI environments were strongly spatially correlated to each other \cite{Peng2010,Li2009a}. For \cz MGs quenched at a faster rate, less FI and \gls{ilike} packing or \gls{ilo} has been found \cite{Yue2018}. \par 

\begin{figure}[!h] \centering
	\includegraphics[width=0.5\linewidth]{mg-sro}
	\mycaption{Short-range order in metallic glasses}{The Voronoi tessellation method \cite{Sheng2006,Fukunaga2006} helps identify the distribution of \gls{sro} polyhedra indexed using Schl\"afli notation (discussed in further detail later in Section~\ref{s:voronoi}). In \czsix glasses, FI-polyhedra (\vi{0}{0}{12}{0}) are the prominently occurring SRO motifs. \reprintfig{Ma2015}{2015}{Nature Publishing Group}.}
	\label{f:voro-sro-mg}
\end{figure}

Icosahedral order dominates local order and influences packing. The link between stability (structural and thermodynamic) and \gls{sro} in MGs is a very well-studied topic in literature. The work by \textcite{Cheng2008} unequivocally clarifies the relationship between increased FI SRO and increased thermodynamic stability in Cu-Zr MGs. Additionally, there is also work on \gls{gum}, which are SRO structures that are known to be the most likely participators in shear transformations, contributing to structural instability of MGs \cite{Ding2014}. Icosahedral SRO units are known to be least likely to be \gls{gum}s. \par

In addition to giving a description of the topological \gls{sro}, \textcite{Sheng2006} proposed a model for \gls{mro}. They suggested that in dilute solutions, unlike the proposed SRO-units by \textcite{Miracle2004}, it would be the solute clusters as derived from the Voronoi that would pack in FCC/HCP/icosahedral-stacking to form a \gls{mro}. When the solute concentration increases above $\sim$20\%, they observed that all of the solute atoms can no longer be placed at the centre of a solute-cluster being completely surrounded only by solvent atoms. In this case, some solute atoms are expected to be first nearest neighbours. Consequently, a different kind of \gls{mro} forms in solute-rich compositions, comprising of  strings of solute atoms. Another recent work discusses the aggregation of FI-clusters to form chains present also with interpenetrating icosahedra to exhibit a cross-linked MRO network \cite{Lee2011,Ritter2012b,Ritter2012}.

\section{Nanoglasses} \label{s:ngs}
The previous section elaborates on the various aspects of the conventionally prepared \gls{mg}s. Alternative routes to preparing glasses have been explored to find new ways to control properties of amorphous materials. As early as 1977, it was expected by the glass community that vapour-deposited materials should have different structures than their monolithic counterparts, due to the effects of boundary conditions on the nanoparticles or atoms being deposited during synthesis \cite{Finney1977}. The discovery of \gls{ncm} \cite{Gleiter1991}---made from compaction of nanometre-sized grains---catalyzed similar advances in 
noncrystalline solids.  Analogous approaches to the \gls{ncm} powder processing, could then be used to generate glasses with tailorable defect microstructures. \par

\begin{figure}[!h] \centering
	\includegraphics[width=0.5\linewidth]{ng_compact.png}
	\label{f:ng-sch}
	\mycaption{Schematic of nanoglass formation by compaction}{Illustrated here is the hydro-static compaction of particles produced via IGC, which leads to the synthesis of nanostructured NGs.}
\end{figure}

\textcite{Gleiter1991} envisioned a new type of non-crystalline solid made from the  consolidation nanometre-sized amorphous nanoparticles, and termed it as nanoglass (NG). He hypothesized that in compacting a multitude of glassy nanoparticles, it would be possible to create a glass with enhanced free volume, which exist amongst in the regions of contact (or interfaces) of adjacent nanoparticles. Expecting defective coordinations at the interfaces, 
Gleiter propounded that the interfacial regions should be structurally and/or chemically distinct from the cores \cite{Gleiter1991, Gleiter1995}. The \gls{sro} and properties of the \gls{ng}s, he said, must then deviate from a \gls{rq} \gls{mg} of a similar chemical composition. \par

\subsection{Experimental Studies}
The group of Gleiter successfully synthesised a PdFeSi \gls{ng} \cite{Jing1989} in 1989. The nanoparticles were prepared by thermal evaporation and \gls{igc} under \gls{uhv}, and later consolidated at gigapascal orders of pressure. The structure of the PdFeSi \gls{ng} was evaluated using the Fe-\gls{ms} technique. In contrast to the \gls{rq} \gls{mg}, which shows a single broad peak in the \gls{qs} distribution, it was observed that the \gls{ng} shows two peaks \cite{Jing1989}: One peak was similar to that of the \gls{rq} \gls{mg}, and hence was interpreted to be originating from the core regions of the glassy nanoparticles, as the \gls{ng} core was expected to have similar structure to the \gls{rq} \gls{mg}. The second peak of the \gls{qs} distribution was unique to the \gls{ng}. This additional component was attributed to the interfaces formed amongst the consolidated nanoparticles \cite{Jing1989}. In addition, the \gls{ms} indicated the interfacial region to possess a reduced electron density. \gls{ng}s of various chemical compositions, such as Au–Si, Au–La, Fe–Si, La–Si, Pd–Si, Ni–Ti, Ni–Zr and Ti–P, were also synthesised \cite{Weissmuller1992}. \par

Sc$_{75}$Fe$_{25}$ NGs were also investigated using \gls{pas} \cite{Fang2012}, as depicted in Figure~\ref{f:ng-evidence}a. The Fe-Sc NGs indicated two distinct positron lifetimes, one of which was the same as of the \gls{rq} \gls{mg}, and the second lifetime was present only in the NGs and interpreted as originating from the interface regions. The intensities of the two lifetimes were used to indicate that the as-prepared NG consisted of 65 vol\% glassy cores and 35 vol\% interfacial regions. Moreover, this Sc$_{75}$Fe$_{25}$  NG demonstrated enhanced plasticity in comparison to \gls{rq} \gls{mg}s, this was also attributed to the presence of glass-glass interfaces. \par

\begin{figure}[!h] \centering
	\begin{subfigure}{0.4\linewidth}
		\begin{subfigure}{\linewidth}
		\includegraphics[width=\linewidth]{positron_ann} \subcaption{}
		\end{subfigure}%
		\vfill	
		\begin{subfigure}{\linewidth}
		\includegraphics[width=\linewidth]{witte-ngmh.jpeg} \subcaption{}
		\end{subfigure}%
	\end{subfigure}%
	\hfill
	\begin{subfigure}{0.6\linewidth} \centering
		\includegraphics[width=0.9\linewidth]{witte-ngms.jpeg} \subcaption{}
	\end{subfigure}%

	\mycaption{Indirect evidence for interfaces in FeSc NGs}{(a) Relative intensities of two positron lifetimes from \gls{pas} of Sc$_{75}$Fe$_{25}$ NG; $\tau_2$ was interpreted to arise from interfaces and increases in intensity with annealing \adaptfig{Fang2012}{2012}{American Chemical Society}. (b) Magnetisation and (c) \gls{ms} spectra of Fe$_{90}$Sc$_{10}$ 
	NG indicate a new distinct Fe-environment in NGs \reprintfig{Witte2013}{2013}{AIP Publishing LLC}.}
	\label{f:ng-evidence}
\end{figure}

The Fe environments in Fe-Sc NGs were studied extensively to identify core and interfaces. The Fe$_{90}$Sc$_{10}$ NG in particular was the subject of many reports indicating the presence and peculiar properties of interfaces \cite{Ghafari2012,Ghafari2012c,Witte2013,Wang2016,Wang2017}. Using high energy \gls{xrd}, it was found that the number of Fe nearest neighbour atoms in the interfacial regions of the Fe$_{90}$Sc$_{10}$ \gls{ng}s is lower than in the corresponding \gls{rq} \gls{mg}s \cite{Ghafari2012c}. \par

It was also demonstrated that the magnetisation curves of the NGs vastly differed from RQ MGs. While the Fe$_{90}$Sc$_{10}$ MG was paramagnetic at 300 K, the corresponding NG was ferromagnetic with a paramagnetic component (See Figure~\ref{f:ng-evidence}b), indicated by its magnetisation not fully saturating even at 4 T \cite{Witte2013}. Like in reference \cite{Jing1989}, once again the \gls{ms} experiment, shown in Figure~\ref{f:ng-evidence}c, 
indicated a heterogeneous magnetic structure in the Fe$_{90}$Sc$_{10}$ NG (there was an occurrence of two sub-spectra: a paramagnetic component and a ferromagnetic sextet). It was also proven that these fascinating properties of the \gls{ng} were observed only after the nanoparticles were compacted, and not as a loose powder, strengthening the idea that the formation of interfaces resulted in the special properties. The concept of interfaces is illustrated in Figure~\ref{f:ng-core-int}, highlighting the idea that the cores and interfaces, while both amorphous, are structurally distinct from one another. \par

\begin{figure}[!h] \centering
	\includegraphics[width=0.65\linewidth]{core-int.jpg}
	\mycaption{Proposed core-interface structure of nanoglasses}{The processing of a \gls{ng} is hypothesised to form the two core and interface glassy phases---both of which resemble the amorphous structure of \gls{rq} \gls{mg}s in terms of exhibiting no \gls{lro}. However, the \gls{sro} and \gls{mro} of core and interface differ from one another.
	\reprintfig{Ivanisenko2018}{2018}{Wiley}.}
	\label{f:ng-core-int}
\end{figure}

The Fe$_{90}$Sc$_{10}$ NG was further investigated with electron energy loss spectroscopy, which indicated a surface segregation of Fe in the glassy nanoparticles. The surface segregation leads to a chemical heterogeneity in the structure of Fe$_{90}$Sc$_{10}$ \gls{ng} \cite{Wang2016}. Such a surface segregation was explained to arise from both the difference in surface energies of Fe and Sc, and the enthalpy of mixing. A segregation model in the Fe-Sc \gls{ng}s was proposed based on experimental results of small- and wide-angle X-ray scattering \cite{Wang2017}. \par

Recently, \cz NGs were also prepared using magnetron sputtering as opposed to thermal evaporation \cite{Nandam2017,Nandam2018}. These \glspl{ng} were characterised to exhibit higher \gls{tg}, and consequently, higher thermal stability as compared to the \gls{rq} \gls{mg}s. Furthermore, a chemical segregation present in the interfaces facilitated better stability against crystallisation (i.e., higher \glsdesc{tx} \gls{tx}) as well. Nanoindentation studies also revealed that the NGs demonstrated better plastic behaviour (homogenous deformation as opposed to formation of shear bands), and also a higher hardness (Young's modulus) than \gls{mg}s. The \gls{stz} of \glspl{ng} were also measured to be \sim3.8 times larger in volume as compared to melt-spun MGs. The segregation of Cu to the interfaces in the NG, as evidenced by \gls{apt}, was used to explain the enhancement of \gls{tx} and \gls{tg} in the \gls{ng}s. \par

\subsection{Simulation Models}
In the previous subsection, the progress made in the \gls{ng} experimental investigations was discussed. To aid these reports, several independent \gls{md} simulation models of \gls{ng} have been made, predominantly by the groups of Karsten Albe \cite{Krasnochtchekov2003, Sopu2009, Ritter2011, Adjaoud2016, Adjaoud2018, Adjaoud2019, Adjaoud2020, Kalcher2017}, Jason Trelewicz \cite{Cheng2019,Cheng2019a} and Paulo S Branicio \cite{Adibi2014,Sha2017,Zheng2020,Zheng2021}. A first report of \gls{ng} simulations was made by \textcite{Sopu2009}, which demonstrated a compaction protocol for glassy Ge nano-sized spheres at 300 K temperature. The investigators observed that the interfaces formed with lesser densities than the cores. However, their attempts at simulating a \cz NG at 300 K failed as the interfaces completely vanished after compaction, indicating that the mechanical and diffusional properties of the materials influenced the stability of interfaces. \par

Later, in 2011, \textcite{Ritter2011} were successful in simulating a planar glass-glass interface in \czsix \gls{mg}s, which were also stable in \gls{md} timescales. The interfaces were characterised with defective local \gls{fi} \gls{sro}, and present with a 1-2\% excess free-volume. The interfaces were also found to promote shear band formation during tensile deformation, indicating a mechanism to explain enhanced plastic deformation in \gls{ng}s. 
Such planar glass-glass interfaces were further studied in \czsix and Pd$_{80}$Si$_{20}$ systems, also with an elemental surface segregation model \cite{Adjaoud2016}. It was found that segregated interfaces were found to be better packed as compared to ordinary interfaces, with higher topological \gls{sro}, number and electron densities. The interfaces in NGs can hence be 
considered as a structurally stable interphase between the core regions. In the same work, a model was proposed to simulate nanoparticles (precursors to \gls{ng}s) with chemical segregation. The nanoparticles, derived from cutting a sphere out of a glassy bulk and subsequent heat treatment, were found to have similar energetic states as a cluster derived from an \gls{igc} simulation \cite{Adjaoud2016}. The cohesive energies of the Zr in \cz, and Si in Pd$_{80}$Si$_{20}$ were found to drive their segregation to the core regions, while the complementary elements segregated to the surface. This method of deriving a glassy cluster from the bulk of an \gls{rq} also offers the advantage of controlling the nanoparticle size, as compared to other successful \gls{igc}-based models reported for generating Ge clusters 
\cite{Krasnochtchekov2003} and \czsix clusters \cite{Zheng2020}. \par

A contemporaneous work to the nanoparticle segregation model discussed above \cite{Adjaoud2016} was an investigation by \textcite{Danilov2016}, that reported a model for \gls{igc} of nanoparticles of an amorphous Kob-Anderson system \cite{Kob1995}. This model, which described atomic forces in a \gls{lj}-like manner substantiated the presence of a distinct chemical segregation in the cores and shells of the \gls{igc}-nanoparticles. Furthermore, a compaction model was also proposed to create NGs from the amorphous \gls{igc}-nanoparticles. The Kob-Anderson model of \gls{igc}-nanoparticles, and the \gls{ng}s made from them were found to demonstrate an enhanced thermal stability in comparison with the \gls{rq} MGs \cite{Danilov2016}. \par

\textcite{Adjaoud2018} further improved the realistic simulation of the NG-compaction model in \czsix and Pd$_{80}$Si$_{20}$ systems by consolidating a polydisperse distribution of nanoparticles, with an empirically derived \gls{eam} interatomic potential. Further details of \gls{eam} are given in Section~\ref{s:ffs}. The consolidation pressure was shown to have significant influence on the porosity and microstructure of the \gls{ng}s. At $\sim$5 GPa of pressure, the porosity was seen to considerably disappear. Unlike in the work of \textcite{Sopu2009}, the temperature of the system was chosen to be 50 K. The closing up of pores was accompanied by the formation of interfaces, indicated by the local shearing of surface atoms of the nanoparticles. The interfaces constituted of surface atoms of the nanoparticles, and possessed defective \gls{sro} compared to the cores (for the \czsix and Pd$_{80}$Si$_{20}$ systems). In \czsix NGs, the presence of chemical segregation in the nanoparticles led to the formation of interfaces which were chemically heterogeneous from the core regions. This heterogeneity from the segregation was found to be more energetically favourable in the \czsix nanoparticles, than in the interfaces. \par

Similar NG models were made by \textcite{Cheng2019,Cheng2019a} in which a compaction model for a monodisperse distribution of nanoparticles in a patterned arrangement was described, much like the work of \textcite{Sopu2009}. In \czsix glasses, the interface SRO and width in NGs was shown to increase with the temperature at which the nanoparticles were consolidated, and even 
annealing temperatures after consolidation \cite{Cheng2019,Cheng2019a}. In these works, the interfaces remained stable even up to 800 K temperature\footnote{This temperature is close to \czsix \gls{tg}, although it is not clear if this temperature is above or below the simulated 
glass-transition temperature, which is shown to vary with the choice of the \gls{eam} potential \cite{Mendelev2009,Mendelev2019}.}, hinting that the attempts of \cz NGs by \textcite{Sopu2009} could have failed due to lack of availability of a good interatomic potential. Reduced average flow stress and shear localisation factor \cite{Shimizu2007} were observed in the 2.5-nm and 
5-nm grain size NGs, hinting at a stable plastic flow and scaling of elastic properties with nanoparticle size. \par

Another alternative means to simulating NGs that exists in literature is the tessellation-based approach. The previously discussed works fill a box with nanoparticles and compact them \cite{Sopu2009,Adjaoud2018,Adjaoud2019,Cheng2019,Cheng2019a}. In the tessellation approach, first a Poisson-Voronoi tessellation is applied to segment the simulation box, and then segmented partitions are filled with amorphous glassy ``grains" \cite{Adibi2014,Sha2017,Ma2020,Zheng2021}. An external hydrostatic pressure is applied on this configuration. The resulting grain microstructure tends to resemble that of nanocrystalline materials albeit with amorphous grains. While this interesting approach was used to explain some mechanical behaviour of the NGs, it is quite different from the compaction model and also from the experiments. Next, the ``amorphous grains" are chunks of 3 nm particles on average, which are aspherical and  fill the space efficiently, rendering the grains to not deform realistically as would the spherical clusters. These tessellation-based NG simulations are not discussed in the thesis beyond this section. \par

\section{Cluster-assembled Metallic Glasses (CAMGs)} \label{s:camgs}

\subsection{Cluster-ion Beam Deposition and Cluster-assembled Materials (CAMs)}
Advances in research on nanocomposites such as \gls{ncm}s and \gls{ng}s inspired the exploration of \gls{cam}. This new class of materials are built from clusters of atoms, in \gls{uhv}, and have been realised in experiments \cite{Takagi1986,Takagi1988,Beuhler1986}. One of the initial works from 1991 reported the development of a \gls{cibd} apparatus. This apparatus epitaxially deposited a beam of cluster ions with a large size range (25-1600 atoms) onto Si substrates. The ion guided deposition process was successful in preparing films mostly from the smaller clusters in the distribution \cite{Brown1991}, as only a few of the larger clusters were ionised. \par

\gls{cam} were also prepared (and simulated) by low energy deposition of atomic clusters \cite{Perez1997}, resulting in highly porous films which grow in fractal-like patterns due to random stacking. Another extensive review of \gls{cam} discusses the cluster-assembly method for tailorability of superconductivity or charge transport in fullerene-based materials, for the tunability of bandgaps in cluster-based semiconductors, and for the devising of superlattices of size-selected inorganic nanocrystals to enhance conductivity and charge mobilities \cite{Claridge2009}. Although \gls{cam} have been studied by various research groups, the detailed mechanisms that affect films deposited from \gls{cibd} sources are still unclear from the experiments. \par

Computer simulations of ion-beam depositions provide preliminary insights into mechanisms of cluster deposition. The first theoretical studies on \gls{cibd} linked cluster deposition energy to the epitaxy of the films, and suggested that the density of the films increases with impact energy \cite{Muller1987,Cleveland1992}. Succeeding atomistic deposition simulations of covalent systems correlated change in bonding and lattice distortions with deposition energy \cite{Albe1998}. Simulations of molybdenum-cluster depositions provided insights into the morphologies adopted by the \gls{cam} \cite{Haberland1993,Haberland1995}. More recent deposition simulations of Cu clusters onto a Si substrate also exist \cite{Hwang2012,Gong2012}: describing the variation of film epitaxy, stress, and surface roughness with the deposition energy of cluster-assembly. However, a deeper understanding of the local atomic structures adopted in metallic \gls{cam} is still lacking. \par

\subsection{State of Art: Size-selected CAMs and CAMGs} \label{s:camg-uhv}
As seen in the previous section, a control over the local chemistry, morphology, and even microstructure in nanocomposites can be envisioned with precise cluster-assembly. Kartouzian et al. \cite{Kartouzian2013,Kartouzian2014} synchronised the idea of \gls{cam} with metallic glasses, and conceptualised the idea of \acrfull{camg}. By designing \gls{mg}s meticulously with size-selected clusters as building blocks, Kartouzian expounded the possibility to better understand the structure of \gls{mg}s. His proposed method was successful in creating an amorphous \gls{camg}, but it was presented with no discernible features and no concrete method of investigating them. The quest of preparing \gls{camg}s as a means to fabricate fully tailorable amorphous nanocomposites remained unexplored. Aided by experience from \gls{ng} experiments, the synthesis of size-selected \gls{cam} and \gls{camg}s for the first time was achieved by the group of Hahn with the development of a unique \gls{uhv} apparatus to perform \gls{cibd} \cite{Fischer2015,Fischer2015a,Benel2018}. 

\begin{figure}[!ht] \centering
	\includegraphics[width=0.5\linewidth]{cams.png}
	\mycaption{Various CAMs made possible by UHV CIBD}{Illustrations of the types of nanocomposites. The cluster-assembly can be used to create (a) purely cluster-composed films, or (b) cluster decorated surfaces. (c) With high impact energies, surface alloys can be created. (d) By co-deposition, clusters can be embedded in a matrix material.} 
	\label{f:camg-sch}
\end{figure}

Figure~\ref{f:camg-sch} depicts a variety of cluster-based nanocomposites that can be synthesised in the UHV machine. The \gls{cam} are made by depositing crystalline or amorphous nanoparticles onto a substrate placed within the UHV CIBD apparatus\footnote{Further details regarding the UHV apparatus are discussed in detail in the works of Fischer \cite{Fischer2015,Fischer2015a}}. A purely cluster-composed \gls{cam} (Figure~\ref{f:camg-sch}a) can be made with both monodisperse and polydisperse clusters. The clusters may be deposited with a low deposition energy, to create cluster decorated surfaces (Figure~\ref{f:camg-sch}b), or directed with high energies onto a substrate to make surface alloys (Figure~\ref{f:camg-sch}c). Furthermore, in the UHV CIBD apparatus, clusters 
can be co-deposited along with \gls{pvd} of a matrix material from a thermal evaporation source. In doing so, clusters deposited at low energies can be embedded in matrices (Figure~\ref{f:camg-sch}d). The co-deposition serves an additional function i.e., to deposit a capping layer on any kind of the \gls{cam} as protection against oxidation. Cluster-composed matrices \cite{Fischer2015a,Benel2018,Gack2020} have been shown to demonstrate tailorable magnetic properties by varying cluster-size and cluster-concentration in the matrices. The embedding of clusters in a matrix at low deposition energies also promises the potential to engineer multiphase nanocomposites with elements that are known to be miscible in thermodynamic 
equilibrium. Conversely, at high deposition energies, an immiscible system can be made miscible. \par

%\clearpage

\subsection{Initial Studies on CAMGs}
As mentioned before, CAMGs are a special class of the CAMs. Initial attempts to synthesise CAMGs were by the deposition of 10-16 atom sized clusters of varying composition, to create a locally heterogeneous chemical structure. Although the films made were characterised to be amorphous, a more detailed study of structure–property relationships was not reported 
\cite{Kartouzian2013,Kartouzian2014}. \par

\begin{figure}[!h] \centering
	\begin{subfigure}{0.5\linewidth} \centering %\addtocounter{subfigure}{-2}
		\includegraphics[height=11cm]{xanes}
		\subcaption{XANES}	
	\end{subfigure}%
	\hfill
	\begin{subfigure}{0.5\linewidth} \centering
		\begin{subfigure}{\linewidth} \centering
			\includegraphics[height=6cm]{exafs}
			\subcaption{EXAFS}
		\end{subfigure}%
		\vfill
		\begin{subfigure}{\linewidth} \centering %\addtocounter{subfigure}{+1}
			\includegraphics[height=5cm]{m_t.png}
			\subcaption{$M(T)$ vs $T$}
		\end{subfigure}%
	\end{subfigure}%
	
	\mycaption{Experimental characterisation of \fs CAMGs}{(a) The XANES spectra show evidence of no oxidation. (b) The EXAFS spectra of CAMGs are distinct from that of the amorphous \fs RQ MG ribbon. (c) Curie temperature (\gls{tc}) varies with deposition energy in CAMGs. \reprintfig{Benel2019}{2019}{Royal Society of Chemistry}.}
	\label{f:benel-camg}
\end{figure}

Previous experiments on Fe-Sc NGs made in the Hahn group \cite{Witte2013,Ghafari2012}, however, confirmed that the magnetism studies can shed light on the structural information. Hence, the study of size-selected \fs CAMGs was a necessary step in understanding the nature of amorphous structures made from cluster assembly \cite{Benel2019}. \fs CAMGs of $\sim$800 atom sized 
\fs clusters were prepared at three different deposition energies (50, 100, 500 eV per cluster), and coated with a Mg layer to prevent oxidation. The samples were studied by \gls{edx} to confirm chemical compositions. Although the Mg capping layer posed a challenge to characterise the CAMG films with \gls{tem}, the samples were found to be amorphous with synchrotron \gls{xrd} \cite{Benel2019}. X-ray absorption spectroscopic studies were conducted to ascertain the average local atomic structure of the constituent atoms. \par

The structural and magnetic characteristics of \fs CAMGs are described in Figure~\ref{f:benel-camg}. X-ray absorption studies were performed at a synchrotron facility to study the fine structure in the extended (\gls{exafs}) or the near-edge region (\gls{xanes}). The normalised Sc K-edge XANES spectra of the three CAMGs, an RQ MG ribbon of the same composition, pure metallic Sc, and Sc$_{2}$O$_{3}$ are all depicted in Figure~\ref{f:benel-camg}a. The K-edge, i.e, the first inflection point in the energy absorption spectra of the CAMGs and the MG were found to be lower than the Sc$_{2}$O$_{3}$ K-edge, indicating that the CAMG samples were not oxidised\footnote{The X-ray absorption energy 
increases with increase in oxidation state.}. Furthermore, the spectra in the CAMGs and the MG ribbon were observed to be dissimilar, hinting at differences in local structures. \par

In Figure~\ref{f:benel-camg}b, the Sc K EXAFS spectra of the three \gls{camg}s and the \gls{rq} \gls{mg} ribbon are compared with pure metallic scandium. The fading of EXAFS oscillations in k-space was identified as an indicator of amorphous structure. The oscillations in the EXAFS spectra of the three \gls{camg}s differ from that of the \gls{mg} ribbon, indicating once again that the local structures in the cluster-assembled samples are different from \gls{rq} \glspl{mg}. \par

The magnetic behaviour of four \fs CAMGs of impact energies 50, 100, 200, and 500 eV per cluster are described in Figure~\ref{f:benel-camg}c.  Firstly, a paramagnetic to ferromagnetic transition was observed in all samples. When the impact energy is reduced from 500 eV per cluster to 50 eV per cluster, the magnetic transition temperature or the Curie temperature (\gls{tc}) shifts by 60 K. This result was surprising, as only a change in the impact energy for the \glspl{camg} can result in significantly different \gls{tc}. Based on the increase in nearest neighbour distances from the EXAFS data, \textcite{Benel2019} speculated that the increase of \gls{tc}, which is known to correspond to an increased strength of the magnetic exchange interactions, was a result of an increase in the number of the Fe nearest neighbors. The study of \fs CAMGs prepared via CIBD has demonstrated the opportunity to tailor the local atomic structure and the \gls{tc} by varying the deposition energy. The astounding promise of the CAMGs is the variation of properties in a glass while keeping the macroscopic composition constant.

%So far we have discussed the conception of \gls{mg}s, and some of the important reports that have led to understanding them better. Physical way of changing properties will be difficult in Mgs 

%\begin{figure}[!h] \centering
%	\includegraphics[width=0.5\linewidth]{cibd.jpeg}
%	\mycaption{Schematic of the UHV CIBD Apparatus}{A visual representation of the UHV cluster deposition apparatus, with the cluster source at Stage-a, and two deposition stages (Stage-c and Stage-g). The manipulation of the cluster-ion beam is made with electromagnetic lenses and magnets at Stages b, e, and f. %See Section~\ref{s:camg-uhv} for more details.
%	\reprintfig{Fischer2015}{2015}{AIP Publishing LLC}}
%	\label{f:cibd-uhv-sch}
%\end{figure}
%Figure~\ref{f:cibd-uhv-sch} describes a schematic of the \gls{uhv} CIBD experiment. \par

%The complete details of the UHV apparatus can be found in the Ph.D. thesis of \textcite{Fischer2015a}. However, some relevant details are described below. The cluster ions are generated (Figure~\ref{f:cibd-uhv-sch}a) using a Haberland source \cite{Haberland1991}, and passed through an adjustable \gls{igc} aggregation chamber to grow a controllable size based on inert-gas (He/Ar) flow rates and the chamber length. Based on the chosen target material in the cluster source, there is a possibility of growing either crystalline or amorphous clusters. The aggregation chamber opens into a supersonic expansion, allowing the clusters to leave without further growth. The clusters are then accelerated and guided as a cluster ion beam using electromagnetic lenses Figure~\ref{f:cibd-uhv-sch}b. The clusters so-synthesized could be characterized by a \gls{tof} (Figure~\ref{f:cibd-uhv-sch}d). The clusters synthesized at Stage-d (Figure~\ref{f:cibd-uhv-sch}d) were found to be present with a mass distribution \cite{Fischer2015a}, which were shown to vary with the He flow-rates \cite{Fischer2015a}. \par
%
%The clusters were also found to be roughly 50\% neutral and 50\% ionized. The ionized clusters were found to be either singly negatively or singly positively charged. Using an electric field at Stage-c, the negatively charged clusters were chosen to be guided towards a substrate (Figure~\ref{f:cibd-uhv-sch}c1). This is the first deposition stage in the experiment. An expanded view of the first deposition stage (Stage-c) is depicted in Figure~\ref{f:cibd-uhv-sch}c2. The cluster beam was also found to be Gaussian in shape. To ensure a uniform film deposition, the electric field was swept across the substrate. A sample holder/mask contraption was designed to also ensure cluster-deposition onto the substrate only within a limited area of the sweeping cluster-ion beam. The cluster ion beam, when not diverted to the substrate at Stage-c, passes through Stage-e (Figure~\ref{f:cibd-uhv-sch}e), which consists of a quadrupole magnet and a Faraday cup to monitor the size-charge distribution. A $90 ^{\circ}$ sector magnet at Stage-f (Figure~\ref{f:cibd-uhv-sch}f), acts as a mass selector for a more precise mass selection of the cluster ion beam. The mass-selected beam is guided by means of another Stage-e, to then be decelerated and deposited at Stage g (Figure~\ref{f:cibd-uhv-sch}g1). This is the second deposition stage in the UHV CIBD experiment. The Stage g is shown greater detail in Figure~\ref{f:cibd-uhv-sch}g2. The deceleration lenses slow down the beam to allow deposition at a desired velocity. A thermal evaporation source is also available to facilitate a co-deposition along with the cluster deposition. \par

%Sometimes, such treatments %may lead to undesirable effects, such as possible phase separation in the %glassy systems. For instance, in binary immiscible systems with a miscibility %gap, the single phase of a glass can phase separate into two phases, whether %by %classical nucleation (when the separating phases are distinct from one %another) %or by spinodal decomposition (if the two phases have a high volume fraction) %\cite{Cahn1961}. Another possibility is also that of surface crystallisation %or %crystalline nucleation, which can drastically change the structural order of 
%the glasses. \par

%\subsection{Functional properties}\label{s:props-mgs}
%Metallic glasses are used for a variety of purposes today. The lack of defects %leads to high strengths and when combined with their low densities finds %applications in aerospace materials. Their subsequently low elastic modulus %results in high stiffness, which when combined with their high hardness which %is useful for industrial machine components. Some Fe and Co-based metallic %glasses are also utilised for soft magnetic properties. Owing to the lack of a %sharp reduction in volume during the liquid-to-solid transition, in comparison %to crystalline materials (See Figure~\ref{f:rq-mg-sch}b), metallic glasses can %be using in casting and thermoplastic forming applications. Some Ti- and Zr- %based \gls{bmg}s are biocompatible, and offering the possibility high strength %and wear resistant medical implants. \par
