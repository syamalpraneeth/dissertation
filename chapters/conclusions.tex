\chapter{Conclusions} \label{c:conclusions}

\section{Summary}
In this dissertation the novel \cz \gls{camg}, prepared by the energetic deposition of amorphous nanoclusters, has been studied by means of \gls{md} simulations. The careful cluster assembly is expected to give rise to new amorphous structures different from the conventional \gls{rq} \gls{mg} of the same composition. The \gls{camg} structure is also expected to be different from the \gls{ng} produced by mechanical compaction of amorphous clusters. Understanding the formation mechanisms and final structures of the so-obtained \gls{camg} opens up possibilities to design and control amorphous structures. The key results obtained in this doctoral work are summarized below. \par

\begin{enumerate}[leftmargin=*]
\item \textbf{First virtual insights into \gls{camg}s}
%\item \textbf{Development of a simulation protocol for \gls{camg}s}
\begin{enumerate}[leftmargin=*]
%\item \textbf{Synthesis of a 3 nm \cz cluster}\\
%%\item \textbf{Deposition of single 3 nm clusters hint at morphologies adopted in CAMGs} \\
%Amorphous \cz clusters (3 nm in diameter) are first derived from the bulk of an \gls{rq} \gls{mg}, followed by a heat-treatment (by the method of \textcite{Adjaoud2016}) to induce a structure that is expected from experimental \gls{igc} preparation. The heat-treated 3 nm cluster is found to have a radial compositional variation, with a core-shell structure. The Cu-concentration is \mbox{56 \%} in a 0.2 nm thick shell region, and \mbox{44 \%} in the core. An \gls{igc} model to prepare clusters is also briefly discussed, however, the bulk-derived clusters are utilized in this dissertation.

\item \textbf{Development of a simulation protocol for cluster deposition} \\
First, a single 3 nm cluster, derived from the bulk of a \gls{rq} \gls{mg} followed by a heat-treatment to induce a structure that is expected from experimental \gls{igc} preparation. This cluster was deposited at various impact energies ranging from 6-6000 meV/atom onto a \cz substrate. Its shape and deviation from its undeposited state were noted to change with impact energy. At low energies (6-60 meV/atom) the cluster adopted a nearly spherical morphology, while at medium impact energy of 300 meV/atom the cluster is disorted further although still adopting a convex drop-like shape. At deposition energies higher than 600 meV/atom the cluster begins to turn concave-like, embedding itself more into the substrate. The cluster core-shell structure disintegrates in the extreme landing (6000 meV/atom energy) case. \par

\item \textbf{Simulating \cz \gls{camg}s} \\
The protocol developed for cluster deposition was extended to simulate \gls{camg}s. It was chosen to deposit 3 nm size \cz clusters in a \gls{hcp} pattern to remove surface artefacts, maximize cluster-cluster interface formation, and improve simulation efficiency. Upon assembling the \gls{camg}s from the 3 nm sized clusters, it was found that the chemically segregated core and shell structure of the undeposited clusters gave rise to two chemically distinct Zr-rich cores and Cu-rich interfaces phases; with the core regions being embedded within an interconnected network of the cluster-cluster interfaces. The interfaces were found to be stable in \gls{md} timescales for impact energies up to 600 meV/atom, however they disappear at extreme landing energy of 6000 meV/atom energy. \par

\item \textbf{Short-to-medium range order tailoring in \cz \gls{camg}s}\\
The \gls{fi} and \gls{ilo} evaluated by Voronoi tessellation were used to describe the \gls{sro} of the glasses. Strings of \gls{fi}-atoms are used to indicate \gls{mro}. The \gls{sro} of the \gls{ng} and \gls{camg} are found to differ from the \gls{rq} \gls{mg}s, irrespective of the quench rate associated with the bulk-derived clusters. In the conventional (\qr{10} quench rate) case, \gls{sro} is lower in the \gls{ng} and \gls{camg} as compared to \gls{rq} \gls{mg}s, for both the cores and interfaces. However, irrespective of the quench rate used, both the \gls{sro} and \gls{mro} in \gls{camg}s recover towards the metallic glass values with increasing impact energies. As a result, adjusting the processing conditions in \gls{camg}s makes it possible to control the local structure of metallic glasses.
\end{enumerate}

\item \textbf{Cluster-size effects in \gls{camg}s and \gls{ng}s}\\
An influence of cluster size on the interplay between interface and cores regions was expected, motivating an investigation of further changes within the \gls{camg} structure.

\begin{enumerate}[leftmargin=*]
\item \textbf{Cluster deposition: 3 nm cluster vs 7 nm \cz nanoparticle}\\
Large 7 nm sized \cz nanoparticles were derived from the bulk, in an identical procedure as for the 3 nm clusters. The 7 nm nanoparticles also showed a core-shell structure, with a shell of 0.3 nm thickness. The Cu-enrichment in the nanoparticle shell was 52 \%, which is 4\% lesser than in the 3 nm cluster. Upon testing the single nanoparticle deposition, the 7 nm nanoparticle was found to retain its morphology better at any given impact energy as compared to the 3 nm cluster. Like in the 3 nm cluster case, the shell atoms of the 7 nm nanoparticles were also found to deviate from their original positions, characterized by the local atomic strain. The amount of deviation was higher at any given energy for the 7 nm nanoparticle. At the deposition site, the atoms were found to increasingly shear (with respect to their un-deposited states) with deposition energy.
%The starkness ??? of the cluster reduces with impact energy, but the amount of reduction is lower for larger clusters. For 7 nm clusters, until 600 meV/atom energy. At the deposition site, the atoms were found to shear, and progressively with deposition energy.
At a given energy, the larger 7 nm nanoparticle was found to cause larger volume of atomic-level deformation at the site of deposition, hinting at formation of larger interfaces during the \gls{cibd} process. \par

\item \textbf{Local order in 7 nm \cz CAMGs}\\
Like the 3 nm CAMGs, the CAMGs from 7 nm nanoparticles were also deposited in a \gls{hcp} pattern. As expected from the single cluster deposition, the interfacial width was larger, and at 60 meV/atom energy, the nanoparticles could not deform enough, leaving large pores in the sample. However, at higher energies, the pores closed up and the \gls{sro} of the 7 nm CAMGs increased with deposition energy. At a fixed impact energy, the \gls{sro} and the average energetic states of CAMGs were found to be invariant with the cluster-size. The difference in the local chemical heterogeneity introduced from the precusor clusters into the simulated CAMGs, brought about a significant variation in the \gls{sro} and the average P.E./atom within the core and interfacial regions with cluster size. 

\item \textbf{Building \gls{ng}s with blocks ranging from atomic clusters to large nanoparticles} \\
Monodisperse clusters of diameters ranging between 1-7 nm were arranged randomly before compaction to prepare \cz NGs. Lowering the cluster size was observed to lower the interfacial width as in the CAMGs. The number of cluster atoms participating in readjustment at the interfaces upon compaction and annealing was also found to increase with decreasing cluster sizes. The change in cluster size appears to not have a significant influence on the \gls{sro}, average energetic states, and the thermal behaviour in the simulated NGs. Regardless, the pronounced increase in readjustment of the interfacial atoms with reduced cluster size offers scope for interesting local structural changes not observed in the present simulations. A deeper investigation is required to better understand these novel glasses. 
\end{enumerate}
\end{enumerate}

The establishment of an \gls{md} simulation protocol for \gls{camg}s and the study of their local characteristics now lends support to the idea of controllable local order and microstructures in metallic glasses. There is an immense need for future work on this class of novel materials to better understand them and harness their properties.

\clearpage
\section{Outlook}
The primary contributions of this work are the initial \textit{in silico} investigations on the cluster-assembly of metallic glasses, as discussed above. While some preliminary questions have been answered, the quest to fully understand \gls{camg}s and the consequences of their tailorability of their local order remains to be explored.

\begin{enumerate}[leftmargin=*]
\item \textbf{Thermal stability of CAMGs made from vapour-condensed clusters}\\
In a previous study, \textcite{Danilov2016} reported the thermal ultrastability of \gls{ng}s from \gls{igc}-derived nanoparticles, as compared to \gls{ng}s from bulk-derived nanoparticles, also also conventional \gls{rq} \glspl{mg}. This remains to be explored in \gls{camg}s. The \gls{igc}-like vapour condensed cluster growth method described in Chapter~\ref{c:dev} can be combined with gas-phase condensation methods \cite{Krasnochtchekov2003,Zheng2020} to prepare spherical \cz clusters. It was also determined in the current work, that the cluster assembly of amorphous nanoclusters leads to the increase of local \gls{sro} with impact energy. The stability gained from the use of \gls{igc}-derived nanoparticles as building blocks, compounded with the increase in \gls{sro} with the deposition energy can result in a synergetic effect to form thermally ultrastable \gls{camg}s.

\item \textbf{Isolating the role of cluster-deposition processing in CAMGs}\\
In the present thesis, a heat-treatment was given to the clusters to allow the cluster-atoms to diffuse, enabling a chemical segregation. The phenomena observed in \gls{camg}s, discussed in Chapters~\ref{c:dev}, \ref{c:camg} and \ref{c:cbmg} are a result of the deposition process and also the chemical segregation. A comparison of \gls{camg}s made from unsegregated clusters can help isolate the role of the deposition processing. Moreover, the difference in local packing and density in comparison to the segregated interfaces and unsegregated interfaces can lead to interesting changes in properties of CAMGs.

\item \textbf{Novel mechanical properties of CAMGs} \\
The tailorable local structure of the CAMGs may result in exciting properties. Previously, reducing the cluster size of \gls{ng}s, has been shown to reduce flow stress \cite{Adibi2014,Cheng2019a} of the material. This change in property was driven by the reduction of interface width and simultaneous increase of interface volume---the nature of the interface plays an important influence on the mechanical properties of bottom-up glasses. Consequently, comparing the mechanical properties of \gls{camg}s made from unsegregated and vapour condensed clusters via nanoindentation studies, specifically observing the stress-strain relations, and strain localization and propagation in the two kinds of CAMG interfaces would be interesting.

\item \textbf{CAMG simulations at mesoscopic timescales}\\
A challenge encountered in this dissertation was the formation of large pores in the \gls{camg}s when the clusters were deposited randomly (discussed in Chapter~\ref{c:dev}). It is very possible that the clusters at long experimental times scales, may reach their preferred states which result in the pores closing up. The short simulation times achieved by \gls{md} thereby hinder the exact replication of \gls{camg} experiments. A recent technique called the \gls{abc} has been developed to traverse the \gls{pel} to general atomic trajectories in the timescale of seconds \cite{Cao2012,Fan2018}. This basin-hopping method finds immense value in \gls{camg} simulations to access mesoscopic timescales and better model the experiments.

\item \textbf{Investigating crystalline \gls{cam}}\\
Recently, polycrystalline materials of small nm-sized grains have been shown to exhibit minimal-interface configurations, exhibiting higher strengths and thermal stability \cite{Li2020,Hu2022} in contrast to the expected \gls{ihp} behavior. The simulation protocols developed within the framework of this thesis are already being applied to simulate \gls{cam} from nm-sized crystalline clusters in the group of Prof. Penghui Cao, University of California-Irvine, to test the \gls{ihp} relationship with cluster-assembly.

\item \textbf{Controlling chemical heterogeneity in \gls{camg}s}\\
As mentioned in the \nameref{c:theory} chapter, one of the first works on \gls{camg}s reported a new amorphous class of materials built from chemically distinct clusters 10-16 atoms in size, in an attempt to engineer a locally heterogenous structure in \gls{mg}s \cite{Kartouzian2013,Kartouzian2014}. Such exotic \gls{camg}s have not been investigated beyond synchrotron surface \gls{xrd} measurements. In Chapter~\ref{c:cbmg}, the possibility of simulating locally heterogenous clusters is briefly discussed. In the present thesis, the simulation of CuZr clusters of varying compositions with $\sim$30 atoms each were not stable upon deposition despite the impact energies being much lower than the cohesive energy of the glassy solid. Presently, the dissolution of the small clusters is attributed to the cohesive forces from the \cz substrate used, and hence using a Si substrate may offer more success. The \gls{camg}s of a given macroscopic composition may then be simulated at soft-landing energies with varying local heterogeneity. The resulting systems could be investigated by their local \gls{sro} and \gls{mro}, but also by the phonon density of states, which is sensitive to the local atomic structure.



%\item \textbf{Soft spots and vibrational modes}\\
%$\alpha$ relaxation, $\beta$ relaxation, PDOS (out of scope of thesis)
%Thermal stability, potential energy states, further expanding our knowledge of metallic glasses in general
%\gls{gum} in \gls{camg}s could be discussed \cite{Ding2014}
\end{enumerate}


