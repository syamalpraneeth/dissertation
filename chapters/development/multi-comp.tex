\section{Modelling Cluster-compacted Glasses}

\begin{selfcite}
One of the aims of this study is to compare CAMGs to metallic glasses prepared by mechanical compaction, i.e., NGs. The results of simulations of CAMGs and NGs using the same clusters as building blocks allows a comparison of the different processing techniques, compaction for NGs and energetic impact for CAMGs. Furthermore, the structure of simulated NGs prepared by compaction of clusters in the size range of 800 atoms has not been reported. For the simulation of the cold compaction, the clusters were inserted in a simulation box and compacted at 50 K temperature under 5 GPa pressure to yield a NG of ∼ 300,000 atoms. In previous works, the compaction of NGs was modelled by inserting the clusters at random positions before compaction [18,29] as this method closely resembles the actual experiments conducted to obtain NGs. The properties of such NGs are described in further detail in Chapter~\ref{c:cbmg}. \par
%\end{selfcite}

%\begin{selfcite}
However, when comparing the NGs with CAMGs, the clusters were inserted in a HCP arrangement prior to compaction in order to resemble the arrangement used for the CAMGs. Once the sample was compacted at 50 K temperature and equilibrated, it was unloaded for 0.2 ns and then equilibrated again for another 2 ns. In the NGs with clusters of sizes 3 nm (described in Chapter~\ref{c:camg}) and 7 nm (described in Chapter~\ref{c:cbmg}) prepared in this way, no pores were present, when examined using a surface mesh with a probe sphere radius of 3 Å [22,30].
\end{selfcite}