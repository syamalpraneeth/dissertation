\chapter{Structure and Packing of Cluster-assembled Metallic Glasses} \label{c:camg}
\scdeclaration

One of the primary objectives of the thesis has been to computationally study the \gls{camg}---which are an alternative route to synthesizing metallic glasses. The novel \gls{cibd} method holds promise of creating distinct nanostructures in amorphous materials, and controlling the local amorphous order. The simulations of \gls{camg}s are finally possible due to protocols established in Chapter~\ref{c:dev}. Some primary characteristics of the deposition process were also gathered. In the current chapter, we dive deeper into analysing the simulated \gls{camg}s. First, the deposition of a single cluster on a substrate is studied to understand the behaviour of cluster depositions. Next, the nanostructure and the local amorphous order of the film made from deposition of a plethora of clusters is understood: the simulated \gls{camg}s are characterized by their \gls{rdf}s, short-to-medium range order, their atomic packing, and the energetic states they occupy. By the above metrics, the \gls{camg}s are also contrasted with the \gls{rq} \gls{mg} that they are derived from, also also the \gls{ng}. Thereby, one can isolate the effects of cluster deposition and cluster compaction processes\footnote{It is noted once again that in this thesis, the \gls{ng}s are made with a monodisperse cluster distribution. They serve as a better counterpart to \gls{camg}s, unlike previous polydisperse-cluster models of \gls{ng} \cite{Adjaoud2018}.}. \par

Furthermore, it is explored whether or not the initial state of the cluster has any influence on the final states of the \gls{camg}s (and also \gls{ng}s). The clusters are derived from \gls{rq} \gls{mg}s of varying quench rates, and \gls{camg}s and \gls{ng}s are produced from them. Consequently, an attempt is made to understand the resulting structural and energetical changes in the \gls{camg}s and \gls{ng}s. \par

%\section{Exploring Deposition Energy Ranges} 
\begin{selfcite}
In order to understand the role of the impact energy on the cluster deposition and to identify the range of impact energy of interest for the preparation of CAMGs, the deposition of a single cluster on a substrate was studied initially. The single \cz  cluster, 3 nm in diameter— prepared as described in Section~\ref{s:clus}—was deposited at various energies ranging from 6 meV to 6000 meV per atom. \par

Figure~\ref{f:clus_single3} shows the cross-sections of the clusters deposited at various energies in the YZ plane parallel to the deposition axis. The snapshots were made 2 ns after deposition, as in Section~ref{s:camgdev} the simulation was determined to have converged by this time. The atoms colored in yellow and magenta, belong to the shell and core atoms of the cluster prior to deposition, respectively (as in Figure~\ref{f:clus_rad-3nm}). No distinction is made in this color code for the constituent elements. All substrate atoms are colored in black. Clearly, the morphology of the cluster after impact varies with deposition energy. \par In the energy range of 6-60 meV/atom, the cluster is in a soft-landing state. In this regime, it is observed that even for the lowest deposition energy of 6 meV/atom the cluster loses the original shape of the free cluster, which was almost perfectly spherical. This change of shape is attributed to a partial wetting due to the cohesive forces at the surface between the cluster and the substrate. No noticeable difference in the final shapes is observed for the cases of 6 meV/atom and 60 meV/atom impact energy. As more drastic changes of the cluster shape are observed at higher impact energies, the deposition energy of 60 meV/atom was chosen to be the upper limit for the soft-landed state.

\begin{figure}
	\centering
	\includegraphics[width=0.5\linewidth]{2021_05/figures/2.pdf}
	\includegraphics[width=0.5\linewidth]{2021_05/figures/3.pdf}
	\mycaption{Single 3 nm cluster deposited states:}{ In Figure~\ref{f:clus_single3}(a), the cross sections of snapshots of the clusters 2 ns after the simulated deposition at various per-atom energies ranging from 6 meV/atom to 6000 meV/atom are shown, with categories of soft, medium, hard and extreme hard landing indicated. The core and shell atoms are marked in magenta and yellow colors, respectively. This is the same color scheme used in Figure~\ref{f:clus_rad-3nm}b. In Figure~\ref{f:clus_single3}(b) the as-deposited states of the clusters (curvature and thickness of the embedded clusters) are represented after equilibration for 2 ns after the deposition.}
	\label{f:clus_single3}
\end{figure}

For all simulated cluster impacts, it is observed that the impact energy clearly influences the final states of the shell atoms in the clusters. The change in state of the deposited clusters, was quantified by means of the root mean square deviation of the shell atoms of the radial coordinate of the shell atoms from the average shell radius (RMSD$ _{shell}$) and the radius of curvature of the cluster R$ _{C} $. Figure~\ref{f:clus_single3}b summarizes the RMSD$ _{shell}$ and R$ _{C} $ as a function of deposition energy, with the clusters being equilibrated for 2 ns. \par

The values of RMSD$ _{shell}$ quantify the degree of distortion of the shell atoms from their original positions, which increases monotonically with the impact energy. The shell region stays intact at energies below 600 meV/atom. However, for impact energies ≥ 600 meV/atom, i.e., in the hard-landed state, the distortion of the cluster increases continuously with increasing impact energy. The deposition at 300 meV/atom energy is then defined as the medium-landed state. The separation between soft, medium and hard landing is assigned arbitrarily. However, these distinctions allow us to understand the broad energy regimes in which the CAMGs retain or lose the signatures of the originally free clusters. In the soft-landed state, the clusters in the CAMGs can be expected to remain mostly spherical. At the higher energies, in the medium-landed state, a lot more deformation of the cluster is expected. In the hard-landing state, not only will the cluster be deformed, but the inter-diffusion of the core-shell atoms in the cluster becomes significant. \par

In line with the changes of RMSD$ _{shell}$, the radius of curvature R$ _{C} $ gradually decreases with increasing energy, indicating that the cluster loses its spherical morphology at higher impact energies. At energies $\geq$ 3000 meV/atom, i.e., extreme hard landing, the cluster embeds itself into the substrate during impact, being reflected in a negative R$ _{C} $. With increasing impact energy, the cluster deforms more and embeds deeper into the substrate. In the context of formation of CAMG films, i.e., when multiple clusters are deposited over each other layer-by-layer, intermixing between clusters is expected at the higher impact energies. Based on the results of the single cluster deposition, impact energies between 60-600 meV/atom were chosen to study the formation of CAMG films. In the following section, as part of a first analysis, the changes of the core-shell structures during multiple cluster deposition will be considered.

\end{selfcite}


\section{Exploring Deposition Energy Ranges} \label{c:cibd_single}
\begin{changebar}
In order to understand the role of the impact energy on the cluster deposition and to identify the range of impact energy of interest for the preparation of \gls{camg}s, the deposition of a single cluster on a substrate was studied initially. A single \cz cluster, 3 nm in diameter— prepared from a \qr{10} \gls{rq} \gls{mg} as described in Section~\ref{s:clus}---was deposited at various energies ranging from 6 meV to 6000 meV per atom. \par

Figure~\ref{f:clus_single3} shows the cross-sections of the clusters deposited at various energies in the YZ plane parallel to the deposition axis. The snapshots were made 2 ns after deposition. As seen in Section~\ref{s:camgdev}, the simulation was determined to have converged by this time. The atoms colored in yellow and magenta, belong to the shell and core atoms of the cluster prior to deposition, respectively (as in Figure~\ref{f:clus_rad-3nm}). No distinction is made in this color code for the constituent elements. All substrate atoms are colored in black. Clearly, it is seen that the morphology of the cluster after impact varies with deposition energy. \par

In the energy range of 6-60 meV/atom, the cluster is in a soft-landing state. In this regime, it is observed that even for the lowest deposition energy of 6 meV/atom the cluster loses the original shape of the free cluster, which was almost perfectly spherical. This change of shape is attributed to a partial wetting due to the cohesive forces at the surface between the cluster and the substrate. No noticeable difference in the final shapes is observed for the cases of 6 meV/atom and 60 meV/atom impact energy. As more drastic changes of the cluster shape are observed at higher impact energies, the deposition energy of 60 meV/atom was chosen to be the upper limit for the soft-landed state. \par
	
For all simulated cluster impacts, it is observed in Figure~\ref{f:clus_single3} that the impact energy clearly influences the final states of the shell atoms in the clusters. The change in state of the deposited clusters, was quantified by means of the root mean square deviation of the radial coordinate (with respect to centroid of the cluster) of the shell atoms from the average shell radius and the largest \gls{rc} of the cluster in the deposition axis. The \gls{rc} is calculated as the largest Z-component of the displacement vector subtended by shell atoms in a spherical sector around the deposition axis, to the surface level of the substrate. The method of evaluation \gls{rc} is illustrated in Figure~\ref{f:rccalc}. The \gls{rc} represents the convexity of the aspherical cluster for the which RC varies as a function of the surface. \par

\begin{figure}[!h]
	\centering
	\begin{subfigure}{0.5\textwidth}
		\includegraphics[width=\linewidth,trim={3.8cm 1cm 4cm 0.5cm},clip]{camg_3nm/2.pdf}
	\end{subfigure}%
	\hfill
	\begin{subfigure}{0.5\textwidth}
		\includegraphics[width=\linewidth,trim={4cm 0.5cm 4cm 0.8cm},clip]{camg_3nm/3.pdf}
	\end{subfigure}
	\mycaption{Single 3 nm Cluster Deposited States}{ In (a), the cross sections of snapshots of the clusters 2 ns after the simulated deposition at various per-atom energies ranging from 6 meV/atom to 6000 meV/atom are shown, with categories of soft, medium, hard and extreme hard landing indicated. The core and shell atoms are marked in magenta and yellow colors, respectively. This is the same color scheme used in Figure~\ref{f:clus_rad-3nm}. In (b) the as-deposited states of the clusters (curvature and thickness of the embedded clusters) are represented after equilibration for 2 ns after the deposition.}
	\label{f:clus_single3}
\end{figure}

Figure~\ref{f:clus_single3}b summarizes the \gls{rmsds} and \gls{rc} as a function of deposition energy, with the clusters being equilibrated for 2 ns after deposition. The values of \gls{rmsds} quantify the degree of distortion of the shell atoms from their original positions, which increases monotonically with the impact energy. The shell region stays intact at energies below 600 meV/atom. However, for impact energies $\geq$ 600 meV/atom, i.e., in the hard-landed state, the distortion of the cluster increases continuously with increasing impact energy. The deposition at 300 meV/atom energy is then defined as the medium-landed state. \par The separation between soft, medium and hard landing is assigned arbitrarily. However, these distinctions allow us to understand the broad energy regimes in which the \gls{camg}s retain or lose the signatures of the originally free clusters. In the soft-landed state, the clusters in the \gls{camg}s can be expected to remain mostly spherical. At the higher energies, in the medium-landed state, the cluster is expected to deform further. In the hard-landing state, not only will the cluster be deformed, but the inter-diffusion of the core-shell atoms in the cluster becomes significant. \par

\begin{figure}[!h]\centering
	\includegraphics[width=0.8\linewidth]{radius_curv.png}
	\mycaption{Calculating \gls{rc} of a Deposited Cluster}{The construction used to evaluate the \gls{rc} of single deposited cluster, as depicted in Figure~\ref{f:clus_single3}b.}
	\label{f:rccalc}
\end{figure}%

In line with the changes of \gls{rmsds}, the \gls{rc} gradually decreases with increasing energy, indicating that the cluster loses its spherical morphology at higher impact energies. At energies $\geq$ 3000 meV/atom, an extreme hard landing is observed: the cluster embeds itself into the substrate during impact, being reflected in a negative \gls{rc}. The cluster adopts a concave
shape on the substrate. With increasing impact energy, the cluster deforms more and embeds deeper into the substrate. \par In the context of formation of \gls{camg} films, i.e., when multiple clusters are deposited over each other layer-by-layer, intermixing between clusters is expected at the higher impact energies. Based on the results of the single cluster deposition, impact energies between 60-600 meV/atom were chosen to study the formation of \gls{camg} films. In the following section, as part of a first analysis, the changes of the core-shell structures during multiple cluster deposition will be considered.
\end{changebar}
%
%\section{Identifying Cores and Interfaces}

\begin{selfcite}
The deposition of multiple clusters, with clusters derived from \qr{10} \cz MG, was simulated at 60 meV/atom, 300 meV/atom, 600 meV/atom impact energies to mimic the soft, medium, and hard-landing in the CAMG film samples, respectively. Additionally, a deposition at extreme energetic conditions was simulated at the impact energy of 6000 meV/atom. The \gls{hcp} arrangement for each impact energy was achieved in the following manner: before each new cluster was deposited, the clusters that were already present in its neighborhood were relaxed for at least 2 ns. Every simulated CAMG had three layers of such cluster depositions, with ~50 clusters in each layer. The deposition sequence and the processes occurring during impact can be followed in the \todo[inline]{Supplementary Video V1}, which shows the simulation of the deposition of CAMGs in comparison to that of the compaction in the NG processing. \par

Figure~\ref{f:film_network} shows cross-sections of the films, similar to Figure~\ref{f:clus_single3}a, after equilibrating the sample for 2 ns after the deposition of the last cluster. The yellow and magenta color coding denotes the shell and core atoms of the clusters prior to deposition, similar to Figure~\ref{f:clus_comp-3nm} and Figure~\ref{f:clus_single3}. No evidence for porosity is observed even for the soft-landing sample (60 meV/atom case) by evaluating a surface mesh with a probe sphere radius of 2.4 \r{A} \cite{Stukowski2010a,Stukowski2014}. At the lowest impact energy of 60 meV/atom, it is observed that, like in Figure~\ref{f:clus_single3}b, the deposited clusters mostly retain their initial sphericity. The cluster sphericity is progressively lost with increasing deposition energy. Next, it should be noted that the first layer of clusters has resided on the substrate for at least 24 ns (using the deposition protocol described) by the time the final layer is deposited. Nevertheless, the interdiffusion of the core and shell atoms is quite low for deposition energies even up to 600 meV/atom. In the energy range of 60-600 meV/atom, the shell atoms, i.e., the former surface atoms of the free cluster prior to deposition, are forming a distinct inter-connected network, which can be interpreted as atomically thin interfacial regions between the cores of the clusters. At the impact energy of 6000 meV/atom the interfacial regions vanish completely. Additionally, this film shows significant atomic intermixing between the substrate and the film (see Figure~\ref{f:film_network}, visualizing the black substrate atoms found in the film, and magenta and yellow atoms from the deposited film embedded in the substrate). For the case of the 6000 meV/atom energy, it is estimated that 8\% of the atoms originally in the film are mixed into the substrate, whereas for the case of the 600 meV/atom impact energy this value is about 1.7\%. Similarly, the mixing of the substrate atoms diffusing into the film sample was also ascertained, by tracking the substrate atoms in the final deposited films. The mixing of the substrate atoms into the film has been estimated to be 6\% for 600 meV/atom case, and is 16\% for the 6000 meV/atom case. For both the film and substrate atoms, the diffusion into the neighboring medium is higher at higher impaction energies. \par
	
\begin{figure*}	\centering
	\includegraphics[width=0.7\textwidth,trim={2cm 2.5cm 1.9cm 1.8cm},clip]{2021_05/figures/4.pdf}
	\includegraphics[width=\textwidth,trim={0.8cm 2cm 0 1.7cm},clip]{2021_05/figures/5.pdf}
	\mycaption{Deposition of CAMG films}{(a) A depiction of the deposition of a cluster onto the substrate and (b) a top view of the clusters deposited in a HCP arrangement (colour coded by the atom height in the deposition axis) (c) A vertical crossection of the deposited films at 60, 300, 600, and 6000 meV/atom energies, with cluster-core atoms in magenta, and cluster-shell atoms in yellow. The substrate atoms are coloured in black. We observe that the shell atoms form a network of interfaces across the film at least up to 600 meV/atom deposition energy. (d) When colour coded with von Mises shear strain, the interfacial atoms correlate with the higher strained atoms.}
	\label{f:film_network}
\end{figure*}
	
Up to now, the locations of the atoms, located at the surfaces of the clusters prior to deposition, have been followed (using the magenta and yellow color scheme for the core and shell atoms, respectively) to determine the interfacial regions in the samples prepared with impact energies in the range below 600 meV/atom. This approach does not provide any information on the energetic state of the atoms in the CAMGs or on the local environments in the cores and interfaces. In a first step towards a more detailed analysis, the von Mises shear strain for each of the CAMG atoms was determined \cite{Stukowski2014}. A cut-off radius of 3.8 \r{A} was chosen to compute the strain tensor. In Figure~\ref{f:film_network}d, the high strain regions can be clearly correlated to the interfacial network shown in Figure 3c for all impact energies below 600 meV/atom. Only for the highest impact energy, no sign for the presence of interfaces can be found, similar to the observation in Figure~\ref{f:film_network}c. The correlation observed for CAMGs is consistent with Gleiter’s original definition of interfaces \cite{Gleiter1991} in NGs, in which the interfaces were assumed to be regions of distorted and sheared coordination among adjacent clusters. The strain maps in Figure~\ref{f:film_network}d confirm that an interfacial structure is formed and is retained in the range of 60-600 meV/atom impact energies. Upon inspection of the simulation snapshots in \gls{ovito}, the cores and interfaces in the CAMG film samples made by soft-to-high landing deposition are found to reflect a chemical heterogeneity similar to what was originally present in the free clusters prior to deposition (described in Figure~\ref{f:clus_comp-3nm}a), with the Cu concentration of ∼46 at. \% in the cores, and ∼54 at. \% in the interfacial regions. The atoms in the CAMG film sample for the extreme hard-landing case of 6000 meV/atom are strained to the point where the presence of core and interface structure is lost, and in this manner resembling the MGs obtained by RQ. It should be mentioned that the processing of NGs and CAMGs seems to differ in one aspect: at the harshest conditions, i.e., at the extreme hard-landing case for CAMGs and at the highest pressures for NGs, the final structures are different. In NGs, the interfacial regions continue to exist even at the highest pressures, while the interfacial regions disappear in CAMGs at the extreme hard-landing. This might be due to the sequential deposition of the clusters in CAMG processing, compared to the compaction process, which occurs in the entire arrangement of clusters. \par

\begin{figure}
	\begin{subfigure}[b]{0.33\textwidth} \includegraphics[width=\textwidth,trim={0 1cm 0 1cm},clip]{1e10/dep_60meV_pers_del}
		\caption{}
	\end{subfigure}%
	\hfill
	\begin{subfigure}[b]{0.33\textwidth} \includegraphics[width=\textwidth,trim={0 1cm 0 1cm},clip]{1e10/dep_300meV_pers_del}
		\caption{}
	\end{subfigure}%
	\hfill
	\begin{subfigure}[b]{0.33\textwidth} \includegraphics[width=\textwidth,trim={0 1cm 0 1cm},clip]{1e10/dep_600meV_pers_del}
		\caption{}
	\end{subfigure}%
\mycaption{Representative slabs of CAMGs}{To avoid surface artifacts, these slabs were cut out of the CAMG films}
\label{f:camg_slabs}
\end{figure}

The CAMG samples are also different from both the MGs and the NG, in the following fashion: the CAMGs only partially fill a simulation box. It is important to clarify that the unfilled volume referred to is not within the film, rather it is between the upper surface of the film and the upper wall of the simulation box. Due to the open surface at the top, and a surface interaction between the cluster atoms and the substrate, it is expected in the CAMG film samples that this gives rise to surface artifacts—including defective surface coordinations, larger atomic occupancy and higher surface energy. It was decided to first analyse the entire CAMG film sample, and then later represent the data from a slab of fixed dimensions present within the inside of the deposited film samples, in order to avoid the surface artifacts, as shown in Figure~\ref{f:camg_slabs} (Tip for \gls{ovito} users: the atoms have to be deleted from the data only after any OVITO-API calls have been performed).  \par
\end{selfcite}

Another possible method of removing surface artifacts is to query the surface atoms by means of a surface mesh, and deleting them from the slab before plotting the analysed data. However, the first method of representing data from slabs is preferred in this thesis. The fixed slab volume and the consistently similar amounts of core and interfacial atoms, allows for a consistent analysis of the effects of core and interface regions. This consistency is lost when deleted surface atoms from the surface mesh, as the deleted atoms are majorly made up of the interface atoms. \par

\begin{selfcite}
Based on the presented results, a structural model for the CAMGs is proposed, similar to that of NGs. In this model, interfacial regions, which are chemically different from the core regions due to the surface segregation observed in the individual clusters, are formed during the cluster deposition in the range of impact energies between 60–600 meV/atom. The only exception is, as mentioned above, the structure for an impact energy of 6000 meV/atom, for which interfaces are totally absent. Therefore, in the following sections, only simulations of CAMGs deposited at the impact energies 60, 300, and 600 meV/atom, are being considered. Incidentally, this energy range corresponds well with the energy range used in the cluster experiments reported in \cite{Benel2019}.
\end{selfcite}
\section{Identifying Cores and Interfaces} \label{s:corint}

\begin{changebar}
The deposition of multiple clusters, with clusters derived from \qr{10} \cz \gls{mg}, was simulated at 60 meV/atom, 300 meV/atom, 600 meV/atom impact energies to mimic the soft, medium, and hard-landing in the \gls{camg} film samples, respectively. Additionally, a deposition at an extreme energetic condition was simulated with an impact energy of 6000 meV/atom. The \gls{hcp} arrangement (chosen for reasons mentioned in Section~\ref{s:camgdev}) for each impact energy was achieved in the following manner: before each new cluster was deposited, the clusters that were already present in its neighborhood were relaxed for at least 2 ns. Every simulated \gls{camg} had three layers of such cluster depositions, with $\sim$50 clusters in each layer. The deposition sequence and the processes occurring during impact can be followed in the Supplementary Video V2 \todo{Find out about adding video to dissertation}, which shows the simulation of the deposition of \gls{camg}s in comparison to that of the compaction in the \gls{ng} processing. \par

Figure~\ref{f:film_network} shows cross-sections of the films, similar to Figure~\ref{f:clus_single3}a, after equilibrating the sample for 2 ns after the deposition of the last cluster. The yellow and magenta color coding denotes the shell and core atoms of the clusters prior to deposition, similar to Figure~\ref{f:clus_comp-3nm} and Figure~\ref{f:clus_single3}. No evidence for porosity is observed even for the soft-landing sample (60 meV/atom case) by evaluating a surface mesh with a probe sphere radius of 2.4 \r{A} \cite{Stukowski2010a,Stukowski2014}. \par

At the lowest impact energy of 60 meV/atom, it is observed that, like in Figure~\ref{f:clus_single3}b, the deposited clusters mostly retain their initial sphericity. The cluster sphericity is progressively lost with increasing deposition energy. Next, it should be noted that the first layer of clusters has resided on the substrate for at least 24 ns (using the deposition protocol described) by the time the final layer is deposited. Nevertheless, the interdiffusion of the core and shell atoms is quite low for deposition energies even up to 600 meV/atom. In the energy range of 60-600 meV/atom, the shell atoms, i.e., the former surface atoms of the free cluster prior to deposition, are forming a distinct inter-connected network, which can be interpreted as atomically thin interfacial regions between the cores of the clusters. The former shell atoms (colored yellow as mentioned above)
of the clusters are hence defined to constitute the cluster-cluster interfaces. At the impact energy of 6000 meV/atom the interfacial regions vanish completely. \par

Additionally, this film shows significant atomic intermixing between the substrate and the film (see Figure~\ref{f:film_network}, visualizing the black substrate atoms found in the film, and magenta and yellow atoms from the deposited film embedded in the substrate). For the case of the 6000 meV/atom energy, it is estimated that 8\% of the atoms originally in the film are mixed into the substrate, whereas for the case of the 600 meV/atom impact energy this value is about 1.7\%. \par

\begin{figure}[!h]	\centering
	\includegraphics[width=0.7\textwidth,trim={2cm 2.5cm 1.9cm 1.8cm},clip]{camg_3nm/4.pdf}
	\includegraphics[width=\textwidth,trim={0.8cm 2cm 0 1.7cm},clip]{camg_3nm/5.pdf}
	\mycaption{Deposition of \gls{camg} Films}{(a) A depiction of the deposition of a cluster onto the substrate and (b) a top view of the clusters deposited in a HCP arrangement (colour coded by the atom height in the deposition axis) (c) A vertical crossection of the deposited films at 60, 300, 600, and 6000 meV/atom energies, with cluster-core atoms in magenta, and cluster-shell atoms in yellow. The substrate atoms are coloured in black. We observe that the shell atoms form a network of interfaces across the film at least up to 600 meV/atom deposition energy. (d) When colour coded with von Mises shear strain, the interfacial atoms correlate with the higher strained atoms.}
	\label{f:film_network}
\end{figure}

Similarly, the mixing of the substrate atoms diffusing into the film sample was also ascertained, by tracking the substrate atoms in the final deposited films. The mixing of the substrate atoms into the film has been estimated to be 6\% for 600 meV/atom case, and is 16\% for the 6000 meV/atom case. For both the film and substrate atoms, the diffusion into the neighboring medium is higher at higher impaction energies. \par

Up to now, the locations of the atoms, located initally at the surfaces of the clusters prior to deposition, have been followed (using the magenta and yellow color scheme for the core and shell atoms, respectively) to determine the interfacial regions in the samples prepared with impact energies in the range below 600 meV/atom. This approach does not provide any information on the energetic state of the atoms in the \gls{camg}s or on the local environments in the cores and interfaces. In a first step towards a more detailed analysis, the von Mises shear strain for each of the \gls{camg} atoms was determined \cite{Stukowski2014}. A cut-off radius of 3.8 \r{A} was chosen to compute the strain tensor. In Figure~\ref{f:film_network}d, the high strain regions can be clearly correlated to the interfacial network shown in Figure~\ref{f:film_network}c for all impact energies below 600 meV/atom. Only for the highest impact energy, no sign for the presence of interfaces can be found, similar to the observation in Figure~\ref{f:film_network}c.  \par

The correlation observed for \gls{camg}s is consistent with Gleiter’s original definition of interfaces \cite{Gleiter1991} in \gls{ng}s, in which the interfaces were assumed to be regions of distorted and sheared coordination among adjacent clusters. The strain maps in Figure~\ref{f:film_network}d confirm that an interfacial structure is formed and is retained in the range of 60-600 meV/atom impact energies.\par

Upon inspection of the simulation snapshots in \gls{ovito}, the cores and interfaces in the \gls{camg} film samples made by soft-to-high landing deposition are found to reflect a chemical heterogeneity similar to what was originally present in the free clusters prior to deposition (described in Figure~\ref{f:clus_comp-3nm}a), with the Cu concentration of ∼46 at. \% in the cores, and ∼54 at. \% in the interfacial regions. The overall composition of the CAMG films are found to be \cz (with 1\% deviation), same as of the original clusters. The atoms in the CAMG film sample for the extreme hard-landing case of 6000 meV/atom indicate a loss of core and interface structure, and in this manner resembling the MGs obtained by RQ. This loss of core-interface structure, also observed from von Mises local strains is indicative of a high deviation of atoms from their as-deposited positions in the clusters, likely due to local melting and resolidification near the deposition sites. It should be mentioned that the processing of \gls{ng}s and \gls{camg}s seems to differ in one aspect: at the harshest conditions, i.e., at the extreme hard-landing case for \gls{camg}s and at the highest pressures for \gls{ng}s, the final structures are different. In NGs, the interfacial regions continue to exist even at the highest pressures, while the interfacial regions disappear in CAMGs at the extreme hard-landing. This most likely is caused by the differences between the
compaction and deposition processes. While the cold compaction shears the entire arrangement of clusters together, the sequential deposition of the clusters along with local heating due to the inelastic collision may result in the dissolution of the cluster structure. Therefore, the energetic impact of clusters is considered as a novel process leading to metallic glasses. CAMGs also contain interfacial regions with a modified structure, which differs from that of RQ MGs and NGs prepared by high pressure compaction. However, it is not clear if the structural details of the interfacial regions of CAMGs and NGs are identical, implying that different properties for RQ MGs, NGs and CAMGs are very well possible. Consequently, there is a need to further investigate the NGs and CAMGs both experimentally and using simulation methods. \par

The \gls{camg} samples are also different from both the \gls{mg}s and the \gls{ng}, in the following fashion: the \gls{camg}s only partially fill a simulation box. It is important to clarify that the unfilled volume referred to is not within the film, rather it is between the upper surface of the film and the upper wall of the simulation box.  The open surface at the top in the CAMG film samples is expected to give rise to surface artifacts—including defective surface coordinations, larger atomic occupancy and excess surface energy. The surface interaction between the cluster atoms and the substrate will also lead to defective coordinations at the border. It was decided to first analyse the entire \gls{camg} film sample, and then later represent the data from a slab of fixed dimensions present within the inside of the deposited film samples, in order to avoid the surface artifacts, as shown in Figure~\ref{f:camg_slabs} (Tip for \gls{ovito} users: the atoms have to be deleted from the data only after any OVITO-API calls have been performed).  \par

\begin{figure}[h]
	\begin{subfigure}[b]{0.33\textwidth} \includegraphics[width=\textwidth,trim={0 0.5cm 0 1cm},clip]{1e10/dep_60meV_pers_del}
		\caption{}
	\end{subfigure}%
	\hfill
	\begin{subfigure}[b]{0.33\textwidth} \includegraphics[width=\textwidth,trim={0 0.5cm 0 1cm},clip]{1e10/dep_300meV_pers_del}
		\caption{}
	\end{subfigure}%
	\hfill
	\begin{subfigure}[b]{0.33\textwidth} \includegraphics[width=\textwidth,trim={0 0.5cm 0 1cm},clip]{1e10/dep_600meV_pers_del}
		\caption{}
	\end{subfigure}%
	\mycaption{Representative Slabs of \gls{camg}s}{To avoid surface artifacts, these slabs were cut out of the \gls{camg} (a) 60 meV/atom (b) 300 meV/atom and (c) 600 meV/atom films.}
	\label{f:camg_slabs}
\end{figure}
\end{changebar}

Another possible method of removing surface artifacts is to query the surface atoms by means of a surface mesh, and deleting them from the slab before plotting the analysed data. When deleting surface atoms by using the surface mesh, the deleted atoms are majorly made up of the interface atoms. Moreover, the \gls{camg}s deposited at lower energies have more distinct morphologies at the top surface of the film, meaning that the number of deleted surface atoms in this case will be larger than for a \gls{camg} deposited at higher energies. This would result in the representative samples not consistently having the same number of core and interface atoms in the \gls{camg}s. For the fixed slab method, however, the volume and the consistently similar amounts of core and interfacial atoms, allows for a consistent analysis of the effects of core and interface regions. Hence, the first method of representing data from fixed slabs is preferred in this thesis. \par

\begin{changebar}
Based on the presented results, a structural model for the \gls{camg}s is proposed, similar to that of \gls{ng}s. In this model, interfacial regions, which are chemically different from the core regions due to the surface segregation observed in the individual clusters, are formed during the cluster deposition in the range of impact energies between 60–600 meV/atom. The only exception is, as mentioned above, the structure for an impact energy of 6000 meV/atom, for which interfaces are totally absent. Therefore, in the following sections, only simulations of \gls{camg}s deposited at the impact energies 60, 300, and 600 meV/atom, are being considered. Incidentally, this energy range corresponds well with the energy range used in the cluster experiments reported in \cite{Benel2019}. \par

The prepared \gls{camg}s in this chapter are compared to corresponding \gls{ng}s and \gls{rq} \gls{mg}s. Moreover, a heat treatment as described for the single cluster (Section~\ref{s:clus}) can also modify the metastable state and structure of a MG. Therefore, the as-prepared MG sample, i.e., quenched directly from the liquid phase, was subjected to a heat treatment identical to the one used for the single cluster to determine the changes in the structure of the MG sample. In addition to the as-prepared MG, the \gls{mght} serves as a reference structure for the NGs and CAMGs prepared by compaction and energetic impact, respectively. In the following discussion up until Section\ref{s:camg_quenchrt}, the NGs and CAMGs derived from the cluster made only from the \qr{10} MG are considered. \par
\end{changebar}


%\section{Local Structure Tailoring in CAMGs}
\begin{changebar}
To obtain a more detailed insight into the structure of CAMGs, in particular the interfacial regions, the normalized \gls{prdf} of the CAMGs, NG, and the MGs were studied (see Figure~\ref{f:camg-rdf} in \nameref{c:supple} for more details). Like in Section~\ref{s:simtestMG}, the Cu-Cu (2.45 \r{A}), Cu-Zr (2.8 \r{A}), and Zr-Zr (3.25 \r{A}) first peak positions of the MGs match well with those previously reported values \cite{Nasu2007,Duan2005}. Even the NG and CAMGs follow the same pair distributions as those of MG. Moreover, no significant change in atomic pair correlations both in the core and the interfacial regions are observed. This prompts the study of the local \gls{sro}. \par

The local atomic environment is typically represented using the Voronoi analysis method, and the polyhedra are represented by a  \vi{n$_{3}$}{n$_{4}$}{n$_{5}$}{n$_{6}$} Scha\"afli index (See Section~\ref{s:voronoi}). Figures~\ref{f:voro_camg}a-c show the top seven frequent Voronoi polyhedra (arranged in the numerical order of the indices) for CAMGs, NGs, \gls{mght}, and the precursor MG quenched at a rate of \qr{10}. In Figure~\ref{f:voro_camg}a, the histograms of the Voronoi polyhedra for all the atoms constituting the entire six sample sets are shown, whereas Figures~\ref{f:voro_camg}b-c denote the histograms for the core and interface atoms, respectively. The index \vi{0}{0}{12}{0}, which represents the FI coordination, is the highest occurring in the MG, amongst all the six glasses. For the \qr{10} MG, the heat-treatment reduces the FI order as seen in \gls{mght}. The next highest occurring index is the \vi{0}{2}{8}{2}, which is known to be an \gls{ilike} polyhedron \cite{Yue2018,Borodin1999}. Its occurrence is highest in MG and \gls{mght} when compared to the NG and CAMGs. As mentioned in Section~\ref{s:voronoi}, the Voronoi polyhedra are known to be classified into four main categories \cite{Yue2018} are repeated here for the benefit of the reader: 1. icosahedral-like: \vi{0}{0}{12}{0}, \vi{0}{0}{10}{x}, and \vi{0}{2}{8}{x}; 2. crystal-like: \vi{0}{4}{4}{x} and \vi{0}{5}{2}{x}; 3. mixed coordinations: \vi{0}{3}{6}{x}, where 0 $\leq$ x $\leq$ 4; and 4. other remaining indices. With this knowledge, it is noticed that in the NG and CAMGs the other prominently occurring polyhedra in Figure~\ref{f:voro_camg}a-c are of the icosahedral-like and mixed coordination types. \par

\begin{figure}[!ht]
	\centering
	\includegraphics[width=\textwidth,trim={1cm 0cm 1cm 0cm},clip]{2021_05/figures/6.pdf}
	\mycaption{SRO recovers in CAMGs with deposition energy:}{The top 7 Voronoi indices arranged based 		on numerical order, without considering any special central atom species (Cu/Zr) in the Voronoi Polyhedra. The Voronoi histograms are shown for (a) the entire representative sample, (b) the core atoms, and (c) the interface atoms, respectively. The magenta and yellow backgrounds in the figures are rendered to represent the core and interface cases, respectively. Figure~\ref{f:voro_camg} (d), (e) and (f) show the indices sorted into known groups of coordinations. Crystalline coordinations are absent in all of the simulated glasses, in particular in the \gls{mght}, NG, and CAMGs where a heat-treatment is involved in the simulation process. The FI and ILO short-range order (SRO) in CAMGs increase systematically with deposition energy}
	\label{f:voro_camg}
\end{figure}

Furthermore, the Voronoi indices are sorted based on the above rules for the six simulated glass samples and represented in Figure~\ref{f:voro_camg}d-f, in order to facilitate the analysis of the dominant indices in comparison of the different metallic glass structures. It is first noticed that crystalline coordinations do not occur in any of the simulated metallic glass samples, especially in the \gls{mght} and the CAMGs, despite the heat treatment involved in their processing. Especially, the lack of any crystalline coordinations confirms that all the simulated CAMGs are fully amorphous, both in the cores and in the interfaces. \par

In the CAMGs and NGs a significantly reduced short-range FI order was observed, consistent with previous studies of NGs \cite{Adjaoud2019}. This trend is also seen in the analysis of the ILO. The present simulations of the \gls{rq} \cz MGs have shown that less stable MGs (prepared at higher quench rates) are accompanied by a reduction in FI fractions and also in the ILO (refer to Figure~\ref{f:voro_qr} in Section~\ref{s:voro-mgs}). The tendency of a decrease of stability in bottom-up metallic glasses (NGs, CAMGs) with decreasing ILO is further discussed in Section~\ref{s:camg-pe}. \par
\end{changebar}

The sorted Voronoi polyhedra histograms for Cu-centered and Zr-centered coordinations are depicted in Figure~\ref{f:vorosort-cu-zr} (The unsorted Voronoi Indices distribution is indicated in Figure~\ref{f:voro_cu-zr} in \nameref{c:supple}). The ILO is observed to be higher for Cu atoms than for Zr atoms for all the six metallic glass samples that have been studied. The Cu-centered atoms make up the majority in contributing to \gls{ilo}. Also seen is the increasing icosahedral-like order in Cu-centered atoms in the CAMGs with deposition energy. \par

\begin{figure}[!ht]
	\centering
	\begin{subfigure}{0.45\linewidth} \centering \includegraphics[width=\textwidth]{1e10/voronoi-sort_3nm_Cu50Zr50_CAMG_vs_MG_1e10_asp_Cu} 
		\subcaption{} \end{subfigure}%
	\hspace{1cm}
	\begin{subfigure}{0.45\linewidth} \centering \includegraphics[width=\textwidth]{1e10/voronoi-sort_3nm_Cu50Zr50_CAMG_vs_MG_1e10_asp_Zr}
		\subcaption{} \end{subfigure}
	\mycaption{Sorted Voronoi Polyhedra for Cu- and Zr- centered atoms in MGs vs in \gls{cbmg}s}{The atomic fractions here are defined w.r.t the same species i.e., at. fraction of Cu is the fraction of Cu atoms exhibiting a certain local order. }
	\label{f:vorosort_cu-zr}
\end{figure}

\todo[inline]{Want to add sorted voronoi core and interface, cu and zr centered in the appendix? decide}

\begin{changebar}
Looking back at Figure~\ref{f:voro_camg}e-f, it is observed that for NGs and CAMGs, both FI order and the ILO indices are respectively at least 1\% and 5\% higher in the interfaces compared to the cores. It is evident that the interfaces, being richer in Cu compared to the core regions, exhibit higher FI as well as higher ILO. This indicates that the interfaces must be better packed than the core regions. \par

Interestingly, a systematic increase in FI order (See Figure~\ref{f:voro_camg}c) in the CAMGs with increasing impact energy is observed. This trend, also seen in the ILO fractions of the entire sample, is especially prominent for the interfaces (see Figure~\ref{f:voro_camg}f). It should be noted that the interfaces for all impact energies have the same chemical composition. The increase in the FI-order and ILO in the interfaces can then be interpreted to be the direct result of the CIBD process. In both the core and interface atoms, another striking feature is the systematic increase of ILO in the CAMGs with increasing deposition energy, recovering towards the ILO of MG and \gls{mght}. The present model demonstrates the possibility of tailoring the local amorphous order with impact energy for metallic glasses synthesized via the CIBD route.

\end{changebar}
%%%%%%%%%%%%%%%%%
%%%%%%%%%%%%%%%%%
%%%%%%%%%%%%%%%%%
%%%%%%%%%%%%%%%%%




\section{Local Structure Tailoring in CAMGs} \label{s:vorocamg}
\begin{changebar}
To obtain a more detailed insight into the structure of \gls{camg}s, in particular the interfacial regions, the normalized \gls{prdf} of the \gls{camg}s, \gls{ng}, and the \gls{mg}s were studied (see Figure~\ref{f:camg-rdf} in \nameref{c:supple} for more details). Like in Section~\ref{s:simtestMG}, the Cu-Cu (2.45 \r{A}), Cu-Zr (2.8 \r{A}), and Zr-Zr (3.25 \r{A}) first peak positions of the \gls{mg}s match well with those previously reported values \cite{Nasu2007,Duan2005}. Even the \gls{ng} and \gls{camg}s follow the same pair distributions as those of \gls{mg}. Moreover, no significant change in atomic pair correlations both in the core and the interfacial regions are observed. This prompts the study of the local \gls{sro}. \par

The local atomic environment is typically represented using the Voronoi analysis method, and the polyhedra are represented by a  \vi{n$_{3}$}{n$_{4}$}{n$_{5}$}{n$_{6}$} Sch\"afli index (See Section~\ref{s:voronoi}). Figures~\ref{f:voro_camg}a-c show the top seven frequent \gls{vp} (arranged in the numerical order of the indices) for \gls{camg}s, \gls{ng}s, \gls{mght}, and the precursor \gls{mg} quenched at a rate of \qr{10}. In Figure~\ref{f:voro_camg}a, the histograms of the \gls{vp} for all the atoms constituting the entire six sample sets are shown, whereas Figures~\ref{f:voro_camg}b-c denote the histograms for the core and interface atoms, respectively. The index \vi{0}{0}{12}{0}, which represents the \gls{fi} coordination, is the highest occurring in the \gls{mg}, amongst all the six glasses. For the \qr{10} \gls{mg}, the heat-treatment reduces the \gls{fi} order as seen in \gls{mght}. The next highest occurring index is the \vi{0}{2}{8}{2}, which is known to be an \gls{ilike} polyhedron \cite{Yue2018,Borodin1999}. Its occurrence is highest in \gls{mg} and \gls{mght} when compared to the \gls{ng} and \gls{camg}s. As mentioned in Section~\ref{s:voronoi}, the \gls{vp} known to be classified into four main categories \cite{Yue2018}. The are mentioned here again for the benefit of the reader: 1. icosahedral-like: \vi{0}{0}{12}{0}, \vi{0}{0}{10}{x}, and \vi{0}{2}{8}{x}; 2. crystal-like: \vi{0}{4}{4}{x} and \vi{0}{5}{2}{x}; 3. mixed coordinations: \vi{0}{3}{6}{x}, where 0 $\leq$ x $\leq$ 4; and 4. other remaining indices. With this knowledge, it is noticed that in the \gls{ng} and \gls{camg}s the other prominently occurring polyhedra in Figure~\ref{f:voro_camg}a-c are of the icosahedral-like and mixed coordination types. \par

\begin{figure}[!ht]
\centering
\includegraphics[width=\textwidth,trim={1.4cm 0cm 1.65cm 0cm},clip]{camg_3nm/6.pdf}
\mycaption{FI SRO Recovers in CAMGs with Deposition Energy}{The top 7 Voronoi indices arranged based on numerical order, without considering any special central atom species (Cu/Zr) in the \gls{vp}. The Voronoi histograms are shown for (a) the entire representative sample, (b) the core atoms, and (c) the interface atoms, respectively. The magenta and yellow backgrounds in the figures are rendered to represent the core and interface cases, respectively. Figure~\ref{f:voro_camg} (d), (e) and (f) show the indices sorted into known groups of coordinations. Crystalline coordinations are absent in all of the simulated glasses, in particular in the \gls{mght}, \gls{ng}, and \gls{camg}s where a heat-treatment is involved in the simulation process. The \gls{fi}-order and \gls{ilo} in \gls{camg}s increase systematically with deposition energy}
\label{f:voro_camg}
\end{figure}

Furthermore, the Voronoi indices are sorted based on the above rules for the six simulated glass samples and represented in Figure~\ref{f:voro_camg}d-f, in order to facilitate the analysis of the dominant index classes in comparison of the different metallic glass structures. It is first noticed that crystalline coordinations do not occur in any of the simulated metallic glass samples, especially in the \gls{mght} and the \gls{camg}s, despite the heat treatment involved in their processing. Especially, the lack of any crystalline coordinations confirms that all the simulated \gls{camg}s are fully amorphous, both in the cores and in the interfaces. \par

In the \gls{camg}s and \gls{ng}s a significantly reduced short-range \gls{fi} order was observed, consistent with previous studies of \gls{ng}s \cite{Adjaoud2019}. This trend is also seen in the analysis of the \gls{ilo}. The present simulations of the \gls{rq} \cz \gls{mg}s have shown that less stable \gls{mg}s (prepared at higher quench rates) are accompanied by a reduction in \gls{fi} fractions and also in the \gls{ilo} (refer to Figure~\ref{f:voro_qr} in Section~\ref{s:voro-mgs}), as reported by \textcite{Yue2018}. The tendency of a decrease of stability in the NGs and \gls{camg}s with decreasing \gls{ilo} is further discussed in Section~\ref{s:camg-pe}. \par
\end{changebar}

The sorted \gls{vp} histograms for Cu-centered and Zr-centered coordinations are depicted in Figure~\ref{f:vorosort_cu-zr} (The unsorted Voronoi indices distribution is indicated in Figure~\ref{f:voro_cu-zr} in \nameref{c:supple}). The \gls{ilo} is observed to be higher for Cu atoms than for Zr atoms for all the six metallic glass samples that have been studied. The Cu-centered atoms make up the majority in contributing to \gls{ilo}. Also seen is the increasing \gls{ilo} in Cu-centered atoms in the \gls{camg}s with deposition energy. \par

\begin{figure}[!ht]
	\centering
	\begin{subfigure}{0.5\linewidth} \centering \includegraphics[width=0.9\textwidth]{1e10/voronoi-sort_3nm_Cu50Zr50_CAMG_vs_MG_1e10_asp_Cu} 
		\subcaption{} \end{subfigure}%
	\hfill
	\begin{subfigure}{0.5\linewidth} \centering \includegraphics[width=0.9\textwidth]{1e10/voronoi-sort_3nm_Cu50Zr50_CAMG_vs_MG_1e10_asp_Zr}
		\subcaption{} \end{subfigure}
	\mycaption{Sorted VP for Cu- and Zr-centered Atoms in MGs, NGs, and CAMGs}{The atomic fractions here are defined w.r.t the same species i.e., at. fraction of Cu is the fraction of Cu atoms exhibiting a certain local order. }
	\label{f:vorosort_cu-zr}
\end{figure}

%\todo[inline]{Want to add sorted voronoi core and interface, cu and zr centered in the appendix? decide}

\begin{changebar}
Looking back at Figure~\ref{f:voro_camg}e-f, it is observed that for \gls{ng}s and \gls{camg}s, both \gls{fi} order and the \gls{ilo} indices are respectively at least 1\% and 5\% higher in the interfaces compared to the cores. It is evident that the interfaces, being richer in Cu compared to the core regions, exhibit higher \gls{fi} as well as higher \gls{ilo}. This indicates that the interfaces must be better packed than the core regions\footnote{The correlation of ILO of cores and interfaces in the CAMGs and NGs, with their stability is discussed in Sections~\ref{s:camg-pe} and \ref{s:camg_quenchrt}}. \par

Interestingly, a systematic increase in \gls{fi} order (See Figure~\ref{f:voro_camg}c) in the \gls{camg}s with increasing impact energy is observed. This trend, also seen in the \gls{ilo} fractions of the entire sample, is especially prominent for the interfaces (see Figure~\ref{f:voro_camg}f). We recall that the interfaces are defined as the surface atoms of the undeposited clusters, and for this reason it should be noted that the interfaces for all impact energies have the same chemical composition. The increase in the \gls{fi}-order and \gls{ilo} in the interfacial atoms can then be interpreted to be the direct result of the \gls{cibd} process. In both the core and interface atoms, another striking feature is the systematic increase of \gls{ilo} in the \gls{camg}s with increasing deposition energy, recovering towards the \gls{ilo} of \gls{mg} and \gls{mght}. The present model demonstrates the possibility of tailoring the local amorphous order with impact energy for metallic glasses synthesized via the \gls{cibd} route.
\end{changebar}

%\section{Atomic Volume Analysis}
\begin{selfcite}
The normalized distribution of the atomic coordination volumes for all the atoms in the six metallic glass samples was studied using the Voronoi analysis. Figure~\ref{f:vol_camg}a shows the volume distribution with two peaks approximately at 13.9 \acu for Cu and 21.8 \acu for Zr. The Cu atoms are observed to have lower occupancy numbers compared to those of Zr atoms. \textcite{Cheng2019} reported similar distributions in \czsix nanoglasses, however, with the volume per atom peaks shifted to the left, likely caused by the fact that \czsix MGs are denser than the \cz MGs \cite{Li2008}. Furthermore, the CAMGs are similar to the NGs and MGs in terms of atomic volume distributions. It is also noticed that the distributions in the core and in the interfaces are not significantly different from each other (see Figure~\ref{f:atvol} in \nameref{c:supple}). Using the volume distributions in Figure~\ref{f:vol_camg}a, the exclusion of the surface atoms in the analysis (detailed in Section~\ref{s:corint}) of CAMGs was cross-verified. The surface atoms occupy higher volume per atom than average, and when included in the volume analysis, are known to alter the volume distributions of Cu and Zr atoms with a shoulder to the right of each main peak \cite{Cheng2019}. The absence of such shoulders indeed ascertains the absence of surface atoms in the representative Figure~\ref{f:vol_camg}a. \par

\begin{figure}[!h] \centering
	\includegraphics[width=0.65\columnwidth,trim={0 0 2cm 0.5cm},clip]{2021_05/figures/7.pdf}
	\mycaption{Reduced volumes in cluster assembled metallic glass samples}{ (a) The normalized volume per atom distribution shows
		similar behavior for the six metallic glass samples. (b) Average volumes of the atoms show that the core regions are less densely packed
		than the interfaces}
	\label{f:vol_camg}
\end{figure}

\todo[inline]{should I be distinguishing between NGs and CCMGs in this chapter, because I will be talking about CCMGs in the next chapter}

In Figure~\ref{f:vol_camg}b, which shows the average volume/atom values for all six metallic glass samples, it can be seen that, on average, the atoms in CAMGs and NGs occupy similar volumes. Furthermore, the impact energy does not have an influence on the average volume of the CAMGs. The core regions in the NGs and CAMGs present a higher volume occupancy. By contrast, the opposite behavior is observed for the interfacial atoms. The increase of volume for core atoms and the decrease of volume for the interface atoms in CAMGs and NGs offset each other to result in similar volume occupancies as MGs prepared by RQ, when all atoms of the are considered together. While the interfaces in NGs have previously been reported to be less dense in the MG \cite{Sopu2009,Witte2013}, this need not hold true for the CAMGs as well. The interfaces are richer in Cu-atoms, which, on an average, occupy lower volumes compared to Zr-atoms. The interfaces in the present CAMGs model have an increased density due to the chemical effects, and this is consistent with previous studies of segregated planar interfaces by Adjaoud and Albe \cite{Adjaoud2016}.
\end{selfcite}
\section{Atomic Volume Analysis} \label{s:atvolcamg}
\begin{changebar}
The normalized distribution of the atomic coordination volumes, or Voronoi volumes \cite{Cheng2019,Lu2018} for all the atoms in the six metallic glass samples was studied using the Voronoi analysis. Figure~\ref{f:vol_camg}a shows the volume distribution with two peaks approximately at 13.9 \acu for Cu and 21.8 \acu for Zr, which indicates the volumes occupied by the Cu and Zr atoms. The Cu atoms are observed to have lower volume occupancies compared to those of Zr atoms. \textcite{Cheng2019} reported similar distributions in \czsix nanoglasses, however, with the volume per atom peaks shifted to the left, likely caused by the fact that \czsix \gls{mg}s are denser than the \cz \gls{mg}s \cite{Li2008}. Furthermore, the \gls{camg}s are similar to the \gls{ng}s and \gls{mg}s in terms of atomic volume distributions. It is also noticed that the distributions in the core and in the interfaces are not significantly different from each other (see Figure~\ref{f:atvol} in \nameref{c:supple}). Using the volume distributions in Figure~\ref{f:vol_camg}a, the exclusion of the surface atoms in the analysis (detailed in Section~\ref{s:corint}) of \gls{camg}s was cross-verified. The surface atoms occupy higher volume per atom than average, and when included in the volume analysis, are known to alter the volume distributions of Cu and Zr atoms with a shoulder to the right of each main peak \cite{Cheng2019}. The absence of such shoulders indeed ascertains the absence of surface atoms in the representative Figure~\ref{f:vol_camg}a. \par

\begin{figure}[!h] \centering
%	\includegraphics[width=0.65\columnwidth,trim={0 0 2cm 0.5cm},clip]{camg_3nm7.pdf}
	\includegraphics[width=\textwidth,trim={0.6cm 2.8cm 0.65cm 2.1cm},clip]{camg_3nm/7.pdf}
	\mycaption{Reduced Volumes in CAMG samples}{(a) The normalized volume per atom distribution shows
		similar behavior for the six metallic glass samples. (b) Average volumes of the atoms show that the core regions are less densely packed
		than the interfaces.}
	\label{f:vol_camg}
\end{figure}

%\todo[inline]{should I be distinguishing between \gls{ng}s and \gls{ng}s in this chapter, because I will be talking about \gls{ng}s in the next chapter}

In Figure~\ref{f:vol_camg}b, which shows the average volume/atom values for all six metallic glass samples, it can be seen that, on average, the atoms in \gls{camg}s and \gls{ng}s occupy similar volumes. Furthermore, the impact energy does not have an influence on the average volume of the \gls{camg}s. The core regions in the \gls{ng}s and \gls{camg}s present a higher volume occupancy. By contrast, the opposite behavior is observed for the interfacial atoms. When all of the atoms are considered together, the increase of volume for core atoms and the decrease of volume for the interface atoms in CAMGs and NGs offset each other to result in similar volume occupancies as MGs prepared by RQ. While the interfaces in \gls{ng}s have previously been reported to be less dense in the \gls{mg} \cite{Sopu2009,Witte2013}, this need not hold true for the \gls{camg}s as well. The interfaces are richer in Cu-atoms, which, on an average, occupy lower volumes compared to Zr-atoms. The interfaces in the present \gls{camg}s model have an increased density due to the chemical effects, and this is consistent with previous studies of segregated planar interfaces by Adjaoud and Albe \cite{Adjaoud2016}.
\end{changebar}

%\section{Potential Energy Inspection} \label{s:camg-pe}

\begin{selfcite}
In Figure~\ref{f:pe_camg}a, the normalized potential energy distribution of the simulated CAMGs, NG and MGs are summarized. The Cu and Zr atoms exhibit two separate distributions, with the peaks of -3.5 eV for Cu atoms and -6.5 eV for Zr atoms. By contrast to the atomic volume distribution behavior, where the Cu atoms occupied lower volumes, the Cu atoms have a higher potential energy overall in comparison to Zr atoms. The absence of right shoulders in the potential energy distribution peaks, like in the volume distributions discussed in Section~\ref{s:atvolcamg}, once again confirms the absence of surface atoms in the representative CAMG slabs. \par

\begin{figure}[!h] \centering
	\includegraphics[width=0.65\columnwidth,trim={0 0 2cm 0.5cm},clip]{2021_05/figures/8.pdf}
	\mycaption{Potential energy states of the CAMGs, NG and MGs: Figure~\ref{f:pe_camg}(a) shows the normalized potential energy distributions for the six glasses, whereas Figure~\ref{f:pe_camg}(b) shows the average potential energy per atom, also in the core and interface regions for all the glasses.}
	\label{f:pe_camg}
\end{figure}

\todo[inline]{write speculations about PEL}

Figure~\ref{f:pe_camg}b summarizes the average potential energies for all atoms in the MG, \gls{mght}, NG, and the CAMGs and the average potential energies of the atoms in the core and interfacial regions in the NG and CAMGs, deposited at the different impact energies. All of these six metallic glasses samples have been made from a \qr{10} \cz glass. It is observed that the core and interfacial atoms in the CAMGs and NGs can be distinguished by their energetic states. The core atoms occupy lower energy states, about 2\% lower than that of the atoms in MGs prepared by RQ, whereas the interfaces possess higher energies ~ 6\% higher compared to those of the MGs. While the interfaces are better packed than the cores, as seen in Section~\ref{s:atvolcamg}, they occupy a higher energy state than the MGs. From liquid quenched traditional metallic glasses simulated in Section~\ref{s:s:simtestMG} it is known that the total potential energy of the glass increases with increasing Cu concentration maintaining the same quenching rate (See Figure~\ref{f:pe_mgs}). Hence, it can be explained that the higher Cu concentration in the interfaces, results in the interface atoms residing at a higher energy state. We conclude that the core regions stabilize the CAMGs and the NGs. Denser packing at the interfaces does not necessarily correspond to a lower energy state in the NGs and CAMGs due to their chemical heterogeneity. \par

In the following section, the medium-range order (MRO) in all six metallic glasses (3 CAMGs, NG, MG and \gls{mght}) will be inspected to better understand the connectivity of the FI units in the MG and \gls{mght} and how the MRO varies in the CAMGs with the deposition energy.
\end{selfcite}
\section{Potential Energy Inspection} \label{s:camg-pe}

\begin{changebar}
In Figure~\ref{f:pe_camg}a, the normalized \gls{pe} distribution of the simulated \gls{camg}s, \gls{ng} and \gls{mg}s are summarized. The Cu and Zr atoms exhibit two separate distributions, with the peaks of -3.5 eV for Cu atoms and $\sim$-6.4 eV for Zr atoms. By contrast to the atomic volume distribution behavior, where the Cu atoms occupied lower volumes, the Cu atoms have a higher P.E. overall in comparison to Zr atoms. The absence of right shoulders in the P.E. distribution peaks, like in the volume distributions discussed in Section~\ref{s:atvolcamg}, once again confirms the absence of surface atoms in the representative \gls{camg} slabs. \par

\begin{figure} %[!h]
	\centering
%	\includegraphics[width=0.65\columnwidth,trim={0 0 2cm 0.5cm},clip]{camg_3nm8.pdf}
	\includegraphics[width=\textwidth,trim={0.5cm 2.5cm 0.4cm 2.05cm},clip]{camg_3nm/8.pdf}
	\mycaption{P.E./atom States of the CAMGs, NG and the MGs}{(a) The normalized P.E. distributions for the six glasses. (b) The average P.E. per atom, also in the core and interface regions for all the glasses.}
	\label{f:pe_camg}
\end{figure}

Figure~\ref{f:pe_camg}b summarizes the average potential energies for all atoms in the \gls{mg}, \gls{mght}, \gls{ng}, and the \gls{camg}s and the average potential energies of the atoms in the core and interfacial regions in the \gls{ng} and \gls{camg}s, deposited at the different impact energies. All of these six metallic glasses samples have been made from a \qr{10} \cz glass. It is observed that the core and interfacial atoms in the \gls{camg}s and \gls{ng}s can be distinguished by their energetic states. The core atoms occupy lower energy states, about 2\% lower than that of the atoms in \gls{mg}s prepared by \gls{rq}, whereas the interfaces possess higher energies ~ 6\% higher compared to those of the \gls{mg}s. While the interfaces are better packed than the cores, as seen in Section~\ref{s:atvolcamg}, they occupy a higher energy state than the \gls{mg}s. From liquid quenched traditional metallic glasses simulated in Section~\ref{s:simtestMG}, it is known that the total P.E. of the glass increases with increasing Cu concentration maintaining the same quenching rate (See Figure~\ref{f:pe_mgs}). Hence, it can be explained that the higher Cu concentration in the interfaces, results in the interface atoms residing at a higher energy state. We conclude that the core regions stabilize the \gls{camg}s and the \gls{ng}s. Denser packing at the interfaces does not necessarily correspond to a lower energy state in the \gls{ng}s and \gls{camg}s due to their chemical heterogeneity. \par

\end{changebar}

%\section{Medium-range order in CAMGs}

\begin{selfcite}
The relative packing of coordination polyhedra centered around solute atoms is used to define MRO in metallic glasses; it has been shown that the solute atoms exhibit string-like connectivity when the solute concentration goes beyond 20-30 at.\%, \cite{Sheng2006}. Similarly, the string-like connectivity of FI-atoms, which are the atoms residing in FI coordinations, have been reported to indicate MRO, as icosahedral \vi{0}{0}{12}{0} clusters have a strong tendency to aggregate with each other \cite{Bernal1959,Miracle2004,Li2008}. Interpenetrating string-like networks of atoms in FI environments have been reported before as indicative of MRO, including the study by Lee et al. \cite{Lee2011} and by Ritter et al. [\cite{Ritter2011}. To visualize these strings for the glasses simulated in this work, bonds were constructed  using \gls{ovito} for FI-atoms with other FI-atoms, present within a cut-off radius of 3.5 \r{A}. \par 
\end{selfcite}

Figure~\ref{f:mro-alt_camg}a shows one of the FI-atom chains found in one of the CAMGs. The yellow coloured atoms are the ones in the FI environment, the blue atoms are the surrounding atoms in the coordination polyhedron. In Figure~\ref{f:mro-alt_camg}b, the number of strings (\% of total strings) below a maximum string length are described for for all FI-atoms (Figure~\ref{f:mro-alt_camg}b1), FI-atoms in the cores (Figure~\ref{f:mro-alt_camg}b2) and FI-atoms in the interfaces (Figure~\ref{f:mro-alt_camg}b3) for the \gls{ng}s and \gls{camg}s. The number of small strings of sizes 3-9 recover in the CAMGs, specifically in the interfaces towards MG and \gls{mght} values with increasing deposition energies. In the glasses, at least 74\% of all linked atoms are in 3-atom or longer string-like networks. This behaviour is seen in both the cores and the interfaces of the \gls{camg}s and \gls{ng}s as well. The \gls{mro} in all the glasses in the context of the larger strings of length 40-100 is similar. \par

\begin{figure} %[!ht]
	\centering
	\begin{subfigure}[b]{0.5\textwidth} \centering
	\includegraphics[width=\textwidth]{mrochain2}	\subcaption{}
	\end{subfigure}%
	\vfill
	\begin{subfigure}{\textwidth} \centering
		\includegraphics[width=\textwidth,trim={0cm 1cm 0cm 1cm},clip]{2021_05/figures/12.pdf}	\subcaption{}
	\end{subfigure}%
	\mycaption{MRO in MG, \gls{mght}, NG and CAMGs}{}
	\label{f:mro-alt_camg}
\end{figure}

\begin{selfcite}
Figure~\ref{f:mro_camg}a shows the fraction of FI-atoms in each of the glasses made from a \qr{10} MG, which are present in a string of a given size. From both Figures~\ref{f:mro-alt_camg}b~and~\ref{f:mro_camg}a, it is evident that most FI-atoms exist in small strings. However, the number of atoms in small strings (3-5 FI-atoms in size) is the lowest in the MG and \gls{mght}. A higher percentage of larger-sized strings is seen in the MG, \gls{mght} and NG. This can be attributed to the geometry of the samples: larger strings can form in MG, \gls{mght}, and NG cases due to periodic boundaries conditions in all directions, and the simulation box being completely filled. Such strings of larger sizes cannot be expected to form in the CAMGs, as the sample considered for analysis without surface artifacts is limited by the dimensions of the representative slabs from within the CAMG films. However, amongst the 3 CAMGs, it can still be noted that a 600 meV/atom CAMG has more 3 FI-atom strings than the 60 meV/atom CAMG. This trend is seen for strings of at least 5 FI-atoms in length. \par

Figure~\ref{f:mro_camg}b shows the average string size for all simulated glasses. The average string size in NG and CAMGs is about 40\% lower than for MG and \gls{mght}. However, with increasing deposition energy, a slight increase in average string size in the CAMGs deposited at 60 and 300 meV/atom compared to the 60 meV/atom CAMG is observed. This indicates that with increasing deposition energies in the CAMGs, the MRO of the strings of FI-atoms can be at least partially recovered. A comparison of the present results with the available experimental data including the structural and magnetic information on \fs CAMGs is not possible as the current simulations are specific to the \cz metallic glass, and also due to the non-availability of an EAM potential for \fs systems. However, some conclusions on the behavior of cluster-assembled glasses, in particular on the of the medium-range order in CAMGs, can help to better understand the experimental results for \fs CAMGs, in particular the comparison to the local motif analysis reported in \cite{Benel2019}. As the local order in CAMGs recovers towards the metallic glass values with increasing impact energies, an increase in the size of the string-like MRO networks is expected. This behavior could explain the strengthening of exchange interactions, and thus the observed increase in the ferromagnetic transition temperature (Curie temperature) with increasing impact energy. \par

\begin{figure}%[!ht]
	\centering
	\includegraphics[width=\textwidth,trim={2cm 0cm 2cm 0cm},clip]{2021_05/figures/9.pdf}
	\includegraphics[width=\textwidth,trim={2cm 0cm 2cm 0cm},clip]{2021_05/figures/10.pdf}
	\mycaption{MRO in 3 nm CAMGs and NGs}{}
	\label{f:mro_camg}
\end{figure}
%\end{selfcite}

%\begin{selfcite}
Next, in Section~\ref{c:camg_quenchrt}, an attempt is made to explain the dependence of the final metastable states of the NG and CAMGs on the initial processing conditions of the clusters themselves. This could help to understand the CIBD process and would allow to traverse the \gls{pel} of metallic glasses. One important processing condition in the simulation is the quenching rate of the MG from which the clusters are formed. Therefore, a comparison of CAMGs and NGs, prepared with different quenching rates is made. \par
\end{selfcite}


%\newpage
\section{Medium-range Order in CAMGs}

\begin{changebar}
In this section, the \gls{mro} in all six metallic glasses (3 \gls{camg}s, \gls{ng}, \gls{mg} and \gls{mght}) shall be inspected to better understand the connectivity of the \gls{fi} units in the \gls{mg} and \gls{mght} and how the \gls{mro} varies in the \gls{camg}s with the deposition energy. \par

\begin{figure}[!h]
	\centering
	\begin{subfigure}[b]{0.5\textwidth} \centering
		\includegraphics[width=\textwidth,trim={0 2.6cm 0 2.1cm},clip]{mrochain2}	\subcaption{}
	\end{subfigure}%
	\vfill
	\begin{subfigure}{\textwidth} \centering
		\includegraphics[width=\textwidth,trim={0cm 1.5cm 0.5cm 1.2cm},clip]{camg_3nm/12.pdf}	\subcaption{}
	\end{subfigure}%
	\mycaption{FI-strings in MGs, MG$_{ht}$, NG and CAMGs}{(a) An FI-string (yellow atoms) seen in the 60 meV/atom \gls{camg} (b) Number of atoms (\%) in the glass samples having strings of sizes less than a cut-off value. See text for more details.}
	\label{f:mro-alt_camg}
\end{figure}

The relative packing of coordination polyhedra centered around solute atoms is used to define \gls{mro} in metallic glasses; it has been shown that the solute atoms exhibit string-like connectivity when the solute concentration goes beyond 20-30 at.\%, \cite{Sheng2006}. Similarly, the string-like connectivity of \gls{fi}-atoms, which are the atoms residing in \gls{fi} coordinations, have been reported to indicate \gls{mro}, as icosahedral \vi{0}{0}{12}{0} clusters have a strong tendency to aggregate with each other \cite{Bernal1959,Miracle2004,Li2008}. Interpenetrating string-like networks of atoms in \gls{fi} environments have been reported before as indicative of \gls{mro}, including the studies by \textcite{Lee2011} and \textcite{Ritter2011}. To visualize these strings for the glasses simulated in this work, bonds were constructed  using \gls{ovito} for \gls{fi}-atoms with other \gls{fi}-atoms, present within a cut-off radius of 3.5 \r{A}. \par 
\end{changebar}

\begin{changebar}
Figure~\ref{f:mro-alt_camg}a shows one of the \gls{fi}-atom chains found in one of the \gls{camg}s. The yellow coloured atoms are the ones in the \gls{fi} environment, the blue atoms are the surrounding atoms in the coordination polyhedron. In Figure~\ref{f:mro-alt_camg}b, the number of strings (\% of total strings) below a maximum string length are described for all \gls{fi}-atoms (Figure~\ref{f:mro-alt_camg}b1), \gls{fi}-atoms in the cores (Figure~\ref{f:mro-alt_camg}b2) and \gls{fi}-atoms in the interfaces (Figure~\ref{f:mro-alt_camg}b3) for the \gls{ng}s and \gls{camg}s. \par The number of small strings of sizes 3-9 recover in the \gls{camg}s, specifically in the interfaces towards \gls{mg} and \gls{mght} values with increasing deposition energies. In the glasses, at least 74\% of all linked atoms are in 3-atom or longer string-like networks. This behaviour is seen in both the cores and the interfaces of the \gls{camg}s and \gls{ng}s as well. The \gls{mro}s of all the glasses in the context of the larger strings of length 40-100 are similar as characterized by the chosen metrics. \par

\begin{figure}[!h]
	\centering
	\begin{subfigure}{0.5\textwidth} \centering
		\includegraphics[width=\textwidth,trim={4cm 1cm 4cm 1cm},clip]{camg_3nm/9.pdf}
	\end{subfigure}%
	\hfill
	\begin{subfigure}{0.5\textwidth} \centering
		\includegraphics[width=\textwidth,trim={4cm 1cm 4cm 1cm},clip]{camg_3nm/10.pdf}
	\end{subfigure}%
	\mycaption{MRO in 3 nm \gls{camg}s and \gls{ng}s}{(a) Distribution of the number (\%) of atoms existing in FI-strings of varying sizes  (b) Average FI string size in the six glasses.}
	\label{f:mro_camg}
\end{figure}

Figure~\ref{f:mro_camg}a shows the fraction of \gls{fi}-atoms in each of the glasses made from a \qr{10} \gls{mg}, which are present in a string of a given size. From both Figures~\ref{f:mro-alt_camg}b~and~\ref{f:mro_camg}a, it is evident that most \gls{fi}-atoms exist in small strings. However, the number of atoms in small strings (3-5 \gls{fi}-atoms in size) is the lowest in the \gls{mg} and \gls{mght}. A higher percentage of larger-sized strings is seen in the \gls{mg}, \gls{mght} and \gls{ng}. This can be attributed to the geometry of the samples: larger strings can form in \gls{mg}, \gls{mght}, and \gls{ng} cases due to periodic boundaries conditions in all directions, and the simulation box being completely filled. Such strings of larger sizes cannot be expected to form in the \gls{camg}s, as the sample considered for analysis without surface artifacts is limited by the dimensions of the representative slabs from within the \gls{camg} films. However, amongst the 3 \gls{camg}s, it can still be noted that a 600 meV/atom \gls{camg} has more 3 \gls{fi}-atom strings than the 60 meV/atom \gls{camg}. This trend is seen for strings of at least 5 \gls{fi}-atoms in length. \par

Figure~\ref{f:mro_camg}b shows the average string size for all simulated glasses. The average string size in \gls{ng} and \gls{camg}s is about 40\% lower than for \gls{mg} and \gls{mght}. However, with increasing deposition energy, a slight increase in average string size in the \gls{camg}s deposited at 300 and 600 meV/atom compared to the 60 meV/atom \gls{camg} is observed. This indicates that with increasing deposition energies in the \gls{camg}s, the \gls{mro} of the strings of \gls{fi}-atoms can be at least partially recovered. \par

A comparison of the present results with the available experimental data including the structural and magnetic information on \fs \gls{camg}s \cite{Benel2019} is not possible as the current simulations are specific to the \cz metallic glass, and also due to the non-availability of an \gls{eam} potential for \fs systems. However, some conclusions on the behavior of cluster-assembled glasses, in particular on the of the medium-range order in \gls{camg}s, can help to better understand the experimental results for \fs \gls{camg}s, in particular the comparison to the local motif analysis reported in \cite{Benel2019}. As the local order in \gls{camg}s recovers towards the metallic glass values with increasing impact energies, an increase in the size of the string-like \gls{mro} networks is expected. This behavior could explain the strengthening of exchange interactions, and thus the observed increase in the \gls{tc}, i.e., the ferromagnetic transition temperature with increasing impact energy. \par

%\end{changebar}

%\begin{changebar}
\end{changebar}

%\section{CAMG behaviour across quench rates} \label{s:camg_quenchrt}

\begin{selfcite}
In this section, an attempt is made to explain the dependence of the final metastable states of the NG and CAMGs on the initial processing conditions of the clusters themselves. This could help to understand the CIBD process and would allow one to traverse the \gls{pel} of metallic glasses. One important processing condition in the simulation is the quenching rate of the MG from which the clusters are formed. Therefore, a comparison of CAMGs and NGs, prepared with different quenching rates is made. \par
\end{selfcite}

The MG, \gls{mght}s, were prepared at three quench rates of \qr{10}, \qr{12} and \qr{14}. Subsequently, clusters were derived for the three quench rates, and the CAMGs and NGs were prepared from them. All the glasses were compared against one another in terms of local icosahedral \gls{sro} and potential energy. As mention in Section~\ref{s:vol-mgs}, the current potential does not correctly reproduce the volume behaviour of RQ MGs with quench rates, and hence the CAMGs volume behaviour with quench rate not attempted to be explained. \par

Figure~\ref{f:camg_fi} describes the \gls{fi} order in the glasses. The fraction of FI in RQ MGs is highest at 7\% for the \qr{10} MG, and at 5.5\% and 3.8\% for the \qr{12} and \qr{14} MGs respectively. The \gls{mght} FI drops at low quench rate of \qr{10}, but increases at \qr{14}, revealing that the heat-treatment causes different effects at different quench rates. The \gls{cbmg}s appear to consistently demonstrate lesser FI-packed states for the three quench rates. For the CBMGs, the FI-order in the core ad interfacial regions are also represented. The interfaces possess higher FI-packing than the cores. The order in entire CBMG sample is depicted as 'All' in the Figure~\ref{f:camg_fi}. Amongst the CBMGs, the NG/CCMGs have the lowest FI-order. For the three quench rates, a trend of increase in the FI-order with increasing deposition energy is hinted; as the FI-600 meV/atom CAMG $\geq$ FI-60 meV/atom CAMG. However the FI-300 meV/atom CAMG does not follow this trend for the \qr{12} and \qr{14} cases. The \gls{ilike} was considered a better candidate to explore this trend. \par

\begin{figure} %[!htp]
	\includegraphics[width=\textwidth]{collated/ico_coll_3nm_Cu50Zr50_CAMG_vs_MG}
	\mycaption{Full-Icosahedral ordering versus quench rates}{Variation of FI in the various CAMGs and MGs, in comparison to their precursor MGs and \gls{mght}s is uninfluenced by the quench rate used in the processing.}
	\label{f:camg_fi}
\end{figure}

\begin{selfcite}
Previously in literature \cite{Ding2014} it has been discussed that the discintinction bewteen FI and CIolike coorindations is a fine line, owing the definition of the threshold in the voronoi tessealation. In the previous sections, it has been mentioned that the ILO is a well-known indicator of glass stability and of packing \cite{Ding2014,Ding2014a,Adelman1976}. Figure~\ref{f:camg_ilo} shows the variation of the ILO for three different quenching rates of the MGs from which the clusters were prepared. Firstly, it can be seen that the ILO of the MG decreases with increasing quenching rates. Secondly, the heat-treatment for the MGs results in different dependence for the different quenching rates. For the MGs prepared with cooling rates \qr{10} and \qr{12} (Figure~\ref{f:camg_ilo}a, Figure~\ref{f:camg_ilo}b), the heat-treated glasses \gls{mght} exhibit lower ILO than the MG. At the highest cooling rate of \qr{14}, the same heat-treatment places the \gls{mght} at a state with higher ILO value (Figure~\ref{f:camg_ilo}c). Given that the clusters undergo the same heat-treatment as \gls{mght}, the NGs and CAMGs can be compared with the \gls{mght}. \par

At all quenching rates, it is noted that the interfaces exhibit higher SRO than the cores, due to the chemical effects discussed in Section 3.5. The CAMGs, however, show an increase in the ILO with increasing impact energies, for all the three quench rates used in the present study. Therefore, it is concluded that for a given cooling rate of the as-prepared clusters, the CIBD process determines the final states of the CAMGs. \par

\begin{figure} %[!htp]
	\includegraphics[width=\textwidth,trim={0.65cm 2.4cm 0.65cm 1.8cm},clip]{2021_05/figures/11.pdf}
	\mycaption{Icosahedral-like ordering versus quench rates in CBMGs}{Variation of ILO in the various CBMGs and MGs, compared to their
		precursor MGs and \gls{mght}s for cluster derived from (a) \qr{10}, (b) \qr{12}, and (c) \qr{14} \cz RQ MGs.}
	\label{f:camg_ilo}
\end{figure}

Unlike in the MG and the \gls{mght}, the ILO packing and stability do not correlate with each other in the CAMGs and NGs. For a CAMG prepared at a given impact energy, the ILO packing increases with quenching rate. The strain energy due to CAMG processing could be dominating the stability gained from the slow quenching rate in the NGs and CAMGs, leading to the observed behavior. In Figure~\ref{f:film_network}d, the strain analysis shows that von-Mises strains for the 3 nm clusters studied here are higher in both NGs and CAMGs, compared to previous reports for 7 nm cluster NGs \cite{Adjaoud2018}. This leads to the conclusion that the size of the clusters plays an important role in the final structures attained by the CAMGs. Further studies with different cluster sizes, including size distributions, and random deposition locations are needed to gain further understanding on the role of the processing parameters in cluster assembled metallic glasses prepared by compaction (NGs) and by energetic impact (CAMGs). \par
\end{selfcite}

\begin{quote}
Figure~\ref{f:camg_pote}a shows the average potential energy per atom for the six glasses, for the three cooling rates. Also, a possible representation of the PEL has been illusterate and overlayed upon the potential energy states, in Figure~\ref{f:camg_pote}b. We observe that while in the low cooling rate cases of \qr{10} and \qr{12} \cz glasses, the \gls{mght}s are at a higher energy states, indicating a rejuvenation process \cite{Wakeda2015,Saida2017}. This is consistent with the ILO behaviour of MG and \gls{mght}. The CAMGs and NGs however, seem to be at higher energy states than that MG \todo[inline]{working out an argument; this is not true for \qr{12} 300 meV/atom case)}. The ILO packing and stability are not correlated in the CAMGs and NGs. While the general increasing trend with deposition energy is seen at all quench rates, the quench rate itself has little influence on the final state of these cluster. This indicates that if \gls{mght} are sitting at one local minima for their respective quench rates in the potential energy landscape, the CAMGs and NGs are displaced to a nearest slightly lower local minima. The answer to why the CAMGs and NGs remain unaffected by quench rates could be the deformation processes in these glasses (competition between strain energy and stability gain from slowing quench rate). We refer back to Figure 4, where the strain analysis shows that von-Mises strains for the 3nm clusters are quite high for both NGs and CAMGs, especially in comparison with previous reports for 7nm cluster NGs \cite{Adjaoud2018}, leading us to believe that size of the clusters may indeed also play a role in those complex phenomena. \par
\end{quote}

\begin{figure}[!htp]
	\begin{subfigure}{\textwidth}
		\includegraphics[width=\textwidth]{collated/pote_coll_3nm_Cu50Zr50_CAMG_vs_MG}
	\end{subfigure}%
	\vfill
	\begin{subfigure}{\textwidth}
	\includegraphics[width=\textwidth]{pe-ldscp.png}
	\end{subfigure}%
	\mycaption{Energy versus quench rates}{(a) Variation of FI in the various CAMGs and MGs, in comparison to their precursor MGs and MG$_{ht}$s is uninfluenced by the quench rate used in the processing. (b) Average potential energy per atom in the metallic glasses, along with an imagined depiction of the \gls{pel}.}
	\label{f:camg_pote}
\end{figure}
\section{Influence of Quench Rates on Final Structure of CAMGs} \label{s:camg_quenchrt}

\begin{changebar}
In this section, we explore the dependence of the final metastable states of the \gls{ng} and \gls{camg}s on the initial processing conditions of the clusters themselves. This could help to understand the \gls{cibd} process and would allow one to traverse the \gls{pel} of metallic glasses. One important processing condition in the simulation is the quenching rate of the \gls{mg} from which the clusters are formed. Therefore a comparison of \gls{camg}s and \gls{ng}s, prepared with different quenching rates, is performed. \par
\end{changebar}

The \gls{mg}, and \gls{mght}s, were prepared at three quench rates of \qr{10}, \qr{12} and \qr{14}. Subsequently, clusters were derived from them for the three quench rates, and the \gls{camg}s and \gls{ng}s were prepared. All the glasses were compared against one another in terms of local icosahedral \gls{sro} and potential energy. As mention in Section~\ref{s:vol-mgs}, the current potential does not correctly reproduce the volume behaviour of \gls{rq} \gls{mg}s with quench rates, and hence the comparison glasses' volume behaviour with quench rates is not attempted. \par

Figure~\ref{f:camg_fi} describes the \gls{fi} order in the glasses. The fraction of \gls{fi} in \gls{rq} \gls{mg}s is highest at 7\% for the \qr{10} \gls{mg}, and at 5.5\% and 3.8\% for the \qr{12} and \qr{14} \gls{mg}s respectively. The \gls{mght} \gls{fi} drops at low quench rate of \qr{10}, but increases at \qr{14}, revealing that the heat-treatment causes different effects at different quench rates. Compared to the \gls{mght}, the NGs and \gls{camg}s appear to consistently demonstrate lesser \gls{fi}-packed states for the three quench rates. For the \gls{camg}s and \gls{ng}s, the \gls{fi}-order in the core and interfacial regions are also represented. As seen in Section~\ref{f:voro_camg}, the interfaces possess higher \gls{fi}-packing than the cores for the \gls{camg}s and \gls{ng}s. The order in the entire NGs and \gls{camg}s sample is depicted as 'All' in the Figure~\ref{f:camg_fi}. Amongst the \gls{camg}s and \gls{ng}s, the \gls{ng}s have the lowest \gls{fi}-order. For the three quench rates, a trend of increase in the \gls{fi}-order with increasing deposition energy is hinted; as the \gls{fi}-order for the 600 meV/atom \gls{camg} is greater than that for the 60 meV/atom \gls{camg}. However the \gls{fi}-300 meV/atom \gls{camg} does not follow this trend for the \qr{12} and \qr{14} cases. The \gls{ilike} was considered as an alternative candidate to explore \gls{sro}. \par

\begin{figure}[!h] \centering
\includegraphics[width=\textwidth]{collated/ico_coll_3nm_Cu50Zr50_CAMG_vs_MG}
\mycaption{Full-icosahedral Ordering versus Quench Rates}{The \gls{fi} in the \gls{camg}s and \gls{mg}s, in comparison to their precursor \gls{mg}s and \gls{mght}s is invariant with the quench rate used.}
\label{f:camg_fi}
\end{figure}

\begin{changebar}
Previously in literature \cite{Ding2014}, it has been discussed that the distinction between \gls{fi} and \gls{ilike} coordinations is ambiguous, owing to the defined threshold of a distortion tolerance set while performing the Voronoi tessellation. In the previous sections, it has also been mentioned that the \gls{ilo} is a well-known indicator of glass stability and of packing \cite{Ding2014,Ding2014a,Yue2018,Cheng2008}. Now, the ILO behavior in CAMGs is observed with respect to the processing conditions of the RQ MGs, from which the clusters are derived. CAMGs made from fast-quenched \qr{12}-MGs and 
\qr{14}-MGs were investigated. For these reasons, \gls{ilo}s of the six glasses are also examined. Figure~\ref{f:camg_ilo} shows the variation of the \gls{ilo} for three different quenching rates of the \gls{mg}s from which the clusters were prepared. Firstly, it can be seen that the \gls{ilo} in the \gls{mg} decreases with increasing quenching rates. Secondly, the heat-treatment for the \gls{mg}s results in different behaviours for the different quenching rates. For the \gls{mg}s prepared with cooling rates \qr{10} (Figure~\ref{f:camg_ilo}a) and \qr{12} (Figure~\ref{f:camg_ilo}b), the heat-treated glasses \gls{mght} exhibit lower \gls{ilo} than the \gls{mg}. At the highest cooling rate of \qr{14}, the same heat-treatment places the \gls{mght} at a state with higher \gls{ilo} value (Figure~\ref{f:camg_ilo}c). Given that the clusters undergo the same heat-treatment as \gls{mght}, the \gls{ng}s and \gls{camg}s can be compared with the \gls{mght}. \par

At all quenching rates, it is noted that the interfaces exhibit higher \gls{ilike} \gls{sro} than the cores, due to the chemical effects discussed in Section~\ref{s:vorocamg}. The \gls{camg}s, however, show an increase in the \gls{ilo} with increasing impact energies, for all the three quench rates used in the present study. Therefore, it is concluded that for a given cooling rate of the as-prepared clusters, the \gls{cibd} process determines the final \gls{ilike} states of the \gls{camg}s. \par

\begin{figure}[!h]
	\includegraphics[width=\textwidth,trim={0.65cm 2.4cm 0.65cm 1.8cm},clip]{camg_3nm/11.pdf}
	\mycaption{Icosahedral-like Ordering versus Quench Rates in CAMGs}{Variation of \gls{ilo} in the various NGs and \gls{camg}s and \gls{mg}s, compared to their precursor \gls{mg}s and \gls{mght}s for cluster derived from (a) \qr{10}, (b) \qr{12}, and (c) \qr{14} \cz \gls{rq} \gls{mg}s.}
	\label{f:camg_ilo}
\end{figure}
\end{changebar}

\begin{changebar}
Figure~\ref{f:camg_pote}a shows the average P.E. per atom for the cores, interfaces and the entire sample for the six glasses, and for the three cooling rates. In Figure~\ref{f:camg_pote}b, the average P.E. of only the entire sample is represented, and a possible representation of the \gls{pel} has also been illustrated and overlayed upon the plots. We observe that while in the low cooling rate cases of \qr{10} and \qr{12} \cz glasses, the \gls{mght}s are at a higher energy states, indicating a rejuvenation process \cite{Wakeda2015,Saida2017}. This is consistent with the \gls{ilo} behaviour of \gls{mg} and \gls{mght}. Furthermore, when examined across the quench rates, the potential energies of the glasses are lower at lower quench rates. \par
\end{changebar}

\begin{figure}[!h] %[!htp]
	\begin{subfigure}{\textwidth}
		\includegraphics[width=\textwidth,trim={0cm 0.45cm 0cm 0cm},clip]{collated/pote_coll_3nm_Cu50Zr50_CAMG_vs_MG}
	\end{subfigure}%
	\vfill
	\begin{subfigure}{\textwidth}
		\includegraphics[width=\textwidth,trim={0cm 0cm 0cm 0.75cm},clip]{pe-ldscp.png}
	\end{subfigure}%
	\mycaption{Average P.E. of \gls{camg}s versus quench rates}{(a)  in the various \gls{camg}s and \gls{mg}s, in comparison to their precursor \gls{mg}s and \gls{mght}s is for the by the quench rate used in the processing. (b) Average P.E. per atom in the metallic glasses, along with an imagined depiction of the \gls{pel}.}
	\label{f:camg_pote}
\end{figure}

The \gls{camg}s and \gls{ng}s however, for any given cooling rate, do not show any clear trend in the potential energies. Unlike in the \gls{mg} and the \gls{mght}, the stability and \gls{ilo} packing do not correlate with each other in the \gls{camg}s and \gls{ng}s. For a \gls{camg} prepared at a given impact energy, the \gls{ilo} packing increases with quenching rate. The \gls{ng} is consistently at a higher energy state than the \gls{mght} at all quench rates. In contrast, the potential energies of the \gls{camg}s decrease with impact energy for the lower quench rate case of \qr{10}, but at higher quench rates of \qr{12} and \qr{14}, the difference in the energy states of the \gls{ng}s and \gls{camg}s is seen to diminish. This seemingly random behaviour could be an effect of the local atomic strain accumulated during the deposition process. \par

\begin{changebar}
It may be that the strain energy has a lower effect for glasses quenched at lower rates. It may also be that the strain energy gained due to \gls{camg} processing could be erratic, and dominating the stability gained from the quenching in the \gls{ng}s and \gls{camg}s, leading to the observed behavior. In Figure~\ref{f:film_network}d, the strain analysis shows that von-Mises strains for the 3 nm clusters studied here are higher in both \gls{ng}s and \gls{camg}s, compared to previous reports for 7 nm cluster \gls{ng}s \cite{Adjaoud2018}. This leads to the conclusion that the size of the clusters plays an important role in the final structures attained by the \gls{camg}s. Further studies with different cluster sizes, including size distributions, and random deposition locations are needed to gain further understanding on the role of the processing parameters in cluster assembled metallic glasses prepared by compaction (\gls{ng}s) and by energetic impact (\gls{camg}s).
\end{changebar} \par

%\section{Summary}

\begin{selfcite}
In this study, a molecular dynamics simulation protocol was developed to study the local structure of \cz cluster-assembled metallic glasses (CAMGs). The present model of CAMGs uses chemically segregated amorphous Cu50Zr50 clusters of ~800 atoms, which are 3 nm in diameter, being deposited onto a substrate at different impact energies. These CAMGs are compared with NGs produced by mechanical compaction of the same clusters, and to the conventionally prepared melt-quenched metallic glasses of the same composition. The main results of the study are as follows:

1. In the CAMGs, two chemically distinct amorphous phases were observed: cores and interfaces, which constitute an interconnected network of interfaces in which the cores with their distinctly different local structure are embedded. The formation of Cu-rich interfaces is observed at impact energies up to 600 meV/atom. The interfaces appear to be completely absent at the extreme impact energy of 6000 meV/atom.
2. The FI and ILO short-range order parameters are lower in the NG and CAMG, for both the cores and interfaces. The interfaces exhibit higher FI and ILO compared to the cores, with a higher density than the cores. Due to the chemical heterogeneity between cores and interfaces, the core regions occupy lower energy states, thus stabilizing the CAMG structures.
3. The local ILO as well as the MRO in CAMGs are found to increase with the impact energy. Furthermore, the ILO increases with impact energy, irrespective of the quenching rates used to prepare the clusters. Consequently, at a fixed overall bulk composition of the metallic glass, control of the local structure is possible by modifying the processing conditions. The SRO and MRO in CAMGs recover towards the metallic glass values with increasing impact energies.
\end{selfcite}



\begin{figure}
	\includegraphics[width=\textwidth]{/home/mj0054/Documents/work/writing/articles/2021_05/figures/graphical_abstract2.pdf}
	\label{f:camg-summary}
	\mycaption{CAMG summary}{CAMG summary}
\end{figure}
\section{Summary}
\begin{changebar}
In this chapter, the simulated \cz \gls{camg}s were studied and characterized. The present model of \gls{camg}s uses chemically segregated amorphous \cz clusters of $\sim$800 atoms, which are 3 nm in diameter, being deposited onto a substrate at different impact energies. These \gls{camg}s are compared with \gls{ng}s produced by mechanical compaction of the same precursor clusters, and to the conventionally prepared melt-quenched metallic glasses of the same composition. \par

In the \gls{camg}s, two chemically distinct amorphous phases were observed: cores and interfaces, which constitute an interconnected network of interfaces in which the cores with their distinctly different local structure are embedded. The formation of Cu-rich interfaces is observed at impact energies up to 600 meV/atom. Due to the chemical heterogeneity between cores and interfaces, the core regions occupy lower energy states, thus stabilizing the \gls{camg} structures. The interfaces appear to be completely absent at the extreme impact energy of 6000 meV/atom. \par

The \gls{fi} and \gls{ilo} short-range order parameters are lower in the \gls{ng} and \gls{camg}, in the conventional (\qr{10} quench rate) case, for both the cores and interfaces. The interfaces exhibit higher \gls{fi} and \gls{ilo} compared to the cores, with a higher density than the cores. A graphical depiction of these findings is represented in Figure~\ref{f:camg-summary}. \par

The present simulations constitute the first model explaining the tailoring of local \gls{sro} as well as the \gls{mro} in \gls{camg}s by increasing the impact energy. Furthermore, the \gls{ilo} increases with impact energy, irrespective of the quenching rates used to prepare the 3 nm sized clusters. Consequently, at a fixed overall bulk composition of the metallic glass, control of the local structure is possible by simply modifying the processing conditions. The \gls{sro} and \gls{mro} in \gls{camg}s recover towards the metallic glass values with increasing impact energies.


\begin{figure}
	\includegraphics[width=\textwidth,trim={0 3cm 0cm 2.1cm},clip]{camg_3nm/graphical_abstract2.pdf}
	\mycaption{SRO and MRO tailoring in CAMGs}{A visual summary of the formation of core-interface structures, and the tailoring of FI-order and FI-string size with deposition energy in \gls{camg}s.}
	\label{f:camg-summary}
\end{figure}
\end{changebar}