\vspace{-0.5cm}
%\hrule
 \begin{table}[!ht]
\begin{tabular}{R{3 cm} m{14.6cm}}
{Nov 2016-present}& {\color{blue} Project Assistantship: Study of bonding in binary metallic systems} \footnotemark[1] \\ %\label{note2}For details, kindly visit my homepage: http://syamalpraneeth.wixsite.com/home}\\
 & University of Hyderabad, India. \vspace{0.2 cm} \\
%& Performing semi-empirical analysis to understand atom-pair bonds in binary alloys.\vspace{0.2 cm} 

{Jul-Nov 2016}& {\color{blue} Project Assistantship: Study of frustrated magnetic systems} \footnotemark[1]\\
 & University of Hyderabad, India. \vspace{0.2 cm} \\
%& Performed further studies on $PbCuTeO_{5}$, which I had synthesised as a part of my master's project work. Also succeeded in preparing an other antiferromagnetic system $KCuTa_{3}O_{9}$ and studied its magnetic properties. \\ \vspace{0.2 cm}
 
{May-July 2015}& {\color{blue} Internship: Study of ferroelectric transitions in $BaSn_{x}Ti_{1-x}O_{3}$} \footnotemark[1]\\
 & National University of Singapore, Singapore. \vspace{0.2 cm} \\
% \vspace{0.2 cm}
% & Interfaced the experimental set-up (to measure pyroelectric current) with a computer using NI-LabVIEW software and wrote codes to calculate entropy change $\Delta S_{E}$ corresponding to the ferro-electric transition of the series of compounds. Synthesized and characterized the samples and worked towards optimization of the experimental procedure. \vspace{0.2 cm}\\
 
{May-July 2014}& {\color{blue} Internship: Calculations of Magnetic Field due to a Helmholtz Coil} \footnotemark[1] \\
 &Indian Institute of Technology-Madras, India.\vspace{0.2 cm} \\
% \vspace{0.2 cm}
% & Coded to calculate magnetic field at any point in space due a Helmholtz coil set up, by considering multiple layers of wiring and discarding approximate approaches. Determined optimal design parameters of the coil. It was a MATLAB learning project. \vspace{0.2 cm}\\
 
{May-July 2013} & {\color{blue} Internship:  $ \tau^{-} \tau^{+} $ Analysis for Decay of Neutral Heavy Higgs in two Higgs Doublet Model} \footnotemark[1] \\
% \vspace{0.2 cm}
& University of Wisconsin-Madison, USA. \\
%& Learnt to handle simulated 14 TeV proton-proton collision data sets generated using PYTHIA 8 and analysed $H/A \longrightarrow \tau\tau$ decay process. Imposed conditions to improve signal to background ratio and employed mass reconstruction techniques in order to test the two-Higgs doublet model.
\end{tabular}
\end{table}
\vspace{-0.4cm}
