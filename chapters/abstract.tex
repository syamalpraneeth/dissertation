\chapter*{Abstract/Zusammenfassung}
\addcontentsline{toc}{chapter}{Abstract/Zusammenfassung}

\section*{Abstract}
Metallic glasses are an exciting class of amorphous materials, primarily known for their interesting properties such as high resilience and superior strength. The opportunity to tailor their local structure is a step towards an increased control of their properties in a manner that is possible with crystalline materials today. Recently, films prepared via an intricate assembly of amorphous nanoclusters by energetic deposition were reported to demonstrate a remarkable change in properties with only the deposition energy. The exotic properties of the so-prepared cluster-assembled metallic glasses are currently believed to arise from the deposition process, and the formation of cluster-cluster interfaces---creating novel microstructures otherwise absent in traditionally prepared glasses. Being in the nascent stages of conception, the nature of the cluster-assembled glasses remains largely unexplored. \par

In the present thesis, molecular dynamics simulations of the cluster-assembled metallic glasses are studied in a model CuZr system. The development and implementation of new simulation protocols uncover the mechanisms of the formation routes to these novel cluster-assembled glassy films, the morphologies adopted by the clusters, and the local topological order in the materials. Two amorphous phases are identified in these glasses: one in the cores of the clusters, and the other in the continuous network of interfaces formed amongst the clusters. The amorphous short- and medium-range orders of cluster-assembled glasses are demonstrated to not only differ considerably from the traditional metallic glasses prepared by rapid quenching, but also to vary with cluster-impact energies in both the core and interface regions. In cluster-assembled glasses the interface regions are better packed than the cores, while the core atoms occupy lower energy states---a surprising outcome when contrasted with the traditional glasses where better packing and lower energetic states occur together.
%
%In the traditional glasses, better packing and lower energetic states go together.
%interfaces are better packed than the cores
%core atoms that occupy lower energy states
%The interfaces are found to be better packed than cores, although the cores occupy lower energy states---a surprising result going against packing-stability correlations in traditional glasses.
Such an interesting occurrence is found to be a consequence of the local chemical structure of the cluster being carried over to the core and interfaces. The inherent chemical heterogeneity of the precursor clusters also plays a role in the variation of local order and energetic states of cluster-assembled glasses made from varying cluster sizes. However, the local short-range order and the thermal evolution of the enthalpy is found to be invariant with cluster size in these materials. These investigations form the groundwork for the computer-aided understanding of amorphous cluster assembly and the synthesis of tailorable non-crystalline architectures in the quest to harness the properties of amorphous materials in the future. %\clearpage

\section*{Zusammenfassung} \todo{Translate Abstract to German}
Will translate Zussammenfassung along with professional proofreading, based on the finalized English version.
%Metallische Gläser sind Materialien, die durch Amorphisierung von Metall-Metalloid-Systemen hergestellt werden und seit ihrer Entdeckung in den 1960er Jahren gut untersucht wurden. Die Möglichkeit, ihre Struktur nach Maß zu gestalten, fördert Fortschritte bei der entsprechenden Kontrolle ihrer Eigenschaften in dem Maße, wie es heute bei kristallinen Materialien möglich ist. Eine kürzlich entwickelte Herstellungsmethode ist die komplizierte Anordnung von amorphen Nanoclustern durch energetische Abscheidung, die eine bemerkenswerte Veränderung der Materialeigenschaften mit der Abscheidungsenergie zeigt. Solche cluster-assemblierten metallischen Gläser sollen neuartige amorphe Strukturen besitzen, die durch die Bildung von Cluster-Cluster-Grenzflächen entstehen, analog zu den Körnern und Korngrenzen in nanokristallinen Materialien. In der vorliegenden Arbeit werden Molekulardynamiksimulationen solcher cluster-assemblierter metallischer Gläser in einem Kupfer-Zirkonium-Modellsystem untersucht. Durch die Entwicklung und Implementierung neuer Simulationsprotokolle werden die Mechanismen der Bildungswege zu diesen neuartigen cluster-assemblierten glasartigen Filmen, die von den Clustern angenommenen Morphologien und die lokale topologische Ordnung in den Materialien aufgedeckt. In diesen Gläsern werden zwei amorphe Phasen identifiziert: eine in den Kernen der Cluster und die andere in dem kontinuierlichen Netzwerk von Grenzflächen, das sich zwischen den Clustern bildet. Die amorphen kurz- und mittelfristigen Ordnungen der cluster-assemblierten Gläser unterscheiden sich nicht nur erheblich von den traditionellen metallischen Gläsern, die durch schnelles Abschrecken hergestellt werden, sondern variieren auch beträchtlich mit den Aufprallenergien der Cluster sowohl in den Kern- als auch in den Grenzflächenregionen. Ein überraschendes Ergebnis ist, dass die Grenzflächen zwar besser gepackt sind als die Kerne, aber die Kernatome niedrigere Energiezustände einnehmen. Dieses interessante Ergebnis ist eine Folge der lokalen chemischen Struktur des Clusters, die sich auf den Kern und die Grenzflächen überträgt. Die inhärente chemische Heterogenität der Vorläufercluster spielt ebenfalls eine Rolle bei der Variation der lokalen Ordnung und der energetischen Zustände von cluster-assemblierten Gläsern, die aus unterschiedlich großen Clustern hergestellt werden. Die lokale Nahbereichsordnung und die thermische Entwicklung der Enthalpie sind in diesen Materialien jedoch unveränderlich mit der Clustergröße. Diese Untersuchungen bilden die Grundlage für ein computergestütztes Verständnis des Zusammenbaus amorpher Cluster und die Synthese maßgeschneiderter nichtkristalliner Architekturen, um in Zukunft die Eigenschaften amorpher Materialien nutzen zu können.