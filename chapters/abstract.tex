\chapter*{Abstract/Zusammenfassung}
\addcontentsline{toc}{chapter}{Abstract/Zusammenfassung}

\section*{Abstract}
Metallic glasses are an exciting class of amorphous materials, primarily known for their interesting properties such as high resilience and superior strength. The opportunity to tailor their local structure is a step towards an increased control of their properties in a manner that is possible with crystalline materials today. Recently, films prepared via an intricate assembly of amorphous nanoclusters by energetic deposition were reported to demonstrate a remarkable change in properties with only the deposition energy. The properties of the so-prepared cluster-assembled metallic glasses are currently believed to arise from the deposition process, and the formation of cluster-cluster interfaces---creating novel microstructures otherwise absent in traditionally prepared glasses. Being in the nascent stages of conception, the nature of the cluster-assembled glasses remains largely unexplored. \par

In the present thesis, molecular dynamics simulations of the cluster-assembled metallic glasses are studied in a model CuZr system. The development and implementation of new simulation protocols uncover the mechanisms of the formation routes to these novel cluster-assembled glassy films, the morphologies adopted by the clusters, and the local topological order in the materials. Two amorphous phases are identified in these glasses: one in the cores of the clusters, and the other in the continuous network of interfaces formed amongst the clusters. The amorphous short- and medium-range orders of cluster-assembled glasses are demonstrated to not only differ considerably from the traditional metallic glasses prepared by rapid quenching, but also to vary with cluster-impact energies in both the core and interface regions. In cluster-assembled glasses the interface regions are better packed than the cores, while the core atoms occupy lower energy states---a surprising outcome when contrasted with the traditional glasses where better packing and lower energetic states occur together. Such an interesting occurrence is found to be a consequence of the local chemical structure of the cluster being carried over to the core and interfaces. The inherent chemical heterogeneity of the precursor clusters also plays a role in the variation of local order and energetic states of cluster-assembled glasses made from varying cluster sizes. However, the local short-range order and the thermal evolution of the enthalpy is found to be invariant with cluster size in these materials. These investigations form the groundwork for the computer-aided understanding of amorphous cluster assembly and the synthesis of tailorable non-crystalline architectures in the quest to harness the properties of amorphous materials in the future. %\clearpage

\section*{Zusammenfassung}
Metallische Gläser sind eine spannende Klasse amorpher Materialien, die vor allem für ihre interessanten Eigenschaften wie hohe Elastizität und überlegene Festigkeit bekannt sind. Die Möglichkeit, ihre lokale Struktur maßzuschneidern, ist ein Schritt hin zu einer besseren Kontrolle ihrer Eigenschaften, als dies heute bei kristallinen Materialien möglich ist. Kürzlich wurde berichtet, dass Filme, die durch eine komplizierte Anordnung von amorphen Nanoclustern durch energetische Abscheidung hergestellt wurden, eine bemerkenswerte Veränderung der Eigenschaften allein durch die Abscheidungsenergie aufweisen. Es wird derzeit angenommen, dass die Eigenschaften der so hergestellten cluster-assemblierten metallischen Gläser durch den Abscheidungsprozess und die Bildung von Cluster-Cluster-Grenzflächen, die neuartige Mikrostrukturen erzeugen, die es in herkömmlich hergestellten Gläsern nicht gibt. Da das Konzept noch in den Kinderschuhen steckt, ist die Natur der cluster-assemblierten Gläser noch weitgehend unerforscht.

In der vorliegenden Arbeit werden Molekulardynamiksimulationen der cluster-assemblierten metallischen Gläser in einem CuZr-Modellsystem untersucht. Die Entwicklung und Implementierung neuer Simulationsprotokolle deckt die Mechanismen der Bildungswege dieser neuartigen cluster-assemblierten glasartigen Filme, die Morphologien, die die Cluster annehmen, und die lokale topologische Ordnung in den Materialien. In diesen Gläsern werden zwei amorphe Phasen identifiziert: eine in den Kernen der Cluster und die andere in dem kontinuierlichen Netzwerk von Grenzflächen, das sich zwischen den Clustern bildet. Die amorphen kurz- und mittelfristigen Ordnungen der cluster-assemblierten Gläser unterscheiden sich nicht nur erheblich von den traditionellen metallischen Gläsern, die durch schnelles Abschrecken hergestellt werden, sondern variieren auch mit den Aufprallenergien der Cluster sowohl im Kern als auch in den Grenzflächenbereichen. In cluster-assemblierten Gläsern sind die Grenzflächenregionen besser gepackt als die Kerne, während die Kernatome niedrigere Energiezustände einnehmen - ein überraschendes Ergebnis, wenn man es mit den traditionellen Gläsern vergleicht, bei denen bessere Packung und niedrigere Energiezustände zusammen auftreten. Dieses interessante Ergebnis ist darauf zurückzuführen, dass sich die lokale chemische Struktur des Clusters auf den Kern und die Grenzflächen überträgt. Die inhärente chemische Heterogenität der Vorläufercluster spielt ebenfalls eine Rolle bei der Variation der lokalen Ordnung und energetischen Zuständen von cluster-assemblierten Gläsern, die aus unterschiedlich großen Clustern bestehen. Die lokale Nahbereichsordnung und die thermische Entwicklung der Enthalpie sind in diesen Materialien jedoch nicht von der Clustergröße abhängig. Diese Untersuchungen bilden die Grundlage für das computergestützte Verständnis amorpher Cluster und die Synthese maßgeschneiderter nichtkristalliner Architekturen, um die Eigenschaften amorpher Materialien in Zukunft nutzbar zu machen.

