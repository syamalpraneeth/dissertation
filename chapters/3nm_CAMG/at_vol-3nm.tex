\section{Atomic Volume Analysis}
\begin{selfcite}
The normalized distribution of the atomic coordination volumes for all the atoms in the six metallic glass samples was studied using the Voronoi analysis. Figure~\ref{f:vol_camg}a shows the volume distribution with two peaks approximately at 13.9 \acu for Cu and 21.8 \acu for Zr. The Cu atoms are observed to have lower occupancy numbers compared to those of Zr atoms. \textcite{Cheng2019} reported similar distributions in \czsix nanoglasses, however, with the volume per atom peaks shifted to the left, likely caused by the fact that \czsix MGs are denser than the \cz MGs \cite{Li2008}. Furthermore, the CAMGs are similar to the NGs and MGs in terms of atomic volume distributions. It is also noticed that the distributions in the core and in the interfaces are not significantly different from each other (see Figure~\ref{f:atvol} in \nameref{c:supple}). Using the volume distributions in Figure~\ref{f:vol_camg}a, the exclusion of the surface atoms in the analysis (detailed in Section~\ref{s:corint}) of CAMGs was cross-verified. The surface atoms occupy higher volume per atom than average, and when included in the volume analysis, are known to alter the volume distributions of Cu and Zr atoms with a shoulder to the right of each main peak \cite{Cheng2019}. The absence of such shoulders indeed ascertains the absence of surface atoms in the representative Figure~\ref{f:vol_camg}a. \par

\begin{figure}[!h] \centering
	\includegraphics[width=0.65\columnwidth,trim={0 0 2cm 0.5cm},clip]{2021_05/figures/7.pdf}
	\mycaption{Reduced volumes in cluster assembled metallic glass samples}{ (a) The normalized volume per atom distribution shows
		similar behavior for the six metallic glass samples. (b) Average volumes of the atoms show that the core regions are less densely packed
		than the interfaces}
	\label{f:vol_camg}
\end{figure}

\todo[inline]{should I be distinguishing between NGs and CCMGs in this chapter, because I will be talking about CCMGs in the next chapter}

In Figure~\ref{f:vol_camg}b, which shows the average volume/atom values for all six metallic glass samples, it can be seen that, on average, the atoms in CAMGs and NGs occupy similar volumes. Furthermore, the impact energy does not have an influence on the average volume of the CAMGs. The core regions in the NGs and CAMGs present a higher volume occupancy. By contrast, the opposite behavior is observed for the interfacial atoms. The increase of volume for core atoms and the decrease of volume for the interface atoms in CAMGs and NGs offset each other to result in similar volume occupancies as MGs prepared by RQ, when all atoms of the are considered together. While the interfaces in NGs have previously been reported to be less dense in the MG \cite{Sopu2009,Witte2013}, this need not hold true for the CAMGs as well. The interfaces are richer in Cu-atoms, which, on an average, occupy lower volumes compared to Zr-atoms. The interfaces in the present CAMGs model have an increased density due to the chemical effects, and this is consistent with previous studies of segregated planar interfaces by Adjaoud and Albe \cite{Adjaoud2016}.
\end{selfcite}