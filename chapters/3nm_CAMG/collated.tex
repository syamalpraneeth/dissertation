\section{CAMG behaviour across quench rates} \label{s:camg_quenchrt}

\begin{selfcite}
In this section, an attempt is made to explain the dependence of the final metastable states of the NG and CAMGs on the initial processing conditions of the clusters themselves. This could help to understand the CIBD process and would allow one to traverse the \gls{pel} of metallic glasses. One important processing condition in the simulation is the quenching rate of the MG from which the clusters are formed. Therefore, a comparison of CAMGs and NGs, prepared with different quenching rates is made. \par
\end{selfcite}

The MG, \gls{mght}s, were prepared at three quench rates of \qr{10}, \qr{12} and \qr{14}. Subsequently, clusters were derived for the three quench rates, and the CAMGs and NGs were prepared from them. All the glasses were compared against one another in terms of local icosahedral \gls{sro} and potential energy. As mention in Section~\ref{s:vol-mgs}, the current potential does not correctly reproduce the volume behaviour of RQ MGs with quench rates, and hence the CAMGs volume behaviour with quench rate not attempted to be explained. \par

Figure~\ref{f:camg_fi} describes the \gls{fi} order in the glasses. The fraction of FI in RQ MGs is highest at 7\% for the \qr{10} MG, and at 5.5\% and 3.8\% for the \qr{12} and \qr{14} MGs respectively. The \gls{mght} FI drops at low quench rate of \qr{10}, but increases at \qr{14}, revealing that the heat-treatment causes different effects at different quench rates. The \gls{cbmg}s appear to consistently demonstrate lesser FI-packed states for the three quench rates. For the CBMGs, the FI-order in the core ad interfacial regions are also represented. The interfaces possess higher FI-packing than the cores. The order in entire CBMG sample is depicted as 'All' in the Figure~\ref{f:camg_fi}. Amongst the CBMGs, the NG/CCMGs have the lowest FI-order. For the three quench rates, a trend of increase in the FI-order with increasing deposition energy is hinted; as the FI-600 meV/atom CAMG $\geq$ FI-60 meV/atom CAMG. However the FI-300 meV/atom CAMG does not follow this trend for the \qr{12} and \qr{14} cases. The \gls{ilike} was considered a better candidate to explore this trend. \par

\begin{figure} %[!htp]
	\includegraphics[width=\textwidth]{collated/ico_coll_3nm_Cu50Zr50_CAMG_vs_MG}
	\mycaption{Full-Icosahedral ordering versus quench rates}{Variation of FI in the various CAMGs and MGs, in comparison to their precursor MGs and \gls{mght}s is uninfluenced by the quench rate used in the processing.}
	\label{f:camg_fi}
\end{figure}

\begin{selfcite}
Previously in literature \cite{Ding2014} it has been discussed that the discintinction bewteen FI and CIolike coorindations is a fine line, owing the definition of the threshold in the voronoi tessealation. In the previous sections, it has been mentioned that the ILO is a well-known indicator of glass stability and of packing \cite{Ding2014,Ding2014a,Adelman1976}. Figure~\ref{f:camg_ilo} shows the variation of the ILO for three different quenching rates of the MGs from which the clusters were prepared. Firstly, it can be seen that the ILO of the MG decreases with increasing quenching rates. Secondly, the heat-treatment for the MGs results in different dependence for the different quenching rates. For the MGs prepared with cooling rates \qr{10} and \qr{12} (Figure~\ref{f:camg_ilo}a, Figure~\ref{f:camg_ilo}b), the heat-treated glasses \gls{mght} exhibit lower ILO than the MG. At the highest cooling rate of \qr{14}, the same heat-treatment places the \gls{mght} at a state with higher ILO value (Figure~\ref{f:camg_ilo}c). Given that the clusters undergo the same heat-treatment as \gls{mght}, the NGs and CAMGs can be compared with the \gls{mght}. \par

At all quenching rates, it is noted that the interfaces exhibit higher SRO than the cores, due to the chemical effects discussed in Section 3.5. The CAMGs, however, show an increase in the ILO with increasing impact energies, for all the three quench rates used in the present study. Therefore, it is concluded that for a given cooling rate of the as-prepared clusters, the CIBD process determines the final states of the CAMGs. \par

\begin{figure} %[!htp]
	\includegraphics[width=\textwidth,trim={0.65cm 2.4cm 0.65cm 1.8cm},clip]{2021_05/figures/11.pdf}
	\mycaption{Icosahedral-like ordering versus quench rates in CBMGs}{Variation of ILO in the various CBMGs and MGs, compared to their
		precursor MGs and \gls{mght}s for cluster derived from (a) \qr{10}, (b) \qr{12}, and (c) \qr{14} \cz RQ MGs.}
	\label{f:camg_ilo}
\end{figure}

Unlike in the MG and the \gls{mght}, the ILO packing and stability do not correlate with each other in the CAMGs and NGs. For a CAMG prepared at a given impact energy, the ILO packing increases with quenching rate. The strain energy due to CAMG processing could be dominating the stability gained from the slow quenching rate in the NGs and CAMGs, leading to the observed behavior. In Figure~\ref{f:film_network}d, the strain analysis shows that von-Mises strains for the 3 nm clusters studied here are higher in both NGs and CAMGs, compared to previous reports for 7 nm cluster NGs \cite{Adjaoud2018}. This leads to the conclusion that the size of the clusters plays an important role in the final structures attained by the CAMGs. Further studies with different cluster sizes, including size distributions, and random deposition locations are needed to gain further understanding on the role of the processing parameters in cluster assembled metallic glasses prepared by compaction (NGs) and by energetic impact (CAMGs). \par
\end{selfcite}

\begin{quote}
Figure~\ref{f:camg_pote}a shows the average potential energy per atom for the six glasses, for the three cooling rates. Also, a possible representation of the PEL has been illusterate and overlayed upon the potential energy states, in Figure~\ref{f:camg_pote}b. We observe that while in the low cooling rate cases of \qr{10} and \qr{12} \cz glasses, the \gls{mght}s are at a higher energy states, indicating a rejuvenation process \cite{Wakeda2015,Saida2017}. This is consistent with the ILO behaviour of MG and \gls{mght}. The CAMGs and NGs however, seem to be at higher energy states than that MG \todo[inline]{working out an argument; this is not true for \qr{12} 300 meV/atom case)}. The ILO packing and stability are not correlated in the CAMGs and NGs. While the general increasing trend with deposition energy is seen at all quench rates, the quench rate itself has little influence on the final state of these cluster. This indicates that if \gls{mght} are sitting at one local minima for their respective quench rates in the potential energy landscape, the CAMGs and NGs are displaced to a nearest slightly lower local minima. The answer to why the CAMGs and NGs remain unaffected by quench rates could be the deformation processes in these glasses (competition between strain energy and stability gain from slowing quench rate). We refer back to Figure 4, where the strain analysis shows that von-Mises strains for the 3nm clusters are quite high for both NGs and CAMGs, especially in comparison with previous reports for 7nm cluster NGs \cite{Adjaoud2018}, leading us to believe that size of the clusters may indeed also play a role in those complex phenomena. \par
\end{quote}

\begin{figure}[!htp]
	\begin{subfigure}{\textwidth}
		\includegraphics[width=\textwidth]{collated/pote_coll_3nm_Cu50Zr50_CAMG_vs_MG}
	\end{subfigure}%
	\vfill
	\begin{subfigure}{\textwidth}
	\includegraphics[width=\textwidth]{pe-ldscp.png}
	\end{subfigure}%
	\mycaption{Energy versus quench rates}{(a) Variation of FI in the various CAMGs and MGs, in comparison to their precursor MGs and MG$_{ht}$s is uninfluenced by the quench rate used in the processing. (b) Average potential energy per atom in the metallic glasses, along with an imagined depiction of the \gls{pel}.}
	\label{f:camg_pote}
\end{figure}