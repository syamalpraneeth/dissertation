\section{Exploring Deposition Energy Ranges} 
\begin{selfcite}
In order to understand the role of the impact energy on the cluster deposition and to identify the range of impact energy of interest for the preparation of CAMGs, the deposition of a single cluster on a substrate was studied initially. The single \cz  cluster, 3 nm in diameter— prepared as described in Section~\ref{s:clus}—was deposited at various energies ranging from 6 meV to 6000 meV per atom. \par

Figure~\ref{f:clus_single3} shows the cross-sections of the clusters deposited at various energies in the YZ plane parallel to the deposition axis. The snapshots were made 2 ns after deposition, as in Section~ref{s:camgdev} the simulation was determined to have converged by this time. The atoms colored in yellow and magenta, belong to the shell and core atoms of the cluster prior to deposition, respectively (as in Figure~\ref{f:clus_rad-3nm}). No distinction is made in this color code for the constituent elements. All substrate atoms are colored in black. Clearly, the morphology of the cluster after impact varies with deposition energy. \par In the energy range of 6-60 meV/atom, the cluster is in a soft-landing state. In this regime, it is observed that even for the lowest deposition energy of 6 meV/atom the cluster loses the original shape of the free cluster, which was almost perfectly spherical. This change of shape is attributed to a partial wetting due to the cohesive forces at the surface between the cluster and the substrate. No noticeable difference in the final shapes is observed for the cases of 6 meV/atom and 60 meV/atom impact energy. As more drastic changes of the cluster shape are observed at higher impact energies, the deposition energy of 60 meV/atom was chosen to be the upper limit for the soft-landed state.

\begin{figure}
	\centering
	\includegraphics[width=0.5\linewidth]{2021_05/figures/2.pdf}
	\includegraphics[width=0.5\linewidth]{2021_05/figures/3.pdf}
	\mycaption{Single 3 nm cluster deposited states:}{ In Figure~\ref{f:clus_single3}(a), the cross sections of snapshots of the clusters 2 ns after the simulated deposition at various per-atom energies ranging from 6 meV/atom to 6000 meV/atom are shown, with categories of soft, medium, hard and extreme hard landing indicated. The core and shell atoms are marked in magenta and yellow colors, respectively. This is the same color scheme used in Figure~\ref{f:clus_rad-3nm}b. In Figure~\ref{f:clus_single3}(b) the as-deposited states of the clusters (curvature and thickness of the embedded clusters) are represented after equilibration for 2 ns after the deposition.}
	\label{f:clus_single3}
\end{figure}

For all simulated cluster impacts, it is observed that the impact energy clearly influences the final states of the shell atoms in the clusters. The change in state of the deposited clusters, was quantified by means of the root mean square deviation of the shell atoms of the radial coordinate of the shell atoms from the average shell radius (RMSD$ _{shell}$) and the radius of curvature of the cluster R$ _{C} $. Figure~\ref{f:clus_single3}b summarizes the RMSD$ _{shell}$ and R$ _{C} $ as a function of deposition energy, with the clusters being equilibrated for 2 ns. \par

The values of RMSD$ _{shell}$ quantify the degree of distortion of the shell atoms from their original positions, which increases monotonically with the impact energy. The shell region stays intact at energies below 600 meV/atom. However, for impact energies ≥ 600 meV/atom, i.e., in the hard-landed state, the distortion of the cluster increases continuously with increasing impact energy. The deposition at 300 meV/atom energy is then defined as the medium-landed state. The separation between soft, medium and hard landing is assigned arbitrarily. However, these distinctions allow us to understand the broad energy regimes in which the CAMGs retain or lose the signatures of the originally free clusters. In the soft-landed state, the clusters in the CAMGs can be expected to remain mostly spherical. At the higher energies, in the medium-landed state, a lot more deformation of the cluster is expected. In the hard-landing state, not only will the cluster be deformed, but the inter-diffusion of the core-shell atoms in the cluster becomes significant. \par

In line with the changes of RMSD$ _{shell}$, the radius of curvature R$ _{C} $ gradually decreases with increasing energy, indicating that the cluster loses its spherical morphology at higher impact energies. At energies $\geq$ 3000 meV/atom, i.e., extreme hard landing, the cluster embeds itself into the substrate during impact, being reflected in a negative R$ _{C} $. With increasing impact energy, the cluster deforms more and embeds deeper into the substrate. In the context of formation of CAMG films, i.e., when multiple clusters are deposited over each other layer-by-layer, intermixing between clusters is expected at the higher impact energies. Based on the results of the single cluster deposition, impact energies between 60-600 meV/atom were chosen to study the formation of CAMG films. In the following section, as part of a first analysis, the changes of the core-shell structures during multiple cluster deposition will be considered.

\end{selfcite}

