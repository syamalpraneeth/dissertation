\section{Identifying Cores and Interfaces}

\begin{selfcite}
The deposition of multiple clusters, with clusters derived from \qr{10} \cz MG, was simulated at 60 meV/atom, 300 meV/atom, 600 meV/atom impact energies to mimic the soft, medium, and hard-landing in the CAMG film samples, respectively. Additionally, a deposition at extreme energetic conditions was simulated at the impact energy of 6000 meV/atom. The \gls{hcp} arrangement for each impact energy was achieved in the following manner: before each new cluster was deposited, the clusters that were already present in its neighborhood were relaxed for at least 2 ns. Every simulated CAMG had three layers of such cluster depositions, with ~50 clusters in each layer. The deposition sequence and the processes occurring during impact can be followed in the \todo[inline]{Supplementary Video V1}, which shows the simulation of the deposition of CAMGs in comparison to that of the compaction in the NG processing. \par

Figure~\ref{f:film_network} shows cross-sections of the films, similar to Figure~\ref{f:clus_single3}a, after equilibrating the sample for 2 ns after the deposition of the last cluster. The yellow and magenta color coding denotes the shell and core atoms of the clusters prior to deposition, similar to Figure~\ref{f:clus_comp-3nm} and Figure~\ref{f:clus_single3}. No evidence for porosity is observed even for the soft-landing sample (60 meV/atom case) by evaluating a surface mesh with a probe sphere radius of 2.4 \r{A} \cite{Stukowski2010a,Stukowski2014}. At the lowest impact energy of 60 meV/atom, it is observed that, like in Figure~\ref{f:clus_single3}b, the deposited clusters mostly retain their initial sphericity. The cluster sphericity is progressively lost with increasing deposition energy. Next, it should be noted that the first layer of clusters has resided on the substrate for at least 24 ns (using the deposition protocol described) by the time the final layer is deposited. Nevertheless, the interdiffusion of the core and shell atoms is quite low for deposition energies even up to 600 meV/atom. In the energy range of 60-600 meV/atom, the shell atoms, i.e., the former surface atoms of the free cluster prior to deposition, are forming a distinct inter-connected network, which can be interpreted as atomically thin interfacial regions between the cores of the clusters. At the impact energy of 6000 meV/atom the interfacial regions vanish completely. Additionally, this film shows significant atomic intermixing between the substrate and the film (see Figure~\ref{f:film_network}, visualizing the black substrate atoms found in the film, and magenta and yellow atoms from the deposited film embedded in the substrate). For the case of the 6000 meV/atom energy, it is estimated that 8\% of the atoms originally in the film are mixed into the substrate, whereas for the case of the 600 meV/atom impact energy this value is about 1.7\%. Similarly, the mixing of the substrate atoms diffusing into the film sample was also ascertained, by tracking the substrate atoms in the final deposited films. The mixing of the substrate atoms into the film has been estimated to be 6\% for 600 meV/atom case, and is 16\% for the 6000 meV/atom case. For both the film and substrate atoms, the diffusion into the neighboring medium is higher at higher impaction energies. \par
	
\begin{figure*}	\centering
	\includegraphics[width=0.7\textwidth,trim={2cm 2.5cm 1.9cm 1.8cm},clip]{2021_05/figures/4.pdf}
	\includegraphics[width=\textwidth,trim={0.8cm 2cm 0 1.7cm},clip]{2021_05/figures/5.pdf}
	\mycaption{Deposition of CAMG films}{(a) A depiction of the deposition of a cluster onto the substrate and (b) a top view of the clusters deposited in a HCP arrangement (colour coded by the atom height in the deposition axis) (c) A vertical crossection of the deposited films at 60, 300, 600, and 6000 meV/atom energies, with cluster-core atoms in magenta, and cluster-shell atoms in yellow. The substrate atoms are coloured in black. We observe that the shell atoms form a network of interfaces across the film at least up to 600 meV/atom deposition energy. (d) When colour coded with von Mises shear strain, the interfacial atoms correlate with the higher strained atoms.}
	\label{f:film_network}
\end{figure*}
	
Up to now, the locations of the atoms, located at the surfaces of the clusters prior to deposition, have been followed (using the magenta and yellow color scheme for the core and shell atoms, respectively) to determine the interfacial regions in the samples prepared with impact energies in the range below 600 meV/atom. This approach does not provide any information on the energetic state of the atoms in the CAMGs or on the local environments in the cores and interfaces. In a first step towards a more detailed analysis, the von Mises shear strain for each of the CAMG atoms was determined \cite{Stukowski2014}. A cut-off radius of 3.8 \r{A} was chosen to compute the strain tensor. In Figure~\ref{f:film_network}d, the high strain regions can be clearly correlated to the interfacial network shown in Figure 3c for all impact energies below 600 meV/atom. Only for the highest impact energy, no sign for the presence of interfaces can be found, similar to the observation in Figure~\ref{f:film_network}c. The correlation observed for CAMGs is consistent with Gleiter’s original definition of interfaces \cite{Gleiter1991} in NGs, in which the interfaces were assumed to be regions of distorted and sheared coordination among adjacent clusters. The strain maps in Figure~\ref{f:film_network}d confirm that an interfacial structure is formed and is retained in the range of 60-600 meV/atom impact energies. Upon inspection of the simulation snapshots in \gls{ovito}, the cores and interfaces in the CAMG film samples made by soft-to-high landing deposition are found to reflect a chemical heterogeneity similar to what was originally present in the free clusters prior to deposition (described in Figure~\ref{f:clus_comp-3nm}a), with the Cu concentration of ∼46 at. \% in the cores, and ∼54 at. \% in the interfacial regions. The atoms in the CAMG film sample for the extreme hard-landing case of 6000 meV/atom are strained to the point where the presence of core and interface structure is lost, and in this manner resembling the MGs obtained by RQ. It should be mentioned that the processing of NGs and CAMGs seems to differ in one aspect: at the harshest conditions, i.e., at the extreme hard-landing case for CAMGs and at the highest pressures for NGs, the final structures are different. In NGs, the interfacial regions continue to exist even at the highest pressures, while the interfacial regions disappear in CAMGs at the extreme hard-landing. This might be due to the sequential deposition of the clusters in CAMG processing, compared to the compaction process, which occurs in the entire arrangement of clusters. \par

\begin{figure}
	\begin{subfigure}[b]{0.33\textwidth} \includegraphics[width=\textwidth,trim={0 1cm 0 1cm},clip]{1e10/dep_60meV_pers_del}
		\caption{}
	\end{subfigure}%
	\hfill
	\begin{subfigure}[b]{0.33\textwidth} \includegraphics[width=\textwidth,trim={0 1cm 0 1cm},clip]{1e10/dep_300meV_pers_del}
		\caption{}
	\end{subfigure}%
	\hfill
	\begin{subfigure}[b]{0.33\textwidth} \includegraphics[width=\textwidth,trim={0 1cm 0 1cm},clip]{1e10/dep_600meV_pers_del}
		\caption{}
	\end{subfigure}%
\mycaption{Representative slabs of CAMGs}{To avoid surface artifacts, these slabs were cut out of the CAMG films}
\label{f:camg_slabs}
\end{figure}

The CAMG samples are also different from both the MGs and the NG, in the following fashion: the CAMGs only partially fill a simulation box. It is important to clarify that the unfilled volume referred to is not within the film, rather it is between the upper surface of the film and the upper wall of the simulation box. Due to the open surface at the top, and a surface interaction between the cluster atoms and the substrate, it is expected in the CAMG film samples that this gives rise to surface artifacts—including defective surface coordinations, larger atomic occupancy and higher surface energy. It was decided to first analyse the entire CAMG film sample, and then later represent the data from a slab of fixed dimensions present within the inside of the deposited film samples, in order to avoid the surface artifacts, as shown in Figure~\ref{f:camg_slabs} (Tip for \gls{ovito} users: the atoms have to be deleted from the data only after any OVITO-API calls have been performed).  \par
\end{selfcite}

Another possible method of removing surface artifacts is to query the surface atoms by means of a surface mesh, and deleting them from the slab before plotting the analysed data. However, the first method of representing data from slabs is preferred in this thesis. The fixed slab volume and the consistently similar amounts of core and interfacial atoms, allows for a consistent analysis of the effects of core and interface regions. This consistency is lost when deleted surface atoms from the surface mesh, as the deleted atoms are majorly made up of the interface atoms. \par

\begin{selfcite}
Based on the presented results, a structural model for the CAMGs is proposed, similar to that of NGs. In this model, interfacial regions, which are chemically different from the core regions due to the surface segregation observed in the individual clusters, are formed during the cluster deposition in the range of impact energies between 60–600 meV/atom. The only exception is, as mentioned above, the structure for an impact energy of 6000 meV/atom, for which interfaces are totally absent. Therefore, in the following sections, only simulations of CAMGs deposited at the impact energies 60, 300, and 600 meV/atom, are being considered. Incidentally, this energy range corresponds well with the energy range used in the cluster experiments reported in \cite{Benel2019}.
\end{selfcite}