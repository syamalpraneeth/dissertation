\section{Potential Energy Inspection} \label{s:camg-pe}

\begin{selfcite}
In Figure~\ref{f:pe_camg}a, the normalized potential energy distribution of the simulated CAMGs, NG and MGs are summarized. The Cu and Zr atoms exhibit two separate distributions, with the peaks of -3.5 eV for Cu atoms and -6.5 eV for Zr atoms. By contrast to the atomic volume distribution behavior, where the Cu atoms occupied lower volumes, the Cu atoms have a higher potential energy overall in comparison to Zr atoms. The absence of right shoulders in the potential energy distribution peaks, like in the volume distributions discussed in Section~\ref{s:atvolcamg}, once again confirms the absence of surface atoms in the representative CAMG slabs. \par

\begin{figure}[!h] \centering
	\includegraphics[width=0.65\columnwidth,trim={0 0 2cm 0.5cm},clip]{2021_05/figures/8.pdf}
	\mycaption{Potential energy states of the CAMGs, NG and MGs: Figure~\ref{f:pe_camg}(a) shows the normalized potential energy distributions for the six glasses, whereas Figure~\ref{f:pe_camg}(b) shows the average potential energy per atom, also in the core and interface regions for all the glasses.}
	\label{f:pe_camg}
\end{figure}

\todo[inline]{write speculations about PEL}

Figure~\ref{f:pe_camg}b summarizes the average potential energies for all atoms in the MG, \gls{mght}, NG, and the CAMGs and the average potential energies of the atoms in the core and interfacial regions in the NG and CAMGs, deposited at the different impact energies. All of these six metallic glasses samples have been made from a \qr{10} \cz glass. It is observed that the core and interfacial atoms in the CAMGs and NGs can be distinguished by their energetic states. The core atoms occupy lower energy states, about 2\% lower than that of the atoms in MGs prepared by RQ, whereas the interfaces possess higher energies ~ 6\% higher compared to those of the MGs. While the interfaces are better packed than the cores, as seen in Section~\ref{s:atvolcamg}, they occupy a higher energy state than the MGs. From liquid quenched traditional metallic glasses simulated in Section~\ref{s:s:simtestMG} it is known that the total potential energy of the glass increases with increasing Cu concentration maintaining the same quenching rate (See Figure~\ref{f:pe_mgs}). Hence, it can be explained that the higher Cu concentration in the interfaces, results in the interface atoms residing at a higher energy state. We conclude that the core regions stabilize the CAMGs and the NGs. Denser packing at the interfaces does not necessarily correspond to a lower energy state in the NGs and CAMGs due to their chemical heterogeneity. \par

In the following section, the medium-range order (MRO) in all six metallic glasses (3 CAMGs, NG, MG and \gls{mght}) will be inspected to better understand the connectivity of the FI units in the MG and \gls{mght} and how the MRO varies in the CAMGs with the deposition energy.
\end{selfcite}