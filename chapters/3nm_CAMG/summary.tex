\section{Summary}

\begin{selfcite}
In this study, a molecular dynamics simulation protocol was developed to study the local structure of \cz cluster-assembled metallic glasses (CAMGs). The present model of CAMGs uses chemically segregated amorphous Cu50Zr50 clusters of ~800 atoms, which are 3 nm in diameter, being deposited onto a substrate at different impact energies. These CAMGs are compared with NGs produced by mechanical compaction of the same clusters, and to the conventionally prepared melt-quenched metallic glasses of the same composition. The main results of the study are as follows:

1. In the CAMGs, two chemically distinct amorphous phases were observed: cores and interfaces, which constitute an interconnected network of interfaces in which the cores with their distinctly different local structure are embedded. The formation of Cu-rich interfaces is observed at impact energies up to 600 meV/atom. The interfaces appear to be completely absent at the extreme impact energy of 6000 meV/atom.
2. The FI and ILO short-range order parameters are lower in the NG and CAMG, for both the cores and interfaces. The interfaces exhibit higher FI and ILO compared to the cores, with a higher density than the cores. Due to the chemical heterogeneity between cores and interfaces, the core regions occupy lower energy states, thus stabilizing the CAMG structures.
3. The local ILO as well as the MRO in CAMGs are found to increase with the impact energy. Furthermore, the ILO increases with impact energy, irrespective of the quenching rates used to prepare the clusters. Consequently, at a fixed overall bulk composition of the metallic glass, control of the local structure is possible by modifying the processing conditions. The SRO and MRO in CAMGs recover towards the metallic glass values with increasing impact energies.
\end{selfcite}



\begin{figure}
	\includegraphics[width=\textwidth]{/home/mj0054/Documents/work/writing/articles/2021_05/figures/graphical_abstract2.pdf}
	\label{f:camg-summary}
	\mycaption{CAMG summary}{CAMG summary}
\end{figure}