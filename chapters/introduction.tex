\chapter{Introduction} \label{c:intro}

\section{Motivation}
A \Gls{mg}, which is a class of amorphous metallic or metal-metalloid alloys which have been well studied for decades. Initially synthesized in the group of Pol Duwez \cite{Klement1960}, \gls{mg}s of various compositions and systems have since been investigated, revealing many structural and functional properties \cite{Greer1995,Suryanarayana2001,Trexler2010,Berthier2016}. These bulk-processed MGs, having a homogeneous distribution of the constituent elements (depicted in Figure~\ref{f:bottom-up}a), possess many interesting properties such as low density, high stiffness, high wear and corrosion resistance, better thermoplasticity, and biocompatibility in comparison to crystalline materials---finding various %structural and functional 
applications in aerospace materials, machine components, casting and thermoplastic forming, and wear-resistant surgical tools and medical implants \cite{Johnson2002,Nu2016}. \par

Conventionally, they are prepared from a melt as \gls{rq} solids, or from \gls{bm} mixtures of elements into powders \cite{Klement1960,Weeber1988,Duwez1960,Greer1995}. In the late 1980s, a new class of \gls{mg}s came to prominence. The advent of \gls{ncm} production via compaction of nanopowders, motivated the synthesis of a novel kind of metallic glass called \gls{ng}---by the mechanical compaction of amorphous nanoparticles \cite{Jing1989,Gleiter2014}. This bottom-up approach to amorphization, shown in Figure~\ref{f:bottom-up}b, and successive research on \gls{ng}s \cite{Gleiter2014,Ivanisenko2018,Fang2012,Ghafari2012,Weissmuller1992,Gleiter2016,Witte2013} indicated that these NGs are present with additional structural features---deduced to arise from nanoparticle interface formation---that are not achievable in MGs prepared by RQ and BM. The \gls{ng} structural model is reminiscent of that of \gls{ncm}s i.e., cores embedded in an interfacial network; however, both the cores and interfaces in the NGs are fully amorphous \cite{Ivanisenko2018,Nandam2017,Sopu2009,Adjaoud2018,Cheng2019}. \par

While the tailoring of structural and functional properties (by varying composition and quenching rate) of conventional metallic glasses is quite challenging, the new processing route of making NGs offers many advantages. The current research indicates that the interface formation and the surface segregation in the nanoparticles lead to the interesting properties of in \gls{ng}s such as reduced density and enhanced plasticity \cite{Adjaoud2016,Adjaoud2018,Ritter2011,Nandam2017,Wang2017,Wang2018,Nandam2020,Nandam2021}.  \par

\begin{figure}[!ht] \centering
	\includegraphics[width=0.9\linewidth]{bottom-up-illus3.png}
%	\mycaption{Building Metallic Glasses from the Bottom-up}{Synthesizing glasses from nanoparticles is a recent approach to tailoring	properties of amorphous materials.}
	\mycaption{Building Metallic Glasses from the Bottom-up}{(a) Traditional metallic glasses have a homogenous amorphous structure. Glasses built from amorphous nanoclusters can by (b) compaction into nanoglasses, or by (c) energetic deposition to create cluster-assembled glasses. The hand-drawn illustrations indicate the broad microstructural differences expected between the bulk processed glasses and those prepared from bottom-up approaches.}
	\label{f:bottom-up}
\end{figure}

More recently, it has been discovered that MGs can also be produced from atomic clusters which are substantially smaller than the building blocks of \gls{ng}s. Cluster-assembled amorphous films were demonstrated to be synthesized by the energetic deposition of small clusters (Illustrated in Figure~\ref{f:bottom-up}) in the size range of 10-2000 atoms per cluster, onto a substrate under \gls{uhv} \cite{Kartouzian2013,Kartouzian2014,Benel2019}. Such a materials is referred to as a \gls{camg}. \par 

A first report on CAMGs detailed their preparation from 10-16 atoms-sized Cu-Zr clusters of varying compositions, characterizing the resulting films as amorphous by synchrotron-based diffraction \cite{Kartouzian2013,Kartouzian2014}. However, a detailed account of the structure was not given. In another study, \fs CAMGs were prepared from 800 atom-sized clusters \cite{Benel2019,Benel2018}. By changing only the deposition energy of the cluster assembly, the average local structure---characterized by synchrotron \gls{exafs}---and the ferromagnetic transition temperatures (\gls{tc}) were found to change significantly in the CAMG of constant macroscopic composition of \fs. This possibility to modify amorphous structure by the novel CAMG preparation methods presents the opportunity to control properties of amorphous solids. \par

Due to difference in the preparation methods, and the intriguing observations mentioned above, the structural features of CAMGs are expected to be different from those of NGs and CAMGs, which are in turn known to differ from each other \cite{Adjaoud2018,Ritter2011,Nandam2017,Wang2017}. The structural characterization of CAMGs is currently limited to the \gls{exafs} study of the \fs CAMGs \cite{Benel2019}, further exploration of CAMGs with advanced \gls{tem} and \gls{apt} methods is pending. At this juncture, complementary computational studies can prove to be a powerful tool to improve existing knowledge on CAMGs. Virtual insights from computer simulations can not only describe the dynamics and the structure at the atomic level, but also perform parametric studies with ease---thereby guiding future experiments on CAMGs. \par

\section{Objective, Scope and Outline of this thesis}
\subsection{Objective}

The objective of the present work is to describe the mechanisms of formation routes and the structures attained by the CAMGs, by developing specific \gls{md} simulations for the same. A special emphasis is given to how cluster deposition as a processing method affects both the structure and packing of the CAMGs in comparison to rapid quenching in MGs and mechanical compaction in NGs. With the intention to connect to various previous studies on MGs and NGs, and also due to availability of a well-known molecular dynamics potential, Cu-Zr has been chosen as a model system to investigate CAMGs.  \par

\subsection{Scope}
\cz clusters were simulated using \gls{md} techniques to replicate a cluster-structure as obtained by \gls{igc} in the experiments. A distinction is made between clusters---which are in the size scales of 1-3 nm---and nanoparticles, which have a diameter greater than 4 nm.

The deposition of single \cz clusters (the protocol for which was developed within the framework of the thesis) were studied with varying impact energies to gain a semi-quantitative understanding on morphologies and distortion of the deposited clusters. \par

Two efficient simulation algorithms of deposition of a monodisperse distribution of clusters were developed to virtually synthesize \gls{camg} films at various deposition energies. The latter of the two algorithms was designed to deposit clusters in a densely packed manner to reduce porosity and maximize cluster-cluster interactions for optimal interface creation. A monodisperse model of \gls{ng}s is designed to compare the effects of cluster-processing routes of deposition (in CAMGs) with that of compaction (in the NGs). Although the cluster/nanoparticle size-distribution in the NG-experiments is polydisperse, the \gls{ng}s in this are limited to mono-disperse cluster distributions. Simulations of Cu-Zr \gls{rq} \gls{mg}s were developed and studied to serve as a reference standard to compare \gls{camg}s and \gls{ng}s with. In this thesis, the terms glass and \gls{mg} shall be used interchangeably unless mentioned otherwise. \par

The CAMGs of 3 nm sized \cz clusters were made at varying impact energies. The surface atoms of the yet-to-be deposited clusters were found to form a network of cluster-cluster interfaces. The simulated 3 nm cluster CAMGs were evaluated and compared with NGs and RQ MGs by means of \gls{sro}, \gls{mro}, and atomic packing and energetic characteristics. %A processing parameter of the cluster synthesis was also identified and varied to understand the influence of the initial cluster states on the final states of the resulting CAMGs and NGs.
Furthermore, the influence of the cluster size on the cluster-cluster interfaces in CAMGs and NGs was investigated. %The simulations of CAMGs and NGs with varying the mondisperse cluster distributions demonstrated an increase in interfacial regions with reduction of cluster size.

\subsection{Outline}
Chapter~\ref{c:theory} introduces the theoretical background required for the thesis, providing a historical narrative of the research on metallic glasses and nanoglasses, which had motivated the study of cluster-assembled glasses. The definitions pertinent to glassy systems, and the concepts relevant to the computational study of Cu-Zr binary metallic glasses, as well as a review of the relevant literature is included. \par

Chapter~\ref{c:methods} elaborates the molecular dynamics technique used to perform the necessary simulations, and presents information about the various characterization methods used to evaluate the simulated data sets. The specific details relevant to the high-performance computing aspects of the simulations are also noted. \par

%Chapter~\ref{c:dev} presents the first-ever development efforts made to simulate and analysis a digital-twin for the cluster-ion beam deposition system. Initially, the simulations of conventional metallic glasses are studied to establish a reference standard for the cluster-assembled metallic glasses and nanoglasses. Next, the modeling details for \cz clusters and single-cluster deposition are covered to deduce the variation of cluster morphology with impact energies. Then, two efficient algorithms to perform cluster-deposition and simulate \gls{camg}s are elucidated in detail. Finally, rthe model chosen to simulate \gls{ng}s is discussed. \par 

In Chapter~\ref{c:camg} the structure of \gls{camg}s is investigated in depth as a function of deposition energy, and compared with \gls{rq} \gls{mg}s and \gls{ng}s. The formation of interfaces, and their evolution with impact energies in \gls{camg}s is noted. The structural characterization and atomic-property inspection of \gls{camg}s unravel the relationships of their \gls{sro} and \gls{mro} with the impact energy. \par

The scaling of interface formation with the cluster size in \gls{camg}s and \gls{ng}s is explored in Chapter~\ref{c:cbmg}. The rise of cluster-size effects, and its influence on their structure and characteristics in the \gls{camg}s are presented. %At first, the case of the \gls{camg}s is presented. Next, a is charted out in \gls{ng}s. \par

The results and findings of the presented research are concluded in Chapter~\ref{c:conclusions} and promising directions for future research are proposed.