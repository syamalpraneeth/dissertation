%% This is file `DEMO-TUDaThesis.tex' version 2.09 (2020/03/13),
%% it is part of
%% TUDa-CI -- Corporate Design for TU Darmstadt
%% ----------------------------------------------------------------------------
%%
%%  Copyright (C) 2018--2020 by Marei Peischl <marei@peitex.de>
%%
%% ============================================================================
%% This work may be distributed and/or modified under the
%% conditions of the LaTeX Project Public License, either version 1.3c
%% of this license or (at your option) any later version.
%% The latest version of this license is in
%% http://www.latex-project.org/lppl.txt
%% and version 1.3c or later is part of all distributions of LaTeX
%% version 2008/05/04 or later.
%%
%% This work has the LPPL maintenance status `maintained'.
%%
%% The Current Maintainers of this work are
%%   Marei Peischl <tuda-ci@peitex.de>
%%   Markus Lazanowski <latex@ce.tu-darmstadt.de>
%%
%% The development respository can be found at
%% https://github.com/tudace/tuda_latex_templates
%% Please use the issue tracker for feedback!
%%
%% ============================================================================
%%
% !TeX program = lualatex
%%


\documentclass[
english,
ruledheaders=section,%level up to which the headings are separated with lines, see DEMO-TUDaPub
class=report,% base document class. Selects the corresponding KOMA-Script class
%thesis={type=bachelor},% Thesis document type, for dissertations see the demo file DEMO-TUDaPhd
thesis={
	% For smaller theses see DEMO-TUDaThesis 
	type=drfinal, dr=ing, department=matgeo},
accentcolor=9c,% Accent color selection
custommargins=true,% Margins are automatically calculated using typearea
marginpar=false,% Header and footer do not extend beyond the margin column
%BCOR=5mm,%binding correction if necessary
parskip=half-,%paragraph identification by spacing see KOMA script
fontsize=11pt,%Base font size according to corporate design is often too small at 9pt
% logofile=example-image, %If the logo files are not available
]{tudapub} 

% The following block is only necessary for pdfTeX on versions before April 2018
\usepackage{iftex}
\ifPDFTeX
\usepackage[utf8]{inputenc}%compatibility with TeX versions before April 2018
\fi

\usepackage{lmodern}
\usepackage{fontspec}
\newfontfamily{\tel}[Script=Telugu]{Pothana2000}

%%%%%%%%%%%%%%%%%%
%Language Customization & Improved Separation Rules
%%%%%%%%%%%%%%%%%%
\usepackage[english, main=german]{babel}
\usepackage[autostyle]{csquotes}% quotes simplified
% If compiling with pdflatex, microtype is loaded automatically, in this case this line has to be removed, and if more options are to be added, this has to be done via
% \PassOptionsToPackage{options}{microtype}
% must be added before \documentclass. 
\usepackage{microtype}

%%%%%%%%%%%%%%%%%%
%Bibliography
%%%%%%%%%%%%%%%%%%
\usepackage{biblatex} % Bibliography
\usepackage{glossaries}
\bibliography{library}

%%%%%%%%%%%%%%%%%%
%Package Suggestion Tables
%%%%%%%%%%%%%%%%%%
%\usepackage{array} % Base package for table configuration, autoloaded by the following
\usepackage{tabularx} % Tables that automatically adjust to the width
%\usepackage{longtable} % Multipage tables
%\usepackage{xltabular} % Multi-page tables with adjustable width
\usepackage{booktabs} % Improved options for table layout using horizontal lines

%%%%%%%%%%%%%%%%%%
%Package suggestions mathematics
%%%%%%%%%%%%%%%%%%
\usepackage{mathtools} % enhanced version of amsmath
\usepackage{amssymb} % extended character set
%\usepackage{siunitx} % units

%Formatting for examples in this document. Generally not necessary!
\let\file\texttt
\let\code\texttt
\let\tbs\textbackslash
\let\pck\textsf
\let\cls\textsf

\usepackage{setspace}	%spacing
\usepackage{enumitem} 
\usepackage{amsfonts,xcolor,graphicx,multicol,titlesec,ragged2e}
\usepackage{float}
\usepackage[a4paper,bindingoffset=0in,top=0.55 in,bottom=0.50in, footskip=.25in]{geometry}
\usepackage{mdframed}
\newmdenv[
topline=false,
bottomline=false,
linecolor = gray,
skipabove=\topsep,
skipbelow=\topsep
]{siderules}
\usepackage[font=sf, labelfont={sf,bf}, margin=1cm]{caption}
\usepackage[font=sf, labelfont={sf,bf}, margin=1cm]{subcaption}

\graphicspath{{/home/mj0054/Documents/work/posters_talks/images/}{/home/mj0054/Documents/work/simulations/results/cuzr/CAMG_3nm/}{/home/mj0054/Documents/work/simulations/projects/cibd/cibd_single/1e10/}{/home/mj0054/Documents/work/simulations/projects/clusters/1e10/}{../figures/}{/home/mj0054/Documents/work/simulations/results/cuzr/MSR_compositions/}{/home/mj0054/Documents/work/simulations/results/cuzr/MSR_cooling_rates/}{/home/mj0054/Documents/work/simulations/projects/bulk/msr/50-50/}}

\loadglsentries{glossary}

\usepackage{pifont}% Zapf-Dingbats Symbole
\newcommand*{\FeatureTrue}{\ding{52}}
\newcommand*{\FeatureFalse}{\ding{56}}

\newcommand{\cz}{\mbox{Cu$_{50}$Zr$_{50}$} }
\newcommand{\czsix}{\mbox{Cu$_{64}$Zr$_{36}$} }
\newcommand{\fs}{\mbox{Fe$_{80}$Sc$_{20}$} }
\newcommand{\vi}[4]{\mbox{$<${#1} {#2} {#3} {#4}$>$}}
\newcommand{\s}[2]{{#1}$_{#2}$}
\newcommand{\qr}[1]{\mbox{$10^{{#1}}$ K/s}}
\newcommand\mycaption[2]{\caption[#1]{\textbf{#1}: #2}}


\begin{document}

\Metadata{
	title=Molecular Dynamics Simulations of Cluster-based Metallic Glasses,
	author=Syamal Praneeth Chilakalapudi
}


\frontmatter

\title{Molecular Dynamics Simulations of Cluster-based Metallic Glasses}
%\subtitle{\LaTeX{} using TU Darmstadt's Corporate Design}
\author[S. P. Chilakalapudi]{M.Sc. Syamal Praneeth Chilakalapudi}%optionales Argument ist die Signatur,
\birthplace{Hyderabad, India}%Geburtsort, bei Dissertationen zwingend notwendig
\reviewer{Prof. Dr.-ing Horst Hahn \and Prof. Dr. Wolfgang Wenzel} % \and noch einer \and falls das immernoch nicht reicht}%Gutachter
%Falls die Bezeichner entsprechend der Promotionsordnung angepasst werden sollen:
%\reviewer*[Erstreferent\_in,Koreferent\_in]{Gutachter 1 \and Gutachter 2}
\publishers{Darmstadt}% Feld für die Ortsangabe oder einen Verlag. Dies ist mit Darmstadt -- D17 vorbelegt, s.u. jedoch wurde die Anforderung für diese Vorgabe reduziert, daher genügt auch die Ortsangabe.


%%Sofern keine passende Option verfügbar ist
%\drtext{}

%Diese Felder werden untereinander auf der Titelseite platziert.
%\department ist eine notwendige Angabe, siehe auch dem Abschnitt `Abweichung von den Vorgaben für die Titelseite'
%\department{phys} %Kürzel werden entsprechend der Liste in diesem Dokument ersetzt.
%\institute{Institut}
\group{Research Unit Nanomaterials}

\submissiondate{\today}
\examdate{\today}

% Hinweis zur Lizenz:
% TUDa-CI verwendet momentan die Lizenz CC BY-NC-ND 2.0 DE als Voreinstellung.
% Die TU Darmstadt hat jedoch die Empfehlung von dieser auf die liberalere
% CC BY 4.0 geändert. Diese erlaubt eine Verwendung bearbeiteter Versionen und
% die kommerzielle Nutzung.
% TUDa-CI wird im nächsten größeren Release ebenfalls diese Anpassung vornehmen.
% Aus diesem Grund wird empfohlen die Lizenz manuell auszuwählen.
%\tuprints{urn=1234,printid=12345,doi=10.25534/tuprints-1234,license=cc-by-4.0}
\tuprints{urn=1234,printid=12345}
% To see furhter information on the license option in English, remove the license= key and pay attention to the warning & help message.

\dedication{{\tel నా  పెద్దల  ఆశీస్సులు  కోరుతూ}}

\maketitle

\affidavit

\addchap{Foreword}
 DEMO-TUDaPub zu Problemen führen. In diesem Fall sollte entweder der Compiler gewechselt oder \code{pdfa=false} aktiviert werden.


\minisec{Unterschiede der Demodateien DEMO-TUDaThesis und DEMO-TUDaPhD}
Zwar basieren alle drei DEMO-Dateien auf der Klasse \code{tudapub}, allerdings sind die Basiseinstelungen dem Dokumententyp angepasst.


\begin{abstract}
	Deutsche Zusammenfassung
\end{abstract}

\begin{abstract}[english]
	Englische Zusammenfassung, falls benötigt
\end{abstract}

\begin{spacing}{1}
	\tableofcontents{}
\end{spacing}
\newpage

\addcontentsline{toc}{chapter}{List of Figures}
\begin{spacing}{1}
	\listoffigures
\end{spacing}
\newpage


\chapter*{Abbreviations}
\addcontentsline{toc}{chapter}{Abbreviations}
%\printglossaries
%\printacronyms

%\chapter*{Abstract/Zusammenfassung}
\addcontentsline{toc}{chapter}{Abstract}

\setcounter{chapter}{0}
\mainmatter

\chapter{Introduction} \label{c:into}
\chapter{Introduction} \label{c:intro}

\section{Motivation}
\glspl{mg}, which are a class of amorphous metallic or metal-metalloid alloys which have been well studied for decades. Initially synthesized in the group of Pol Duwez \cite{Klement1960}, \gls{mg}s of various compositions and systems 
have since been investigated, revealing many structural and functional properties \cite{Greer1995,Suryanarayana2001,Trexler2010,Berthier2016}. These bulk-processed MGs, have a homogeneous distribution of the constituent elements (depicted in Figure~\ref{f:bottom-up}a), and are possess many interesting characteristics such as low density, high stiffness, high wear and corrosion resistance, better thermoplasticity, and biocompatibility in comparison to crystalline materials---finding various applications in aerospace materials, machine components, casting and thermoplastic forming, wear-resistant surgical tools, and medical implants \cite{Johnson2002,Nu2016}. \par

Conventionally, \glspl{mg} are prepared from a melt as \gls{rq} solids, or by \gls{bm} mixtures of elements \cite{Klement1960,Weeber1988,Duwez1960,Greer1995}. In the late 1980s, a new class of \gls{mg}s came to prominence. The advent of \gls{ncm} production via compaction of nanopowders, motivated the synthesis of a novel kind of metallic 
glass called \gls{ng}---by the mechanical compaction of amorphous nanoparticles \cite{Jing1989,Gleiter2014}. This bottom-up approach to amorphisation, shown in Figure~\ref{f:bottom-up}b, and successive research on \gls{ng}s 
\cite{Gleiter2014,Ivanisenko2018,Fang2012,Ghafari2012,Weissmuller1992,Gleiter2016,Witte2013} indicated that \glspl{ng} are present with additional structural features---deduced to arise from nanoparticle interface formation---that are not achievable in MGs prepared by RQ and BM. The \gls{ng} structural model is reminiscent of that of \gls{ncm}s i.e., cores embedded in an interfacial network; however, both the cores and interfaces in the NGs are fully amorphous \cite{Ivanisenko2018,Nandam2017,Sopu2009,Adjaoud2018,Cheng2019}. \par

While the tailoring of structural and functional properties (by varying composition and quenching rate) of conventional metallic glasses is quite challenging, the new processing route of making NGs offers many advantages. The current research indicates that the interface formation and the surface segregation in the nanoparticles lead to interesting features and properties of \gls{ng}s such as reduced density and enhanced plasticity \cite{Adjaoud2016,Adjaoud2018,Ritter2011,Nandam2017,Wang2017,Wang2018,Nandam2020,Nandam2021}.
  \par

\begin{figure}[!ht] \centering
	\includegraphics[width=\linewidth]{bottom-up-illus3.png}
	\mycaption{Building metallic glasses from the bottom-up}{(a) Traditional metallic glasses have a homogenous amorphous structure. Glasses built from amorphous nanoclusters by (b) compaction form nanoglasses, and by (c) energetic deposition create cluster-assembled glasses. The hand-drawn illustrations indicate the broad microstructural differences expected between the bulk processed glasses and those prepared from bottom-up approaches.}
	\label{f:bottom-up}
\end{figure}

More recently, it has been discovered that MGs can also be produced from atomic clusters which are substantially smaller than the building blocks of \gls{ng}s. Cluster-assembled amorphous films were demonstrated to be synthesised by the 
energetic deposition of small clusters (Illustrated in Figure~\ref{f:bottom-up}) in the size range of 10-2000 atoms per cluster, onto a substrate under \gls{uhv} \cite{Kartouzian2013,Kartouzian2014,Benel2019}. Such materials have been referred to as \glspl{camg}. \par 

A first report on CAMGs detailed their preparation from 10-16 atoms-sized Cu-Zr clusters of varying compositions, characterising the resulting films as amorphous by synchrotron-based diffraction \cite{Kartouzian2013,Kartouzian2014}. However, a detailed account of the structure was not given. In another study, \fs CAMGs were prepared from 800 atom-sized clusters \cite{Benel2019,Benel2018}. By changing only the deposition energy of the cluster assembly, the average local structure---characterised by synchrotron \gls{exafs}---and the ferromagnetic transition temperatures (\gls{tc}) were found to change significantly in the CAMG of constant macroscopic composition of \fs. This possibility to modify the amorphous structure by the novel CAMG preparation methods presents a novel opportunity to 
control properties of amorphous solids. \par

Due to the difference in the preparation methods, and the intriguing observations mentioned above, the structural features of CAMGs are expected to be different from those of NGs and CAMGs, which are in turn known to differ from each other \cite{Adjaoud2018,Ritter2011,Nandam2017,Wang2017}. The structural characterisation of CAMGs is currently limited to the \gls{exafs} study of the \fs CAMGs \cite{Benel2019}, further exploration of CAMGs with advanced \gls{tem} and \gls{apt} methods is pending. At this juncture, complementary computational studies can prove to be a powerful tool to improve existing knowledge on CAMGs. Virtual insights from computer simulations can not only describe the dynamics and the structure at the atomic level, but also perform parametric studies with ease---thereby guiding future experiments on 
CAMGs. \par

\section{Objective, Scope and Outline of this thesis}
\subsection{Objective}

The objective of the present work is to describe the mechanisms of formation routes and the structures attained by the CAMGs, by developing specific \gls{md} simulations for the same. A special emphasis is given to how cluster deposition as a processing method affects both the structure and packing of the CAMGs in comparison to rapid quenching in MGs and mechanical compaction in NGs. With the intention to connect to various previous studies on MGs and NGs, and also due to availability of a well-known molecular dynamics potential, Cu-Zr has been chosen as a model system to investigate CAMGs.  \par

\subsection{Scope}
\cz clusters were simulated using \gls{md} techniques to replicate a cluster-structure as obtained by \gls{igc} in the experiments. A distinction is made between clusters---which are in the size scales of 1-3 nm---and nanoparticles, which have a diameter greater than 4 nm.

The deposition of single \cz clusters (the protocol for which was developed within the framework of the thesis) was studied with varying impact energies to gain a semi-quantitative understanding on morphologies and distortion of the deposited clusters. \par

Two efficient simulation algorithms of deposition of a monodisperse clusters were developed to virtually synthesise \gls{camg} films at various deposition energies. The latter of the two algorithms was designed to deposit clusters in 
a densely packed manner to reduce porosity and maximise cluster-cluster interactions for optimal interface creation. A monodisperse model of \gls{ng}s is designed to compare the effects of cluster-processing routes of deposition 
(in CAMGs) with that of compaction (in the NGs). Although the cluster/nanoparticle size-distribution in the NG-experiments is polydisperse, the \gls{ng}s in this are limited to mono-disperse cluster distributions. Simulations of Cu-Zr \gls{rq} \gls{mg}s were developed and studied to serve as a reference standard to compare \gls{camg}s and \gls{ng}s with. In this thesis, the terms glass and \gls{mg} shall be used interchangeably unless mentioned otherwise. \par

The CAMGs of 3 nm sized \cz clusters were made at varying impact energies. The surface atoms of the yet-to-be deposited clusters were found to form a network of cluster-cluster interfaces. The simulated 3 nm cluster CAMGs were evaluated and compared with NGs and RQ MGs by means of \gls{sro}, \gls{mro}, and atomic packing and energetic characteristics. Furthermore, the influence of the cluster size on the cluster-cluster interfaces in CAMGs and NGs was investigated.

\subsection{Outline}
Chapter~\ref{c:theory} introduces the theoretical background required for the thesis, providing a historical narrative of the research on metallic glasses and nanoglasses, which had motivated the study of cluster-assembled glasses. The definitions pertinent to glassy systems, and the concepts relevant to the computational study of Cu-Zr binary metallic glasses, as well as a review of the relevant literature is included. \par

Chapter~\ref{c:methods} elaborates the molecular dynamics technique used to perform the necessary simulations, and presents information about the various characterisation methods used to evaluate the simulated data sets. \par 

Chapter~\ref{c:dev} presents the development of two efficient protocols to simulate the \glspl{camg}. Simulations of \gls{rq} \glspl{mg} are discussed to establish a standard of reference. The model chosen to simulate \gls{ng}s is also mentioned. \par 

In Chapter~\ref{c:camg} the structure of \gls{camg}s is investigated in depth as a function of deposition energy, and compared with \gls{rq} \gls{mg}s and \gls{ng}s. The formation of interfaces, and their evolution with impact energies in \gls{camg}s is noted. The structural characterisation and atomic-property inspection of \gls{camg}s unravel the relationships of their 
\gls{sro} and \gls{mro} with the impact energy. \par

The scaling of interface formation with the cluster size in \gls{camg}s and \gls{ng}s is explored in Chapter~\ref{c:cbmg}. The rise of cluster-size effects, and its influence on their structure and characteristics in the \gls{camg}s are presented. \par

The results and findings of the presented research are concluded in Chapter~\ref{c:conclusions} and promising directions for future research are proposed.

%A processing parameter of the cluster synthesis was also identified and varied to understand the influence of the initial cluster states on the final states of the resulting CAMGs and NGs.%The simulations of CAMGs and NGs with varying the mondisperse cluster distributions demonstrated an increase in interfacial regions with reduction of cluster size.

%Next, the modelling details for \cz clusters and single-cluster deposition are discussed. Finally, the model chosen to simulate \gls{ng}s is also mentioned. \par 

\chapter{Theoretical background} \label{c:theory}
\chapter{Scientific Background} \label{c:theory}
This chapter discusses the relevant historical overview and scientific knowledge that forms the basis of this thesis. First and foremost, the traditional \gls{mg}, which is made by rapid quenching is discussed. The fundamentals of MGs synthesis, their properties and the computation-aided structural models are provided. Next, the \gls{ng}---which is a recent class of \glspl{mg}--is introduced. The details on how to prepare the NGs, and their relevant structural models are elaborated. Equipped with the fundamentals of MGs and NGs, the reader is then acquainted with the concept of the novel \gls{camg}. The details of the initial \gls{camg} experiments, and the current advances and limits of the understanding of \gls{camg} is covered.  \par 

\section{Metallic Glasses} \label{s:mg}
In their natural state, metallic solids exhibit crystalline order. However, when an alloy is cooled from a liquid to solid state fast enough i.e., supercooled, typically at high cooling rates of $10^5-10^6$ K/s, it gives rise to a metallic solid with an amorphous structure \cite{Debenedetti2001}. This is because the atomic mobility of the liquid decreases drastically before any crystallisation nucleation events can occur during the cooling, and the so-obtained solid is trapped in a state with no \gls{lro}. These materials, exhibiting a lack of \gls{lro}, are broadly termed as glasses. They are already known to occur in nature (as obsdian and amber), and have also been artificially produced from silicates, polymers and even dextrose \cite{Doremus1994,Berthier2016}. Primarily, gls{mg}s are supercooled metallic materials exhibiting amorphous atomic arrangement. In 1960, Duwez et al. \cite{Klement1960,Duwez1960} devised an apparatus to rapidly quench molten materials. Their successful synthesis of an entirely amorphous \mbox{Au$_{75}$Si$_{25}$} flake led the foray into \gls{mg} research \cite{Klement1960}. The interesting physics in the \gls{mg}s will be discussed in the following sections. \par

\subsection{Glass Transition and Free Volume} \label{s:gt-fv}
The liquid to solid transition of glasses, known as glass transition, is very different from that exhibited in crystalline materials---as shown in Figure~\ref{f:rq-mg-sch}a. For the crystalline material, the liquid upon solidification undergoes a sharp phase transition in the specific volume or enthalpy at the \glsdesc{tm} (\gls{tm}). The glass transition is much more gradual, occurring at the \glsdesc{tg} (\gls{tg}).  A slower cooled glass (glass 2 in Figure~\ref{f:rq-mg-sch}a) demonstrates a lower glass transition temperature ($T_{g2} < T_{g1}$). \par

In general, physical changes in materials can be classified based on the change in the phases, typically described by an order parameter. By the Ehrenfest classification, the order of the lowest derivative of the \glsdesc{gibbs} (defined as \gls{gibbs} = H - TS, where H, T, and S are enthalpy, temperature, and entropy respectively) exhibiting a discontinuity upon crossing the phase boundary is the order of a phase transition \cite{Jaeger1998}. By this definition, \gls{gibbs}(T,p) is continuous in a first order phase transition but its first derivatives are discontinuous:
\begin{equation}
	S = - \left( \frac{\partial G}{\partial T} \right)_p \text{, and } V = \left( \frac{\partial G}{\partial p} \right)_T = \frac{\partial H}{\partial p} =  \frac{\partial (pV)}{\partial p}
\end{equation}
where S, p and V are the entropy, pressure and volume of the system, and H = pV. In the second order phase transition, the first derivatives are continuous but the second-order derivatives of \gls{gibbs}(T,p), and also the following response functions are discontinuous:
\begin{equation}
\begin{gathered}
C_p = T\left(\frac{\partial S}{\partial T} \right)_p = -T \left( \frac{\partial^2 G}{\partial T^2} \right)_p \text{, } \alpha = \frac{1}{V} \left(\frac{\partial V}{\partial T} \right) = \frac{1}{V} \left(\frac{\partial^2 G}{\partial T \partial p} \right), \\
\text{and }\kappa_T = -\frac{1}{V} \left(\frac{\partial V}{\partial p} \right)_T = -\frac{1}{V} \left( \frac{\partial^2 G}{\partial p^2} \right)_T\text{,}
\end{gathered}
\end{equation}
where $C_p$, $\alpha$, and $\kappa _T$ are the isobaric heat capacity, coefficient of thermal expansion, and isothermal compressibility, respectively. The first- and second-order phase transitions are illustrated in Figure~\ref{f:rq-mg-sch}b. The first order transition is discontinuous in entropy (S) and volume (V) (See Figure~\ref{f:rq-mg-sch}bi-bii). The second order transition, while continuous in S, is discontinuous in heat capacity ($C_p$) (See Figure~\ref{f:rq-mg-sch}biii-biv). It can now be clearly noticed that the transition from the liquid to the crystalline material depicted in Figure~\ref{f:rq-mg-sch}a is a first-order phase transition. The glasses, on the other hand, exhibit the glass transition, which is continuous in the volume or enthalpy order parameters, but discontinuous in viscosity and specific heat. The glass transition is a second order phase-transition \cite{Cohen1959,Berthier2016} and has been experimentally demonstrated in many materials \cite{Kauzmann1948}. \par

\begin{figure}[!h] \centering
	\begin{subfigure}{0.4\textwidth} \vfill
		\includegraphics[width=0.9\linewidth]{glasstrans_mod}
		\subcaption{}
	\end{subfigure}%
	\begin{subfigure}{0.6\textwidth}
		\includegraphics[width=\linewidth]{phasetrans.png}
		\subcaption{}
	\end{subfigure}
	\mycaption{Glass transition and phase transitions}{(a) Specific volume of a glass upon cooling demonstrates a glass transition at \gls{tg}, different from the liquid-to-solid transition in its crystalline counterpart \adaptfig{Ediger1996}{1996}{American Chemical Society}. (b) First-order (i-ii) and second-order (iii-iv) phase transitions. \adaptfig{Nishimori2010}{2010}{Oxford University Press}.}
	\label{f:rq-mg-sch}
\end{figure}

Contrary to the liquid-to-crystalline solid transition, the glass transition is not an equilibrium transition, rather is an effect of failure of kinetic readjustments due to change in temperatures \cite{Fox1950,Ramachandrarao1977,Ediger1996}. The observation that the product of viscosity and total volume is constant with temperature \cite{Batschinski1913} leads to the concept of free volume. One of the popular definitions of free volume of a solid system at a given temperature is the excess measured specific volume compared to its specific volume at 0 K temperature \cite{Ramachandrarao1977}. The glass transition in supercooled liquids is said to originate from the reduction of relative free volume in the bulk \cite{Fox1951}. \textcite{Cohen1959} proposed a model of free volume: the atoms were assumed to be transported into the voids that appeared, when the void volume was greater than a critical volume, and no energy is required to free volume redistribution. The free volume would be negligible in low temperature regimes, but with increasing temperature, the volume gained upon expansion is ``free'' to be redistributed into the entire bulk of the solid. After the volume is distributed, the system would attain a minimum free energy configuration \cite{Turnbull1961}. The glass transition is a change in the viscosity, it is not really a physical change. It is for this reason that a glass transition is not considered to be a thermodynamic phase change like melting. \par

\subsection{Glass Forming Ability and Energy Landscape} \label{s:gfa}
Some popular routes today to synthesise \gls{mg}s are \gls{rq} by melt-spinning \cite{Greer1995}, ball-milling \cite{Suryanarayana2001,Weeber1988} and solid-state reactions \cite{Schwarz1983}. 
The \gls{mg} synthesis involves accessing a metastable amorphous state, and avoiding nucleation of crystals is not without difficulties. There is hence a need to search for good glass formers, and enhancing the \gls{gfa}. \textcite{Turnbull1969} proposed criteria to bypass crystallisation when supercooling, one of the most important being the reduced glass transition temperature $T_r = T_g/T_m$, where \gls{tg} and \gls{tm} are the \glsdesc{tg} and \glsdesc{tm} at the composition respectively \cite{Turnbull1969,Wang2004a}. It was predicted that glasses around the deep eutectic compositions would have good \gls{gfa}. For $T_r=0.5-0.67$, the system becomes sluggish in crystallisation at experimental timescales \cite{Turnbull1969}. \par 

While Turnbull's criterion is a good rule of thumb for \gls{gfa}, the experimentally realisable size of metallic glass samples is another challenge. The larger the dimensions of the glass sample, the greater the chance of non-uniform cooling rates, promoting crystallisation within the bulk of the undercooled sample. This greatly limits the size of the \gls{mg}s. For instance, the first AuSi glass could be made in very small dimensions in the order of 0.1 mm$^2$ area and 10 \gls{um} thickness\footnote{Today, industrial production of melt-spun amorphous ribbons of even 150 mm in width is possible \cite{Wu2014}.}. The interest to produce glasses of larger critical sizes in the 1990s led to research on \gls{bmg}, which are up to several millimetres in thickness \cite{Johnson1999,Greer2007,Inoue2000}. In 1999, \textcite{Johnson1999} developed the first \glspl{bmg}. Shortly after, it was empirically determined in 2000 by \textcite{Inoue2000} that the crystallisation in larger glasses could be avoided using the following criteria:
\begin{enumerate}[noitemsep]
	\item \textit{Confusion principle}: Choosing multi-component alloys ($\ge$ 3 components) can cause frustration in the undercooled liquid, delaying crystallisation
	\item The difference in radii of atoms should be more than 12-15\% 
	\item Negative heat of mixing amongst the main constituent elements, 
\end{enumerate}
These conditions increasing solid/liquid interfacial energy, and make it difficult for atomic rearrangement on a long-range scale. Thereby, the visosity and the \gls{tg} increase, and the driving force for crystallisation is reduced. By choosing multi-component systems and along with these empirical rules, \textcite{Inoue2000} successfully prepared \gls{bmg} of 72 mm in size. The good \gls{gfa} of \gls{bmg}s is attributed to their Arrhenius-like behaviour, demonstrating strong-liquid-like viscosities as classified by the empirical Vogel–Fulcher–Tammann (VFT) relation \cite{Debenedetti2001,Wang2004a}. In the quest to produce better glasses, which are also thermally stable, a better understanding of the underlying thermodynamic mechanisms is warranted. \par

One of the first observations about glass stability was its dependence on its quench rate. Figure~\ref{f:rq-mg-sch}b shows the specific volume behaviour with temperature for a liquid forming a glass. At lower quench rates, glasses can reach a state with lower specific volume, which translates to lower enthalpy and entropy. Furthermore, the \glsdesc{tg} (\gls{tg}) of the glasses reduces when lowering the cooling rates. However, there is a limit on the slowest cooling rate possible in the glasses. It was pointed out that theoretically, at an infinitely slow quench rate, the entropy of the system can be lowered below the entropy of the solid, forming an \textit{ideal glass} \cite{Kauzmann1948}. Visually, this can be understood from Figure~\ref{f:rq-mg-sch}a, where the liquid transition is extrapolated to low temperatures, and where the extended (dotted) line meets the crystal entropy is the \glsdesc{tk} (\gls{tk}). However, this renders the entropy of the liquid to attain a value below that of the crystalline solid at absolute zero temperature. This is called the Kauzmann paradox \cite{Kauzmann1948,Debenedetti2001,Berthier2016}. A resolution to this entropy paradox was postulated by Kauzmann himself, suggesting that all supercooled liquids must crystallise before the \gls{tk} is reached. In actual experiments there is a limit to quench rates possible \cite{Kauzmann1948}, and at very slow quench rates, the system will crystallise. \par

\begin{figure}[!h]
	\begin{subfigure}{0.5\textwidth}
		\includegraphics[width=\linewidth]{pelglass2}
		\subcaption{}
	\end{subfigure}%
	\begin{subfigure}{0.5\textwidth}
		\includegraphics[width=\linewidth]{alphabeta}
		\subcaption{}
	\end{subfigure}
	\mycaption{Energy landscape of metallic glasses}{(a) The potential-energy landscape of MGs in configurational space, depicted with basins and sub-basins \adaptfig{Debenedetti2001}{2001}{Springer Nature} (b) $\alpha$ and $\beta$ transitions in configurational space, $\alpha$ relaxations constitute a change of the all the particle coordinates. \reprintfig{Stillinger1995}{1995}{AAAS}.}
	\label{f:mgpel}
\end{figure}

The synthesis of \gls{mg}s is evidently complicated because of their inherent metastability, as the preferred states of the systems are either crystalline (at low temperatures) or liquid (at high temperatures). To better understand the process, the system of atoms that constitute the glass can be imagined to have many accessible states, which are potential wells (also called basins) in configurational space---often referred to as a \gls{pel}. In Figure~\ref{f:mgpel}a, an illustration of a \gls{pel} is depicted. The various amorphous states are local minima in the \gls{pel}, with the crystalline structure at the global minimum of the configurational coordinates. While various processing treatments bring the system to a certain state, it is possible for the system to hop from one state to another by transitions or relaxation processes. This is visualised in Figure~\ref{f:mgpel}b. $\beta$-relaxations are the hopping of a system amongst subbasins, and involves a local rearrangement of atoms. The larger and relatively slower hopping events, which constitute a rearrangement of the entire system, are referred to as $\alpha$-relaxations, transporting a system to an entirely different basin in the \gls{pel}. \par

In order to stabilise the glasses at the nearest local minimum, sometimes heat-treatment processes are implemented. A low temperature thermal cycling such as annealing below \gls{tg} for instance, lowers the enthalpy and potential energy of the glassy system. This is referred to as aging. In contrast, when a treatment of the system increases the total energy of the system, the process is called rejuvenation. The rejuvenation and aging processes are depicted in Figure~\ref{f:mgpel}a. 

\subsection{Structural Models for Amorphous Metals}\label{s:sro-mgs}
As mentioned in the previous section, a system can attain multiple glassy confiurations (see Figure~\ref{f:mgpel}); the thermodynamic state of a glass depends upon the formation routes---and hence affects its properties. This leads to the age-old problem of understanding the structure of \gls{mg}s, to explore structure-property relationships. At a first glance, glasses appear to have no particular structural order, only to be characterised by their general lack of \gls{lro}. A unifying structural definition of the amorphous structures, requires understanding them beyond simply defining the lack of \gls{lro}. Recent studies discuss models in which metallic glasses possess ordered substructures \cite{Sheng2006,Fukunaga2006,Greer2007}. The first kind, termed as \gls{sro} is defined over the length scales of $\leq$ 0.5 nm. \gls{sro} is the structure at the level of an atom and its first and second nearest neighbours. Beyond the \gls{sro}, the \gls{mro} is defined at length scales of $\sim$1 nm, and describe how the SRO-motifs pack together. An experimentalist may already be familiar with these terms upon calculating \gls{rdf} (described in Chapter~\ref{c:methods}) from electron diffraction, \gls{xrd} or \gls{exafs}, however the \gls{rdf} provides only an average 
picture of the atomic structure. A description of the local structure of atoms provides essential clues to understand the amorphous \gls{mg}s, and finds value particularly in Chapters~\ref{c:camg}~and~\ref{c:cbmg} of this thesis.  \par

The local atomic arrangements and their contribution to the packing and, eventually, to the stability of glassy structures have been discussed in previous studies. A first theory of \gls{sro} in random packing in solids was from empirical evidence of compressing densely packed plasticine (modelling clay) balls by \textcite{Bernal1959}, in connection with an investigation of the structure of liquids. It was noticed that the hard spheres upon compression deformed into a wide variety of polyhedra, which were mostly made up of pentagonal faces. The convex polyhedra so formed in the plasticine models could also be constructed in atomic structures, as shapes enclosed by planes are drawn to bisect distances between every geometrical neighbour of all the atoms \cite{Bernal1959,Finney1970a}. Such a construction has been known in the mathematics and physics communities by various names, such as the Voronoi tessellation \cite{Voronoi1908,Coxeter1973}, Dirichlet region \cite{Dirichlet1850}, Delaunay triangulation \cite{Delaunay1934} or as a Wigner-Seitz cell \cite{Kittel2004}. The Bernal plasticine model for irregular packing in liquids, fails to describe multicomponent glasses with large size differences between the atoms, and also the realistic atomic systems, which interact differently than hard spheres. Another model was proposed by \textcite{Gaskell1978} for metal-metalloid systems, arguing that the nearest neighbour units resemble the structure of the crystalline phases at the same compositions. The structure factors of the glasses obtained from this model were in agreement with those obtained from neutron diffraction experiments. However, this model did not support metal-metal glass systems, and failed to explain the stability of \gls{bmg}s in their supercooled states \cite{Chen2011a}. \par

In recent times, \textcite{Miracle2004} proposed an alternative theory for metallic glasses. He supposed that the structure of glasses could be constructed by efficient packing of solute-centered clusters (or solute clusters) in \gls{fcc} or \gls{hcp} layouts, and called this the \gls{ecp} 
model. By this model, the solute atoms ($\alpha$ type) exist inside the solvent atom clusters ($\Omega$ type), and also in the cluster-octahedral interstices ($\beta$ type) and cluster-tetrahedral interstices ($\gamma$ type) depending upon the ratio of atomic radii of the solute and solvent atoms. This makes up four topologically distinct classes of atoms in the glass. The solute clusters were deemed to be predominantly face-sharing, but also exhibit edge- and 
vertex-sharing to accommodate for internal strains. The model theorised that the solute clusters in turn were packed in \gls{fcc} and \gls{hcp} configurations, giving rise to the \gls{mro}. This \gls{ecp} model was successful in predicting the compositions and also \gls{mro} up to 1 nm length 
scale. However, this model has some major deficiencies. The model fails to explain the evolution of the local structure of \gls{mg}s, which are known to vary with the quench rate. Furthermore, the \gls{ecp} is a static model, so it also can not explain the dynamical behaviour of glasses. \par

A more recent method to describe simulated local amorphous structure and \gls{sro} is the evaluation of the local topological order of the atoms. Unlike the \textcite{Miracle2004,Miracle2013} approach, first an amorphous structure is prepared either by \gls{md} or \gls{rmc} simulations. Then, a Voronoi tessellation is performed to study the topology \cite{Sheng2006,Fukunaga2006}. This results in a broad distribution of polyhedra of various shapes and sizes, reminiscent of the polyhedra obtained by \textcite{Bernal1959}. Such a method has various advantages over the Miracle model, i.e., the dynamic evolution of \gls{sro} can be studied. \par

Initially, \textcite{Honeycutt1987} calculated the stability of free-standing agglomerations of 13 atoms in size—arranged in an icosahedral packing with five-fold symmetry. The stability in supercooled liquids was suggested to be a result of icosahedral clusters by \textcite{Frank1952}. It was later reported that the occurrence of icosahedral packing–or, \gls{fi} order---can be correlated with increased packing fraction in model metallic glasses \cite{Clarke1993}. Recently, performing Voronoi tessellation on simulated glasses \cite{Sheng2006,Fukunaga2006} showed that the partitioned \gls{3d} space assigned to atoms were equivalent to Kasper polyhedra \cite{Frank1958,Doye1996}, which constitute the local SRO in MGs. \textcite{Sheng2006} showed that the glass structure need not be made from a single \gls{fi} motif, but that every alloy system has distribution of coordinations, dominated by a certain coordination polyhedra. For the case of \czsix MGs, icosahedral atomic packing was observed to be the highest occurring structural motif \cite{Ding2014,Ding2014a}. Furthermore, for \cz MGs, it was observed that these FI environments were strongly spatially correlated to each other \cite{Peng2010,Li2009a}. For \cz MGs quenched at a faster rate, less FI and \gls{ilike} packing or \gls{ilo} has been found \cite{Yue2018}. \par 

\begin{figure}[!h] \centering
	\includegraphics[width=0.5\linewidth]{mg-sro}
	\mycaption{Short-range order in metallic glasses}{The Voronoi tessellation method \cite{Sheng2006,Fukunaga2006} helps identify the distribution of \gls{sro} polyhedra indexed using Schl\"afli notation (discussed in further detail later in Section~\ref{s:voronoi}). In \czsix glasses, FI-polyhedra (\vi{0}{0}{12}{0}) are the prominently occurring SRO motifs. \reprintfig{Ma2015}{2015}{Nature Publishing Group}.}
	\label{f:voro-sro-mg}
\end{figure}

Icosahedral order dominates local order and influences packing. The link between stability (structural and thermodynamic) and \gls{sro} in MGs is a very well-studied topic in literature. The work by \textcite{Cheng2008} unequivocally clarifies the relationship between increased FI SRO and increased thermodynamic stability in Cu-Zr MGs. Additionally, there is also work on \gls{gum}, which are SRO structures that are known to be the most likely participators in shear transformations, contributing to structural instability of MGs \cite{Ding2014}. Icosahedral SRO units are known to be least likely to be \gls{gum}s. \par

In addition to giving a description of the topological \gls{sro}, \textcite{Sheng2006} proposed a model for \gls{mro}. They suggested that in dilute solutions, unlike the proposed SRO-units by \textcite{Miracle2004}, it would be the solute clusters as derived from the Voronoi that would pack in FCC/HCP/icosahedral-stacking to form a \gls{mro}. When the solute concentration increases above $\sim$20\%, they observed that all of the solute atoms can no longer be placed at the centre of a solute-cluster being completely surrounded only by solvent atoms. In this case, some solute atoms are expected to be first nearest neighbours. Consequently, a different kind of \gls{mro} forms in solute-rich compositions, comprising of  strings of solute atoms. Another recent work discusses the aggregation of FI-clusters to form chains present also with interpenetrating icosahedra to exhibit a cross-linked MRO network \cite{Lee2011,Ritter2012b,Ritter2012}.

\section{Nanoglasses} \label{s:ngs}
The previous section elaborates on the various aspects of the conventionally prepared \gls{mg}s. Alternative routes to preparing glasses have been explored to find new ways to control properties of amorphous materials. As early as 1977, it was expected by the glass community that vapour-deposited materials should have different structures than their monolithic counterparts, due to the effects of boundary conditions on the nanoparticles or atoms being deposited during synthesis \cite{Finney1977}. The discovery of \gls{ncm} \cite{Gleiter1991}---made from compaction of nanometre-sized grains---catalyzed similar advances in 
noncrystalline solids.  Analogous approaches to the \gls{ncm} powder processing, could then be used to generate glasses with tailorable defect microstructures. \par

\begin{figure}[!h] \centering
	\includegraphics[width=0.5\linewidth]{ng_compact.png}
	\label{f:ng-sch}
	\mycaption{Schematic of nanoglass formation by compaction}{Illustrated here is the hydro-static compaction of particles produced via IGC, which leads to the synthesis of nanostructured NGs.}
\end{figure}

\textcite{Gleiter1991} envisioned a new type of non-crystalline solid made from the  consolidation nanometre-sized amorphous nanoparticles, and termed it as nanoglass (NG). He hypothesized that in compacting a multitude of glassy nanoparticles, it would be possible to create a glass with enhanced free volume, which exist amongst in the regions of contact (or interfaces) of adjacent nanoparticles. Expecting defective coordinations at the interfaces, 
Gleiter propounded that the interfacial regions should be structurally and/or chemically distinct from the cores \cite{Gleiter1991, Gleiter1995}. The \gls{sro} and properties of the \gls{ng}s, he said, must then deviate from a \gls{rq} \gls{mg} of a similar chemical composition. \par

\subsection{Experimental Studies}
The group of Gleiter successfully synthesised a PdFeSi \gls{ng} \cite{Jing1989} in 1989. The nanoparticles were prepared by thermal evaporation and \gls{igc} under \gls{uhv}, and later consolidated at gigapascal orders of pressure. The structure of the PdFeSi \gls{ng} was evaluated using the Fe-\gls{ms} technique. In contrast to the \gls{rq} \gls{mg}, which shows a single broad peak in the \gls{qs} distribution, it was observed that the \gls{ng} shows two peaks \cite{Jing1989}: One peak was similar to that of the \gls{rq} \gls{mg}, and hence was interpreted to be originating from the core regions of the glassy nanoparticles, as the \gls{ng} core was expected to have similar structure to the \gls{rq} \gls{mg}. The second peak of the \gls{qs} distribution was unique to the \gls{ng}. This additional component was attributed to the interfaces formed amongst the consolidated nanoparticles \cite{Jing1989}. In addition, the \gls{ms} indicated the interfacial region to possess a reduced electron density. \gls{ng}s of various chemical compositions, such as Au–Si, Au–La, Fe–Si, La–Si, Pd–Si, Ni–Ti, Ni–Zr and Ti–P, were also synthesised \cite{Weissmuller1992}. \par

Sc$_{75}$Fe$_{25}$ NGs were also investigated using \gls{pas} \cite{Fang2012}, as depicted in Figure~\ref{f:ng-evidence}a. The Fe-Sc NGs indicated two distinct positron lifetimes, one of which was the same as of the \gls{rq} \gls{mg}, and the second lifetime was present only in the NGs and interpreted as originating from the interface regions. The intensities of the two lifetimes were used to indicate that the as-prepared NG consisted of 65 vol\% glassy cores and 35 vol\% interfacial regions. Moreover, this Sc$_{75}$Fe$_{25}$  NG demonstrated enhanced plasticity in comparison to \gls{rq} \gls{mg}s, this was also attributed to the presence of glass-glass interfaces. \par

\begin{figure}[!h] \centering
	\begin{subfigure}{0.4\linewidth}
		\begin{subfigure}{\linewidth}
		\includegraphics[width=\linewidth]{positron_ann} \subcaption{}
		\end{subfigure}%
		\vfill	
		\begin{subfigure}{\linewidth}
		\includegraphics[width=\linewidth]{witte-ngmh.jpeg} \subcaption{}
		\end{subfigure}%
	\end{subfigure}%
	\hfill
	\begin{subfigure}{0.6\linewidth} \centering
		\includegraphics[width=0.9\linewidth]{witte-ngms.jpeg} \subcaption{}
	\end{subfigure}%

	\mycaption{Indirect evidence for interfaces in FeSc NGs}{(a) Relative intensities of two positron lifetimes from \gls{pas} of Sc$_{75}$Fe$_{25}$ NG; $\tau_2$ was interpreted to arise from interfaces and increases in intensity with annealing \adaptfig{Fang2012}{2012}{American Chemical Society}. (b) Magnetisation and (c) \gls{ms} spectra of Fe$_{90}$Sc$_{10}$ 
	NG indicate a new distinct Fe-environment in NGs \reprintfig{Witte2013}{2013}{AIP Publishing LLC}.}
	\label{f:ng-evidence}
\end{figure}

The Fe environments in Fe-Sc NGs were studied extensively to identify core and interfaces. The Fe$_{90}$Sc$_{10}$ NG in particular was the subject of many reports indicating the presence and peculiar properties of interfaces \cite{Ghafari2012,Ghafari2012c,Witte2013,Wang2016,Wang2017}. Using high energy \gls{xrd}, it was found that the number of Fe nearest neighbour atoms in the interfacial regions of the Fe$_{90}$Sc$_{10}$ \gls{ng}s is lower than in the corresponding \gls{rq} \gls{mg}s \cite{Ghafari2012c}. \par

It was also demonstrated that the magnetisation curves of the NGs vastly differed from RQ MGs. While the Fe$_{90}$Sc$_{10}$ MG was paramagnetic at 300 K, the corresponding NG was ferromagnetic with a paramagnetic component (See Figure~\ref{f:ng-evidence}b), indicated by its magnetisation not fully saturating even at 4 T \cite{Witte2013}. Like in reference \cite{Jing1989}, once again the \gls{ms} experiment, shown in Figure~\ref{f:ng-evidence}c, 
indicated a heterogeneous magnetic structure in the Fe$_{90}$Sc$_{10}$ NG (there was an occurrence of two sub-spectra: a paramagnetic component and a ferromagnetic sextet). It was also proven that these fascinating properties of the \gls{ng} were observed only after the nanoparticles were compacted, and not as a loose powder, strengthening the idea that the formation of interfaces resulted in the special properties. The concept of interfaces is illustrated in Figure~\ref{f:ng-core-int}, highlighting the idea that the cores and interfaces, while both amorphous, are structurally distinct from one another. \par

\begin{figure}[!h] \centering
	\includegraphics[width=0.65\linewidth]{core-int.jpg}
	\mycaption{Proposed core-interface structure of nanoglasses}{The processing of a \gls{ng} is hypothesised to form the two core and interface glassy phases---both of which resemble the amorphous structure of \gls{rq} \gls{mg}s in terms of exhibiting no \gls{lro}. However, the \gls{sro} and \gls{mro} of core and interface differ from one another.
	\reprintfig{Ivanisenko2018}{2018}{Wiley}.}
	\label{f:ng-core-int}
\end{figure}

The Fe$_{90}$Sc$_{10}$ NG was further investigated with electron energy loss spectroscopy, which indicated a surface segregation of Fe in the glassy nanoparticles. The surface segregation leads to a chemical heterogeneity in the structure of Fe$_{90}$Sc$_{10}$ \gls{ng} \cite{Wang2016}. Such a surface segregation was explained to arise from both the difference in surface energies of Fe and Sc, and the enthalpy of mixing. A segregation model in the Fe-Sc \gls{ng}s was proposed based on experimental results of small- and wide-angle X-ray scattering \cite{Wang2017}. \par

Recently, \cz NGs were also prepared using magnetron sputtering as opposed to thermal evaporation \cite{Nandam2017,Nandam2018}. These \glspl{ng} were characterised to exhibit higher \gls{tg}, and consequently, higher thermal stability as compared to the \gls{rq} \gls{mg}s. Furthermore, a chemical segregation present in the interfaces facilitated better stability against crystallisation (i.e., higher \glsdesc{tx} \gls{tx}) as well. Nanoindentation studies also revealed that the NGs demonstrated better plastic behaviour (homogenous deformation as opposed to formation of shear bands), and also a higher hardness (Young's modulus) than \gls{mg}s. The \gls{stz} of \glspl{ng} were also measured to be \sim3.8 times larger in volume as compared to melt-spun MGs. The segregation of Cu to the interfaces in the NG, as evidenced by \gls{apt}, was used to explain the enhancement of \gls{tx} and \gls{tg} in the \gls{ng}s. \par

\subsection{Simulation Models}
In the previous subsection, the progress made in the \gls{ng} experimental investigations was discussed. To aid these reports, several independent \gls{md} simulation models of \gls{ng} have been made, predominantly by the groups of Karsten Albe \cite{Krasnochtchekov2003, Sopu2009, Ritter2011, Adjaoud2016, Adjaoud2018, Adjaoud2019, Adjaoud2020, Kalcher2017}, Jason Trelewicz \cite{Cheng2019,Cheng2019a} and Paulo S Branicio \cite{Adibi2014,Sha2017,Zheng2020,Zheng2021}. A first report of \gls{ng} simulations was made by \textcite{Sopu2009}, which demonstrated a compaction protocol for glassy Ge nano-sized spheres at 300 K temperature. The investigators observed that the interfaces formed with lesser densities than the cores. However, their attempts at simulating a \cz NG at 300 K failed as the interfaces completely vanished after compaction, indicating that the mechanical and diffusional properties of the materials influenced the stability of interfaces. \par

Later, in 2011, \textcite{Ritter2011} were successful in simulating a planar glass-glass interface in \czsix \gls{mg}s, which were also stable in \gls{md} timescales. The interfaces were characterised with defective local \gls{fi} \gls{sro}, and present with a 1-2\% excess free-volume. The interfaces were also found to promote shear band formation during tensile deformation, indicating a mechanism to explain enhanced plastic deformation in \gls{ng}s. 
Such planar glass-glass interfaces were further studied in \czsix and Pd$_{80}$Si$_{20}$ systems, also with an elemental surface segregation model \cite{Adjaoud2016}. It was found that segregated interfaces were found to be better packed as compared to ordinary interfaces, with higher topological \gls{sro}, number and electron densities. The interfaces in NGs can hence be 
considered as a structurally stable interphase between the core regions. In the same work, a model was proposed to simulate nanoparticles (precursors to \gls{ng}s) with chemical segregation. The nanoparticles, derived from cutting a sphere out of a glassy bulk and subsequent heat treatment, were found to have similar energetic states as a cluster derived from an \gls{igc} simulation \cite{Adjaoud2016}. The cohesive energies of the Zr in \cz, and Si in Pd$_{80}$Si$_{20}$ were found to drive their segregation to the core regions, while the complementary elements segregated to the surface. This method of deriving a glassy cluster from the bulk of an \gls{rq} also offers the advantage of controlling the nanoparticle size, as compared to other successful \gls{igc}-based models reported for generating Ge clusters 
\cite{Krasnochtchekov2003} and \czsix clusters \cite{Zheng2020}. \par

A contemporaneous work to the nanoparticle segregation model discussed above \cite{Adjaoud2016} was an investigation by \textcite{Danilov2016}, that reported a model for \gls{igc} of nanoparticles of an amorphous Kob-Anderson system \cite{Kob1995}. This model, which described atomic forces in a \gls{lj}-like manner substantiated the presence of a distinct chemical segregation in the cores and shells of the \gls{igc}-nanoparticles. Furthermore, a compaction model was also proposed to create NGs from the amorphous \gls{igc}-nanoparticles. The Kob-Anderson model of \gls{igc}-nanoparticles, and the \gls{ng}s made from them were found to demonstrate an enhanced thermal stability in comparison with the \gls{rq} MGs \cite{Danilov2016}. \par

\textcite{Adjaoud2018} further improved the realistic simulation of the NG-compaction model in \czsix and Pd$_{80}$Si$_{20}$ systems by consolidating a polydisperse distribution of nanoparticles, with an empirically derived \gls{eam} interatomic potential. Further details of \gls{eam} are given in Section~\ref{s:ffs}. The consolidation pressure was shown to have significant influence on the porosity and microstructure of the \gls{ng}s. At $\sim$5 GPa of pressure, the porosity was seen to considerably disappear. Unlike in the work of \textcite{Sopu2009}, the temperature of the system was chosen to be 50 K. The closing up of pores was accompanied by the formation of interfaces, indicated by the local shearing of surface atoms of the nanoparticles. The interfaces constituted of surface atoms of the nanoparticles, and possessed defective \gls{sro} compared to the cores (for the \czsix and Pd$_{80}$Si$_{20}$ systems). In \czsix NGs, the presence of chemical segregation in the nanoparticles led to the formation of interfaces which were chemically heterogeneous from the core regions. This heterogeneity from the segregation was found to be more energetically favourable in the \czsix nanoparticles, than in the interfaces. \par

Similar NG models were made by \textcite{Cheng2019,Cheng2019a} in which a compaction model for a monodisperse distribution of nanoparticles in a patterned arrangement was described, much like the work of \textcite{Sopu2009}. In \czsix glasses, the interface SRO and width in NGs was shown to increase with the temperature at which the nanoparticles were consolidated, and even 
annealing temperatures after consolidation \cite{Cheng2019,Cheng2019a}. In these works, the interfaces remained stable even up to 800 K temperature\footnote{This temperature is close to \czsix \gls{tg}, although it is not clear if this temperature is above or below the simulated 
glass-transition temperature, which is shown to vary with the choice of the \gls{eam} potential \cite{Mendelev2009,Mendelev2019}.}, hinting that the attempts of \cz NGs by \textcite{Sopu2009} could have failed due to lack of availability of a good interatomic potential. Reduced average flow stress and shear localisation factor \cite{Shimizu2007} were observed in the 2.5-nm and 
5-nm grain size NGs, hinting at a stable plastic flow and scaling of elastic properties with nanoparticle size. \par

Another alternative means to simulating NGs that exists in literature is the tessellation-based approach. The previously discussed works fill a box with nanoparticles and compact them \cite{Sopu2009,Adjaoud2018,Adjaoud2019,Cheng2019,Cheng2019a}. In the tessellation approach, first a Poisson-Voronoi tessellation is applied to segment the simulation box, and then segmented partitions are filled with amorphous glassy ``grains" \cite{Adibi2014,Sha2017,Ma2020,Zheng2021}. An external hydrostatic pressure is applied on this configuration. The resulting grain microstructure tends to resemble that of nanocrystalline materials albeit with amorphous grains. While this interesting approach was used to explain some mechanical behaviour of the NGs, it is quite different from the compaction model and also from the experiments. Next, the ``amorphous grains" are chunks of 3 nm particles on average, which are aspherical and  fill the space efficiently, rendering the grains to not deform realistically as would the spherical clusters. These tessellation-based NG simulations are not discussed in the thesis beyond this section. \par

\section{Cluster-assembled Metallic Glasses (CAMGs)} \label{s:camgs}

\subsection{Cluster-ion Beam Deposition and Cluster-assembled Materials (CAMs)}
Advances in research on nanocomposites such as \gls{ncm}s and \gls{ng}s inspired the exploration of \gls{cam}. This new class of materials are built from clusters of atoms, in \gls{uhv}, and have been realised in experiments \cite{Takagi1986,Takagi1988,Beuhler1986}. One of the initial works from 1991 reported the development of a \gls{cibd} apparatus. This apparatus epitaxially deposited a beam of cluster ions with a large size range (25-1600 atoms) onto Si substrates. The ion guided deposition process was successful in preparing films mostly from the smaller clusters in the distribution \cite{Brown1991}, as only a few of the larger clusters were ionised. \par

\gls{cam} were also prepared (and simulated) by low energy deposition of atomic clusters \cite{Perez1997}, resulting in highly porous films which grow in fractal-like patterns due to random stacking. Another extensive review of \gls{cam} discusses the cluster-assembly method for tailorability of superconductivity or charge transport in fullerene-based materials, for the tunability of bandgaps in cluster-based semiconductors, and for the devising of superlattices of size-selected inorganic nanocrystals to enhance conductivity and charge mobilities \cite{Claridge2009}. Although \gls{cam} have been studied by various research groups, the detailed mechanisms that affect films deposited from \gls{cibd} sources are still unclear from the experiments. \par

Computer simulations of ion-beam depositions provide preliminary insights into mechanisms of cluster deposition. The first theoretical studies on \gls{cibd} linked cluster deposition energy to the epitaxy of the films, and suggested that the density of the films increases with impact energy \cite{Muller1987,Cleveland1992}. Succeeding atomistic deposition simulations of covalent systems correlated change in bonding and lattice distortions with deposition energy \cite{Albe1998}. Simulations of molybdenum-cluster depositions provided insights into the morphologies adopted by the \gls{cam} \cite{Haberland1993,Haberland1995}. More recent deposition simulations of Cu clusters onto a Si substrate also exist \cite{Hwang2012,Gong2012}: describing the variation of film epitaxy, stress, and surface roughness with the deposition energy of cluster-assembly. However, a deeper understanding of the local atomic structures adopted in metallic \gls{cam} is still lacking. \par

\subsection{State of Art: Size-selected CAMs and CAMGs} \label{s:camg-uhv}
As seen in the previous section, a control over the local chemistry, morphology, and even microstructure in nanocomposites can be envisioned with precise cluster-assembly. Kartouzian et al. \cite{Kartouzian2013,Kartouzian2014} synchronised the idea of \gls{cam} with metallic glasses, and conceptualised the idea of \acrfull{camg}. By designing \gls{mg}s meticulously with size-selected clusters as building blocks, Kartouzian expounded the possibility to better understand the structure of \gls{mg}s. His proposed method was successful in creating an amorphous \gls{camg}, but it was presented with no discernible features and no concrete method of investigating them. The quest of preparing \gls{camg}s as a means to fabricate fully tailorable amorphous nanocomposites remained unexplored. Aided by experience from \gls{ng} experiments, the synthesis of size-selected \gls{cam} and \gls{camg}s for the first time was achieved by the group of Hahn with the development of a unique \gls{uhv} apparatus to perform \gls{cibd} \cite{Fischer2015,Fischer2015a,Benel2018}. 

\begin{figure}[!ht] \centering
	\includegraphics[width=0.5\linewidth]{cams.png}
	\mycaption{Various CAMs made possible by UHV CIBD}{Illustrations of the types of nanocomposites. The cluster-assembly can be used to create (a) purely cluster-composed films, or (b) cluster decorated surfaces. (c) With high impact energies, surface alloys can be created. (d) By co-deposition, clusters can be embedded in a matrix material.} 
	\label{f:camg-sch}
\end{figure}

Figure~\ref{f:camg-sch} depicts a variety of cluster-based nanocomposites that can be synthesised in the UHV machine. The \gls{cam} are made by depositing crystalline or amorphous nanoparticles onto a substrate placed within the UHV CIBD apparatus\footnote{Further details regarding the UHV apparatus are discussed in detail in the works of Fischer \cite{Fischer2015,Fischer2015a}}. A purely cluster-composed \gls{cam} (Figure~\ref{f:camg-sch}a) can be made with both monodisperse and polydisperse clusters. The clusters may be deposited with a low deposition energy, to create cluster decorated surfaces (Figure~\ref{f:camg-sch}b), or directed with high energies onto a substrate to make surface alloys (Figure~\ref{f:camg-sch}c). Furthermore, in the UHV CIBD apparatus, clusters 
can be co-deposited along with \gls{pvd} of a matrix material from a thermal evaporation source. In doing so, clusters deposited at low energies can be embedded in matrices (Figure~\ref{f:camg-sch}d). The co-deposition serves an additional function i.e., to deposit a capping layer on any kind of the \gls{cam} as protection against oxidation. Cluster-composed matrices \cite{Fischer2015a,Benel2018,Gack2020} have been shown to demonstrate tailorable magnetic properties by varying cluster-size and cluster-concentration in the matrices. The embedding of clusters in a matrix at low deposition energies also promises the potential to engineer multiphase nanocomposites with elements that are known to be miscible in thermodynamic 
equilibrium. Conversely, at high deposition energies, an immiscible system can be made miscible. \par

%\clearpage

\subsection{Initial Studies on CAMGs}
As mentioned before, CAMGs are a special class of the CAMs. Initial attempts to synthesise CAMGs were by the deposition of 10-16 atom sized clusters of varying composition, to create a locally heterogeneous chemical structure. Although the films made were characterised to be amorphous, a more detailed study of structure–property relationships was not reported 
\cite{Kartouzian2013,Kartouzian2014}. \par

\begin{figure}[!h] \centering
	\begin{subfigure}{0.5\linewidth} \centering %\addtocounter{subfigure}{-2}
		\includegraphics[height=11cm]{xanes}
		\subcaption{XANES}	
	\end{subfigure}%
	\hfill
	\begin{subfigure}{0.5\linewidth} \centering
		\begin{subfigure}{\linewidth} \centering
			\includegraphics[height=6cm]{exafs}
			\subcaption{EXAFS}
		\end{subfigure}%
		\vfill
		\begin{subfigure}{\linewidth} \centering %\addtocounter{subfigure}{+1}
			\includegraphics[height=5cm]{m_t.png}
			\subcaption{$M(T)$ vs $T$}
		\end{subfigure}%
	\end{subfigure}%
	
	\mycaption{Experimental characterisation of \fs CAMGs}{(a) The XANES spectra show evidence of no oxidation. (b) The EXAFS spectra of CAMGs are distinct from that of the amorphous \fs RQ MG ribbon. (c) Curie temperature (\gls{tc}) varies with deposition energy in CAMGs. \reprintfig{Benel2019}{2019}{Royal Society of Chemistry}.}
	\label{f:benel-camg}
\end{figure}

Previous experiments on Fe-Sc NGs made in the Hahn group \cite{Witte2013,Ghafari2012}, however, confirmed that the magnetism studies can shed light on the structural information. Hence, the study of size-selected \fs CAMGs was a necessary step in understanding the nature of amorphous structures made from cluster assembly \cite{Benel2019}. \fs CAMGs of $\sim$800 atom sized 
\fs clusters were prepared at three different deposition energies (50, 100, 500 eV per cluster), and coated with a Mg layer to prevent oxidation. The samples were studied by \gls{edx} to confirm chemical compositions. Although the Mg capping layer posed a challenge to characterise the CAMG films with \gls{tem}, the samples were found to be amorphous with synchrotron \gls{xrd} \cite{Benel2019}. X-ray absorption spectroscopic studies were conducted to ascertain the average local atomic structure of the constituent atoms. \par

The structural and magnetic characteristics of \fs CAMGs are described in Figure~\ref{f:benel-camg}. X-ray absorption studies were performed at a synchrotron facility to study the fine structure in the extended (\gls{exafs}) or the near-edge region (\gls{xanes}). The normalised Sc K-edge XANES spectra of the three CAMGs, an RQ MG ribbon of the same composition, pure metallic Sc, and Sc$_{2}$O$_{3}$ are all depicted in Figure~\ref{f:benel-camg}a. The K-edge, i.e, the first inflection point in the energy absorption spectra of the CAMGs and the MG were found to be lower than the Sc$_{2}$O$_{3}$ K-edge, indicating that the CAMG samples were not oxidised\footnote{The X-ray absorption energy 
increases with increase in oxidation state.}. Furthermore, the spectra in the CAMGs and the MG ribbon were observed to be dissimilar, hinting at differences in local structures. \par

In Figure~\ref{f:benel-camg}b, the Sc K EXAFS spectra of the three \gls{camg}s and the \gls{rq} \gls{mg} ribbon are compared with pure metallic scandium. The fading of EXAFS oscillations in k-space was identified as an indicator of amorphous structure. The oscillations in the EXAFS spectra of the three \gls{camg}s differ from that of the \gls{mg} ribbon, indicating once again that the local structures in the cluster-assembled samples are different from \gls{rq} \glspl{mg}. \par

The magnetic behaviour of four \fs CAMGs of impact energies 50, 100, 200, and 500 eV per cluster are described in Figure~\ref{f:benel-camg}c.  Firstly, a paramagnetic to ferromagnetic transition was observed in all samples. When the impact energy is reduced from 500 eV per cluster to 50 eV per cluster, the magnetic transition temperature or the Curie temperature (\gls{tc}) shifts by 60 K. This result was surprising, as only a change in the impact energy for the \glspl{camg} can result in significantly different \gls{tc}. Based on the increase in nearest neighbour distances from the EXAFS data, \textcite{Benel2019} speculated that the increase of \gls{tc}, which is known to correspond to an increased strength of the magnetic exchange interactions, was a result of an increase in the number of the Fe nearest neighbors. The study of \fs CAMGs prepared via CIBD has demonstrated the opportunity to tailor the local atomic structure and the \gls{tc} by varying the deposition energy. The astounding promise of the CAMGs is the variation of properties in a glass while keeping the macroscopic composition constant.

%So far we have discussed the conception of \gls{mg}s, and some of the important reports that have led to understanding them better. Physical way of changing properties will be difficult in Mgs 

%\begin{figure}[!h] \centering
%	\includegraphics[width=0.5\linewidth]{cibd.jpeg}
%	\mycaption{Schematic of the UHV CIBD Apparatus}{A visual representation of the UHV cluster deposition apparatus, with the cluster source at Stage-a, and two deposition stages (Stage-c and Stage-g). The manipulation of the cluster-ion beam is made with electromagnetic lenses and magnets at Stages b, e, and f. %See Section~\ref{s:camg-uhv} for more details.
%	\reprintfig{Fischer2015}{2015}{AIP Publishing LLC}}
%	\label{f:cibd-uhv-sch}
%\end{figure}
%Figure~\ref{f:cibd-uhv-sch} describes a schematic of the \gls{uhv} CIBD experiment. \par

%The complete details of the UHV apparatus can be found in the Ph.D. thesis of \textcite{Fischer2015a}. However, some relevant details are described below. The cluster ions are generated (Figure~\ref{f:cibd-uhv-sch}a) using a Haberland source \cite{Haberland1991}, and passed through an adjustable \gls{igc} aggregation chamber to grow a controllable size based on inert-gas (He/Ar) flow rates and the chamber length. Based on the chosen target material in the cluster source, there is a possibility of growing either crystalline or amorphous clusters. The aggregation chamber opens into a supersonic expansion, allowing the clusters to leave without further growth. The clusters are then accelerated and guided as a cluster ion beam using electromagnetic lenses Figure~\ref{f:cibd-uhv-sch}b. The clusters so-synthesized could be characterized by a \gls{tof} (Figure~\ref{f:cibd-uhv-sch}d). The clusters synthesized at Stage-d (Figure~\ref{f:cibd-uhv-sch}d) were found to be present with a mass distribution \cite{Fischer2015a}, which were shown to vary with the He flow-rates \cite{Fischer2015a}. \par
%
%The clusters were also found to be roughly 50\% neutral and 50\% ionized. The ionized clusters were found to be either singly negatively or singly positively charged. Using an electric field at Stage-c, the negatively charged clusters were chosen to be guided towards a substrate (Figure~\ref{f:cibd-uhv-sch}c1). This is the first deposition stage in the experiment. An expanded view of the first deposition stage (Stage-c) is depicted in Figure~\ref{f:cibd-uhv-sch}c2. The cluster beam was also found to be Gaussian in shape. To ensure a uniform film deposition, the electric field was swept across the substrate. A sample holder/mask contraption was designed to also ensure cluster-deposition onto the substrate only within a limited area of the sweeping cluster-ion beam. The cluster ion beam, when not diverted to the substrate at Stage-c, passes through Stage-e (Figure~\ref{f:cibd-uhv-sch}e), which consists of a quadrupole magnet and a Faraday cup to monitor the size-charge distribution. A $90 ^{\circ}$ sector magnet at Stage-f (Figure~\ref{f:cibd-uhv-sch}f), acts as a mass selector for a more precise mass selection of the cluster ion beam. The mass-selected beam is guided by means of another Stage-e, to then be decelerated and deposited at Stage g (Figure~\ref{f:cibd-uhv-sch}g1). This is the second deposition stage in the UHV CIBD experiment. The Stage g is shown greater detail in Figure~\ref{f:cibd-uhv-sch}g2. The deceleration lenses slow down the beam to allow deposition at a desired velocity. A thermal evaporation source is also available to facilitate a co-deposition along with the cluster deposition. \par

%Sometimes, such treatments %may lead to undesirable effects, such as possible phase separation in the %glassy systems. For instance, in binary immiscible systems with a miscibility %gap, the single phase of a glass can phase separate into two phases, whether %by %classical nucleation (when the separating phases are distinct from one %another) %or by spinodal decomposition (if the two phases have a high volume fraction) %\cite{Cahn1961}. Another possibility is also that of surface crystallisation %or %crystalline nucleation, which can drastically change the structural order of 
%the glasses. \par

%\subsection{Functional properties}\label{s:props-mgs}
%Metallic glasses are used for a variety of purposes today. The lack of defects %leads to high strengths and when combined with their low densities finds %applications in aerospace materials. Their subsequently low elastic modulus %results in high stiffness, which when combined with their high hardness which %is useful for industrial machine components. Some Fe and Co-based metallic %glasses are also utilised for soft magnetic properties. Owing to the lack of a %sharp reduction in volume during the liquid-to-solid transition, in comparison %to crystalline materials (See Figure~\ref{f:rq-mg-sch}b), metallic glasses can %be using in casting and thermoplastic forming applications. Some Ti- and Zr- %based \gls{bmg}s are biocompatible, and offering the possibility high strength %and wear resistant medical implants. \par


\chapter{Methods} \label{c:methods}
\chapter{Methods} \label{c:methods}
As discussed in Chapter~\ref{c:theory}, simulating a \gls{mg} is an effective route to understanding its structure-property relationships. Atomistic simulations complement experiments by providing insights into functional and mechanical behaviour. Traditionally, \gls{mg}s have been simulated using atomistic classical molecular dynamics to understand their structure-property relationships and time-evolution of local atomic processes \cite{Schuh2007,Cheng2008}, and local atomic \gls{sro} and \gls{mro} \cite{Sheng2006}. In the current chapter, the fundamentals of the classical molecular dynamics method, as well as techniques used to characterize the simulated samples are discussed. \par

\section{Molecular Dynamics Simulations} \label{s:md}
\subsection{Overview}
The many-body approach is adopted to describe the mechanisms of a system of atoms or molecules. The most accurate solution would be achieved with a quantum mechanical treatment. However, beyond a two-body problem, an analytic quantum mechanics solution is not possible. While the local density approximation class of \gls{dft} methods can help predict structure and thermodynamics, they are highly computationally intensive; one is limited to systems of not more than hundreds of atoms for the simulations to complete in a reasonable time with the computing resources available today \cite{Burke2012}. A reasonable compromise is to perform \gls{md} simulations by treating atoms as the lowest level of discretization to study the properties of materials. Today, \gls{md} simulations of metallic glasses are performed using readily available codes such as \gls{lmp} \cite{Plimpton1995,Thompson2022}. The basics of the techniques implemented in the \gls{md} solvers are addressed.  \par 

The Born–Oppenheimer approximation \cite{Born1927}, allows one to treat the dynamics of the nuclei and electrons of atoms separately due to their large mass differences. Consequently, in the many-body problem, the heavier and slowly moving nuclei are described as point masses using Newton's equations of motion. The electrons are assumed to adapt instantaneously to change in the nuclei. For studies in which one is interested in atomic arrangements and interactions, the atomistic molecular dynamics methods allows discretization of simulation to atomic scales \footnote{Further approximations, such as coarse-graining: where certain molecules or functional groups are treated as one unit, or implicit solvent models: where the solvent interactions in a solute-solvent system are implicit in the equations of motion, are also employed to further reduce simulation costs.}. Being able to calculate both the particle velocities and trajectories, \gls{md} emerges a powerful tool to not only observe microscopic processes, but also compute macroscopic experimental observables of a given system. \par

For a system of N-particles of masses $\{m_i\}$ with a set of initial positions $\{ \vec{r_i} \}$ and momenta $\{ \vec{p_i} \}$, the time evolution of the system can be obtained by solving for the Newton's equation of motion:
\begin{equation} \label{e:eqm}
\vec{F_i} = m_i \frac{\partial^2 \vec{r_i}}{\partial t^2}, \text{ i=1...N}
\end{equation}

The forces $\vec{F_i}$ on the particles are obtained from the gradient of a potential V($\{r_j\}$)
 \begin{equation} \label{e:fpote}
\vec{F_i} = -\vec{\nabla} V_i (\{r_j\}), \text{ i,j=1...N}.
\end{equation}

The mathematical function for V($\{r_j\}$ is decided based on the type of interatomic interactions being modeled. The equations of motions are then time-integrated over short time steps $\delta t$. The aspects of the potentials and the technique of performing the time-integration are discussed in the following sections.

\subsection{Interatomic Potentials} \label{s:ffs}
As seen from Equations~\ref{e:eqm}~and~\ref{e:fpote}, the time-evolution of the system in \gls{md} are crucially dependant on the interatomic potentials (also called force-fields). Different mathematical functions have been prescribed to describe the various physical interactions. The simplest kind of interatomic potential is the \gls{lj} potential and it is a pair potential taking the following form:

\begin{equation} \label{e:lj}
		U_{ij}(r_{ij}) = 4 \epsilon \left[\left(\frac{\sigma}{r_{ij}}\right)^{12} -\left(\frac{\sigma}{r_{ij}}\right)^{6}\right],
\end{equation}

where $U_{ij}$ is the interatomic potential between the i$^{th}$ and j$^{th}$ atoms, $r_{ij}$ is the distance between the particles, $\epsilon$ is the strength of the potential, and $\sigma$ is the distance between the two particles at which the potential energy between them is zero \cite{Lee2016}. The \gls{lj} potentials models the interatomic forces in a simple fashion: The first term, contributing energy to the system replicated the close-range repulsions. The second term contributes to the attractive forces. The balance between the two terms results in the binding of atoms. This definition of an interatomic potential was adopted for supercooled liquids \cite{Kob1995}, and it still used as a model system for glass simulations \cite{Danilov2016}. \par

However, such a pair potential does not include the many-body effects that occur in metallic systems. \textcite{Daw1993}, in 1993, elaborate on the limitations of the pair potential and a solution for it. In metals, bonds between atoms are not independent of each other. Consequently, the pair potentials do not replicate metallic bonding because the \glsdesc{ecoh} \gls{ecoh}() does not scale with negative of \glsdesc{z} (\gls{z}), which is the expected result from pair potential. Rather, \gls{ecoh} scales more weakly as $E_{coh} \propto Z^{1/2}$ \cite{Daw1993}. \par

To overcome the limitations discussed above, and to capture many-body interactions in metallic solids, the \gls{eam} model was proposed by \textcite{Daw1983}. This model takes into consideration the realistic forces on a central atom resulting from the sea of electrons from the atoms that the central atom is embedded in. Each atom is considered as an impurity in the host of the other atoms. Now, the \gls{eam} potential U$_{EAM}$ takes the following form:

\begin{equation} \label{e:eam}
 	U_{EAM} = \sum_{i} \mathcal{F}_{i}(\rho_{i}) + \sum_{i \neq j} \phi_{ij}(r_{ij})  %F \left( \sum_{i \neq j} \rho_{} \right)
\end{equation}

where $\mathbb{F}_i$ is the embedding energy functional, $ \rho_{i}$ is the average electron density of the i$^{th}$ atom, $\phi_{ij} $ is a short-range electrostatic interaction, and $ r_{ij} $ is the scalar distance between the i and j atoms. The general form of the functional $\mathcal{F}$ and $\phi$ are not known, but are fitted to replicate elastic properties known from experiments.\par  

The \gls{eam} interatomic potential used in this thesis was developed by \textcite{Mendelev2009,Mendelev2019}. The potential form used was of a Finnis–Sinclair-type: which differs only slightly from the original Daw and Baskes \cite{Daw1983,Daw1993} formalism in that the electron density $\rho_{i}$ is element dependant. These interatomic potentials are semi-empirically derived; the Cu-Zr potential utilised in this thesis \cite{Mendelev2019} was developed from pure Cu and Zr potentials, modifying the $\rho_{i}$ to fit to the experimental liquid density, mixing enthalpy, and \gls{rdf} \cite{Mendelev2009} (Information on \gls{rdf} is given in Section~\ref{s:rdf}). \par 

Later, the CuZr \gls{eam} potential was improved to predict the Laves phases better by fitting the  \gls{eam} \gls{prdf} to those from experiments and \textit{ab-initio} \gls{md} simulations. Only the terms associated with Cu-Zr interactions were modified; the Cu-Cu and Zr-Zr interactions remained unchanged \cite{Mendelev2019}. Such semi-empirical potentials are system specific and provide reasonably accurate results for large system sizes. \par 

Although the \gls{eam} interatomic potential has a complicated form, at the time of calculating the forces, one only needs the tabulated values of $\mathcal{F}$, $\rho$ and $\phi$ to calculate $U_{EAM}$ and $\vec{F_i}$ from Equations~\ref{e:eam}~and~\ref{e:fpote} respectively. The final expression for $U_{EAM}$ bears resemblance to a pair-wise calculation, leading to the \gls{eam} potentials being often mistaken to be similar to classical potentials like the \gls{lj} model. Owing to the additional calculation of the embedding forces, the \gls{eam} potentials are not only more accurate, but are also computationally expensive than the \gls{lj} potentials. \par

\subsection{Periodic Boundary Conditions and Pair Cutoffs}
Simulations are usually carried out in a cubic or cuboid simulation box, which is a virtual space in which the particles exist. A \gls{2d} simulation box is depicted in Figure~\ref{f:pbc}. While \gls{md} techniques can simulate millions of atoms in today's supercomputers, the number is still quite small in comparison to realistic systems where the particle numbers are in the orders of the Avogadro's number ($6.022 \times 10^{23}$). For this reason, any edge effects or surface effects observed in simulations will be significantly larger than observed in the experiments. \par

\begin{figure}[!h]
	\includegraphics[width=\linewidth]{pbc.png}
	\mycaption{Simulation Box and Periodic Boundary Conditions}{(a) A simulation box and eight of its infinite images are depicted, atoms moving across the boundary can be mapped back within the original box as well. (b) The atoms in the simulation box are partitioned into grids, which can be assigned to independent processors for parallel computing. Ghost atoms of the central grid, associated with the highlighted central atom are also depicted.}
	\label{f:pbc}
\end{figure}

In order to circumvent this problem, \gls{pbc} are used, instead of closed boundaries. The periodic boundaries are visualized in Figure~\ref{f:pbc}.: the simulation box is connected across boundaries such that any atoms that exits the simulation box on side side, enters from the opposite side while maintaining the same trajectory. \footnote{An unacquainted reader can draw parallels from the Pac-Man arcade game, in the way that Pac-Man and the ghosts traverse across the edge of the maze, only to appear on the other side.} Additionally, a minimum-image convection \cite{Frenkel1997,Lee2016}, for each atom that crosses the boundary in either of the \textbf{$\hat{i}$}, \textbf{$\hat{j}$} and \textbf{$\hat{k}$} directions, an ``image" flag is assigned to each atom. Forces on a particle are calculated with the nearest images of the other particles. Even with periodic boundary conditions, the number of atoms in the system should be large enough, to avoid any finite-size effects. \par

In \gls{md} simulations, forces acting on every atom need to be calculated, and subsequently the computation times scale as \On{N^2} with the number of particles N in the N-body system. To improve simulation efficiency, short-range interactions between particles are truncated within cut-off distances (See Figure~\ref{f:pbc}b). For long-range interactions such as Coloumbic forces, a technique called Ewald summation is used to approximate the long-range forces (forces which decay slower than $r^{-3}$) as well allowing the simulations to scale as \On{N^{3/2}} or \On{Nlog(N)} \cite{Lee2016,Frenkel1997}. \par

Furthermore, defining a cut-off radius also allows one to divide a simulation box into multiple grids (See Figure~\ref{f:pbc}b), which allows for the \gls{md} solvers to be run in parallel on multiple processors \cite{Thompson2022}. Each grid of the simulation box is assigned to a single processor. The atoms within a grid, and associated ``ghost" atoms---which are atoms outside the given grid but within the cut-off distances of the atoms inside---exist within the memory of the processor. In this way, the interactions of an N-body simulations can be computed independently within each processor grid. \par

\subsection{Time Integrators}
In an \gls{md} simulations box that was discussed in the previous section, the intial coordinates and even velocities of the particles can be defined. However, to get information on the time evolution of the system, the equations of motion---typically described by the Newtonian formalism---still need to be numerically time-integrated. Based on an algorithm first proposed by \textcite{Verlet1967}, the modified Velocity-Verlet algorithm \cite{Swope1982} is the standard technique to perform time integration in \gls{md} \footnote{Some alternatives like r-RESPA, Runge-Kutta methods also exist but are out of the scope of this thesis.}. At a given time t, the position of an i$^{th}$  particle $\vec{r}_i$ in  the time intervals $t \pm \delta t$ can be expressed as a Taylor expansion:

\begin{equation} \label{e:verini}
\begin{gathered}
	\vec{r}_i(t+\delta t) = \vec{r}_i(t) + \vec{v}_i(t)\delta t + \frac{\vec{F}_i(t)}{2m_i}\delta t^2 + \dddot{\vec{r}_i}(t) \frac{\delta t^3}{3!} + \mathcal{O}(\delta t ^4) \\
	\vec{r}_i(t-\delta t) = \vec{r}_i(t) - \vec{v}_i(t)\delta t + \frac{\vec{F}_i(t)}{2m_i}\delta t^2 - \dddot{\vec{r}_i}(t) \frac{\delta t^3}{3!} + \mathcal{O}(\delta t ^4) 
\end{gathered}
\end{equation}

Where $\vec{F}_i$, $\vec{v}_i$ and m$_i$ are the force, velocity, and mass of the i$^{th}$ particle respectively. Adding and subtracting the two Equations~\ref{e:verini}, one gets:
\begin{equation}
\vec{r}_i(t+\delta t) + \vec{r}_i(t-\delta t) = 2 \left(\vec{r}_i(t) + \frac{\vec{F}_i(t)}{2m_i}\delta t^2 + \mathcal{O}(\delta t ^4) \right)
\end{equation}

\begin{equation}
\vec{r}_i(t+\delta t) - \vec{r}_i(t-\delta t) = 2 \left(\vec{v}_i(t)\delta t + \dddot{\vec{r}_i}(t) \frac{\delta t^3}{3!}  \right)
\end{equation}

These two equations, have been approximated to give the following relation:
\begin{equation} \label{e:rvelver}
\vec{r}_i(t+\delta t) \approx \vec{r}_i(t) + \vec{v}_i(t)\delta t + \frac{\vec{F}_i(t)}{2m}\delta t^2 +\mathcal{O}(\delta t ^3)  % \dddot{\vec{r}}(t) \frac{\delta t^3}{3!} + \mathcal{O}(\delta t ^4)
\end{equation}
For a more detailed treatment, the reader is referred to References \cite{Frenkel1997,Lee2016}.

Similarly, an expression for velocities is obtained:
\begin{equation} \label{e:vvelver}
\vec{v}_i(t+\delta t) = \vec{v}_i(t) + \frac{\vec{a}_i(t) + \vec{a}_i(t+\delta t)}{2}\delta t
\end{equation}

Where $\vec{a}_i$ denotes the acceleration. With the Equations~\ref{e:rvelver}~and~\ref{e:vvelver}, the Velocity-Verlet algorithm can finally be described:

\begin{itemize}[noitemsep]
\item Get intital coordinates
\item Calculate new position $\vec{r}_i(t+\delta t)$ using Equation~\ref{e:rvelver}, and computing forces, or $\vec{a}(t)$
\item At t=$\delta t/2$, calculate intermediate velocity \begin{equation} \vec{v}_i(t+\delta t/2) = \vec{v}_i(t) + \frac{1}{2}\vec{a}_i \delta t \end{equation}
\item Using $\vec{r}_i(t+\delta t)$, compute acceleration $\vec{a}_i(t+\delta t)$
\item Calculate new velocity $\vec{v}_i(t+\delta t)$ \begin{equation} \vec{v}_i(t+\delta) = \vec{v}_i(t+\delta t/2) + \frac{1}{2}\vec{a}_i(t+\delta t) \delta t \end{equation}
\end{itemize}

As seen above, the time integration can be started with just the initial coordinates and velocities. Furthermore, the Velocity-verlet algorithm allows for calculation of forces just once per time step. The error in estimating $\vec{r}_i(t+\delta t)$ is of order \On{t^3}. The error dies down exponentially with the time of the simulation, this effect is termed as Lyapunov instability \cite{Frenkel1997}. This error loses significance at long simulation time scales. \par

The shorter the length of the timestep, the higher the accuracy. However, a longer timestep results in lesser calculations and consequently is less computationally expensive. For this reason, the choice of the timestep is quite important. For \gls{mg} simulations with the \gls{eam} potential, usually a timestep of is 1 \gls{fs}. A good timestep should be at least less than half the timescale of the fastest vibration in the system, by the Nyquist theorem \cite{Shannon1949}.


\subsection{Thermostats and Barostats}
Once a system has been assigned initial coordinates, and velocities, the \gls{md} simulation can be run as an isolated system or a microcanonical ensemble. To simulate the system at a specific temperature, or to avoid accumulation of numerical errors over time, temperature control is required, and it is done by implementing thermostat algorithms. \par

The simplest form of temperature control in \gls{md} can be done by harnessing the equipartition theorem. The average kinetic energy ($\langle E_{kin} \rangle $) and temperature (T) are associated by the following relation, for a system with 3 degrees of freedom:

\begin{equation}
\langle E_{kin} \rangle = \frac{1}{2} \langle \sum_{i} m_i \vec{v_i}^2 \rangle = \frac{3}{2} k_B T
\end{equation}

where m$_i$ and $\vec{v_i}^2$ are the mass and velocity of the i$^{th}$ particle respectively. Knowing this, temperature control can be achieved by simple scaling of velocities. Multiplying velocities by a certain factor can change the temperature of system. This, however, only sets the temperature but does not replicate behaviour of a closed system, or a canonical ensemble. In this thesis, a Nos\'e-Hoover thermostat \cite{Hoover1985} is implemented as thermostat to create an isothermal ensemble. The Nos\'e-Hoover thermostat couples the system with an infinite heat bath, and a fictitious mass (or coupling strength) that indicates how quickly the system's temperature can be set to the target temperature. \par
 
Similarly, the pressure of a \gls{md} system can also be controlled to then simulate constant or varying pressures. Such an algorithm is referred to as a barostat. The chosen barostat in this thesis is the \textcite{Parrinello1980} implementation, which introduces a time-dependent metric tensor, in additional to introducing volume as a thermodynamic variable. In the \gls{lmp} code, the barostat can be coupled with the Nos\'e–Hoover thermostat to approximately simulate an isoenthalpic-isobaric ensemble. \par

\section{Characterization}
The previous section describes the fundamentals of the \gls{md} techniques. The implementation of \gls{md} to simulate the \gls{rq} \gls{mg}s, \gls{ng}, and \gls{camg}, will be discussed in Chapter~\ref{c:dev}. The novel properties of \gls{ng}s and \gls{camg}s are then evaluated in Chapters~\ref{c:camg}~and~\ref{c:cbmg}. \par

In order to aid the development of \gls{camg} simulation protocols, and to unravel the exciting properties of \gls{camg}s, all glasses simulated in this dissertation are characterized based on their structural and energetic features. \par %Being that the systems of our interest are amorphous in nature, they are structurally characterized by their atomic pair-correlations, and their \gls{sro} and \gls{mro}. Additionally, the \gls{camg}s are evaluated by \par

\subsection{Radial Distribution Function} \label{s:rdf}
In amorphous materials, which do not possess Long-range Order, the \gls{rdf}---or the pair correlation function---is to describe atomic structures. The \gls{rdf} usually is denoted as g(r), and is defined as the average probablility of finding a neighbouring atom within a spherical shell $dr$ at distance $r$ of a given atom. Mathematically, the number of particles in the shell around an atom can be expressed as:

\begin{equation}
dn(r) = \frac{N}{V} g(r) 4\pi r^2dr
\end{equation}
Where, N is the total number of particles, and V is the volume occupied by the system. From the above equation, the expression of g(r) around a single atom is derived:
\begin{equation}
g(r) = \frac{V}{N} \frac{dn(r)}{4 \pi r^{2}dr}
\end{equation}

In the case of a two-component system where $\alpha$ and $\beta$ are the two species, the \gls{prdf} of the $\alpha$ species with neighbouring atoms of $\beta$ species is written as: 
\begin{equation}
g_{\alpha \beta}(r) = \frac{V}{N_\alpha} \frac{dn_{\alpha \beta}(r)}{4 \pi r^{2}dr}
\end{equation}

Integrating the \gls{rdf} $g(r)$ function within a shell of r$_0$ gives coordination number in the shell---which is the number of neighbours inside a coordination sphere. In \gls{lmp}, the g(r) is calculated as a histogram: by binning pairwise distances of all atoms within an \gls{md} pair cutoff value. \par

\begin{figure}[h]
	\begin{subfigure}{0.5\textwidth} \centering
		\includegraphics[height=4.5cm,trim={3cm 1.25cm 3cm 1.25cm},clip]{2dmelt.png}
		\subcaption{}
	\end{subfigure}%
	\hfill
	\begin{subfigure}{0.5\textwidth} \centering
		\includegraphics[height=5.5cm]{rdf-lj.png}
		\subcaption{}
	\end{subfigure}%
	\mycaption{RDF of a 2D LJ Melt}{(a) A simulation snapshot of a \gls{2d} LJ melt using \gls{ovito} (b) The \gls{rdf} of the liquid.}
	\label{f:ljrdf}
\end{figure}

Figure~\ref{f:ljrdf}a visualizes a simulated 2 dimensional \gls{lj} melt with \gls{pbc}. In Figure~\ref{f:ljrdf}b, the corresponding \gls{rdf} is described. One way to interpret the $g(r)$ is as an indication of the local density. At the length scale of the order of atomic radii, the local density ($\rho (r)$) is a modulated function of average density ($\rho_{bulk}$). \begin{equation} \rho (r) = \rho_{bulk} \cdot g(r) \end{equation}
Supposing the system simulated in Figure~\ref{f:ljrdf} was a crystalline material, then RDF would have looked a series of peaks, each n$^{th}$ peak corresponding to the position of the n$^{th}$ nearest neighbour. However, since it is a liquid, and lacking long-range order, the peaks that would have been seen in crystalline materials broaden and show the average \gls{sro} and \gls{mro}. The first peak of g(r) (1.2 LJ distance units in Figure~\ref{f:ljrdf}) is the first nearest-neighbour distance. At large values of r, the probability of finding a nearest neighbor within the shell of radius r is a sure event. Hence, $g(r)\rightarrow 1$ for such large r values, and $ \rho (r) \rightarrow \rho_{bulk} $, as seen in Figure~\ref{f:ljrdf}b. \par 

Since the \gls{rdf} is a projection of the positions of nearest neighbours onto a radius space, it can give the average information of the bonding. It can not give a full description of the local atomic order. \par

%\url{http://isaacs.sourceforge.net/phys/rdfs.html} \\
%\url{http://rkt.chem.ox.ac.uk/lectures/liqsolns/liquids.html} \\
%\url{https://en.wikibooks.org/wiki/Molecular_Simulation}
%%https://en.wikibooks.org/wiki/Molecular_Simulation/Radial_Distribution_Functions



\subsection{Voronoi Analysis} \label{s:voronoi}
The Voronoi Tesselation method is used to partition space into mutually exclusive volumes around a finite set of points in space. While applied to metallic glasses, it can assign a finite volume---a polyhedron---to each atom in a simulation. In doing so, one can described the local orders and topology of the simulated amorphous structures. The three dimensional Voronoi polyhedra in the present thesis were determined using the methods described by \textcite{Brostow1978,Brostow1998} and \textcite{Borodin1999} and implemented on \gls{ovito} \cite{Stukowski2010a}. \par

\begin{figure}[h] \centering
	\includegraphics[width=0.4\textwidth,trim={1.8cm 0.5cm 1.2cm 0.8cm},clip]{voronoi-lj.png}
	\mycaption{Voronoi Tesselation of a \gls{2d} LJ Melt}{\gls{2d} Voronoi polygons constructed for the \gls{2d} LJ melt shown in Figure~\ref{f:ljrdf}a using Python and the \href{https://freud.readthedocs.io/en/latest/gettingstarted/examples.html}{\textit{freud}} Python library.}
	\label{f:ljvoro}
\end{figure}

The \gls{vp} are constructed as the shapes that are bound by intersecting planes perpendicular to lines connecting every atom pair. This method is reminiscent of the Wigner-Seitz construction of primitive cells in solid-state physics \cite{Kittel2004}. The representation of Voronoi tesselation in \gls{2d} is shown in Figure~\ref{f:ljvoro}. For 3 dimensional systems, the polyhedra are described by the Sch\"afli notation \cite{Coxeter1973}: \vi{n$_3$}{n$_4$}{n$_5$}{n$_6$}, where n$_i$ are the number of i-edged faces of the polyhedra \cite{Coxeter1973,Brostow1998}. The choices of the polyhedra is not artibrary, rather they are restricted by two constraints: 1. total number of edges is equal to \gls{z} of the central atom. 2. the Volyhedra satisfy the relation $\sum _i (6-i)n_i = 0$, known as the Euler relation \cite{Finney1970}. The \vi{n$_3$}{n$_4$}{n$_5$}{n$_6$}  Sch\"afli restricted for n$_i$, where $3 \leq i \leq 6$. This is because polyhedra cannot have faces with less than three edges, and because coordinates with edges of sizes four to six dominate dense packed structures \cite{Borodin1999}. \par

\begin{figure}
	%\begin{subfigure}{\textwidth}
	\begin{subfigure}{0.33\textwidth}	 \centering		\includegraphics[width=0.7\textwidth]{sc_bonds} 
		\subcaption{Nearest neighbours} \end{subfigure}%
	\hfill
	\begin{subfigure}{0.33\textwidth}	\centering  	\includegraphics[width=0.7\textwidth]{sc_coord}
	\subcaption{Coordination polyhedron} \end{subfigure}%
	\hfill
	\begin{subfigure}{0.33\textwidth}	\centering  	 \includegraphics[width=0.7\textwidth]{sc_voronoi} 
		\subcaption{Voronoi polyheron}  \end{subfigure}%
	\mycaption{Voronoi Polyhedra for an SC coordination}{For a (a) simple cubic coordination, the corresponding (b) coordination polyhedra and (c) Voronoi polyhedra are shown. The volumes of the polyhedra and the bonds are not drawn to scale.}
	\label{f:voronoi-sch}
\end{figure}

In Figure~\ref{f:voronoi-sch}a, a \gls{sc} coordination is depicted as an example. The central atom is surrounded by six nearest neighbours. Figure~\ref{f:voronoi-sch}b depicts the corresponding coordination polyhedron. The corresponding \gls{vp} (Figure~\ref{f:voronoi-sch}a) is equivalent to the coordination polyhedron, with the total number of faces in a \gls{vp} being the same as the number of vertices in the coordination polyhedron. The \gls{vp} has six sides with four faces each, and hence it's Sch\"afli index would be \vi{0}{6}{0}{0}. The indices add up to six, viz. the coordination number of the central atom. \par

For metallic glasses in general, the Voronoi polyhedra are known to be classified into four main categories as reported by \textcite{Yue2018}: 1. icosahedral-like: \vi{0}{0}{12}{0}, \vi{0}{0}{10}{x}, and \vi{0}{2}{8}{x}; 2. crystal-like: \vi{0}{4}{4}{x} and \vi{0}{5}{2}{x}; 3. mixed coordinations: \vi{0}{3}{6}{x}, where 0 $\leq$ x $\leq$ 4; and 4. other remaining indices. \par


\subsection{Surface Mesh} \label{s:surmesh}
The open surfaces of the simulated samples need to be identified in order to evaluate porosity in \gls{ng}s and \gls{camg}s, and to compute the volumes enclosed by the \gls{camg}s. In the current dissertation, this is done by means of constructing a surface mesh, by the alpha-shape algorithm \cite{Stukowski2014} implemented in \gls{ovito}. The algorithm utilizes a method similar to Voronoi tesselation called the Delaunay triangulation, that connects every three nearest neighbouring atoms as vertices of a triangle. Then, a ``probe sphere" is determined, which defines the smoothness of the surface mesh. The circumcirles of the constructed triangles are compares with the probe sphere. If the radius is smaller than that of the circumcircle, the surface is determined to be closed. If larger, then the surface is considered to be open. The alpha-shape method takes care of the \gls{pbc}.

\subsection{Local Atomic Strain} \label{s:vonMises}
The local shear strain or the von Mises Shear Strain, has been traditionally used as a model to explain failure theories in metallic materials, and for the visualization of shear transformations in metallic glasses. One of the aims of this dissertation is to identify the interfaces formed amongst clusters, in which atoms are sheared due to either compaction in \gls{ng}s, or deposition in \gls{camg}s.  \par

To calculate the local deformations, two system configurations are required: one being the sample to be characterized, and the second being an initial undeformed reference configuration. Using these, the von Mises shear strain is calculated by an algorithm proposed by \textcite{Shimizu2007}: from the two configurations, a local atomic deformation gradient tensor (or a transformation tensor) is determined, from with the strain tensor is calculated. The invariant of the strain tensor is the von Mises local shear strain, and is given by the following expression:

\begin{equation}
\eta_i ^{Mises} = \sqrt{\eta_{yz}^2 + \eta_{xz}^2 + \eta_{xy}^2 + \frac{1}{6} \left[ \left(\eta_{yy} - \eta_{zz}\right)^2 + \left(\eta_{xx} - \eta_{zz}\right)^2 + \left(\eta_{xx} - \eta_{yy}\right)^2 \right]}
\end{equation}

where, $\eta_i$ is the strain tensor for each i$^{th}$ atom. The von Mises strain was calculated and visualized using the \gls{ovito} software.

%\subsection{Thermal Stability}
%Annealing by giving a thermal ramp, and equilibrating at intermediate steps
%Enthalpy measured as $H = U + \Delta (PV)$
%
%\begin{equation} C_p = \left( \frac{\partial H}{\partial T} \right)_p	\end{equation}
%Heat Capacity Fitting with a sum of a sigmoid and a Gaussian peak
%\begin{equation}
%\tilde{C}(T) = C_s + \frac{\Delta C}{2} \left(1 + erf\left(\frac{T-T_s}{\sqrt{2\sigma^2_s}} \right) \right) + \frac{\Delta H}{\sigma_p \sqrt{2\pi}} \left(exp\left(-\frac{(T-T_p)^2}{\sqrt{2\sigma^2_p}} \right) \right)
%\end{equation}
%where $C_s$ , $\Delta C$, $T_s$, $\sigma_s$, $T_p$, $\sigma_p$
%Onset temperature $T_R$

\clearpage
\section{Synopsis}
This chapter discusses the fundamentals of the \gls{md} technique, which was used to simulate the glassy systems in this dissertation. The simulations were run using the \gls{lmp} code on the \gls{fh2} and \gls{hrk} \gls{hpc} clusters available through the \gls{kit}. The \gls{mpi} protocol using MPICH2 was implemented to perform parallel programming. Furthermore, some characterization techniques such as \gls{rdf}, Voronoi Tesselation and the von Mises local shear strain---used to evaluate the simulated data sets---were elucidated. The characterization was done predominantly using Python scripts, and the \gls{ovito} Python library. \par

The iterative workflow for the development of the simulations and the software involved, are described in the \nameref{c:supple} Section, Figure~\ref{f:workflow}. The simulation codes developed in the framework of this thesis, and the necessary post-processing scripts can be accessed in Section~\ref{s:github}, also in the \nameref{c:supple}. The concepts addressed in this chapter will be implemented in Chapters~\ref{c:camg}~and~\ref{c:cbmg} to simulate and characterize \gls{camg}s and \gls{ng}s.

\chapter{Simulating Cluster-assembled Materials} \label{c:dev}
\chapter{Development of Simulations of Cluster-based Metallic Glasses} \label{c:dev}
\scdeclaration

It has been discussed in Chapter \ref{c:intro} that a simulation study of the \gls{camg} is critical in understanding their local structure and related properties. %\gls{rq} \gls{mg} in general are traditionally modelled using the \gls{eam} potential and the \gls{md} approach (see Section \ref{c:methods}).
The present chapter elucidates the \gls{md} models developed within the framework of this thesis to produce simulated data sets of \gls{camg}s. \par

First, the \Gls{rq} \gls{mg} is simulated and characterized, to test the \gls{eam} potential, and to serve as a reference. The protocols are then extended to study the \gls{camg} and \gls{ng}. \par

Next, the simulation of amorphous clusters---the building blocks of the \gls{camg}s---is discussed. The reader will then be introduced to novel simulation protocols, developed to investigate the \gls{camg}s. Further, a simulation protocol for \gls{ng}s is also discussed.  The challenges encountered in modelling the non-trivial deposition process of the \gls{camg}, and solutions on how to overcome these hurdles are presented; thereby allowing one to simulate \gls{camg}s in a computationally efficient manner. %by their \gls{rdf}s, local atomic (\gls{vp}) structure, and their atomic volume and \gls{pe} occupancies. \par

\section{Simulating Metallic Glasses} \label{s:simtestMG}
The simulated behaviour of Cu-Zr \gls{rq} \gls{mg}s are compared with results from previously known works. The variation of the local packing, final energy states, and volume occupancy in the Cu-Zr glasses are studied as a function of quench rates (\qr{10}, \qr{12}, \qr{13}, \qr{14}) and composition (\cz, \czsix, \czsf). \par

\begin{sidewaysfigure} %[!h]
	\centering
	\begin{subfigure}{\textheight} \centering \includegraphics[width=\textheight]{rdf_CuZr_MG_asp_All.png}
		\subcaption{\glspl{prdf} vs compositions: \cz, \czsf, and \czsix.} \end{subfigure}%
	\vfill
	\begin{subfigure}{\textheight} \centering \includegraphics[width=\textheight]{rdf_Cu50Zr50_MG_asp_All.png}
		\subcaption{PRDFs vs quench rates: \qr{10}, \qr{12}, \qr{13} and \qr{14}} \end{subfigure}
	\mycaption{Partial pair correlations in RQ MGs}{Cu-Cu, Cu-Zr, and Zr-Zr pair correlations in the RQ MGs for varying MG-synthesis compositions and quenching rates.}
	\label{f:rdf_mgs}
\end{sidewaysfigure}

\begin{changebar}
Atomistic \gls{md} simulations of binary Cu-Zr glasses have been performed using the \gls{lmp} code \cite{Plimpton1995,Thompson2022}. A semi-empirical potential developed with data from Cu$ _{46}$Zr$ _{54} $ alloys \cite{Mendelev2019}, based on the \gls{eam} model proposed by \textcite{Daw1993}, was used to model the Cu-Zr interactions. The \cz  \gls{mg}s were simulated by quenching from the 2000 K (liquid state) to 50 K temperature. The liquid state was simulated by equilibrating an equimolar mixture of Cu and Zr atoms placed at random coordinates at 2000 K for 2 ns. The quenching to the glassy state was performed at zero pressure at four cooling rates of \qr{10}, \qr{12}, \qr{13} and \qr{14}. The system temperature and pressure were controlled by using a Nos\'e-Hoover thermostat (NPT in \gls{lmp}). The MGs will be referred to by the quench rate: for instance, an MG quenched at \qr{10} will be referred to as a \qr{10} MG, and so on. For the case of the \qr{10} \gls{mg}, which requires a much longer and computationally expensive simulation, a sample of $\sim$16,000 atoms was chosen, and replicated in all three dimensions, resulting in a larger sample of $\sim$150,000 atoms in total. For the \qr{12} and \qr{14} MGs, in which the simulation costs are less, the larger samples of $\sim$150,000 atoms were prepared directly. Periodic boundary conditions were used in all three directions, and a time step of 1 \gls{fs} was chosen for all simulations. After quenching, the metallic glasses were equilibrated for 2 ns at 50 K. \par
\end{changebar}

\subsection{Pair-correlations in RQ MGs} \label{s:rdf-mgs}
The \glspl{prdf} were computed from the snapshots of the \gls{rq} \gls{mg}s, with a cutoff radius of 10 \r{A}. The lack of any sharp peaks indicates lack of crystalline order. The first peak positions of Cu-Cu (2.45 \r{A}), Cu-Zr (2.8 \r{A}), and Zr-Zr (3.25 \r{A}) in the MGs match well with those previously reported values \cite{Duan2005,Nasu2007}. Beyond this information, the PRDFs fail to capture the change in the structure of RQ MGs with cooling rate or with composition. \par

%\subsection{Local Atomic Packing} \label{s:voro-mgs}
The local amorphous order of \gls{mg}s, i.e., the description of the coordinations exhibited by the atoms were learnt via Voronoi Tesselation, a process that has been described in Chapter \ref{c:methods}. The reader is suggested to refer to the Section~ \ref{s:voronoi}, wherein the method is described. \par

\begin{selfcite}
While it is understood that the \gls{ilo} is prominent in binary metallic glasses, the exact occurrence of \gls{ilo} and \gls{fi} is quite sensitive to the simulation protocols and potentials used \cite{Adibi2014,Avchaciov2013, Lu2018, Li2009a}. To standardise the simulation protocols presented in this thesis, Voronoi analysis was performed on these already well studied RQ MGs. \textbf{The coordinates of the final positions of the atoms obtained from the \gls{lmp} based simulations are imported into } \par

\begin{figure}[!ht]
	\centering
	\begin{subfigure}{0.45\linewidth} \centering \includegraphics[height=0.4\textheight]{voronoi_Cu50Zr50_MG_asp_All}
		\subcaption{} \end{subfigure}%
	\begin{subfigure}{0.45\linewidth} \centering \includegraphics[height=0.4\textheight]{voronoi-sort_Cu50Zr50_MG_asp_All}
		\subcaption{} \end{subfigure}
	\mycaption{Local SRO vs. Quench Rate in RQ MGs}{(a) and (b) Show systematic increase in full-icosahedra and icosahedral-like fractions of \cz glasses with decrease in cooling rate, from \qr{10} to \qr{14}.}
	\label{f:voro_qr}
\end{figure}
\end{selfcite}

First, the behaviour of the simulated RQ MGs with varying quench rate, but at a fixed composition (\cz), is verified. Figure~\ref{f:voro_qr}a shows the top seven highest occurring Voronoi indices in \cz RQ glasses quenched with cooling rates \qr{10}, \qr{12}, \qr{13}, and \qr{14}. The percentage of atoms \vi{0}{0}{12}{0} index, representing the full-icosahedral coordinations, is seen to increase with lowering quench rates. Thjs was in agreement with previous works on metallic glasses \textbf{cite}, where it was observed that lowering the glass quench rate correlates with increased icosaheral packign and stabilty. In Figure~\ref{f:voro_qr}b, the sorted categories of Voronoi coordinations are described. While it can be seen that the the Icosahedral like (ICO-like) coordinations also increase with decreasing quench rate, concurrently one can also notice the decrease in the "Other"-or-miscellaneous coordinations, which are known to be indicators of glass instability. (for details on sorting, see Section~\ref{s:voronoi}).  decreasing the quenching rate improves the icosahedral fractions \textbf{cite} \par

Next, the Voronoi behaviour was contrasted with composition at a fixed quecnh rate. cz czsx, and czsix glasses made a quench rate of 12 are studied. From Figure~\ref{f:voro_comp}a-b, two observations can be made: 1. With increasing Cu composition in CuZr metallic glasses while keeping the quench rate fixed, the \gls{fi} anf the ilo increase. This observation is consistent with results from previous works \cite{Peng2010}. 2. At the same time, the defective coordinations marked as "Other" decrease. In CuZr glasses, increase in Cu seems to improve \gls{gfa}.

\begin{selfcite}
\begin{figure}[!ht]
	\centering
	\begin{subfigure}{0.45\linewidth} \centering \includegraphics[height=0.4\textheight]{voronoi_CuZr_MG_asp_All} 
		\subcaption{} \end{subfigure}%
	\begin{subfigure}{0.45\linewidth} \centering \includegraphics[height=0.4\textheight]{voronoi-sort_CuZr_MG_asp_All} 
		\subcaption{} \end{subfigure}%
	\mycaption{Local SRO vs. Composition in RQ MGs}{(a) and (b) show the increase of \vi{0}{0}{12}{0} full-icosahedral and icosahedral-like fractions with increase in composition from \cz, to Cu$_{60}$Zr$_{40}$ to \czsix for \qr{12} MGs.}
	\label{f:voro_comp}
\end{figure}
\end{selfcite}

\subsection{Potential Energy of MGs}  \label{s:pe-mgs}
\begin{figure}[!ht]
	\centering
	\begin{subfigure}{\linewidth} \centering
	\includegraphics[width=0.7\linewidth]{pe-atom_CuZr_MG_asp}
	\subcaption{heading}
	\end{subfigure}%
	\vfill
	\begin{subfigure}{\linewidth} \centering
	\includegraphics[width=0.7\linewidth]{pe-atom_Cu50Zr50_MG_asp}
	\subcaption{heading}
	\end{subfigure}	
	\mycaption{P.E. vs Composition and Quench Rate in RQ MGs}{Potential Energy per Atom distribution for \qr{12} metallic glasses of varying composition: \cz, Cu$_{60}$Zr$_{40}$ and \czsix. The inset describes the average potential energy (or potential energy per atom) for the three glasses. For each of the glasses, this is the total area under the curve from both Cu and Zr distributions.}
	\label{f:pe_mgs}
\end{figure}

The potential energy of the simulated RQ MGs were analysed with respect to both the cooling rate and the compositional changes in CuZr metalliglasses. Figure~\ref{f:pe_mgs} shows the normalized Potential Enegry distributions of the MGs. Two distinct distributions are observed; one for the Cu atoms, and one for the Zr atoms. The Zr atoms are found to occupy lower energies on average compared to the Cu atoms. For the \cz \qr{10} MGs, Cu peak occurs at \sim-3.52 eV and the Zr peak at ~-6.45 eV. In the inset of the figures, the Average potential energy of each atom; i.e, the area under the graph is represented. In Figure~\ref{f:pe_mgs}a, it is noticed that the average P.E. of the RQ MGs reduces with quench rate, this is as expected in literature. The relative peak shifts are not noticeably different from one another; however no further analysis has been attempted to characterize the nature of these distributions. Yet another trend is observed, in Figure~\ref{f:pe_mgs}b, where CuZr RQ MGs of \qr{12} are constrasted with one another: with increasing Cu concentration, the average P.E. notably increases. This change is correspondingly noticed in the P.E./atom distributions: the peak height of Zr drops with composition (and that of Cu increases), owing to the decrease in stoichiometric population of Zr atoms. Furthermore, with the peak centers also being shifted to the right, it is confirmed that both the Cu and Zr atoms, on an average, occupy high energy states in glasses with higher Cu concentration. The relative increase in Cu concentration has a stronger influence in the change of energy states, than does the quench rate. \par

\subsection{Atomic Volume Distributions of MGs}  \label{s:vol-mgs}
Similar to the Sections~\ref{s:voro-mgs}~and~\ref{s:pe-mgs}, the RQ MGs are also contrasted with each other in terms of atomic volume, while varying both composition and quench rates. In Figure~\ref{f:vol_comp}, the dristributions of atomic volume occupancy is shown; in the inset, the average volume per atom (area under the curve) is described. Like in Figure~\ref{f:pe_mgs}, two separate distributions are once again observed for the Cu and Zr atoms. The Cu atoms are seen to occupy a lower volume on average in comparison to Zr atoms, influenced by their respective atomic radii (Cu: 1.35 \r{A}, Zr: 1.55 \r{A}). With increasing Cu composition, the Cu peak shifts higher, yet it moves to the left. The opposite in observed for Zr atoms. The resulting effect is seen on the average atomic volume: which reduces with increase in Cu composition. CuZr with a higher concentration in the range of compositions explored tend to be better packed. \par

\begin{figure}[!ht]
	\centering
	\begin{subfigure}{\linewidth} \centering
		\includegraphics[width=0.7\linewidth]{pe-atom_CuZr_MG_asp}
		\subcaption{heading}
	\end{subfigure}%

	\mycaption{Volume Occupancy vs Composition in RQ \qr{12} MGs}{Potential Energy per Atom distribution for \qr{12} metallic glasses of varying composition: \cz, \czsf and \czsix. The inset describes the average potential energy (or potential energy per atom) for the three glasses. For each of the glasses, this is the total area under the curve from both Cu and Zr distributions.}
	\label{f:vol_comp}
\end{figure}

Next, the relationship between volume distributions of \cz glasses and their quench rates were investigated. Figure~\ref{f:vol_quench}a shows the volume distributions of the Cu and Zr atoms. The occurence of the peaks is similar to that inFigure~\ref{f:vol_comp}. In the inset of Figure~\ref{f:vol_quench}a are the calculated values of the average atomic volumes per atom in the RQ MGs. Here, no clear trend is observed. Furthermore, it noticed that the \qr{10} glass has the highest volume, i.e, the lowest density. These findings are in disagreement with previous knowledge. \par

It was initially assumed that the simulation technique mentioned in Section~\ref{s:simtestMG} was a mistake. Firstly, the box sizes chosen were not equal. Next, the melting of the metallic glass was not simulation before the quench. To reveal the influences of these two processing steps, some additional simulations were performed. In Figure~\ref{f:vol_quench}b-d, the volumes were recorded with respect to the temperature, as the quenching process occurred. In the inset, the Temperature (T) is also plotted as a function of time (t). Figure~\ref{f:vol_quench}b MGs were quenched after a 2 ns melt step (same process as in Section~\ref{s:simtestMG}). Figure~\ref{f:vol_quench}c, all RQ MGs had the same number atoms $\sim$ 8000. Next, Figure~\ref{f:vol_quench} all RQ MGs had the same number of atoms, but additionally a 2 ns melting step was performed. The initial random mixture of atoms were first equilibrated at \textbf{300 K}, melted to 2000 K, equlibrated for 2ns, and then quenched. In the three processes, however, the glass behaviour is not reproduced correctly. Moreover, at 50 K, volume fluctuations--significantly higher than a volume change occuring by a temperature increase of 50 K--are seen as indicated in the insets of Figures~\ref{f:vol_quench}b-d. Visually, the glass transition is estimated to be around 600 K, however, a more rigorous estimated has not been attempted for these glasses. \par

\begin{figure}
	\begin{subfigure}{\linewidth} \centering
		\includegraphics[width=0.7\linewidth]{pe-atom_Cu50Zr50_MG_asp}
		\subcaption{heading}
	\end{subfigure}	%
	\vfill
	\begin{subfigure}{0.33\textwidth}
		\includegraphics[width=0.8\textwidth]{50-50/post/volume_8000}
		\subcaption{All MGs except 1e10 have $\sim 100K$ atoms}
	\end{subfigure}%
	\hfill
	\begin{subfigure}{0.33\textwidth}
		\includegraphics[width=0.8\textwidth]{50-50/post_8000/volume_8000}
		\subcaption{8192 atoms box, quench \textit{without} melting}
	\end{subfigure}%
	\hfill
	\begin{subfigure}{0.33\textwidth}
		\includegraphics[width=0.8\textwidth]{50-50/post_8000m/volume_8000}
		\subcaption{8192 atoms box, quench \textit{with} melting}
	\end{subfigure}
	\mycaption{Volume Occupancy vs Quench rate in \cz RQ MGs}{text}
	\label{f:vol_quench}
\end{figure}

To cross-verify the observations made from Figure~\ref{f:vol_quench}, the volume vs temperature behaviour was also studied as a function of quench rates in \czsix RQ MGs, which is close to the Cu$_{64.5}$Zr$_{36}$ composition validated by the developers of the CuZr glass potential \cite{Mendelev2019}. Starting with a box of $\sim$ 8000 atoms in the box, the glass quenching was performed for atoms equilibrated at an inital temperature of 2000 K (Figure~\ref{f:vol_quench64}a), and also atoms which were first set to \textbf{300 K}, melted to 2000 K, and then quenched to 50 K (Figure~\ref{f:vol_quench64}b). For both treatments, the volume of the RQ MGs presented lots of fluctuations during cooling. For the case of direct quenching from the 2000 K, the expected trend of enchanced packing with lower quench rates is not seen below 75 K. The desired effects were observed when the \czsix RQ MGs were first melted from 300 K to 2000 K before quenching. In both the cases, however, the final average volumes of the glasses fluctuate significanlty in comparison to thermal effects as in Figures Figures~\ref{f:vol_quench}b-d.  \par 

\begin{figure}
	\begin{subfigure}{0.44\textwidth}
		\includegraphics[width=0.8\textwidth]{64-36/post_8000/volume_8000}
		\subcaption{8192 atoms box,\\ quench \textit{without} melting}
	\end{subfigure}%
	\hfill
	\begin{subfigure}{0.44\textwidth}
		\includegraphics[width=0.8\textwidth]{64-36/post_8000m/volume_8000}
		\subcaption{8192 atoms box, quench \textit{with} melting}
	\end{subfigure}
	\label{f:vol_quench64}
	\mycaption{Volume Occupancy vs Quench Rate in \czsix RQ MGs}{text}
\end{figure}

Based on the above simluations, it was inferred that the while the RQ MG volume behaviour is greatly influenced by compositional effects,  the effects of quenching rate are not well reproduced by the EAM potential used. The effects of alternatively available, but older potentials is not in the scope of thesis \cite{Cheng2008,Mendelev2009}.

\subsection{Local Atomic Short-range Order} \label{s:voro-mgs}
The local amorphous \gls{sro} of \gls{mg}s, i.e., the description of the local coordinations exhibited by the atoms were studied via Voronoi tessellation, a process that has been described in Chapter~\ref{c:methods},  Section~\ref{s:voronoi}. % The reader is suggested to refer to the Section~\ref{s:voronoi}, wherein the method is detailed.
The volumes occupied by atoms upon Voronoi tessellation were shown by \textcite{Sheng2006} to be analogs of the Frank-Casper polyhedra \cite{Frank1958}. These shapes were used to define the local \gls{sro}. Moreover, it was observed that certain polyhedra were more prominently occurring. In CuZr glasses, atoms in icoshedral coordinations---or demonstrating \gls{fi} order---and \gls{ilike} coordinations---or demonstrating \gls{ilo}---were found to be the highest occuring polyhedra, and the most relevant motifs that correlate to glass stability \cite{Ding2014,Yue2018}. Other kinds of polyhedra could also possibly exist in the glasses, describing other motifs, such as crystalline packing (should crystalline coordinations be present in the solids), or the liquid-like packing (which constitute sites for shearing under stresses) \cite{Ding2014,Cheng2009,Yue2018}. However, these polyhedra are not discussed in detail in this thesis. \par

\textcite{Ding2014} mention how the distinction between \gls{fi} and \gls{ilike} lies in the distortion tolerance set in the polyhedron designation algorithm during the tessellation. For this reason, the quantified \gls{fi} is quite sensitive for differences in the simulation protocols and potentials used.\cite{Adibi2014,Avchaciov2013, Lu2018, Li2009a}. This variability also prompts the inclusion of the \gls{ilo} in the study of the local amorphous \gls{sro}. To standardise the simulation protocols presented in this thesis, Voronoi analysis was performed on these already well studied RQ MGs. \par

\begin{changebar}
\begin{figure}[h] %[!h]
	\centering
	\begin{subfigure}{0.45\linewidth} \centering \includegraphics[height=0.4\textheight]{voronoi_Cu50Zr50_MG_asp_All}
		\subcaption{} \end{subfigure}%
	\begin{subfigure}{0.45\linewidth} \centering \includegraphics[height=0.4\textheight]{voronoi-sort_Cu50Zr50_MG_asp_All}
		\subcaption{} \end{subfigure}
	\mycaption{Local \acrlong{sro} vs. quench rate in RQ MGs}{(a) and (b) Show systematic increase in full-icosahedra and icosahedral-like fractions of \cz  RQ glasses with decrease in cooling rate, from \qr{14} to \qr{10}.}
	\label{f:voro_qr}
\end{figure}
\end{changebar}

First, the behaviour of the simulated RQ MGs with varying quench rate, but at a fixed composition (\cz), is verified. Figure~\ref{f:voro_qr}a shows the top seven highest occurring Voronoi indices in \cz  RQ glasses quenched with cooling rates \qr{10}, \qr{12}, \qr{13}, and \qr{14}. The percentage of atoms exhibiting the \vi{0}{0}{12}{0} index, representing the \gls{fi} coordinations, is seen to increase with lowering quench rates. In Figure~\ref{f:voro_qr}b, the sorted categories (as described in Section~\ref{s:voronoi}) of \gls{vp} are described. Firstly, the lack of any crystalline coordinations indicates that all these \gls{rq} \gls{mg}s have amorphous \gls{sro}. While it can be seen that the \gls{ilike} coordinations also increase with decreasing quench rate, concurrently one can also notice the decrease in the ``other"-or-miscellaneous coordinations, which are known to be indicators of glass instability (for more details on the four prominent classes of \gls{vp}, see Section~\ref{s:voronoi}). Decreasing the quenching rate improves the \gls{fi} and \gls{ilike} fractions. These results are in agreement with previous knowledge \cite{Yue2018,Berthier2016}, that lowering the glass quench rate correlates with increased \gls{ilike} packing and stability. \par

\begin{changebar}
	\begin{figure}[h]
		\centering
		\begin{subfigure}{0.45\linewidth} \centering \includegraphics[height=0.4\textheight]{voronoi_CuZr_MG_asp_All} 
			\subcaption{} \end{subfigure}%
		\begin{subfigure}{0.45\linewidth} \centering \includegraphics[height=0.4\textheight]{voronoi-sort_CuZr_MG_asp_All} 
			\subcaption{} \end{subfigure}%
		\mycaption{Local \acrlong{sro} vs. composition in RQ MGs}{(a) and (b) show the increase of full-icosahedral (\vi{0}{0}{12}{0}) and icosahedral-like fractions with increase in composition from \cz, to Cu$_{60}$Zr$_{40}$ to \czsix  for \qr{12} MGs.}
		\label{f:voro_comp}
	\end{figure}
\end{changebar}

Next, the \gls{fi} and \gls{ilike} fractions in RQ MGs was studied with respect to composition at a fixed quench rate. \cz, \czsf, and \czsix  glasses made at a quench rate of \qr{12} were studied. From Figure~\ref{f:voro_comp}a-b, two observations can be made. Firstly, with increasing Cu composition in CuZr metallic glasses while keeping the quench rate fixed, the \gls{fi} and the \gls{ilo} increase. This observation is consistent with results from previous works \cite{Peng2010,Ritter2012,Li2009a}. Secondly, a concurrently decrease in the defective coordinations marked as ``other" is observed. In CuZr glasses, increase in Cu seems to improve \gls{gfa}. \par
%\newpage

\begin{figure} %[h] %[!h]
	\centering
	\begin{subfigure}{\linewidth} \centering
		\includegraphics[width=0.7\linewidth]{pe-atom_Cu50Zr50_MG_asp}
		\subcaption{\gls{pe} vs Quench Rate}
	\end{subfigure}%
	\vfill
	\begin{subfigure}{\linewidth} \centering
		\includegraphics[width=0.7\linewidth]{pe-atom_CuZr_MG_asp}
		\subcaption{\gls{pe} vs Composition}
	\end{subfigure}	
	\mycaption{\Glsdesc{pe} vs composition and quench rate in RQ MGs}{\gls{pe} per atom distribution for (a) \cz  MGs with changing quench rates \qr{10}, \qr{12}, and \qr{14}; and (b) \qr{12} RQ metallic glasses of varying composition: \cz,  \czsf  and \czsix. The insets describes the average potential energies of the glasses.}
	\label{f:pe_mgs}
\end{figure}

\subsection{Potential Energy of MGs}  \label{s:pe-mgs}
The \gls{pe} of the simulated RQ MGs were analysed with respect to both the cooling rate and the compositional changes in CuZr \gls{rq} \gls{mg}s. Figure~\ref{f:pe_mgs} shows the normalized \gls{pe} distributions of the \gls{mg}s. Two distinct distributions are observed; one for the Cu atoms, and one for the Zr atoms. The Zr atoms are found to occupy lower energies on average compared to the Cu atoms. For the \cz  \qr{10} MGs, the Cu peak occurs at $\sim$-3.52 eV/atom and the Zr peak at $\sim$-6.45 eV/atom. In the inset of the figures, the average \gls{pe} of each atom; i.e, the area under the graph is represented. In Figure~\ref{f:pe_mgs}a, it is noticed that the average \gls{pe} of the RQ MGs reduces with quench rate, this in alignment with the expected physical behaviour of glasses \cite{Yue2018,Berthier2016}. The relative peak shifts are not noticeably different from one another. However, no further analysis has been attempted to characterize the nature of these distributions. Yet another trend is observed, in Figure~\ref{f:pe_mgs}b, where Cu-Zr RQ MGs of \qr{12} are contrasted with one another: with increasing Cu concentration, the average \gls{pe} notably increases. This change is correspondingly noticed in the \gls{pe}/atom distributions: the peak height of Zr drops with composition (and that of Cu increases), owing to the decrease in stoichiometric population of Zr atoms. Furthermore, with the peak centres also being shifted to the right, it is confirmed that both the Cu and Zr atoms, on an average, occupy high energy states in glasses with higher Cu concentration. The relative increase in Cu concentration has a stronger influence in the change of energy states of metallic glasses, than does the quench rate. \par

\subsection{Atomic Volume Distributions of MGs}  \label{s:vol-mgs}
Similar to the Sections~\ref{s:voro-mgs}~and~\ref{s:pe-mgs}, the RQ MGs are also contrasted with each other in terms of atomic volume, while varying both composition and quench rates. In Figure~\ref{f:vol_comp}, the normalized distributions of atomic volume occupancy are shown; in the inset, the average volume per atom (area under the curve) is described. Like in Figure~\ref{f:pe_mgs}, two separate distributions are once again observed for the Cu and Zr atoms. The Cu atoms are seen to occupy a lower volume on average in comparison to Zr atoms, influenced by their respective atomic radii (Cu: 1.35 \r{A}, Zr: 1.55 \r{A}). With increasing Cu composition, the Cu peak height increases, yet it shifts to the left on the x-axis. The opposite trend is observed for Zr atoms. The resulting effect is seen on the average atomic volume which reduces with increase in Cu composition (see Figure~\ref{f:vol_comp} inset). CuZr glasses with a higher concentration in the range of compositions explored tend to be better packed. \par

Next, the relationship between volume distributions of \cz  glasses and the quenching rates was investigated. Figure~\ref{f:vol_quench}a shows the volume distributions of the Cu and Zr atoms. The occurence of the peaks is similar to that in Figure~\ref{f:vol_comp}. In the inset of Figure~\ref{f:vol_quench}a are the calculated values of the average atomic volumes per atom in the RQ MGs. Here, no clear trend is observed. Furthermore, it is noticed that the \qr{10} glass has the highest volume, i.e, the lowest density. These findings are in disagreement with previous knowledge \cite{Berthier2016,Yue2018}. \par

\begin{figure}
	\centering
	\includegraphics[width=0.7\linewidth]{pe-atom_CuZr_MG_asp}
	\mycaption{Volume occupancy vs composition in RQ \qr{12} MGs}{Volume distribution for \qr{12} metallic glasses of varying composition: \cz, \czsf and \czsix. The inset describes the average \gls{pe} (or \gls{pe} per atom) for the three glasses.}
	\label{f:vol_comp}
\end{figure}

\begin{figure}%[!h]
	\centering
	\begin{subfigure}{\linewidth} \centering
		\includegraphics[width=0.9\linewidth]{vol-atom_Cu50Zr50_MG_asp_All}
		\subcaption{Volume vs Quench Rate}
	\end{subfigure}	%
	\vfill
	\begin{subfigure}{\textwidth} \centering
		\begin{subfigure}{0.33\textwidth} \centering \renewcommand\thesubfigure{\alph{subfigure}1}
			\includegraphics[width=\textwidth]{50-50/post/volume_2} \caption{}
		\end{subfigure}%
		%	\hfill
		\begin{subfigure}{0.33\textwidth} \centering \renewcommand\thesubfigure{\alph{subfigure}2} 	\addtocounter{subfigure}{-1}
			\includegraphics[width=\textwidth]{50-50/post_8000/volume_8000} \caption{}
		\end{subfigure}%
		%	\hfill
		\begin{subfigure}{0.33\textwidth} \centering  \renewcommand\thesubfigure{\alph{subfigure}3} \addtocounter{subfigure}{-1}
			\includegraphics[width=\textwidth]{50-50/post_8000m/volume_8000m} \caption{}
		\end{subfigure}
		\addtocounter{subfigure}{-1}
		\subcaption{Volume vs Temperature: Varying RQ MG Preparation}
	\end{subfigure}
	\mycaption{Volume occupancy vs quench rate in \cz  RQ MGs}{(a) Volume occupancy in RQ MGs quenched at rates between \qr{10}-\qr{14} (b) Volume evolution during quenching (b1) as in Section~\ref{s:simtestMG} (b2) starting with 8192 atoms at 2000 K (b3) 8192 atoms melted from 50 K before quenching.}
	\label{f:vol_quench}
\end{figure}

For this reason, it was important to verify the simulation technique chosen to prepare \gls{rq} \gls{mg}s in Section~\ref{s:simtestMG}. Firstly, in the method used, the total number of atoms in the simulation were not equal for the MGs of varying quench rates. Next, the melt of the binary alloy was obtained by directly setting the temperature of the atoms to 2000 K for 2 ns. It could be suspected that the melting of the metallic glass before the quenching was not simulated well enough. To reveal the influences of these two processing steps, some additional simulations sets were performed for each of the \qr{10}, \qr{12} and \qr{14} MGs: 1. The MGs were simulated with equal number of atoms, 2. an additional melting step was performed before quenching. \par 

In Figure~\ref{f:vol_quench}b1-b3, the volumes were recorded with respect to the temperature, as the quenching process occurred. In the inset, the temperature (T) is also plotted as a function of the simulation timesteps. Figure~\ref{f:vol_quench}b1 depicts MGs that were treated by the same process as in Section~\ref{s:simtestMG}. In Figure~\ref{f:vol_quench}b2, all RQ MGs had the 8192 atoms in the box. Next, in Figure~\ref{f:vol_quench}b3, all RQ MGs had the same number of atoms and additionally a 2 ns melting step was performed. The initial random mixture of atoms were first equilibrated at 50 K, melted at 2000 K, equlibrated for 2 ns, and then quenched. In the three processes, however, the volume behaviour during cooling is not reproduced correctly for the glasses. Moreover, at 50 K, volume fluctuations---significantly higher than a volume change occuring by a temperature increase of 50 K--are seen as indicated in the insets of Figures~\ref{f:vol_quench}b1-b3. Visually, the glass transition is estimated to be around 600 K, however, a more rigorous estimate has not been attempted for these RQ MGs in this section. However, the enthalpy evolution with temperature during cooling for the \gls{rq} \gls{mg}s prepared at varying quench rates, described in Figure~\ref{f:enth_quench} in the \nameref{c:supple}, and also briefly in Chapter~\ref{c:cbmg} follows the expected trends \cite{Berthier2016,Ediger1996}. \par

\begin{figure}%[!h]
	\centering
	\begin{subfigure}{0.5\textwidth}
		\includegraphics[width=\textwidth]{64-36/post_8000/volume_8000}
		\caption{}
	\end{subfigure}%
	\begin{subfigure}{0.5\textwidth}
		\includegraphics[width=\textwidth]{64-36/post_8000m/volume_8000}
		\caption{}
	\end{subfigure}
	\mycaption{Volume evolution vs quench rate in \czsix RQ MGs}{Volume evolution during cooling of RQ MGs quenched at rates between \qr{12}-\qr{14} with (a) 8192 atoms melted directly at 2000 K, and (b) 8192 atoms melted with a temperature ramping from 50 K to 2000 K before quenching.}
	\label{f:vol_quench64}
\end{figure}

To cross-verify the observations made from Figure~\ref{f:vol_quench}b, the volume vs temperature behaviour was also studied as a function of quench rates in \czsix RQ MGs, which is close to the Cu$_{64.5}$Zr$_{35.5}$ composition validated by the developers of the Cu-Zr glass potential \cite{Mendelev2019}. Starting with a box of 8192 atoms in the box, the glass quenching was performed with the atoms equilibrated at an inital temperature of 2000 K (Figure~\ref{f:vol_quench64}a), and also in a second case where the atoms were first set to 50 K temperature, melted at 2000 K, and then quenched to 50 K (Figure~\ref{f:vol_quench64}b). For both treatments, the volume of the RQ MGs presented lots of fluctuations during cooling. For the case of direct quenching from 2000 K, the expected trend of enhanced packing with lower quench rates is not seen below 75 K. The desired effects were only observed in the \czsix RQ MGs, which were first melted from 50 K to 2000 K before quenching. In both the cases, however, the final average volumes of the glasses fluctuate significantly at 50 K, in comparison to thermal effects as seen in Figures~\ref{f:vol_quench}b1-b3. \par 

Based on the above simulations, it was inferred that while the influence of compositional changes in the RQ MG volume behaviour is easily observable, the effects of quenching rate on the glass volumes are not well reproduced by the \gls{eam} potential used. The effects of alternative but older \gls{eam} potentials \cite{Cheng2008,Mendelev2009} has not been attempted in this thesis.

%\section{Cluster Synthesis}

\begin{selfcite}
In the present work, the structure of CAMGs with a specific size of the clusters as a function of impact energy is studied using MD simulations. The structure of CAMGs will be compared to MGs prepared by RQ. The CAMGs structure will also be compared to NGs prepared by compaction process using the same original clusters as for the simulation of the CAMGs. \par
\end{selfcite}

The clusters generated in the CIBD experiments \cite{Benel2018,Benel2019} were generated via \gls{igc} (see Section \ref{c:theory} for more details). This was simulated earlier for a Kob-Anderson model to simulate "PVD-Nanoglasses" \cite{Danilov2016}. For a more expensive simulation such as with the EAM, the procedure was to be optimized. Pressure, i.e., growth rate of gases, algorithm to periodically delete straying atoms, didn't work. Furthermore, when adding in new atoms, their chemical potential effects would affect the thermodynamics of this otherwise closed system. This complicated matters. Additional modelling and testing of the \gls{igc} simulation proved to be detrimental to the timeline of the thesis. As a workaround, it was chosen to pursue the alternative method of deriving clusters from the bulk of a simulated RQ MG. \par

\begin{selfcite}
Consequently, in a first step, a free-standing cluster was prepared by cutting a sphere of 3 nm diameter (with ~800 atoms) from a \qr{10} \cz-MG held at 50 K temperature. As reported earlier by Adjaoud and Albe \cite{Adjaoud2016}, any cluster develops surface stresses immediately after cutting. The slow kinetics at 50 K prevent the atoms from relaxing to their lowest energetic state. Therefore, a short-time increase of the temperature of the cluster, which increases the mobility of the atoms, allows to obtain a configuration similar to a cluster synthesized in a real experiment by \gls{igc}. Thus, the protocol developed in reference \cite{Adjaoud2016}, viz., heating the cluster shortly to 1000 K, i.e., beyond the glass-transition temperature \gls{tg}, followed by cooling it back to 50 K, was employed. Both the heating and cooling was performed at a rate of 2.5×\qr{12}. Although \gls{tg} is crossed in the simulation, crystallization is avoided (see Section 3.5) due to the short heating time, but sufficient diffusion occurs over the short distances to establish the equilibrium concentration profile in the cluster. In addition, the cluster was equilibrated for 2 ns both after the cutting and after the heat treatment. The heat treatment and its effect on the structure of the cluster is visualized in Figures \ref{f:clus_rad-3nm} and \ref{f:clus_comp-3nm} for a cluster derived from the \qr{10} MG. \par
\end{selfcite}

It was reported in earlier experiments of small CuZr clusters, of 20-30 atoms in size, that the Cu atoms segregated to the surface \cite{Kartouzian2014}. This chemical segregation, also observed in granular matter and dubbed the "Brazil-nut" effect, influences the local chemical homogeneity in the length scales comparable to the 3 nm cluster in the simulations. To verify the nature of the chemical segregation, the Cu and Zr compositions in 0.2 nm thick bands at various radii within the simluated spherical cluster were plotted as a function of the said radii in Figure~\ref{f:clus_rad-3nm}a. The inverse of the square root of the total population $1/\sqrt{N_{band}}$ is also depicted to estimate the error. At lower radii ($\leq$ 8 \r{A}) the population in the band is low, and the error is high. In the intermediate radii ranges of 8 \r{A}$\leq r \geq$ 13 \r{A}, the Cu and Zr compositions fluctuate around 50 at. \%. From 13-15 \r{A}, Cu and Zr compositions are seen to increase and decrease with increasing $r$. This is indicative of the Cu segregation. Beyond 15 \r{A}, the bands extend outside the volume of the shell, capture only a few of the outer atoms of the cluster, which happen to be predominantly Cu as well. For this reason, the error estimate $1/\sqrt{N_{band}}$ increases for $r \geq $ 15 \r{A}. This behaviour seen in Figure~\ref{f:clus_rad-3nm}a was also replicated in Figure~\ref{f:clus_rad-3nm}b, where the bands were chosen to be equipopulated with \textbf{100} atoms, instead of having the same thickness. In this case, the error estimate remains constant. Nevertheless, the presence of a chemical segregation is observed.  \par

\begin{figure}[!ht]
	\centering
	\begin{subfigure}{0.5\textwidth} 	\centering
		\includegraphics[width=\textwidth]{3nm/50-50/post/comp_3nm}
		\label{fig:radial_3nm}
	\end{subfigure}%
	\vfill
	\begin{subfigure}{0.5\columnwidth} 	\centering
		\includegraphics[width=\textwidth]{3nm/50-50/post/comp_3nm (copy)}
		\label{fig:radial_3nm_alt}
	\end{subfigure}%
	\mycaption{3nm \cz  cluster chemical substructure}{ }
	\label{f:clus_rad-3nm}
\end{figure}

\begin{selfcite}
In Figure~\ref{f:clus_comp-3nm} the evolution of the Cu composition in the 0.2 nm thick shell at a radius of 1.3 nm is shown. Up to a time t = 2 ns, when the cluster is equilibrated at 50 K, the Cu composition remains constant. The heating and cooling spike of the cluster between t = 2 ns and t = ∼3 ns results in a sharp increase of the Cu-concentration in the outer shell compared to the bulk composition, eventually leveling off at about 56 at. \%. In the remaining core volume, the Cu concentration decreases to 44 at. \%. While the overall composition of the 3 nm cluster remains unchanged, two distinct regions are seen in the equilibrated cluster–a core region with a lower Cu concentration, and a shell region with a substantially higher Cu concentration. The inset in Figure \ref{f:clus_comp-3nm}a depicts the core (colored magenta) and shell (colored yellow) regions of the cluster. As in other reports, Cu-atoms segregate towards the cluster surface—increasing the Cu concentration by 9 at. \% as compared to the initial homogeneous composition, while the Zr-atoms are enriched in the core. This compositional variation on the length scale of the cluster size is carried over to the interfacial regions between clusters upon compaction or energetic impact. From previous studies it is known that such chemical heterogeneities in compacted NGs on the nanometer length scale stabilize the amorphous structure \cite{Adjaoud2016}. \par

\begin{figure}[!ht]
	\begin{subfigure}{0.5\columnwidth} 	\centering
		\includegraphics[width=\textwidth]{3nm/50-50/post/comp-cu_3nm.png}
		\label{fig:clus_cu_diff}
	\end{subfigure}% 
	\hfill
	\begin{subfigure}{0.5\columnwidth} 	\centering
		\includegraphics[width=\textwidth]{3nm/50-50/post/comp-zr_3nm.png}
		\label{fig:clus_zr_diff}
	\end{subfigure}% 
	\mycaption{3nm \cz  cluster core-shell composition evolution}{ %(a) shows the radial variation of composition in the 3nm cluster and the clear presence of a shell region from 14 \r{A} radius. 
		Copper atoms diffusing out of a 13 \r{A} shell (the core-shell structure is depicted in the inset) of 2\r{A} thickness. The Cu composition decreases in the core and correspondingly increases in the shell, as the cluster is heated up to and beyond $T_{g}$.}
	%In (d), we see the average potential energy per atom increases near the surface of the cluster.}
	\label{f:clus_comp-3nm}
\end{figure}
\end{selfcite}
\section{Cluster Synthesis} \label{s:clus}

\begin{changebar}
In the present work, the structure of CAMGs with a specific size of the clusters as a function of impact energy is studied using MD simulations, with the intention of comparing the local atomic structure of CAMGs with that of the MGs prepared by RQ. The atomic structure of the CAMGs will also be compared to an equivalent NG prepared by mechanical compaction of clusters as opposed to cluster deposition. \par
\end{changebar}

The clusters generated in the CIBD experiments \cite{Benel2018,Benel2019} were generated via \gls{igc} (see Chapter \ref{c:theory} for more details). The cluster growth was simulated earlier to simulate \gls{ng}s using a Kob-Anderson model in reference \cite{Danilov2016}, by what the authors termed as \gls{pvd} of the nanoparticles. To replicate this method with the more expensive \gls{eam} potentials, the procedure had to be optimized for computational time. An algorithm to periodically condense gaseous Cu and Zr atoms and to form a cluster was designed. Typically, inert gas atoms have also been modelled in previous \gls{igc} simulations \cite{Krasnochtchekov2003,Krasnochtchekov2005}, but this present model does not take it into consideration. Instead, the gaseous Cu and Zr are periodically cooled before aggregation into the cluster. The temperature of the cluster and the gaseous atoms are controlled using a thermostat (NPT in LAMMPS). To reduce computational time, regular checks were made during the condensation process to delete atoms straying away from the cluster. The growth rate of the cluster was chosen arbitrarily, and the chemical potential effects were not accounted for. Upon testing, the simulation protocol was found to not be stable. As a workaround to additional modelling and testing of the \gls{igc} simulation, an alternative method of deriving clusters from the bulk of a simulated RQ MG was chosen. \par

\begin{figure}[t]
	\centering
	\begin{subfigure}{0.5\textwidth} 	\centering
		\includegraphics[width=\textwidth]{3nm/50-50/post/comp_3nm} \caption{}
		\label{f:radial_3nm}
	\end{subfigure}%
	%\vfill
	\begin{subfigure}{0.5\columnwidth} 	\centering
		\includegraphics[width=\textwidth]{3nm/50-50/post/comp_3nm (copy)} \caption{}
		\label{f:radial_3nm_alt}
	\end{subfigure}%
	\mycaption{3 nm \cz  cluster chemical substructure}{A radial compositional analysis with concentric bands chosen with (a) 0.2 nm thickness and (b) fixed population of $\sim$50 atoms, reveals Cu segregration to the surface, which form the basis to define a core-shell structure.}
	\label{f:clus_rad-3nm}
\end{figure}

\begin{changebar}
Consequently, in a first step, a free-standing cluster was prepared by cutting a sphere of 3 nm diameter (with ~800 atoms) from a \qr{10} \cz-MG\footnote{The \qr{10} quench rate is the conventional value used in literature \cite{Ritter2011,Adjaoud2016,Adjaoud2018}. Apart from Sections~\ref{s:simtestMG}, \ref{s:camg_quenchrt}, and \ref{s:mgsquench}, the all the simulated glassy systems in this dissertation are made from \qr{10} \gls{rq} \gls{mg}s.} held at 50 K temperature. The resulting cluster was found to be approximately at a \cz composition (with a ~1\% deviation). This compositional deviation is not significant to the following studies on glass \gls{sro}, as evident from previous \gls{md} studies \cite{Peng2010}. As reported earlier by \textcite{Adjaoud2016}, any cluster develops surface stresses  immediately after cutting. The slow kinetics at 50 K prevent the atoms from relaxing to their lowest energetic state. Therefore, a short-time increase of the temperature of the cluster, which increases the mobility of the atoms, allows to obtain a configuration similar to a cluster synthesised in a real experiment by \gls{igc}. Thus, the protocol developed in reference \cite{Adjaoud2016}, viz., heating the cluster shortly to 1000 K, i.e., beyond the glass-transition temperature \gls{tg}, followed by cooling it back to 50 K, was employed. Both the heating and cooling were performed at a rate of 2.5 $\times$ \qr{12}. Although \gls{tg} is crossed in the simulation, crystallisation is avoided (see Section~\ref{s:vorocamg}) due to the short heating time; sufficient diffusion occurs over the short distances to establish a concentration profile in the cluster. In addition, the cluster was equilibrated for 2 \gls{ns} both after the cutting and after the heat treatment. The heat treatment on the cluster and its effect on the structure of the cluster is visualised in Figures \ref{f:clus_rad-3nm} and \ref{f:clus_comp-3nm} for a cluster derived from the \qr{10} MG, and are discussed below. \par
\end{changebar}

It was reported in earlier experiments of small CuZr clusters, of 20-30 atoms in size, that the Cu atoms segregated to the surface \cite{Kartouzian2014}. This chemical segregation, also observed in granular matter and dubbed the ``Brazil-nut" effect \cite{Rosato1987}, influences the local chemical homogeneity in the length scales comparable to the 3 nm cluster in the simulations. This surface segregation is driven by the different surface energies of the Cu and Zr atoms, and the negativity of their heat of mixing \cite{Wang2016,Adjaoud2016}. To verify the radial variation of the chemical segregation in the cluster, the Cu and Zr compositions in 0.2 nm thick bands at various radii within the simulated spherical cluster were plotted as a function of the said radii in Figure~\ref{f:clus_rad-3nm}a. \par 

\begin{changebar}
	\begin{figure}%[!h]
		\centering
		\includegraphics[width=\textwidth,trim={0 1cm 0 1cm},clip]{camg_3nm/1.pdf}
		\mycaption{3 nm \cz cluster core-shell composition evolution}{Copper atoms diffusing out of a 13 \r{A} shell (the core-shell structure is depicted in the inset) of 2\r{A} thickness. The Cu composition decreases in the core and correspondingly increases in the shell, as the cluster is heated up to and beyond \gls{tg}.}
		\label{f:clus_comp-3nm}
	\end{figure}
\end{changebar}

The inverse of the square root of the total population $1/\sqrt{N_{band}}$ is also depicted to estimate the error. At lower radii ($\leq$ 8 \r{A}) the population in the band is low, and the error is high. In the intermediate radii ranges of 8 \r{A}$\leq r \leq$ 13 \r{A}, the Cu and Zr compositions fluctuate around 50 at. \%. From 13-15 \r{A}, Cu and Zr compositions are seen to increase and decrease with increasing band radius. This is indicative of the Cu segregation. Beyond 15 \r{A}, the bands extend outside the volume of the shell, capture only a few of the outer atoms of the cluster, which happen to be predominantly Cu as well. For this reason, the error estimate $1/\sqrt{N_{band}}$ increases for $r \geq $ 15 \r{A}. This behaviour seen in Figure~\ref{f:clus_rad-3nm}a was also replicated in Figure~\ref{f:clus_rad-3nm}b, where the bands were chosen to be equi-populated with $\sim$50 atoms, instead of having the same band thickness. In this case, the error estimate remains constant. Nevertheless, the presence of a chemical segregation is observed. \par

%\newpage

\begin{changebar}
In Figure~\ref{f:clus_comp-3nm} the evolution of the Cu composition in the 0.2 nm thick shell at a radius of 1.3 nm is shown. Up to a time t = 2 ns, when the cluster is equilibrated at 50 K, the Cu composition remains constant. The heating and cooling spike of the cluster between t = 2 ns and t=$\sim$3 ns results in a sharp increase of the Cu-concentration in the outer shell compared to the bulk composition, eventually leveling off at about 56.48 at. \%. In the remaining core volume, the Cu concentration decreases to 44.59 at. \%. While the overall composition of the 3 nm cluster remains unchanged, two distinct regions are seen in the equilibrated cluster---a core region with a lower Cu concentration, and a shell region with a substantially higher Cu concentration. The inset in Figure \ref{f:clus_comp-3nm}a depicts the core (colored magenta) and shell (colored yellow) regions of the cluster. As in other reports, Cu-atoms segregate towards the cluster surface---increasing the Cu concentration by 9 at. \% as compared to the initial homogeneous composition. \end{changebar} In Figure~\ref{f:clus_comp-3nm}b, the Zr-concentration in the core and shell are plotted as a function of time. A drop in the Zr-concentration in the shell is seen, which is concurrent with the Cu-enrichment. The Cu-atoms are enriched in the shell, while the Zr-atoms are enriched in the core. This elemental segregation was also confirmed by subjecting the cluster multiple heat-spikes (see Figure~\ref{f:heatspike-100} in \nameref{c:supple}). 100 additional heat-spikes were given to the \cz cluster, and Cu-enrichment in the shell was evaluated over the course of the heat-treatments. In contrast to the average Cu-concentration of 57\% in the first initial heat-treatment, the Cu concentration oscillated around 59\% through the subsequent 100 heat-spikes. Even with additional heat-treatment, the average increase is only 2\%, compared to an initial increase of 7\%. \begin{changebar} This compositional variation on the length scale of the cluster size is carried over to the interfacial regions between clusters upon compaction or energetic impact. From previous studies it is known that such chemical heterogeneities in compacted NGs on the nanometre length scale stabilise the amorphous structure \cite{Adjaoud2016}. \par
\end{changebar}

Apart from heating the cluster shortly above the \gls{tg}, it was attempted within this doctoral work to find an equilibrium cluster structure using \gls{mc} techniques on \gls{lmp}, to perform MC swaps with the Metropolis criterion. However, for the cluster size of 800 atoms, the MC simulation did not converge. Another alternative is to perform MC using the \texttt{vcsgc} package on LAMMPS, which is specifically designed for atomistic precipitation in alloys \cite{Sadigh2012}. However, this package is unfortunately optimised for multi-million atom simulations and maybe useful to simulate chemical segregation in large nanoparticles. \par

\section{Modelling Cluster-assembled Metallic Glasses} \label{s:camgdev}
%\subsection{Single Cluster Deposition}
\begin{selfcite}
With the cluster prepared, the next course of study towards understanding \gls{camg}s is the simulation of deposition of single clusters on a surface, as depicted in Figure~\ref{f:cibdsmod}a. These simulation conditions were performed to represent closely the experimental conditions in the cluster ion beam deposition (CIBD) experiments, in terms of cluster size and range of impact energies [10,11]. In the CIBD experiments, CuZr clusters are generated as charged cluster-ions and then guided as a particle-beam towards the substrate using an electric field. The strength of the said electric field determines the impact energy of the cluster ions onto the substrate.  In the present simulation, a classical momentum was given to the cluster to mimic the cluster-acceleration in the experiments, when they pass through the electric field. Furthermore, in the experimental CIBD set-up, the substrate is electrically grounded to prevent any charge buildup on the surface [24]. Therefore, the deposition process can be modelled with classical molecular dynamics without taking electrodynamics into account. \par

\begin{figure}[!ht] \centering
	\begin{subfigure}{0.45\textwidth}
		\includegraphics[width=\textwidth,trim={0 0 4cm 0.5cm},clip]{cibd.png}
		\caption{}
		\label{fig:single}
	\end{subfigure}%
	%	\hfill
	\hspace{1cm}
	%	\hspace{-4cm}
	\begin{subfigure}{0.45\textwidth}
		\includegraphics[width=\textwidth,trim={4cm 0 0 0.5cm},clip]{cibd-xsec-slice.png}
		\caption{}
		\label{f:single_xsec}
	\end{subfigure}
	\mycaption{Substrate model for deposition of 3 nm cluster}{The cluster is deposited onto a substrate, with a given energy as shown in \ref{fig:single}. The cross-sectional view of the film is as in \ref{f:single_xsec}, with a mixing/buffer layer and thermostatted layer. The third fixed layer gives rigidity to the substrate.}
	\label{f:cibdsmod}
\end{figure}

In terms of the thermodynamics, the cluster is modelled as a closed system (micro-canonical ensemble). A simplification, which was made in the MD simulations, is the replacement of the oxidized Si-substrate used in the experiments [10,11] with an amorphous \cz substrate, equilibrated for 2 ns. The crossection of Figure~\ref{f:cibdsmod}a, illustrated in Figure~\ref{f:cibdsmod}b shows a layered thermal model with the following configuration used to represent the substrate: 1. the top layer (modelled as a micro-canonical ensemble) serving as a buffer between the clusters and the substrate, 2. the middle layer being coupled with a heat sink, using a Nosé-Hoover Thermostat to hold the substrate temperature at 50 K, and 3. the bottom layer, with atoms held fixed to mimic the rigidity of the substrate. The buffer and thermostatted layers had a minimum thickness of two-atom layers. It is important to note that all three layers are essential to model the substrate. Without the first layer, the deposited atoms would immediately quench onto the substrate. The second layer accounts for temperature control, the lack of which would have led to a thermally unstable (explosive) substrate caused by its inability to expel sufficient amounts of energy from the system. Furthermore, without the third layer, the substrate would have no mechanical rigidity, and the clusters will simply pass through the substrate at higher energies. The layer model was configured in accordance with previous MD thin film studies [25-27]. For the case of the single cluster depositions, a semi-hemispherical layout was utilized for the thermostatted layer to account for a spherical shockwave that passes through the substrate. For these single cluster depositions, the substrate length and width were chosen to be 6 nm: two times as wide as the cluster diameter. \par

\begin{figure}
	\includegraphics[width=0.5\linewidth]{3nm/post/clus_asph.png}
	\mycaption{Convergence of cluster deposition convergence}{}
	\label{f:cibdsasph}
\end{figure}%

The thickness of the first two layers (buffer and thermostatted layers) can affect the heat absorption and also the hardness of the substrate. Consequently, the dissipation of the energy introduced to the film-substrate system by the cluster deposition is influenced by the specific design of the layers. For the present substrate model, the deposition of a single cluster was inspected at large timescales. Figure~\ref{f:cibdsasph} shows the evolution of asphericity of the single cluster upon deposition. The asphericity of the cluster is defined as the ratio of the radii of the cluster in the deposition direction ($R_{Z}$ in Z-axis) to that the deposition plane ($R_{XY}$ in the XY plane). The undeposited cluster, which is spherical, intially has a $R_{Z}/R_{XY}$ =1, and as the cluster deforms monotonically with the impact energy, the $R_{Z}/R_{XY}$ decreases further. After deposition, the $R_{Z}/R_{XY}$ demonstrates a dip, and eventually the simulation converges, as seen at even 2 ns after the cluster deposition. \par
\end{selfcite}


\subsection{Single Cluster Deposition}
\begin{changebar}
With the cluster prepared, the next course of study towards understanding \gls{camg}s is the simulation of deposition of single clusters on a surface, as depicted in Figure~\ref{f:cibdsmod}a. These simulation conditions were performed to represent closely the experimental conditions in the \gls{cibd} experiments, in terms of cluster size and range of impact energies \cite{Benel2018,Benel2019}. In the \gls{cibd} experiments, CuZr clusters are generated as charged cluster-ions and then guided as a particle-beam towards the substrate using an electric field. The strength of the said electric field determines the impact energy of the cluster ions onto the substrate.  In the present simulation, a classical momentum was given to the cluster to mimic the cluster-acceleration in the experiments, when they pass through the electric field. Furthermore, in the experimental \gls{cibd} set-up, the substrate is electrically grounded to prevent any charge build-up on the surface \cite{Fischer2015}. Therefore, the deposition process can be modelled with classical molecular dynamics without taking electrodynamics into account. \par

\begin{figure}[h]
	\centering
\begin{subfigure}{0.45\textwidth}
	\includegraphics[width=\textwidth,trim={0 0 5cm 0.5cm},clip]{cibd.png}
	\caption{}
\end{subfigure}%
%	\hfill
\hspace{1cm}
%	\hspace{-4cm}
\begin{subfigure}{0.45\textwidth}
	\includegraphics[width=\textwidth,trim={5cm 0 0 0.5cm},clip]{cibd-xsec-slice.png}
	\caption{}
\end{subfigure}
\mycaption{Substrate model for deposition of a 3 nm cluster}{(a) The cluster is deposited onto a substrate, with a given energy. (b) The cross-sectional view of the film is as shown, with a mixing/buffer layer and thermostatted layer. The third fixed layer gives rigidity to the substrate.}
\label{f:cibdsmod}
\end{figure} 

In terms of the thermodynamics, the cluster is modelled as a closed system (micro-canonical ensemble). A simplification, which was made in the MD simulations, is the replacement of the oxidised Si-substrate used in the experiments \cite{Benel2018,Benel2019} with an amorphous \cz  substrate, equilibrated for 2 ns. The cross-section of Figure~\ref{f:cibdsmod}a, illustrated in Figure~\ref{f:cibdsmod}b, shows a layered thermal model with the following configuration used to represent the substrate: 1. the top layer (modelled as a micro-canonical ensemble) serving as a buffer between the clusters and the substrate, 2. the middle layer being coupled with a heat sink, using a Nosé-Hoover thermostat to hold the substrate temperature at 50 K, and 3. the bottom layer, with atoms held fixed to mimic the rigidity of the substrate. The buffer and thermostatted layers had a minimum thickness of two-atom layers. \par 

It is important to note that all three layers are essential to model the substrate. Without the first layer, the deposited atoms would immediately quench onto the substrate. The second layer accounts for temperature control, the lack of which would have led to a thermally unstable (explosive) substrate caused by its inability to expel sufficient amounts of energy from the system. Furthermore, without the third layer, the substrate would have no mechanical rigidity, and the clusters will simply pass through the substrate at higher energies. The layer model was configured in accordance with previous \gls{md} thin film studies \cite{Haberland1993,Haberland1995,Rahmati2020}. For the case of single cluster deposition, a semi-hemispherical layout was utilised for the thermostatted layer to account for a spherical shockwave that passes through the substrate. For these single cluster depositions, the substrate length and width were chosen to be 6 nm: two times as wide as the cluster diameter. \par

\begin{figure}[!h]
\centering
\includegraphics[width=0.45\linewidth]{3nm/post/clus_asph.png}
\mycaption{Convergence of single cluster deposition simulation}{The asphericity of the cluster serves as a metric to evaluate that 2 ns after deposition, a single deposited cluster deforms no further (See text for more details)}
\label{f:cibdsasph}
\end{figure}%

The thickness of the first two layers (buffer and thermostatted layers) can affect the heat absorption and also the hardness of the substrate. Consequently, the dissipation of the energy introduced to the film-substrate system by the cluster deposition is influenced by the specific design of the layers.  For the present substrate model, the deposition of a single cluster was inspected at large timescales. \end{changebar} Figure~\ref{f:cibdsasph} shows the evolution of asphericity of the single cluster upon deposition. The asphericity of the cluster is defined as the ratio of the radii of the cluster in the deposition direction ($R_{Z}$ in Z-axis) to that the deposition plane ($R_{XY}$ in the XY plane). The undeposited cluster, which is spherical, intially has a $R_{Z}/R_{XY}$ =1, and as the cluster deforms monotonically with the impact energy, the $R_{Z}/R_{XY}$ decreases further. After deposition, the $R_{Z}/R_{XY}$ demonstrates a dip, and eventually the simulation converges, as seen at even 2 ns after the cluster deposition. \par


%\subsection{Multiple Cluster Deposition}

\begin{selfcite}
	\begin{figure}[!ht]
	\begin{subfigure}{\textwidth}
		\includegraphics[width=\textwidth,trim={0cm 1.5cm 0cm 4cm},clip]{subs_multi.png}
	\end{subfigure}%
	\vfill
	\begin{subfigure}{\textwidth}
		\includegraphics[width=\textwidth,trim={0cm 1.8cm 0cm 2.5cm},clip]{subs_multi_xsec.png}
	\end{subfigure}
	\mycaption{Substrate thermal model for multiple cluster deposition}{Similar to Figure \ref{f:cibdsmod}, this figure shows the substrate model for the multiple cluster deposition. The substrate is divided into three layers, one more buffer, one as a thermostat to provide temperature to the cluster, an the third fixed layer to provide rigidity. These layers are flat, in comparison to Figure \ref{f:cibdsmod}.}
	% it does not make sense to have individual spherical layers at each deposition site.}
	\label{f:cibdmmod}
	\end{figure}

Following the simulations of deposition of single clusters, the deposition of multiple clusters to form CAMG films was modelled. A large Cu50Zr50 substrate of dimensions 25 nm × 25 nm × 3 nm, consisting of about ∼75,000 atoms was chosen. As described in the previous section, the substrate model is tri-layered with flat substrate layers. Both the buffer and thermostatted-layer are set to an initial temperature of 50 K. As an initial test, single cluster depositions on this larger substrate were also found to relax after 2 ns (see Figure S4a in Supplementary Information). Therefore, each cluster is allowed to relax for 2 ns after deposition before another cluster is deposited next or on top to it. In the CAMG experiment, the clusters are polydisperse in nature, with a Gaussian size distribution, while in the present simulations, however, each cluster is chosen to be of the same size. In addition, each cluster is allowed to rotate by three random Euler angles before deposition to ensure a random configuration in the CAMG film samples. The simulation was performed with periodic boundary conditions in the XY plane. \par
\end{selfcite}

In the \gls{cibd} experiments, the electric field which finally directs the beam of clusters onto the substrate is swept across the substrate surface to ensure uniform particle coverage. Deposition of multiple clusters at a given time complicates the coding aspects on LAMMPS. To replicate the experimental conditions, however, it was first attempted to sequentially deposit clusters at random locations on the substrate. First, the deposition of a single cluster onto the substrate at 60 meV/atom was studied. As seen in Figure~\ref{f:randalgo}, the simulation was found to converge, when inspecting the average potential energy 2 ns after the deposition. Hence, 2 ns is determined as the wait time between each sequential deposition. Such a simulation of clusters being sequentially deposited turned out to be computationally expensive, costing \sim \textbf{hours} to deposit $\sim$ 50 clusters. Hence, an algorithm was developed to shorten simulation times. It described in the following steps:

\begin{enumerate}
	\item A neighbourhood of a cluster is defined as the region enclosed within a 2 cluster diameter lengths
	\item Once a cluster is deposited, its neighbourhood is noted
	\item If a cluster has not been equilibrated for at least 2 ns, no depositions are allowed in its neighbourhood
	\item Therfore, f an i+1$^{th}$ cluster is determined to fall in the neighbourhood of any of the $i$ clusters, attempts are made to deposit the i+1$^{th}$ cluster elsewhere
	\item If no such region exists on the substrate, then the already deposited film is equilibrated for 2 ns, and the i+1$^{th}$ cluster is allowed to land.
\end{enumerate}

The schematic of the deposition algorithm is illustrated in Figure~\ref{f:randalgo}\textbf{b??}. This method greatly cut down the simulation time as the number of long equilibration steps would be reduced; and this method scales inversely with larger substrates, as the probability of a cluster landing in the neighbourhood of an unequilibrated cluster is greatly reduced.

\begin{figure}[!ht] 
	\centering
	\begin{subfigure}{\textwidth} \centering \includegraphics[width=0.7\textwidth]{pe-system_3nm.png}
	\end{subfigure}%
	\vfill
	\begin{subfigure}{0.33\textwidth} \includegraphics[height=0.15\textheight]{grid1} \end{subfigure}%
%	\hfill
	\begin{subfigure}{0.33\textwidth} \includegraphics[height=0.15\textheight]{grid2} \end{subfigure}%
%	\hfill
	\begin{subfigure}{0.33\textwidth} \includegraphics[height=0.15\textheight]{grid3} \end{subfigure}%
	\vfill
	\begin{subfigure}{0.33\textwidth} \includegraphics[height=0.15\textheight]{grid4} \end{subfigure}%
%	\hfill
	\begin{subfigure}{0.33\textwidth} \includegraphics[height=0.15\textheight]{grid5} \end{subfigure}%
%	\hfill
	\begin{subfigure}{0.33\textwidth} \includegraphics[height=0.15\textheight]{grid6} \end{subfigure}%
	\vfill
	\begin{subfigure}{0.33\textwidth} \includegraphics[height=0.15\textheight]{grid7} \end{subfigure}%
%	\hfill
	\begin{subfigure}{0.33\textwidth} \includegraphics[height=0.15\textheight]{grid8} \end{subfigure}%
%	\hfill
	\begin{subfigure}{0.33\textwidth} \includegraphics[height=0.15\textheight]{grid9} \end{subfigure}%
	\mycaption{Schematic of Algorithm for random deposition of clusters}{ }
	\label{f:randalgo}
\end{figure}

\begin{selfcite}
This deposition of the clusters on the substrate at random locations in the XY plane, is visualized in Figure~\ref{f:random_multi}b. The random deposition of clusters, resulted in the formation of pillars, thus shadowing certain regions of the film and leading to porous films. The growth of the film bore resemblance to previous statistical studies on ballistic deposition [28]. Although such behavior is very likely to occur in the experiments, here a model was needed to maximize inter-cluster interactions and to reduce the level of porosity. In order to achieve reducing the porosity, the clusters were deposited in a hexagonal close-packed (HCP) arrangement onto the substrate.

\begin{figure}[!ht]
	\centering
	\begin{subfigure}{0.50\textwidth} \includegraphics[width=\columnwidth]{pe-system_3nm.png}
	\subcaption{} \end{subfigure}%
	\hfill
	\begin{subfigure}{0.50\textwidth} \includegraphics[width=0.9\textwidth]{film_long}
	\subcaption{} \end{subfigure}%
	\label{f:random_multi}
	\mycaption{Multiple Cluster Deposition with a random deposition algorithm}{Pores etc}
\end{figure}

Figure~\ref{f:hcpalgo}a shows the schematic of the alogrithm employed to make the HCP depositions. The HCP patterned film is first considered to be made up of one layer of clusters on the XZ plane. This layer is divided into four sublayers as depicted in Figure~\ref{f:hcpalgo}b.: the red and pink clusters of types 1 and 2 each. The alorithm is as follwos:

\begin{enumerate}
	\item Deposit Red clusters of type 1
	\item Equilibrate for 2 ns
	\item Deposit Red clusters of type 2
	\item Equilibrate for 2 ns
	\item Repeat steps 1-4 for the Pink clusters
\end{enumerate}

This sequential deposition of clusters by the order of the sublayer they belong to ensures that the newly deposited clusters are not in the neighbourhood of unequilibrated clusters. The algorithm allows for a nearly parallel deposition, the only latency between each deposition being the time taken for the cluster to reach the surface of the film. On the 24 nm x 24nm XT plane of the substrate, 52 clusters can be arranged in an HCP pattern, meaning that the deposition of the four sublayers can be done with four equilbration steps, instead of a 52 times as would be the case in a completely sequential deposition. This speeds up the deposition compared to single deposition algorithm process by a factor of 13 for the given substrate and cluster combination. Like with the random deposition algorthim, this algorithm also scales better with the increase in XY dimensions of the film. \par

\begin{figure}[!ht] 
	\centering
	\begin{subfigure}{0.50\textwidth}
		\includegraphics[width=\textwidth]{hcp_alg}
		\subcaption{}
		\label{fig:hcp_dep_algo}
	\end{subfigure}%
	\vfill
	%\begin{figure}[!ht]
	\begin{subfigure}{0.48\textwidth}
		\includegraphics[width=\textwidth]{cibd_hcp}
		\subcaption{}
		\label{fig:hcp_top}
	\end{subfigure}
	\mycaption{Patterned deposition of multiple clusters}{ (\ref{fig:single_dep_long}) The potential energy of the cluster stablizes with time and converges to its minimum value, the single cluster after deposition was found to relax after 2 million timesteps (2 ns) of equilibration. } %\ref{fig:hcp_dep_algo} HCP patterned deposition. This algorithm (see Figure \ref{fig:hcp_dep_algo}) also allows for parallel deposition, speeding up the deposition compared to single deposition algorithm process by a factor of 13 , and this scales better as the XY dimensions of the film are increased. 
	%	\ref{fig:hcp_top} Top view of one layer of deposited film atoms in a HCP, pattern colored coded by their height in Z-axis}
	\label{f:hcpalgo}
\end{figure}
\end{selfcite}


\subsection{Multiple Cluster Deposition}
\begin{changebar}
Following the simulations of deposition of single clusters, the deposition of multiple clusters to form \gls{camg} films was modelled. A large \cz  substrate of dimensions 25 nm $\times$ 25 nm $\times$ 3 nm, consisting of about $\sim$75,000 atoms was chosen. As described in the previous section, the substrate model is tri-layered with flat substrate layers (see Figure~\ref{f:cibdmmod}). Both the buffer and thermostatted-layer are set to an initial temperature of 50 K. % As an initial test, single cluster depositions on this larger substrate were also found to relax after 2 ns (see Figure S4a in Supplementary Information). %Therefore, each cluster is allowed to relax for 2 ns after deposition before another cluster is deposited next or on top to it.
In the \gls{camg} experiment \cite{Benel2019}, the clusters are polydisperse in nature, with a log-normal size distribution, while in the present simulations, however, each cluster is chosen to be of the same size. In addition, each cluster is allowed to rotate by three random Euler angles before deposition to ensure a random configuration in the \gls{camg} film samples. The simulation was performed with periodic boundary conditions in the XY plane. \par

\begin{figure}[h]
%	\begin{subfigure}{\textwidth}
%		\includegraphics[width=\textwidth,trim={0cm 2.5cm 0cm 4.5cm},clip]{subs_multi.png}
%	\end{subfigure}%
%	\vfill
	\begin{subfigure}{\textwidth}
		\includegraphics[width=\textwidth,trim={0cm 1.8cm 0cm 2.5cm},clip]{subs_multi_xsec.png}
	\end{subfigure}
	\mycaption{Substrate thermal model for multiple cluster deposition}{Similar to Figure \ref{f:cibdsmod}, this figure shows the substrate model for the multiple cluster deposition. The substrate is divided into three layers, in this thermal model.}
	% it does not make sense to have individual spherical layers at each deposition site.}
	\label{f:cibdmmod}
\end{figure}

\begin{figure}[h] \centering \includegraphics[width=0.7\textwidth]{pe-system_3nm.png}
	\mycaption{Convergence of deposition on the 25 nm $\times$ 25 nm substrate}{The \gls{pe} of the cluster stabilises with time and converges to its minimum value, the single cluster after deposition was found to relax after 2 million timesteps (2 ns) of equilibration.}
	\label{f:multiconverge}
\end{figure}%
\end{changebar}

In the \gls{cibd} experiments, the electric field, which finally directs the beam of clusters onto the substrate is swept across the substrate surface to ensure uniform particle coverage. To replicate the experimental conditions, however, it was first attempted to sequentially deposit clusters at random locations on the substrate. First, the deposition of a single cluster onto the substrate at 60 meV/atom was studied. As seen in Figure~\ref{f:multiconverge}, the simulation was found to converge, when inspecting the average \gls{pe} 2 ns after the deposition. Hence, 2 ns is determined as the relaxation time between each sequential deposition. \par Such a simulation of clusters being sequentially deposited turned out to be computationally expensive, costing $\sim$30,000 CPU hours to deposit a layer of 50 clusters. Hence, an algorithm was developed to shorten simulation times. The schematic of the deposition algorithm is illustrated in Figure~\ref{f:randalgo} and it is described in the following steps:

\begin{enumerate}[noitemsep]
	\item A neighbourhood of a cluster is defined as the region enclosed within two-cluster-diameter-lengths.
	\item Once a cluster is deposited, its neighbourhood is noted.
	\item If a cluster has not been equilibrated for at least 2 ns, no depositions are allowed in its neighbourhood.
%	\item Therefore, if an i+1$^{th}$ cluster is determined to fall in the neighbourhood of any of the $i$ clusters that is  unequilibrated, attempts are made to deposit the i+1$^{th}$ cluster elsewhere
	\item If no deposit-able region exist on the substrate, then the already deposited film is equilibrated for 2 ns, and the next cluster is allowed to land.
\end{enumerate}

\begin{figure}[!h]
	\centering
	\begin{subfigure}{0.6\textwidth} \centering
		\begin{subfigure}{0.33\textwidth} \includegraphics[width=0.9\textwidth]{grid1} \caption{} \end{subfigure}%
		%	\hfill
		\begin{subfigure}{0.33\textwidth} \includegraphics[width=0.9\textwidth]{grid2} \caption{} \end{subfigure}%
		%	\hfill
		\begin{subfigure}{0.33\textwidth} \includegraphics[width=0.9\textwidth]{grid3} \caption{} \end{subfigure}%
		\vfill
		\begin{subfigure}{0.33\textwidth} \includegraphics[width=0.9\textwidth]{grid4} \caption{} \end{subfigure}%
		%	\hfill
		\begin{subfigure}{0.33\textwidth} \includegraphics[width=0.9\textwidth]{grid5} \caption{} \end{subfigure}%
		%	\hfill
		\begin{subfigure}{0.33\textwidth} \includegraphics[width=0.9\textwidth]{grid6} \caption{} \end{subfigure}%
		\vfill
		\begin{subfigure}{0.33\textwidth} \includegraphics[width=0.9\textwidth]{grid7} \caption{} \end{subfigure}%
		%	\hfill
		\begin{subfigure}{0.33\textwidth} \includegraphics[width=0.9\textwidth]{grid8} \caption{} \end{subfigure}%
		%	\hfill
		\begin{subfigure}{0.33\textwidth} \includegraphics[width=0.9\textwidth]{grid9} \caption{} \end{subfigure}%
	\end{subfigure}
	\mycaption{Schematic of the algorithm for random deposition of clusters}{(a) Substrate depicted as 64 \gls{aru} grids with \gls{pbc} (b). An unequilibrated cluster (c) and its neighbourhood (d) block an area (e) of 9 \gls{aru}, resulting in a deposition in a random position (f). The new neighbourhood (g) and blocked area (h) are shown. When the cluster is equilibrated on the substrate, a new cluster (i) can be deposited in its neighbourhood.}
	\label{f:randalgo}
\end{figure}

As seen in Figure~\ref{f:randalgo}, if the substrate is divided into chunks of a cluster-diameter length, it results in an 8$\times$8 grid i.e. 64 \gls{aru}. Then, the undepositable area around an unequilibrated cluster is 9 (3$\times$3) \gls{aru}. When the first cluster is deposited, the probability of the deposition algorithm finding a depositable-area is high (p=55/64). This probability value gradually decreases as new clusters randomly land across the substrate plane. Neverthless, the algorithm reduces the number of equilibration steps, and consequently the simulation time. Further, this method scales inversely with larger substrates, as the probability of a cluster landing in the neighbourhood of an unequilibrated cluster is greatly reduced. \par

\begin{changebar}
\begin{figure}[!h]
	\centering
	\includegraphics[width=0.53\textwidth]{film_long}
	\mycaption{Pores in randomly deposited cluster films}{Multiple cluster deposition with the random deposition algorithm results in porous CAMG films.}
	\label{f:random_multi}
\end{figure}

%\begin{changebar}
	\begin{figure}[!h] 
		\centering
		\begin{subfigure}{0.5\textwidth} \centering
			\includegraphics[width=\textwidth]{hcp_alg}
			\subcaption{}
			\label{f:hcp_dep_algo}
		\end{subfigure}%
		\hfill
		%\begin{figure}[!ht]
		\begin{subfigure}{0.5\textwidth} \centering
			\includegraphics[width=\textwidth]{cibd_hcp}
			\subcaption{}
			\label{f:hcp_top}
		\end{subfigure}
		\mycaption{Patterned deposition of multiple clusters}{(a) HCP deposition algorithm; the red (1 \& 2) and pink (1 \& 2) atoms are deposited separately. The clear and striped circles represent the clusters in the next layer. (b) Top view of one layer of deposited film atoms in the HCP pattern, colored coded by their height in Z-axis}
		\label{f:hcpalgo}
	\end{figure}
%\end{changebar}

Such a deposition of the clusters on the substrate at random locations in the XY plane, is visualised in Figure~\ref{f:random_multi}. The random deposition of clusters, resulted in the formation of pillars, thus shadowing certain regions of the film and leading to porous films. The growth of the porous film bore resemblance to previous statistical studies on ballistic deposition \cite{Meakin1986}. Although such pores may occur in the experiments, here a model was used to maximise inter-cluster interactions and to avoid formation of pores. An absence of surface effects from the pore formation and an increase of cluster-cluster interfaces is desirable to study the effects of interfaces in CAMGs. In order to achieve reducing the porosity, the clusters were deposited in a \gls{hcp} arrangement onto the substrate. \par
\end{changebar}

Figure~\ref{f:hcpalgo}a shows the schematic of the algorithm employed to make the HCP depositions. The HCP patterned film is first considered to be made up of one layer of clusters on the XZ plane. This layer is divided into four sublayers as depicted in Figure~\ref{f:hcpalgo}a: the red and pink clusters of types 1 and 2 each. Then the following algorithm is implemented:

\begin{enumerate}[noitemsep]
	\item Deposit Red clusters of type 1. Equilibrate for 2 ns.
	\item Deposit Red clusters of type 2. Equilibrate for 2 ns.
	\item Repeat steps 1-4 for the Pink clusters
\end{enumerate}

This sequential deposition of clusters by the order of the sublayer they belong to ensures that the newly deposited clusters are not in the neighbourhood of unequilibrated clusters. The algorithm allows for a nearly simultaneous deposition, the only latency between each deposition being the time taken for the cluster to reach the surface of the film. On the \mbox{24 nm} $\times$ \mbox{24 nm} XY plane of the substrate, 52 clusters can be arranged in an HCP pattern, meaning that the deposition of the four sublayers can be done with four equilbration steps, instead of a 52 times as would be the case in a completely sequential deposition. This speeds up the deposition compared to single deposition algorithm process by a factor of 13 for the given substrate and cluster combination. Like the random deposition algorithm, this algorithm also scales better with the increase in XY dimensions of the film. A top view of a \gls{hcp} patterned deposition of 3 nm sized clusters at 60 meV/atom energy can be seen in Supplementary Video V1. \par

%\section{Modelling Cluster-compacted Glasses}

\begin{selfcite}
One of the aims of this study is to compare CAMGs to metallic glasses prepared by mechanical compaction, i.e., NGs. The results of simulations of CAMGs and NGs using the same clusters as building blocks allows a comparison of the different processing techniques, compaction for NGs and energetic impact for CAMGs. Furthermore, the structure of simulated NGs prepared by compaction of clusters in the size range of 800 atoms has not been reported. For the simulation of the cold compaction, the clusters were inserted in a simulation box and compacted at 50 K temperature under 5 GPa pressure to yield a NG of ∼ 300,000 atoms. In previous works, the compaction of NGs was modelled by inserting the clusters at random positions before compaction [18,29] as this method closely resembles the actual experiments conducted to obtain NGs. The properties of such NGs are described in further detail in Chapter~\ref{c:cbmg}. \par
%\end{selfcite}

%\begin{selfcite}
However, when comparing the NGs with CAMGs, the clusters were inserted in a HCP arrangement prior to compaction in order to resemble the arrangement used for the CAMGs. Once the sample was compacted at 50 K temperature and equilibrated, it was unloaded for 0.2 ns and then equilibrated again for another 2 ns. In the NGs with clusters of sizes 3 nm (described in Chapter~\ref{c:camg}) and 7 nm (described in Chapter~\ref{c:cbmg}) prepared in this way, no pores were present, when examined using a surface mesh with a probe sphere radius of 3 Å [22,30].
\end{selfcite}
\section{Modelling Nanoglasses}
\begin{changebar}
One of the aims of this study is to compare \gls{camg}s to metallic glasses prepared by mechanical compaction, i.e., \gls{ng}s. The results of simulations of \gls{camg}s and \gls{ng}s using the same clusters as building blocks allows a comparison of the different processing techniques, compaction for \gls{ng}s and energetic impact for \gls{camg}s. Furthermore, the structure of simulated \gls{ng}s prepared by compaction of clusters in the size range of 800 atoms has not been reported in the literature. For the simulation of the cold compaction, the clusters were inserted in a simulation box and compacted at 50 K temperature under 5 GPa pressure to yield a \gls{ng} of $\sim$300,000 atoms. In previous works, the compaction of NGs was modelled by inserting the clusters at random positions before compaction  \cite{Adjaoud2018,Kalcher2017} as this method closely resembles the actual experiments conducted to obtain NGs. The properties of such NGs are described in further detail, in Chapter~\ref{c:cbmg}. \par

However, when comparing the NGs with CAMGs, the clusters were inserted in a \gls{hcp} arrangement prior to compaction in order to resemble the arrangement used for the CAMGs. Such a regular cluster arrangement to make NGs has been employed in previous works as well \cite{Sopu2009,Cheng2019,Cheng2019a,Zheng2021}. Once the sample was compacted at 50 K temperature and equilibrated, it was unloaded for 0.2 ns and then equilibrated again for another 2 ns. In the \gls{ng}s with clusters of sizes 3 nm (described in Chapter~\ref{c:camg}) and 7 nm (described in Chapter~\ref{c:cbmg}) prepared in this way, no pores were present, when examined using a surface mesh with a probe sphere radius of 3 \r{A} \cite{Stukowski2010a,Stukowski2014}. \par
\end{changebar}

\clearpage

\section{Summary}
The current chapter laid out the details of protocols implemented to simulate the various kinds of metallic glasses discussed in this dissertation. The simulation of CAMGs and NGs were performed using \gls{md} and \gls{eam} potential. The authenticity of the atomic potential and the modelling methods were tested by simulating \gls{rq} \gls{mg}s. The \gls{eam} potential effectively captures the vitrification of the quenched glasses, as evidenced by the PRDFs. It was also possible to reproduce the known theories for the local atomic \gls{sro} and \gls{pe} states for the MGs. Additionally, it was determined that the volume behaviour upon quenching is not accurately modelled using the potential. \par

Consequently, models used to prepare \gls{camg}s and \gls{ng}s were developed. First a cluster was simulated cutting out a spherical volume from an RQ MG. With a short heat-treatment above \gls{tg}, it was possible to induce Cu-atoms to segregate out to the cluster surface---an effect that better replicates the experiments, and desirable to create stable cluster-cluster interfaces created during deposition (for CAMGs) or compaction (for NGs). \par

To understand the mechanisms of \gls{camg}s synthesis, the deposition of a single cluster onto a substrate is modelled, and the simulation is determined to have converged upon inspecting the atomic trajectories 2 ns after the deposition. To generate an entire film of clusters, one requires a deposition of multiple clusters onto the substrate. Depositing the clusters in a random fashion resulted in formation of porous films: leading to the reduction of number of cluster-cluster interfaces. This reduction of interfaces is undesirable as it is intended to closely examine the nature of the interfaces in the CAMGs. For this reason, an optimal HCP patterned multi-cluster parallel deposition algorithm was developed to maximise cluster-packing and removal of pores, increase cluster-cluster interfaces, and also leading to a 13$\times$ improvement in simulation speed. \par

Two methods of simulating \gls{ng}s were discussed. The compaction of clusters was done upon: 1. clusters being randomly inserted in a box---in the manner described by \textcite{Adjaoud2018} for NG simulations, and 2. clusters being arranged in an HCP pattern to emulate the pattern formed by the CAMGs deposited in the HCP layout. \par

The \gls{lmp}-based simulation codes and workflows can be accessed with information available in Section~\ref{s:github}. Armed with this arsenal of simulation techniques, it is then possible to study the \gls{camg}s  and \gls{ng}s, which motivates the work discussed in the following Chapters~\ref{c:camg}~and~\ref{c:cbmg}. \par

\chapter{Structure and Packing of Cluster-assembled Metallic Glasses (CAMG)} \label{c:camg}
\chapter{Structure and Packing of Cluster-assembled Metallic Glasses} \label{c:camg}
\scdeclaration

One of the primary objectives of the thesis has been to computationally study the \gls{camg}---which are metallic glasses sunthesised by an alternative route. The novel \gls{cibd} method holds promise of creating distinct nanostructures in amorphous materials, and controlling the local amorphous order. The simulations of \gls{camg}s are finally possible due to protocols established in Chapter~\ref{c:dev}. Some primary characteristics of the deposition process were also gathered. In the current chapter, we dive deeper into analysing the simulated \gls{camg}s. First, the deposition of a single cluster on a substrate is studied to understand the behaviour of cluster depositions. Next, the nanostructure and the local amorphous order of the film made from deposition of a plethora of clusters is understood: the simulated \gls{camg}s are characterised by their \gls{rdf}s, short-to-medium range order, their atomic packing, and the energetic states they occupy. By the above metrics, the \gls{camg}s are also contrasted with the \gls{rq} \gls{mg} that they are derived from, also the \gls{ng}. Thereby, one can isolate the effects of cluster deposition and cluster compaction processes\footnote{It is noted once again that in this thesis, the \gls{ng}s are made with a monodisperse cluster distribution. They serve as a better counterpart to \gls{camg}s, unlike previous polydisperse-cluster models of \gls{ng} \cite{Adjaoud2018}.}. \par

Furthermore, it is explored whether or not the initial state of the cluster has any influence on the final states of the \gls{camg}s (and also \gls{ng}s). The clusters are derived from \gls{rq} \gls{mg}s of varying quench rates, and \gls{camg}s and \gls{ng}s are produced from them. Consequently, an attempt is made to understand the resulting structural and energetical changes in the \gls{camg}s and \gls{ng}s. \par

%\section{Exploring Deposition Energy Ranges} 
\begin{selfcite}
In order to understand the role of the impact energy on the cluster deposition and to identify the range of impact energy of interest for the preparation of CAMGs, the deposition of a single cluster on a substrate was studied initially. The single \cz  cluster, 3 nm in diameter— prepared as described in Section~\ref{s:clus}—was deposited at various energies ranging from 6 meV to 6000 meV per atom. \par

Figure~\ref{f:clus_single3} shows the cross-sections of the clusters deposited at various energies in the YZ plane parallel to the deposition axis. The snapshots were made 2 ns after deposition, as in Section~ref{s:camgdev} the simulation was determined to have converged by this time. The atoms colored in yellow and magenta, belong to the shell and core atoms of the cluster prior to deposition, respectively (as in Figure~\ref{f:clus_rad-3nm}). No distinction is made in this color code for the constituent elements. All substrate atoms are colored in black. Clearly, the morphology of the cluster after impact varies with deposition energy. \par In the energy range of 6-60 meV/atom, the cluster is in a soft-landing state. In this regime, it is observed that even for the lowest deposition energy of 6 meV/atom the cluster loses the original shape of the free cluster, which was almost perfectly spherical. This change of shape is attributed to a partial wetting due to the cohesive forces at the surface between the cluster and the substrate. No noticeable difference in the final shapes is observed for the cases of 6 meV/atom and 60 meV/atom impact energy. As more drastic changes of the cluster shape are observed at higher impact energies, the deposition energy of 60 meV/atom was chosen to be the upper limit for the soft-landed state.

\begin{figure}
	\centering
	\includegraphics[width=0.5\linewidth]{2021_05/figures/2.pdf}
	\includegraphics[width=0.5\linewidth]{2021_05/figures/3.pdf}
	\mycaption{Single 3 nm cluster deposited states:}{ In Figure~\ref{f:clus_single3}(a), the cross sections of snapshots of the clusters 2 ns after the simulated deposition at various per-atom energies ranging from 6 meV/atom to 6000 meV/atom are shown, with categories of soft, medium, hard and extreme hard landing indicated. The core and shell atoms are marked in magenta and yellow colors, respectively. This is the same color scheme used in Figure~\ref{f:clus_rad-3nm}b. In Figure~\ref{f:clus_single3}(b) the as-deposited states of the clusters (curvature and thickness of the embedded clusters) are represented after equilibration for 2 ns after the deposition.}
	\label{f:clus_single3}
\end{figure}

For all simulated cluster impacts, it is observed that the impact energy clearly influences the final states of the shell atoms in the clusters. The change in state of the deposited clusters, was quantified by means of the root mean square deviation of the shell atoms of the radial coordinate of the shell atoms from the average shell radius (RMSD$ _{shell}$) and the radius of curvature of the cluster R$ _{C} $. Figure~\ref{f:clus_single3}b summarizes the RMSD$ _{shell}$ and R$ _{C} $ as a function of deposition energy, with the clusters being equilibrated for 2 ns. \par

The values of RMSD$ _{shell}$ quantify the degree of distortion of the shell atoms from their original positions, which increases monotonically with the impact energy. The shell region stays intact at energies below 600 meV/atom. However, for impact energies ≥ 600 meV/atom, i.e., in the hard-landed state, the distortion of the cluster increases continuously with increasing impact energy. The deposition at 300 meV/atom energy is then defined as the medium-landed state. The separation between soft, medium and hard landing is assigned arbitrarily. However, these distinctions allow us to understand the broad energy regimes in which the CAMGs retain or lose the signatures of the originally free clusters. In the soft-landed state, the clusters in the CAMGs can be expected to remain mostly spherical. At the higher energies, in the medium-landed state, a lot more deformation of the cluster is expected. In the hard-landing state, not only will the cluster be deformed, but the inter-diffusion of the core-shell atoms in the cluster becomes significant. \par

In line with the changes of RMSD$ _{shell}$, the radius of curvature R$ _{C} $ gradually decreases with increasing energy, indicating that the cluster loses its spherical morphology at higher impact energies. At energies $\geq$ 3000 meV/atom, i.e., extreme hard landing, the cluster embeds itself into the substrate during impact, being reflected in a negative R$ _{C} $. With increasing impact energy, the cluster deforms more and embeds deeper into the substrate. In the context of formation of CAMG films, i.e., when multiple clusters are deposited over each other layer-by-layer, intermixing between clusters is expected at the higher impact energies. Based on the results of the single cluster deposition, impact energies between 60-600 meV/atom were chosen to study the formation of CAMG films. In the following section, as part of a first analysis, the changes of the core-shell structures during multiple cluster deposition will be considered.

\end{selfcite}


\section{Exploring Deposition Energy Ranges} \label{c:cibd_single}
\begin{changebar}
In order to understand the role of the impact energy on the cluster deposition and to identify the range of impact energy of interest for the preparation of \gls{camg}s, the deposition of a single cluster on a substrate was studied initially. A single \cz cluster, 3 nm in diameter— prepared from a \qr{10} \gls{rq} \gls{mg} as described in Section~\ref{s:clus}---was deposited at various energies ranging from 6 meV to 6000 meV per atom. \par

Figure~\ref{f:clus_single3} shows the cross-sections of the clusters deposited at various energies in the YZ plane parallel to the deposition axis. The snapshots were made 2 ns after deposition. As seen in Section~\ref{s:camgdev}, the simulation was determined to have converged by this time. The atoms coloured in yellow and magenta, belong to the shell and core atoms of the cluster prior to deposition, respectively (as in Figure~\ref{f:clus_rad-3nm}). No distinction is made in this colour code for the constituent elements. All substrate atoms are coloured in black. Clearly, it is seen that the morphology of the cluster after impact varies with deposition energy. \par

In the energy range of 6-60 meV/atom, the cluster is in a soft-landing state. In this regime, it is observed that even for the lowest deposition energy of 6 meV/atom the cluster loses the original shape of the free cluster, which was almost perfectly spherical. This change of shape is attributed to a partial wetting due to the cohesive forces at the surface between the cluster and the substrate. No noticeable difference in the final shapes is observed for the cases of 6 meV/atom and 60 meV/atom impact energy. As more drastic changes of the cluster shape are observed at higher impact energies, the deposition energy of 60 meV/atom was chosen to be the upper limit for the soft-landed state. \par
	
For all simulated cluster impacts, it is observed in Figure~\ref{f:clus_single3} that the impact energy clearly influences the final states of the shell atoms in the clusters. The change in state of the deposited clusters was quantified by means of the root mean square deviation of the radial coordinate (with respect to centroid of the cluster) of the shell atoms from the average shell radius and the largest \gls{rc} of the cluster in the deposition axis. The \gls{rc} is calculated as the largest Z-component of the displacement vector subtended by shell atoms in a spherical sector around the deposition axis, to the surface level of the substrate. The method of evaluation \gls{rc} is illustrated in Figure~\ref{f:rccalc}. The \gls{rc} represents the convexity of the aspherical cluster for the which RC varies as a function of the surface. \par

\begin{figure}[!h]
	\centering
	\begin{subfigure}{0.5\textwidth}
		\includegraphics[width=\linewidth,trim={3.8cm 1cm 4cm 0.5cm},clip]{camg_3nm/2.pdf}
	\end{subfigure}%
	\hfill
	\begin{subfigure}{0.5\textwidth}
		\includegraphics[width=\linewidth,trim={4cm 0.5cm 4cm 0.8cm},clip]{camg_3nm/3.pdf}
	\end{subfigure}
	\mycaption{Single 3 nm cluster deposited states}{ In (a), the cross sections of snapshots of the clusters 2 ns after the simulated deposition at various per-atom energies ranging from 6 meV/atom to 6000 meV/atom are shown, with categories of soft, medium, hard and extreme hard landing indicated. The core and shell atoms are marked in magenta and yellow colours, respectively. This is the same colour scheme used in Figure~\ref{f:clus_rad-3nm}. In (b) the as-deposited states of the clusters (curvature and thickness of the embedded clusters) are represented after equilibration for 2 ns after the deposition.}
	\label{f:clus_single3}
\end{figure}

Figure~\ref{f:clus_single3}b summarises the \gls{rmsds} and \gls{rc} as a function of deposition energy, with the clusters being equilibrated for 2 ns after deposition. The values of \gls{rmsds} quantify the degree of distortion of the shell atoms from their original positions, which increases monotonically with the impact energy. The shell region stays intact at energies below 600 meV/atom. However, for impact energies $\geq$ 600 meV/atom, i.e., in the hard-landed state, the distortion of the cluster increases continuously with increasing impact energy. The deposition at 300 meV/atom energy is then defined as the medium-landed state. \par The separation between soft, medium and hard landing is assigned arbitrarily. However, these distinctions allow us to understand the broad energy regimes in which the \gls{camg}s retain or lose the signatures of the originally free clusters. In the soft-landed state, the clusters in the \gls{camg}s can be expected to remain mostly spherical. At the higher energies, in the medium-landed state, the cluster is expected to deform further. In the hard-landing state, not only will the cluster be deformed, but the inter-diffusion of the core-shell atoms in the cluster becomes significant. \par

\begin{figure}[!h]\centering
	\includegraphics[width=0.8\linewidth]{radius_curv.png}
	\mycaption{Calculating \gls{rc} of a deposited cluster}{The construction used to evaluate the \gls{rc} of single deposited cluster, as depicted in Figure~\ref{f:clus_single3}b.}
	\label{f:rccalc}
\end{figure}%

In line with the changes of \gls{rmsds}, the \gls{rc} gradually decreases with increasing energy, indicating that the cluster loses its spherical morphology at higher impact energies. At energies $\geq$ 3000 meV/atom, an extreme hard landing is observed: the cluster embeds itself into the substrate during impact, being reflected in a negative \gls{rc}. The cluster adopts a concave
shape on the substrate. With increasing impact energy, the cluster deforms more and embeds deeper into the substrate. \par In the context of formation of \gls{camg} films, i.e., when multiple clusters are deposited over each other layer-by-layer, intermixing between clusters is expected at the higher impact energies. Based on the results of the single cluster deposition, impact energies between 60-600 meV/atom were chosen to study the formation of \gls{camg} films. In the following section, as part of a first analysis, the changes of the core-shell structures during multiple cluster deposition will be considered.
\end{changebar}
%
%\section{Identifying Cores and Interfaces}

\begin{selfcite}
The deposition of multiple clusters, with clusters derived from \qr{10} \cz MG, was simulated at 60 meV/atom, 300 meV/atom, 600 meV/atom impact energies to mimic the soft, medium, and hard-landing in the CAMG film samples, respectively. Additionally, a deposition at extreme energetic conditions was simulated at the impact energy of 6000 meV/atom. The \gls{hcp} arrangement for each impact energy was achieved in the following manner: before each new cluster was deposited, the clusters that were already present in its neighborhood were relaxed for at least 2 ns. Every simulated CAMG had three layers of such cluster depositions, with ~50 clusters in each layer. The deposition sequence and the processes occurring during impact can be followed in the \todo[inline]{Supplementary Video V1}, which shows the simulation of the deposition of CAMGs in comparison to that of the compaction in the NG processing. \par

Figure~\ref{f:film_network} shows cross-sections of the films, similar to Figure~\ref{f:clus_single3}a, after equilibrating the sample for 2 ns after the deposition of the last cluster. The yellow and magenta color coding denotes the shell and core atoms of the clusters prior to deposition, similar to Figure~\ref{f:clus_comp-3nm} and Figure~\ref{f:clus_single3}. No evidence for porosity is observed even for the soft-landing sample (60 meV/atom case) by evaluating a surface mesh with a probe sphere radius of 2.4 \r{A} \cite{Stukowski2010a,Stukowski2014}. At the lowest impact energy of 60 meV/atom, it is observed that, like in Figure~\ref{f:clus_single3}b, the deposited clusters mostly retain their initial sphericity. The cluster sphericity is progressively lost with increasing deposition energy. Next, it should be noted that the first layer of clusters has resided on the substrate for at least 24 ns (using the deposition protocol described) by the time the final layer is deposited. Nevertheless, the interdiffusion of the core and shell atoms is quite low for deposition energies even up to 600 meV/atom. In the energy range of 60-600 meV/atom, the shell atoms, i.e., the former surface atoms of the free cluster prior to deposition, are forming a distinct inter-connected network, which can be interpreted as atomically thin interfacial regions between the cores of the clusters. At the impact energy of 6000 meV/atom the interfacial regions vanish completely. Additionally, this film shows significant atomic intermixing between the substrate and the film (see Figure~\ref{f:film_network}, visualizing the black substrate atoms found in the film, and magenta and yellow atoms from the deposited film embedded in the substrate). For the case of the 6000 meV/atom energy, it is estimated that 8\% of the atoms originally in the film are mixed into the substrate, whereas for the case of the 600 meV/atom impact energy this value is about 1.7\%. Similarly, the mixing of the substrate atoms diffusing into the film sample was also ascertained, by tracking the substrate atoms in the final deposited films. The mixing of the substrate atoms into the film has been estimated to be 6\% for 600 meV/atom case, and is 16\% for the 6000 meV/atom case. For both the film and substrate atoms, the diffusion into the neighboring medium is higher at higher impaction energies. \par
	
\begin{figure*}	\centering
	\includegraphics[width=0.7\textwidth,trim={2cm 2.5cm 1.9cm 1.8cm},clip]{2021_05/figures/4.pdf}
	\includegraphics[width=\textwidth,trim={0.8cm 2cm 0 1.7cm},clip]{2021_05/figures/5.pdf}
	\mycaption{Deposition of CAMG films}{(a) A depiction of the deposition of a cluster onto the substrate and (b) a top view of the clusters deposited in a HCP arrangement (colour coded by the atom height in the deposition axis) (c) A vertical crossection of the deposited films at 60, 300, 600, and 6000 meV/atom energies, with cluster-core atoms in magenta, and cluster-shell atoms in yellow. The substrate atoms are coloured in black. We observe that the shell atoms form a network of interfaces across the film at least up to 600 meV/atom deposition energy. (d) When colour coded with von Mises shear strain, the interfacial atoms correlate with the higher strained atoms.}
	\label{f:film_network}
\end{figure*}
	
Up to now, the locations of the atoms, located at the surfaces of the clusters prior to deposition, have been followed (using the magenta and yellow color scheme for the core and shell atoms, respectively) to determine the interfacial regions in the samples prepared with impact energies in the range below 600 meV/atom. This approach does not provide any information on the energetic state of the atoms in the CAMGs or on the local environments in the cores and interfaces. In a first step towards a more detailed analysis, the von Mises shear strain for each of the CAMG atoms was determined \cite{Stukowski2014}. A cut-off radius of 3.8 \r{A} was chosen to compute the strain tensor. In Figure~\ref{f:film_network}d, the high strain regions can be clearly correlated to the interfacial network shown in Figure 3c for all impact energies below 600 meV/atom. Only for the highest impact energy, no sign for the presence of interfaces can be found, similar to the observation in Figure~\ref{f:film_network}c. The correlation observed for CAMGs is consistent with Gleiter’s original definition of interfaces \cite{Gleiter1991} in NGs, in which the interfaces were assumed to be regions of distorted and sheared coordination among adjacent clusters. The strain maps in Figure~\ref{f:film_network}d confirm that an interfacial structure is formed and is retained in the range of 60-600 meV/atom impact energies. Upon inspection of the simulation snapshots in \gls{ovito}, the cores and interfaces in the CAMG film samples made by soft-to-high landing deposition are found to reflect a chemical heterogeneity similar to what was originally present in the free clusters prior to deposition (described in Figure~\ref{f:clus_comp-3nm}a), with the Cu concentration of ∼46 at. \% in the cores, and ∼54 at. \% in the interfacial regions. The atoms in the CAMG film sample for the extreme hard-landing case of 6000 meV/atom are strained to the point where the presence of core and interface structure is lost, and in this manner resembling the MGs obtained by RQ. It should be mentioned that the processing of NGs and CAMGs seems to differ in one aspect: at the harshest conditions, i.e., at the extreme hard-landing case for CAMGs and at the highest pressures for NGs, the final structures are different. In NGs, the interfacial regions continue to exist even at the highest pressures, while the interfacial regions disappear in CAMGs at the extreme hard-landing. This might be due to the sequential deposition of the clusters in CAMG processing, compared to the compaction process, which occurs in the entire arrangement of clusters. \par

\begin{figure}
	\begin{subfigure}[b]{0.33\textwidth} \includegraphics[width=\textwidth,trim={0 1cm 0 1cm},clip]{1e10/dep_60meV_pers_del}
		\caption{}
	\end{subfigure}%
	\hfill
	\begin{subfigure}[b]{0.33\textwidth} \includegraphics[width=\textwidth,trim={0 1cm 0 1cm},clip]{1e10/dep_300meV_pers_del}
		\caption{}
	\end{subfigure}%
	\hfill
	\begin{subfigure}[b]{0.33\textwidth} \includegraphics[width=\textwidth,trim={0 1cm 0 1cm},clip]{1e10/dep_600meV_pers_del}
		\caption{}
	\end{subfigure}%
\mycaption{Representative slabs of CAMGs}{To avoid surface artifacts, these slabs were cut out of the CAMG films}
\label{f:camg_slabs}
\end{figure}

The CAMG samples are also different from both the MGs and the NG, in the following fashion: the CAMGs only partially fill a simulation box. It is important to clarify that the unfilled volume referred to is not within the film, rather it is between the upper surface of the film and the upper wall of the simulation box. Due to the open surface at the top, and a surface interaction between the cluster atoms and the substrate, it is expected in the CAMG film samples that this gives rise to surface artifacts—including defective surface coordinations, larger atomic occupancy and higher surface energy. It was decided to first analyse the entire CAMG film sample, and then later represent the data from a slab of fixed dimensions present within the inside of the deposited film samples, in order to avoid the surface artifacts, as shown in Figure~\ref{f:camg_slabs} (Tip for \gls{ovito} users: the atoms have to be deleted from the data only after any OVITO-API calls have been performed).  \par
\end{selfcite}

Another possible method of removing surface artifacts is to query the surface atoms by means of a surface mesh, and deleting them from the slab before plotting the analysed data. However, the first method of representing data from slabs is preferred in this thesis. The fixed slab volume and the consistently similar amounts of core and interfacial atoms, allows for a consistent analysis of the effects of core and interface regions. This consistency is lost when deleted surface atoms from the surface mesh, as the deleted atoms are majorly made up of the interface atoms. \par

\begin{selfcite}
Based on the presented results, a structural model for the CAMGs is proposed, similar to that of NGs. In this model, interfacial regions, which are chemically different from the core regions due to the surface segregation observed in the individual clusters, are formed during the cluster deposition in the range of impact energies between 60–600 meV/atom. The only exception is, as mentioned above, the structure for an impact energy of 6000 meV/atom, for which interfaces are totally absent. Therefore, in the following sections, only simulations of CAMGs deposited at the impact energies 60, 300, and 600 meV/atom, are being considered. Incidentally, this energy range corresponds well with the energy range used in the cluster experiments reported in \cite{Benel2019}.
\end{selfcite}
\section{Identifying Cores and Interfaces} \label{s:corint}

\begin{changebar}
The deposition of multiple clusters, with clusters derived from \qr{10} \cz \gls{mg}, was simulated at 60 meV/atom, 300 meV/atom, 600 meV/atom impact energies to mimic the soft, medium, and hard-landing in the \gls{camg} film samples, respectively. Additionally, a deposition at an extreme energetic condition was simulated with an impact energy of 6000 meV/atom. The \gls{hcp} arrangement (chosen for reasons mentioned in Section~\ref{s:camgdev}) for each impact energy was achieved in the following manner: before each new cluster was deposited, the clusters that were already present in its neighborhood were relaxed for at least 2 ns. Every simulated \gls{camg} had three layers of such cluster depositions, with $\sim$50 clusters in each layer. The deposition sequence and the processes occurring during impact can be followed in the Supplementary Video V2, which shows the simulation of the deposition of \gls{camg}s in comparison to that of the compaction in the \gls{ng} processing. \par

Figure~\ref{f:film_network} shows cross-sections of the films, similar to Figure~\ref{f:clus_single3}a, after equilibrating the sample for 2 ns after the deposition of the last cluster. The yellow and magenta colour coding denotes the shell and core atoms of the clusters prior to deposition, similar to Figure~\ref{f:clus_comp-3nm} and Figure~\ref{f:clus_single3}. No evidence for porosity is observed even for the soft-landing sample (60 meV/atom case) by evaluating a surface mesh with a probe sphere radius of 2.4 \r{A} \cite{Stukowski2010a,Stukowski2014}. \par

At the lowest impact energy of 60 meV/atom, it is observed that, like in Figure~\ref{f:clus_single3}b, the deposited clusters mostly retain their initial sphericity. The cluster sphericity is progressively lost with increasing deposition energy. Next, it should be noted that the first layer of clusters has resided on the substrate for at least 24 ns (using the deposition protocol described) by the time the final layer is deposited. Nevertheless, the interdiffusion of the core and shell atoms is quite low for deposition energies even up to 600 meV/atom. In the energy range of 60-600 meV/atom, the shell atoms, i.e., the former surface atoms of the free cluster prior to deposition, are forming a distinct inter-connected network, which can be interpreted as atomically thin interfacial regions between the cores of the clusters. The former shell atoms (coloured yellow as mentioned above)
of the clusters are hence defined to constitute the cluster-cluster interfaces. At the impact energy of 6000 meV/atom the interfacial regions vanish completely. \par

Additionally, at 6000 meV/atom deposition energy, the film shows significant atomic intermixing between the substrate and the film (see Figure~\ref{f:film_network}, visualising the black substrate atoms found in the film, and magenta and yellow atoms from the deposited film embedded in the substrate). For the case of the 6000 meV/atom energy, it is estimated that 8\% of the atoms originally in the film are mixed into the substrate, whereas for the case of the 600 meV/atom impact energy this value is about 1.7\%. \par

\begin{figure}[!h]	\centering
	\includegraphics[width=0.7\textwidth,trim={2cm 2.5cm 1.9cm 1.8cm},clip]{camg_3nm/4.pdf}
	\includegraphics[width=\textwidth,trim={0.8cm 2cm 0 1.7cm},clip]{camg_3nm/5.pdf}
	\mycaption{Deposition of \gls{camg} cilms}{(a) A depiction of the deposition of a cluster onto the substrate and (b) a top view of the clusters deposited in a HCP arrangement (colour coded by the atom height in the deposition axis) (c) A vertical crossection of the deposited films at 60, 300, 600, and 6000 meV/atom energies, with cluster-core atoms in magenta, and cluster-shell atoms in yellow. The substrate atoms are coloured in black. We observe that the shell atoms form a network of interfaces across the film at least up to 600 meV/atom deposition energy. (d) When colour coded with von Mises shear strain, the interfacial atoms correlate with the higher strained atoms.}
	\label{f:film_network}
\end{figure}

Similarly, the mixing of the substrate atoms diffusing into the film was also ascertained, by tracking the substrate atoms in the final deposited films. The mixing of the substrate atoms into the film has been estimated to be 6\% for 600 meV/atom case, and is 16\% for the 6000 meV/atom case. For both the film and substrate atoms, the diffusion into the neighboring medium is higher at higher impaction energies. \par

Up to now, the locations of the atoms, located initally at the surfaces of the clusters prior to deposition, have been followed (using the magenta and yellow colour scheme for the core and shell atoms, respectively) to determine the interfacial regions in the samples prepared with impact energies in the range below 600 meV/atom. This approach does not provide any information on the energetic state of the atoms in the \gls{camg}s or on the local environments in the cores and interfaces. In a first step towards a more detailed analysis, the von Mises shear strain for each of the \gls{camg} atoms was determined \cite{Stukowski2014}. A cut-off radius of 3.8 \r{A} was chosen to compute the strain tensor. In Figure~\ref{f:film_network}d, the high strain regions can be clearly correlated to the interfacial network shown in Figure~\ref{f:film_network}c for all impact energies below 600 meV/atom. Only for the highest impact energy, no sign for the presence of interfaces can be found, similar to the observation in Figure~\ref{f:film_network}c.  \par

The correlation observed for \gls{camg}s is consistent with Gleiter’s original definition of interfaces \cite{Gleiter1991} in \gls{ng}s, in which the interfaces were assumed to be regions of distorted and sheared coordination among adjacent clusters. The strain maps in Figure~\ref{f:film_network}d confirm that an interfacial structure is formed and is retained in the range of 60-600 meV/atom impact energies.\par

Upon inspection of the simulation snapshots in \gls{ovito}, the cores and interfaces in the \gls{camg} film samples made by soft-to-hard landing deposition are found to reflect a chemical heterogeneity similar to what was originally present in the free clusters prior to deposition (described in Figure~\ref{f:clus_comp-3nm}a), with the Cu concentration of ∼46 at. \% in the cores, and ∼54 at. \% in the interfacial regions. The overall composition of the CAMG films are found to be \cz (with 1\% deviation), same as of the original clusters. The atoms in the CAMG film sample for the extreme hard-landing case of 6000 meV/atom indicate a loss of core and interface structure, and in this manner resembling the MGs obtained by RQ. This loss of core-interface structure, also observed from von Mises local strains is indicative of a high deviation of atoms from their as-deposited positions in the clusters, likely due to local melting and resolidification near the deposition sites. It should be mentioned that the processing of \gls{ng}s and \gls{camg}s seems to differ in one aspect: at the harshest conditions, i.e., at the extreme hard-landing case for \gls{camg}s and at the highest pressures for \gls{ng}s, the final structures are different. In NGs, the interfacial regions continue to exist even at the highest pressures, while the interfacial regions disappear in CAMGs at the extreme hard-landing. This most likely is caused by the differences between the
compaction and deposition processes. While the cold compaction shears the entire arrangement of clusters together, the sequential deposition of the clusters along with local heating due to the inelastic collision may result in the dissolution of the cluster structure. Therefore, the energetic impact of clusters is considered as a novel process leading to metallic glasses. CAMGs also contain interfacial regions with a modified structure, which differs from that of RQ MGs and NGs prepared by high pressure compaction. However, it is not clear if the structural details of the interfacial regions of CAMGs and NGs are identical, implying that different properties for RQ MGs, NGs and CAMGs are very well possible. Consequently, there is a need to further investigate the NGs and CAMGs both experimentally and using simulation methods. \par

The \gls{camg} samples are also different from both the \gls{mg}s and the \gls{ng}, in the following fashion: the \gls{camg}s only partially fill a simulation box. It is important to clarify that the unfilled volume referred to is not within the film, rather it is between the upper surface of the film and the upper wall of the simulation box.  The open surface at the top in the CAMG film samples is expected to give rise to surface artefacts—including defective surface coordinations, larger atomic occupancy and excess surface energy. The surface interaction between the cluster atoms and the substrate will also lead to defective coordinations at the border. It was decided to first analyse the entire \gls{camg} film sample, and then later represent the data from a slab of fixed dimensions present within the inside of the deposited film samples, in order to avoid the surface artefacts, as shown in Figure~\ref{f:camg_slabs} (Tip for \gls{ovito} users: the atoms have to be deleted from the data only after any OVITO-API calls have been performed).  \par

\begin{figure}[h]
	\begin{subfigure}[b]{0.33\textwidth} \includegraphics[width=\textwidth,trim={0 0.5cm 0 1cm},clip]{1e10/dep_60meV_pers_del}
		\caption{}
	\end{subfigure}%
	\hfill
	\begin{subfigure}[b]{0.33\textwidth} \includegraphics[width=\textwidth,trim={0 0.5cm 0 1cm},clip]{1e10/dep_300meV_pers_del}
		\caption{}
	\end{subfigure}%
	\hfill
	\begin{subfigure}[b]{0.33\textwidth} \includegraphics[width=\textwidth,trim={0 0.5cm 0 1cm},clip]{1e10/dep_600meV_pers_del}
		\caption{}
	\end{subfigure}%
	\mycaption{Representative slabs of \gls{camg}s}{To avoid surface artefacts, these slabs were cut out of the \gls{camg} (a) 60 meV/atom (b) 300 meV/atom and (c) 600 meV/atom films.}
	\label{f:camg_slabs}
\end{figure}
\end{changebar}

Another possible method of removing surface artefacts is to query the surface atoms by means of a surface mesh, and deleting them from the slab before plotting the analysed data. When deleting surface atoms by using the surface mesh, the deleted atoms are majorly made up of the interface atoms. Moreover, the \gls{camg}s deposited at lower energies have more distinct morphologies at the top surface of the film, meaning that the number of deleted surface atoms in this case will be larger than for a \gls{camg} deposited at higher energies. This would result in the representative samples not consistently having the same number of core and interface atoms in the \gls{camg}s. For the fixed slab method, however, the volume and the consistently similar amounts of core and interfacial atoms, allows for a consistent analysis of the effects of core and interface regions. Hence, the first method of representing data from fixed slabs is preferred in this thesis. \par

\begin{changebar}
Based on the presented results, a structural model for the \gls{camg}s is proposed, similar to that of \gls{ng}s. In this model, interfacial regions, which are chemically different from the core regions due to the surface segregation observed in the individual clusters, are formed during the cluster deposition in the range of impact energies between 60–600 meV/atom. The only exception is, as mentioned above, the structure for an impact energy of 6000 meV/atom, for which interfaces are totally absent. Therefore, in the following sections, only simulations of \gls{camg}s deposited at the impact energies 60, 300, and 600 meV/atom, are being considered. Incidentally, this energy range corresponds well with the energy range used in the cluster experiments reported in \cite{Benel2019}. \par

The prepared \gls{camg}s in this chapter are compared to corresponding \gls{ng}s and \gls{rq} \gls{mg}s. Moreover, a heat treatment as described for the single cluster (Section~\ref{s:clus}) can also modify the metastable state and structure of a MG. Therefore, the as-prepared MG sample, i.e., quenched directly from the liquid phase, was subjected to a heat treatment identical to the one used for the single cluster to determine the changes in the structure of the MG sample. In addition to the as-prepared MG, the \gls{mght} serves as a reference structure for the NGs and CAMGs prepared by compaction and energetic impact, respectively. In the following discussion up until Section\ref{s:camg_quenchrt}, the NGs and CAMGs derived from the cluster made only from the \qr{10} MG are considered. \par
\end{changebar}


%\section{Local Structure Tailoring in CAMGs}
\begin{changebar}
To obtain a more detailed insight into the structure of CAMGs, in particular the interfacial regions, the normalized \gls{prdf} of the CAMGs, NG, and the MGs were studied (see Figure~\ref{f:camg-rdf} in \nameref{c:supple} for more details). Like in Section~\ref{s:simtestMG}, the Cu-Cu (2.45 \r{A}), Cu-Zr (2.8 \r{A}), and Zr-Zr (3.25 \r{A}) first peak positions of the MGs match well with those previously reported values \cite{Nasu2007,Duan2005}. Even the NG and CAMGs follow the same pair distributions as those of MG. Moreover, no significant change in atomic pair correlations both in the core and the interfacial regions are observed. This prompts the study of the local \gls{sro}. \par

The local atomic environment is typically represented using the Voronoi analysis method, and the polyhedra are represented by a  \vi{n$_{3}$}{n$_{4}$}{n$_{5}$}{n$_{6}$} Scha\"afli index (See Section~\ref{s:voronoi}). Figures~\ref{f:voro_camg}a-c show the top seven frequent Voronoi polyhedra (arranged in the numerical order of the indices) for CAMGs, NGs, \gls{mght}, and the precursor MG quenched at a rate of \qr{10}. In Figure~\ref{f:voro_camg}a, the histograms of the Voronoi polyhedra for all the atoms constituting the entire six sample sets are shown, whereas Figures~\ref{f:voro_camg}b-c denote the histograms for the core and interface atoms, respectively. The index \vi{0}{0}{12}{0}, which represents the FI coordination, is the highest occurring in the MG, amongst all the six glasses. For the \qr{10} MG, the heat-treatment reduces the FI order as seen in \gls{mght}. The next highest occurring index is the \vi{0}{2}{8}{2}, which is known to be an \gls{ilike} polyhedron \cite{Yue2018,Borodin1999}. Its occurrence is highest in MG and \gls{mght} when compared to the NG and CAMGs. As mentioned in Section~\ref{s:voronoi}, the Voronoi polyhedra are known to be classified into four main categories \cite{Yue2018} are repeated here for the benefit of the reader: 1. icosahedral-like: \vi{0}{0}{12}{0}, \vi{0}{0}{10}{x}, and \vi{0}{2}{8}{x}; 2. crystal-like: \vi{0}{4}{4}{x} and \vi{0}{5}{2}{x}; 3. mixed coordinations: \vi{0}{3}{6}{x}, where 0 $\leq$ x $\leq$ 4; and 4. other remaining indices. With this knowledge, it is noticed that in the NG and CAMGs the other prominently occurring polyhedra in Figure~\ref{f:voro_camg}a-c are of the icosahedral-like and mixed coordination types. \par

\begin{figure}[!ht]
	\centering
	\includegraphics[width=\textwidth,trim={1cm 0cm 1cm 0cm},clip]{2021_05/figures/6.pdf}
	\mycaption{SRO recovers in CAMGs with deposition energy:}{The top 7 Voronoi indices arranged based 		on numerical order, without considering any special central atom species (Cu/Zr) in the Voronoi Polyhedra. The Voronoi histograms are shown for (a) the entire representative sample, (b) the core atoms, and (c) the interface atoms, respectively. The magenta and yellow backgrounds in the figures are rendered to represent the core and interface cases, respectively. Figure~\ref{f:voro_camg} (d), (e) and (f) show the indices sorted into known groups of coordinations. Crystalline coordinations are absent in all of the simulated glasses, in particular in the \gls{mght}, NG, and CAMGs where a heat-treatment is involved in the simulation process. The FI and ILO short-range order (SRO) in CAMGs increase systematically with deposition energy}
	\label{f:voro_camg}
\end{figure}

Furthermore, the Voronoi indices are sorted based on the above rules for the six simulated glass samples and represented in Figure~\ref{f:voro_camg}d-f, in order to facilitate the analysis of the dominant indices in comparison of the different metallic glass structures. It is first noticed that crystalline coordinations do not occur in any of the simulated metallic glass samples, especially in the \gls{mght} and the CAMGs, despite the heat treatment involved in their processing. Especially, the lack of any crystalline coordinations confirms that all the simulated CAMGs are fully amorphous, both in the cores and in the interfaces. \par

In the CAMGs and NGs a significantly reduced short-range FI order was observed, consistent with previous studies of NGs \cite{Adjaoud2019}. This trend is also seen in the analysis of the ILO. The present simulations of the \gls{rq} \cz MGs have shown that less stable MGs (prepared at higher quench rates) are accompanied by a reduction in FI fractions and also in the ILO (refer to Figure~\ref{f:voro_qr} in Section~\ref{s:voro-mgs}). The tendency of a decrease of stability in bottom-up metallic glasses (NGs, CAMGs) with decreasing ILO is further discussed in Section~\ref{s:camg-pe}. \par
\end{changebar}

The sorted Voronoi polyhedra histograms for Cu-centered and Zr-centered coordinations are depicted in Figure~\ref{f:vorosort-cu-zr} (The unsorted Voronoi Indices distribution is indicated in Figure~\ref{f:voro_cu-zr} in \nameref{c:supple}). The ILO is observed to be higher for Cu atoms than for Zr atoms for all the six metallic glass samples that have been studied. The Cu-centered atoms make up the majority in contributing to \gls{ilo}. Also seen is the increasing icosahedral-like order in Cu-centered atoms in the CAMGs with deposition energy. \par

\begin{figure}[!ht]
	\centering
	\begin{subfigure}{0.45\linewidth} \centering \includegraphics[width=\textwidth]{1e10/voronoi-sort_3nm_Cu50Zr50_CAMG_vs_MG_1e10_asp_Cu} 
		\subcaption{} \end{subfigure}%
	\hspace{1cm}
	\begin{subfigure}{0.45\linewidth} \centering \includegraphics[width=\textwidth]{1e10/voronoi-sort_3nm_Cu50Zr50_CAMG_vs_MG_1e10_asp_Zr}
		\subcaption{} \end{subfigure}
	\mycaption{Sorted Voronoi Polyhedra for Cu- and Zr- centered atoms in MGs vs in \gls{cbmg}s}{The atomic fractions here are defined w.r.t the same species i.e., at. fraction of Cu is the fraction of Cu atoms exhibiting a certain local order. }
	\label{f:vorosort_cu-zr}
\end{figure}

\todo[inline]{Want to add sorted voronoi core and interface, cu and zr centered in the appendix? decide}

\begin{changebar}
Looking back at Figure~\ref{f:voro_camg}e-f, it is observed that for NGs and CAMGs, both FI order and the ILO indices are respectively at least 1\% and 5\% higher in the interfaces compared to the cores. It is evident that the interfaces, being richer in Cu compared to the core regions, exhibit higher FI as well as higher ILO. This indicates that the interfaces must be better packed than the core regions. \par

Interestingly, a systematic increase in FI order (See Figure~\ref{f:voro_camg}c) in the CAMGs with increasing impact energy is observed. This trend, also seen in the ILO fractions of the entire sample, is especially prominent for the interfaces (see Figure~\ref{f:voro_camg}f). It should be noted that the interfaces for all impact energies have the same chemical composition. The increase in the FI-order and ILO in the interfaces can then be interpreted to be the direct result of the CIBD process. In both the core and interface atoms, another striking feature is the systematic increase of ILO in the CAMGs with increasing deposition energy, recovering towards the ILO of MG and \gls{mght}. The present model demonstrates the possibility of tailoring the local amorphous order with impact energy for metallic glasses synthesized via the CIBD route.

\end{changebar}
%%%%%%%%%%%%%%%%%
%%%%%%%%%%%%%%%%%
%%%%%%%%%%%%%%%%%
%%%%%%%%%%%%%%%%%




\section{Local Structure Tailoring in CAMGs} \label{s:vorocamg}
\begin{changebar}
To obtain a more detailed insight into the structure of \gls{camg}s, in particular the interfacial regions, the normalised \gls{prdf} of the \gls{camg}s, \gls{ng}, and the \gls{mg}s were studied (see Figure~\ref{f:camg-rdf} in \nameref{c:supple} for more details). Like in Section~\ref{s:simtestMG}, the Cu-Cu (2.45 \r{A}), Cu-Zr (2.8 \r{A}), and Zr-Zr (3.25 \r{A}) first peak positions of the \gls{mg}s match well with those previously reported values \cite{Nasu2007,Duan2005}. Even the \gls{ng} and \gls{camg}s follow the same pair distributions as those of \gls{mg}. Moreover, no significant change in atomic pair correlations both in the core and the interfacial regions are observed. This prompts the study of the local \gls{sro}. \par

The local atomic environment is typically represented using the Voronoi analysis method, and the polyhedra are represented by a  \vi{n$_{3}$}{n$_{4}$}{n$_{5}$}{n$_{6}$} Sch\"afli index (See Section~\ref{s:voronoi}). Figures~\ref{f:voro_camg}a-c show the top seven frequent \gls{vp} (arranged in the numerical order of the indices) for \gls{camg}s, \gls{ng}s, \gls{mght}, and the precursor \gls{mg} quenched at a rate of \qr{10}. In Figure~\ref{f:voro_camg}a, the histograms of the \gls{vp} for all the atoms constituting the entire six sample sets are shown, whereas Figures~\ref{f:voro_camg}b-c denote the histograms for the core and interface atoms, respectively. The index \vi{0}{0}{12}{0}, which represents the \gls{fi} coordination, is the highest occurring in the \gls{mg}, amongst all the six glasses. For the \qr{10} \gls{mg}, the heat-treatment reduces the \gls{fi} order as seen in \gls{mght}. The next highest occurring index is the \vi{0}{2}{8}{2}, which is known to be an \gls{ilike} polyhedron \cite{Yue2018,Borodin1999}. Its occurrence is highest in \gls{mg} and \gls{mght} when compared to the \gls{ng} and \gls{camg}s. As mentioned in Section~\ref{s:voronoi}, the \gls{vp} known to be classified into four main categories \cite{Yue2018}. The are mentioned here again for the benefit of the reader: 1. icosahedral-like: \vi{0}{0}{12}{0}, \vi{0}{0}{10}{x}, and \vi{0}{2}{8}{x}; 2. crystal-like: \vi{0}{4}{4}{x} and \vi{0}{5}{2}{x}; 3. mixed coordinations: \vi{0}{3}{6}{x}, where 0 $\leq$ x $\leq$ 4; and 4. other remaining indices. With this knowledge, it is noticed that in the \gls{ng} and \gls{camg}s the other prominently occurring polyhedra in Figure~\ref{f:voro_camg}a-c are of the icosahedral-like and mixed coordination types. \par

\begin{figure}[!ht]
\centering
\includegraphics[width=\textwidth,trim={1.4cm 0cm 1.65cm 0cm},clip]{camg_3nm/6.pdf}
\mycaption{\acrlong{fi} \acrlong{sro} recovers in CAMGs with deposition energy}{The top 7 Voronoi indices arranged based on numerical order, without considering any special central atom species (Cu/Zr) in the \gls{vp}. The Voronoi histograms are shown for (a) the entire representative sample, (b) the core atoms, and (c) the interface atoms, respectively. The magenta and yellow backgrounds in the figures are rendered to represent the core and interface cases, respectively. Figure~\ref{f:voro_camg} (d), (e) and (f) show the indices sorted into known groups of coordinations. Crystalline coordinations are absent in all of the simulated glasses, in particular in the \gls{mght}, \gls{ng}, and \gls{camg}s where a heat-treatment is involved in the simulation process. The \gls{fi}-order and \gls{ilo} in \gls{camg}s increase systematically with deposition energy}
\label{f:voro_camg}
\end{figure}

Furthermore, the Voronoi indices are sorted based on the above rules for the six simulated glass samples and represented in Figure~\ref{f:voro_camg}d-f, in order to facilitate the analysis of the dominant index classes in comparison of the different metallic glass structures. It is first noticed that crystalline coordinations do not occur in any of the simulated metallic glass samples, especially in the \gls{mght} and the \gls{camg}s, despite the heat treatment involved in their processing. Especially, the lack of any crystalline coordinations confirms that all the simulated \gls{camg}s are fully amorphous, both in the cores and in the interfaces. \par

In the \gls{camg}s and \gls{ng}s a significantly reduced short-range \gls{fi} order was observed, consistent with previous studies of \gls{ng}s \cite{Adjaoud2019}. This trend is also seen in the analysis of the \gls{ilo}. The present simulations of the \gls{rq} \cz \gls{mg}s have shown that less stable \gls{mg}s (prepared at higher quench rates) are accompanied by a reduction in \gls{fi} fractions and also in the \gls{ilo} (refer to Figure~\ref{f:voro_qr} in Section~\ref{s:voro-mgs}), as reported by \textcite{Yue2018}. The tendency of a decrease of stability in the NGs and \gls{camg}s with decreasing \gls{ilo} is further discussed in Section~\ref{s:camg-pe}. \par
\end{changebar}

The sorted \gls{vp} histograms for Cu-centred and Zr-centred coordinations are depicted in Figure~\ref{f:vorosort_cu-zr} (The unsorted Voronoi indices distribution is indicated in Figure~\ref{f:voro_cu-zr} in \nameref{c:supple}). The \gls{ilo} is observed to be higher for Cu atoms than for Zr atoms for all the six metallic glass samples that have been studied. The Cu-centred atoms make up the majority in contributing to \gls{ilo}. Also seen is the increasing \gls{ilo} in Cu-centred atoms in the \gls{camg}s with deposition energy. \par

\begin{figure}[!ht]
	\centering
	\begin{subfigure}{0.5\linewidth} \centering \includegraphics[width=0.9\textwidth]{1e10/voronoi-sort_3nm_Cu50Zr50_CAMG_vs_MG_1e10_asp_Cu} 
		\subcaption{} \end{subfigure}%
	\hfill
	\begin{subfigure}{0.5\linewidth} \centering \includegraphics[width=0.9\textwidth]{1e10/voronoi-sort_3nm_Cu50Zr50_CAMG_vs_MG_1e10_asp_Zr}
		\subcaption{} \end{subfigure}
	\mycaption{Sorted \acrlong{vp} for Cu- and Zr-centred atoms in MGs, NGs, and CAMGs}{The atomic fractions here are defined w.r.t the same species i.e., at. fraction of Cu is the fraction of Cu atoms exhibiting a certain local order. }
	\label{f:vorosort_cu-zr}
\end{figure}

%\todo[inline]{Want to add sorted voronoi core and interface, cu and zr centered in the appendix? decide}

\begin{changebar}
Looking back at Figure~\ref{f:voro_camg}e-f, it is observed that for \gls{ng}s and \gls{camg}s, both \gls{fi} order and the \gls{ilo} indices are respectively at least 1\% and 5\% higher in the interfaces compared to the cores. It is evident that the interfaces, being richer in Cu compared to the core regions, exhibit higher \gls{fi} as well as higher \gls{ilo}. This indicates that the interfaces must be better packed than the core regions\footnote{The correlation of ILO of cores and interfaces in the CAMGs and NGs, with their stability is discussed in Sections~\ref{s:camg-pe} and \ref{s:camg_quenchrt}}. \par

Interestingly, a systematic increase in \gls{fi} order (See Figure~\ref{f:voro_camg}c) in the \gls{camg}s with increasing impact energy is observed. This trend, also seen in the \gls{ilo} fractions of the entire sample, is especially prominent for the interfaces (see Figure~\ref{f:voro_camg}f). We recall that the interfaces are defined as the surface atoms of the undeposited clusters, and for this reason it should be noted that the interfaces for all impact energies have the same chemical composition. The increase in the \gls{fi}-order and \gls{ilo} in the interfacial atoms can then be interpreted to be the direct result of the \gls{cibd} process. In both the core and interface atoms, another striking feature is the systematic increase of \gls{ilo} in the \gls{camg}s with increasing deposition energy, recovering towards the \gls{ilo} of \gls{mg} and \gls{mght}. The present model demonstrates the possibility of tailoring the local amorphous order with impact energy for metallic glasses synthesised via the \gls{cibd} route.
\end{changebar}

%\section{Atomic Volume Analysis}
\begin{selfcite}
The normalized distribution of the atomic coordination volumes for all the atoms in the six metallic glass samples was studied using the Voronoi analysis. Figure~\ref{f:vol_camg}a shows the volume distribution with two peaks approximately at 13.9 \acu for Cu and 21.8 \acu for Zr. The Cu atoms are observed to have lower occupancy numbers compared to those of Zr atoms. \textcite{Cheng2019} reported similar distributions in \czsix nanoglasses, however, with the volume per atom peaks shifted to the left, likely caused by the fact that \czsix MGs are denser than the \cz MGs \cite{Li2008}. Furthermore, the CAMGs are similar to the NGs and MGs in terms of atomic volume distributions. It is also noticed that the distributions in the core and in the interfaces are not significantly different from each other (see Figure~\ref{f:atvol} in \nameref{c:supple}). Using the volume distributions in Figure~\ref{f:vol_camg}a, the exclusion of the surface atoms in the analysis (detailed in Section~\ref{s:corint}) of CAMGs was cross-verified. The surface atoms occupy higher volume per atom than average, and when included in the volume analysis, are known to alter the volume distributions of Cu and Zr atoms with a shoulder to the right of each main peak \cite{Cheng2019}. The absence of such shoulders indeed ascertains the absence of surface atoms in the representative Figure~\ref{f:vol_camg}a. \par

\begin{figure}[!h] \centering
	\includegraphics[width=0.65\columnwidth,trim={0 0 2cm 0.5cm},clip]{2021_05/figures/7.pdf}
	\mycaption{Reduced volumes in cluster assembled metallic glass samples}{ (a) The normalized volume per atom distribution shows
		similar behavior for the six metallic glass samples. (b) Average volumes of the atoms show that the core regions are less densely packed
		than the interfaces}
	\label{f:vol_camg}
\end{figure}

\todo[inline]{should I be distinguishing between NGs and CCMGs in this chapter, because I will be talking about CCMGs in the next chapter}

In Figure~\ref{f:vol_camg}b, which shows the average volume/atom values for all six metallic glass samples, it can be seen that, on average, the atoms in CAMGs and NGs occupy similar volumes. Furthermore, the impact energy does not have an influence on the average volume of the CAMGs. The core regions in the NGs and CAMGs present a higher volume occupancy. By contrast, the opposite behavior is observed for the interfacial atoms. The increase of volume for core atoms and the decrease of volume for the interface atoms in CAMGs and NGs offset each other to result in similar volume occupancies as MGs prepared by RQ, when all atoms of the are considered together. While the interfaces in NGs have previously been reported to be less dense in the MG \cite{Sopu2009,Witte2013}, this need not hold true for the CAMGs as well. The interfaces are richer in Cu-atoms, which, on an average, occupy lower volumes compared to Zr-atoms. The interfaces in the present CAMGs model have an increased density due to the chemical effects, and this is consistent with previous studies of segregated planar interfaces by Adjaoud and Albe \cite{Adjaoud2016}.
\end{selfcite}
\section{Atomic Volume Analysis} \label{s:atvolcamg}
\begin{changebar}
The normalised distribution of the atomic coordination volumes, or Voronoi volumes \cite{Cheng2019,Lu2018} for all the atoms in the six metallic glass samples was studied using the Voronoi analysis. Figure~\ref{f:vol_camg}a shows the volume distribution with two peaks approximately at 13.9 \acu for Cu and 21.8 \acu for Zr, which indicates the volumes occupied by the Cu and Zr atoms. The Cu atoms are observed to have lower volume occupancies compared to those of Zr atoms. \textcite{Cheng2019} reported similar distributions in \czsix nanoglasses, however, with the volume per atom peaks shifted to the left, likely caused by the fact that \czsix \gls{mg}s are denser than the \cz \gls{mg}s \cite{Li2008}. Furthermore, the \gls{camg}s are similar to the \gls{ng}s and \gls{mg}s in terms of atomic volume distributions. It is also noticed that the distributions in the core and in the interfaces are not significantly different from each other (see Figure~\ref{f:atvol} in \nameref{c:supple}). Using the volume distributions in Figure~\ref{f:vol_camg}a, the exclusion of the surface atoms in the analysis (detailed in Section~\ref{s:corint}) of \gls{camg}s was cross-verified. The surface atoms occupy higher volume per atom than average, and when included in the volume analysis, are known to alter the volume distributions of Cu and Zr atoms with a shoulder to the right of each main peak \cite{Cheng2019}. The absence of such shoulders indeed ascertains the absence of surface atoms in the representative Figure~\ref{f:vol_camg}a. \par

\begin{figure}[!h] \centering
%	\includegraphics[width=0.65\columnwidth,trim={0 0 2cm 0.5cm},clip]{camg_3nm7.pdf}
	\includegraphics[width=\textwidth,trim={0.6cm 2.8cm 0.65cm 2.1cm},clip]{camg_3nm/7.pdf}
	\mycaption{Reduced volumes in CAMG samples}{(a) The normalised volume per atom distribution shows
		similar behavior for the six metallic glass samples. (b) Average volumes of the atoms show that the core regions are less densely packed
		than the interfaces.}
	\label{f:vol_camg}
\end{figure}

%\todo[inline]{should I be distinguishing between \gls{ng}s and \gls{ng}s in this chapter, because I will be talking about \gls{ng}s in the next chapter}

In Figure~\ref{f:vol_camg}b, which shows the average volume/atom values for all six metallic glass samples, it can be seen that, on average, the atoms in \gls{camg}s and \gls{ng}s occupy similar volumes. Furthermore, the impact energy does not have an influence on the average volume of the \gls{camg}s. The core regions in the \gls{ng}s and \gls{camg}s present a higher volume occupancy. By contrast, the opposite behavior is observed for the interfacial atoms. When all of the atoms are considered together, the increase of volume for core atoms and the decrease of volume for the interface atoms in CAMGs and NGs offset each other to result in similar volume occupancies as MGs prepared by RQ. While the interfaces in \gls{ng}s have previously been reported to be less dense in the \gls{mg} \cite{Sopu2009,Witte2013}, this need not hold true for the \gls{camg}s as well. The interfaces are richer in Cu-atoms, which, on an average, occupy lower volumes compared to Zr-atoms. The interfaces in the present \gls{camg}s model have an increased density due to the chemical effects, and this is consistent with previous studies of segregated planar interfaces by Adjaoud and Albe \cite{Adjaoud2016}.
\end{changebar}

%\section{Potential Energy Inspection} \label{s:camg-pe}

\begin{selfcite}
In Figure~\ref{f:pe_camg}a, the normalized potential energy distribution of the simulated CAMGs, NG and MGs are summarized. The Cu and Zr atoms exhibit two separate distributions, with the peaks of -3.5 eV for Cu atoms and -6.5 eV for Zr atoms. By contrast to the atomic volume distribution behavior, where the Cu atoms occupied lower volumes, the Cu atoms have a higher potential energy overall in comparison to Zr atoms. The absence of right shoulders in the potential energy distribution peaks, like in the volume distributions discussed in Section~\ref{s:atvolcamg}, once again confirms the absence of surface atoms in the representative CAMG slabs. \par

\begin{figure}[!h] \centering
	\includegraphics[width=0.65\columnwidth,trim={0 0 2cm 0.5cm},clip]{2021_05/figures/8.pdf}
	\mycaption{Potential energy states of the CAMGs, NG and MGs: Figure~\ref{f:pe_camg}(a) shows the normalized potential energy distributions for the six glasses, whereas Figure~\ref{f:pe_camg}(b) shows the average potential energy per atom, also in the core and interface regions for all the glasses.}
	\label{f:pe_camg}
\end{figure}

\todo[inline]{write speculations about PEL}

Figure~\ref{f:pe_camg}b summarizes the average potential energies for all atoms in the MG, \gls{mght}, NG, and the CAMGs and the average potential energies of the atoms in the core and interfacial regions in the NG and CAMGs, deposited at the different impact energies. All of these six metallic glasses samples have been made from a \qr{10} \cz glass. It is observed that the core and interfacial atoms in the CAMGs and NGs can be distinguished by their energetic states. The core atoms occupy lower energy states, about 2\% lower than that of the atoms in MGs prepared by RQ, whereas the interfaces possess higher energies ~ 6\% higher compared to those of the MGs. While the interfaces are better packed than the cores, as seen in Section~\ref{s:atvolcamg}, they occupy a higher energy state than the MGs. From liquid quenched traditional metallic glasses simulated in Section~\ref{s:s:simtestMG} it is known that the total potential energy of the glass increases with increasing Cu concentration maintaining the same quenching rate (See Figure~\ref{f:pe_mgs}). Hence, it can be explained that the higher Cu concentration in the interfaces, results in the interface atoms residing at a higher energy state. We conclude that the core regions stabilize the CAMGs and the NGs. Denser packing at the interfaces does not necessarily correspond to a lower energy state in the NGs and CAMGs due to their chemical heterogeneity. \par

In the following section, the medium-range order (MRO) in all six metallic glasses (3 CAMGs, NG, MG and \gls{mght}) will be inspected to better understand the connectivity of the FI units in the MG and \gls{mght} and how the MRO varies in the CAMGs with the deposition energy.
\end{selfcite}
\section{Potential Energy Inspection} \label{s:camg-pe}

\begin{changebar}
In Figure~\ref{f:pe_camg}a, the normalised \gls{pe} distribution of the simulated \gls{camg}s, \gls{ng} and \gls{mg}s are summarised. The Cu and Zr atoms exhibit two separate distributions, with the peaks of -3.5 eV for Cu atoms and $\sim$-6.4 eV for Zr atoms. By contrast to the atomic volume distribution behavior, where the Cu atoms occupied lower volumes, the Cu atoms have a higher \gls{pe} overall in comparison to Zr atoms. The absence of right shoulders in the \gls{pe} distribution peaks, like in the volume distributions discussed in Section~\ref{s:atvolcamg}, once again confirms the absence of surface atoms in the representative \gls{camg} slabs. \par

\begin{figure} %[!h]
	\centering
%	\includegraphics[width=0.65\columnwidth,trim={0 0 2cm 0.5cm},clip]{camg_3nm8.pdf}
	\includegraphics[width=\textwidth,trim={0.5cm 2.5cm 0.4cm 2.05cm},clip]{camg_3nm/8.pdf}
	\mycaption{\gls{pe}/atom states of the CAMGs, NG and the MGs}{(a) The normalised \gls{pe} distributions for the six glasses. (b) The average \gls{pe} per atom, also in the core and interface regions for all the glasses.}
	\label{f:pe_camg}
\end{figure}

Figure~\ref{f:pe_camg}b summarises the average potential energies for all atoms in the \gls{mg}, \gls{mght}, \gls{ng}, and the \gls{camg}s and the average potential energies of the atoms in the core and interfacial regions in the \gls{ng} and \gls{camg}s, deposited at the different impact energies. All of these six metallic glasses samples have been made from a \qr{10} \cz glass. It is observed that the core and interfacial atoms in the \gls{camg}s and \gls{ng}s can be distinguished by their energetic states. The core atoms occupy lower energy states, about 2\% lower than that of the atoms in \gls{mg}s prepared by \gls{rq}, whereas the interfaces possess higher energies ~ 6\% higher compared to those of the \gls{mg}s. While the interfaces are better packed than the cores, as seen in Section~\ref{s:atvolcamg}, they occupy a higher energy state than the \gls{mg}s. From liquid quenched traditional metallic glasses simulated in Section~\ref{s:simtestMG}, it is known that the total \gls{pe} of the glass increases with increasing Cu concentration maintaining the same quenching rate (See Figure~\ref{f:pe_mgs}). Hence, it can be explained that the higher Cu concentration in the interfaces, results in the interface atoms residing at a higher energy state. We conclude that the core regions stabilise the \gls{camg}s and the \gls{ng}s. Denser packing at the interfaces does not necessarily correspond to a lower energy state in the \gls{ng}s and \gls{camg}s due to their chemical heterogeneity. \par
\end{changebar}
\clearpage

%\section{Medium-range order in CAMGs}

\begin{selfcite}
The relative packing of coordination polyhedra centered around solute atoms is used to define MRO in metallic glasses; it has been shown that the solute atoms exhibit string-like connectivity when the solute concentration goes beyond 20-30 at.\%, \cite{Sheng2006}. Similarly, the string-like connectivity of FI-atoms, which are the atoms residing in FI coordinations, have been reported to indicate MRO, as icosahedral \vi{0}{0}{12}{0} clusters have a strong tendency to aggregate with each other \cite{Bernal1959,Miracle2004,Li2008}. Interpenetrating string-like networks of atoms in FI environments have been reported before as indicative of MRO, including the study by Lee et al. \cite{Lee2011} and by Ritter et al. [\cite{Ritter2011}. To visualize these strings for the glasses simulated in this work, bonds were constructed  using \gls{ovito} for FI-atoms with other FI-atoms, present within a cut-off radius of 3.5 \r{A}. \par 
\end{selfcite}

Figure~\ref{f:mro-alt_camg}a shows one of the FI-atom chains found in one of the CAMGs. The yellow coloured atoms are the ones in the FI environment, the blue atoms are the surrounding atoms in the coordination polyhedron. In Figure~\ref{f:mro-alt_camg}b, the number of strings (\% of total strings) below a maximum string length are described for for all FI-atoms (Figure~\ref{f:mro-alt_camg}b1), FI-atoms in the cores (Figure~\ref{f:mro-alt_camg}b2) and FI-atoms in the interfaces (Figure~\ref{f:mro-alt_camg}b3) for the \gls{ng}s and \gls{camg}s. The number of small strings of sizes 3-9 recover in the CAMGs, specifically in the interfaces towards MG and \gls{mght} values with increasing deposition energies. In the glasses, at least 74\% of all linked atoms are in 3-atom or longer string-like networks. This behaviour is seen in both the cores and the interfaces of the \gls{camg}s and \gls{ng}s as well. The \gls{mro} in all the glasses in the context of the larger strings of length 40-100 is similar. \par

\begin{figure} %[!ht]
	\centering
	\begin{subfigure}[b]{0.5\textwidth} \centering
	\includegraphics[width=\textwidth]{mrochain2}	\subcaption{}
	\end{subfigure}%
	\vfill
	\begin{subfigure}{\textwidth} \centering
		\includegraphics[width=\textwidth,trim={0cm 1cm 0cm 1cm},clip]{2021_05/figures/12.pdf}	\subcaption{}
	\end{subfigure}%
	\mycaption{MRO in MG, \gls{mght}, NG and CAMGs}{}
	\label{f:mro-alt_camg}
\end{figure}

\begin{selfcite}
Figure~\ref{f:mro_camg}a shows the fraction of FI-atoms in each of the glasses made from a \qr{10} MG, which are present in a string of a given size. From both Figures~\ref{f:mro-alt_camg}b~and~\ref{f:mro_camg}a, it is evident that most FI-atoms exist in small strings. However, the number of atoms in small strings (3-5 FI-atoms in size) is the lowest in the MG and \gls{mght}. A higher percentage of larger-sized strings is seen in the MG, \gls{mght} and NG. This can be attributed to the geometry of the samples: larger strings can form in MG, \gls{mght}, and NG cases due to periodic boundaries conditions in all directions, and the simulation box being completely filled. Such strings of larger sizes cannot be expected to form in the CAMGs, as the sample considered for analysis without surface artifacts is limited by the dimensions of the representative slabs from within the CAMG films. However, amongst the 3 CAMGs, it can still be noted that a 600 meV/atom CAMG has more 3 FI-atom strings than the 60 meV/atom CAMG. This trend is seen for strings of at least 5 FI-atoms in length. \par

Figure~\ref{f:mro_camg}b shows the average string size for all simulated glasses. The average string size in NG and CAMGs is about 40\% lower than for MG and \gls{mght}. However, with increasing deposition energy, a slight increase in average string size in the CAMGs deposited at 60 and 300 meV/atom compared to the 60 meV/atom CAMG is observed. This indicates that with increasing deposition energies in the CAMGs, the MRO of the strings of FI-atoms can be at least partially recovered. A comparison of the present results with the available experimental data including the structural and magnetic information on \fs CAMGs is not possible as the current simulations are specific to the \cz metallic glass, and also due to the non-availability of an EAM potential for \fs systems. However, some conclusions on the behavior of cluster-assembled glasses, in particular on the of the medium-range order in CAMGs, can help to better understand the experimental results for \fs CAMGs, in particular the comparison to the local motif analysis reported in \cite{Benel2019}. As the local order in CAMGs recovers towards the metallic glass values with increasing impact energies, an increase in the size of the string-like MRO networks is expected. This behavior could explain the strengthening of exchange interactions, and thus the observed increase in the ferromagnetic transition temperature (Curie temperature) with increasing impact energy. \par

\begin{figure}%[!ht]
	\centering
	\includegraphics[width=\textwidth,trim={2cm 0cm 2cm 0cm},clip]{2021_05/figures/9.pdf}
	\includegraphics[width=\textwidth,trim={2cm 0cm 2cm 0cm},clip]{2021_05/figures/10.pdf}
	\mycaption{MRO in 3 nm CAMGs and NGs}{}
	\label{f:mro_camg}
\end{figure}
%\end{selfcite}

%\begin{selfcite}
Next, in Section~\ref{c:camg_quenchrt}, an attempt is made to explain the dependence of the final metastable states of the NG and CAMGs on the initial processing conditions of the clusters themselves. This could help to understand the CIBD process and would allow to traverse the \gls{pel} of metallic glasses. One important processing condition in the simulation is the quenching rate of the MG from which the clusters are formed. Therefore, a comparison of CAMGs and NGs, prepared with different quenching rates is made. \par
\end{selfcite}


%\newpage
\section{Medium-range Order in CAMGs}

\begin{changebar}
In this section, the \gls{mro} in all six metallic glasses (3 \gls{camg}s, \gls{ng}, \gls{mg} and \gls{mght}) shall be inspected to better understand the connectivity of the \gls{fi} units in the \gls{mg} and \gls{mght} and how the \gls{mro} varies in the \gls{camg}s with the deposition energy. \par

\begin{figure}[!h]
	\centering
	\begin{subfigure}[b]{0.5\textwidth} \centering
		\includegraphics[width=\textwidth,trim={0 2.6cm 0 2.1cm},clip]{mrochain2}	\subcaption{}
	\end{subfigure}%
	\vfill
	\begin{subfigure}{\textwidth} \centering
		\includegraphics[width=\textwidth,trim={0cm 1.5cm 0.5cm 1.2cm},clip]{camg_3nm/12.pdf}	\subcaption{}
	\end{subfigure}%
	\mycaption{FI-strings in MGs, \gls{mght}, NG and CAMGs}{(a) An FI-string (yellow atoms) seen in the 60 meV/atom \gls{camg} (b) Number of atoms (\%) in the glass samples having strings of sizes less than a cut-off value. See text for more details.}
	\label{f:mro-alt_camg}
\end{figure}

The relative packing of coordination polyhedra centred around solute atoms is used to define \gls{mro} in metallic glasses; it has been shown that the solute atoms exhibit string-like connectivity when the solute concentration goes beyond 20-30 at.\%, \cite{Sheng2006}. Similarly, the string-like connectivity of \gls{fi}-atoms, which are the atoms residing in \gls{fi} coordinations, have been reported to indicate \gls{mro}, as icosahedral \vi{0}{0}{12}{0} clusters have a strong tendency to aggregate with each other \cite{Bernal1959,Miracle2004,Li2008}. Interpenetrating string-like networks of atoms in \gls{fi} environments have been reported before as indicative of \gls{mro}, including the studies by \textcite{Lee2011} and \textcite{Ritter2011}. To visualise these strings for the glasses simulated in this work, bonds were constructed  using \gls{ovito} for \gls{fi}-atoms with other \gls{fi}-atoms, present within a cut-off radius of 3.5 \r{A}. \par 
\end{changebar}

\begin{changebar}
Figure~\ref{f:mro-alt_camg}a shows one of the \gls{fi}-atom chains found in one of the \gls{camg}s. The yellow coloured atoms are the ones in the \gls{fi} environment, the blue atoms are the surrounding atoms in the coordination polyhedron. In Figure~\ref{f:mro-alt_camg}b, the number of strings (\% of total strings) below a maximum string length are described for all \gls{fi}-atoms (Figure~\ref{f:mro-alt_camg}b1), \gls{fi}-atoms in the cores (Figure~\ref{f:mro-alt_camg}b2) and \gls{fi}-atoms in the interfaces (Figure~\ref{f:mro-alt_camg}b3) for the \gls{ng}s and \gls{camg}s. \par The number of small strings of sizes 3-9 recover in the \gls{camg}s, specifically in the interfaces towards \gls{mg} and \gls{mght} values with increasing deposition energies. In the glasses, at least 74\% of all linked atoms are in 3-atom or longer string-like networks. This behaviour is seen in both the cores and the interfaces of the \gls{camg}s and \gls{ng}s as well. The \gls{mro}s of all the glasses in the context of the larger strings of length 40-100 are similar as characterised by the chosen metrics. \par

\begin{figure}[!h]
	\centering
	\begin{subfigure}{0.5\textwidth} \centering
		\includegraphics[width=\textwidth,trim={4cm 1cm 4cm 1cm},clip]{camg_3nm/9.pdf}
	\end{subfigure}%
	\hfill
	\begin{subfigure}{0.5\textwidth} \centering
		\includegraphics[width=\textwidth,trim={4cm 1cm 4cm 1cm},clip]{camg_3nm/10.pdf}
	\end{subfigure}%
	\mycaption{MRO in 3 nm \gls{camg}s and \gls{ng}s}{(a) Distribution of the number (\%) of atoms existing in FI-strings of varying sizes  (b) Average FI string size in the six glasses.}
	\label{f:mro_camg}
\end{figure}

Figure~\ref{f:mro_camg}a shows the fraction of \gls{fi}-atoms in each of the glasses made from a \qr{10} \gls{mg}, which are present in a string of a given size. From both Figures~\ref{f:mro-alt_camg}b~and~\ref{f:mro_camg}a, it is evident that most \gls{fi}-atoms exist in small strings. However, the number of atoms in small strings (3-5 \gls{fi}-atoms in size) is the lowest in the \gls{mg} and \gls{mght}. A higher percentage of larger-sized strings is seen in the \gls{mg}, \gls{mght} and \gls{ng}. This can be attributed to the geometry of the samples: larger strings can form in \gls{mg}, \gls{mght}, and \gls{ng} cases due to periodic boundaries conditions in all directions, and the simulation box being completely filled. Such strings of larger sizes cannot be expected to form in the \gls{camg}s, as the sample considered for analysis without surface artefacts is limited by the dimensions of the representative slabs from within the \gls{camg} films. However, amongst the 3 \gls{camg}s, it can still be noted that a 600 meV/atom \gls{camg} has more 3 \gls{fi}-atom strings than the 60 meV/atom \gls{camg}. This trend is seen for strings of at least 5 \gls{fi}-atoms in length. \par

Figure~\ref{f:mro_camg}b shows the average string size for all simulated glasses. The average string size in \gls{ng} and \gls{camg}s is about 40\% lower than for \gls{mg} and \gls{mght}. However, with increasing deposition energy, a slight increase in average string size in the \gls{camg}s deposited at 300 and 600 meV/atom compared to the 60 meV/atom \gls{camg} is observed. This indicates that with increasing deposition energies in the \gls{camg}s, the \gls{mro} of the strings of \gls{fi}-atoms can be at least partially recovered. \par

A comparison of the present results with the available experimental data including the structural and magnetic information on \fs \gls{camg}s \cite{Benel2019} is not possible as the current simulations are specific to the \cz metallic glass, and also due to the non-availability of an \gls{eam} potential for \fs systems. However, some conclusions on the behavior of cluster-assembled glasses, in particular on the of the medium-range order in \gls{camg}s, can help to better understand the experimental results for \fs \gls{camg}s, in particular the comparison to the local motif analysis reported in \cite{Benel2019}. As the local order in \gls{camg}s recovers towards the metallic glass values with increasing impact energies, an increase in the size of the string-like \gls{mro} networks is expected. This behavior could explain the strengthening of exchange interactions, and thus the observed increase in the \gls{tc}, i.e., the ferromagnetic transition temperature with increasing impact energy. \par

%\end{changebar}

%\begin{changebar}
\end{changebar}

%\section{CAMG behaviour across quench rates} \label{s:camg_quenchrt}

\begin{selfcite}
In this section, an attempt is made to explain the dependence of the final metastable states of the NG and CAMGs on the initial processing conditions of the clusters themselves. This could help to understand the CIBD process and would allow one to traverse the \gls{pel} of metallic glasses. One important processing condition in the simulation is the quenching rate of the MG from which the clusters are formed. Therefore, a comparison of CAMGs and NGs, prepared with different quenching rates is made. \par
\end{selfcite}

The MG, \gls{mght}s, were prepared at three quench rates of \qr{10}, \qr{12} and \qr{14}. Subsequently, clusters were derived for the three quench rates, and the CAMGs and NGs were prepared from them. All the glasses were compared against one another in terms of local icosahedral \gls{sro} and potential energy. As mention in Section~\ref{s:vol-mgs}, the current potential does not correctly reproduce the volume behaviour of RQ MGs with quench rates, and hence the CAMGs volume behaviour with quench rate not attempted to be explained. \par

Figure~\ref{f:camg_fi} describes the \gls{fi} order in the glasses. The fraction of FI in RQ MGs is highest at 7\% for the \qr{10} MG, and at 5.5\% and 3.8\% for the \qr{12} and \qr{14} MGs respectively. The \gls{mght} FI drops at low quench rate of \qr{10}, but increases at \qr{14}, revealing that the heat-treatment causes different effects at different quench rates. The \gls{cbmg}s appear to consistently demonstrate lesser FI-packed states for the three quench rates. For the CBMGs, the FI-order in the core ad interfacial regions are also represented. The interfaces possess higher FI-packing than the cores. The order in entire CBMG sample is depicted as 'All' in the Figure~\ref{f:camg_fi}. Amongst the CBMGs, the NG/CCMGs have the lowest FI-order. For the three quench rates, a trend of increase in the FI-order with increasing deposition energy is hinted; as the FI-600 meV/atom CAMG $\geq$ FI-60 meV/atom CAMG. However the FI-300 meV/atom CAMG does not follow this trend for the \qr{12} and \qr{14} cases. The \gls{ilike} was considered a better candidate to explore this trend. \par

\begin{figure} %[!htp]
	\includegraphics[width=\textwidth]{collated/ico_coll_3nm_Cu50Zr50_CAMG_vs_MG}
	\mycaption{Full-Icosahedral ordering versus quench rates}{Variation of FI in the various CAMGs and MGs, in comparison to their precursor MGs and \gls{mght}s is uninfluenced by the quench rate used in the processing.}
	\label{f:camg_fi}
\end{figure}

\begin{selfcite}
Previously in literature \cite{Ding2014} it has been discussed that the discintinction bewteen FI and CIolike coorindations is a fine line, owing the definition of the threshold in the voronoi tessealation. In the previous sections, it has been mentioned that the ILO is a well-known indicator of glass stability and of packing \cite{Ding2014,Ding2014a,Adelman1976}. Figure~\ref{f:camg_ilo} shows the variation of the ILO for three different quenching rates of the MGs from which the clusters were prepared. Firstly, it can be seen that the ILO of the MG decreases with increasing quenching rates. Secondly, the heat-treatment for the MGs results in different dependence for the different quenching rates. For the MGs prepared with cooling rates \qr{10} and \qr{12} (Figure~\ref{f:camg_ilo}a, Figure~\ref{f:camg_ilo}b), the heat-treated glasses \gls{mght} exhibit lower ILO than the MG. At the highest cooling rate of \qr{14}, the same heat-treatment places the \gls{mght} at a state with higher ILO value (Figure~\ref{f:camg_ilo}c). Given that the clusters undergo the same heat-treatment as \gls{mght}, the NGs and CAMGs can be compared with the \gls{mght}. \par

At all quenching rates, it is noted that the interfaces exhibit higher SRO than the cores, due to the chemical effects discussed in Section 3.5. The CAMGs, however, show an increase in the ILO with increasing impact energies, for all the three quench rates used in the present study. Therefore, it is concluded that for a given cooling rate of the as-prepared clusters, the CIBD process determines the final states of the CAMGs. \par

\begin{figure} %[!htp]
	\includegraphics[width=\textwidth,trim={0.65cm 2.4cm 0.65cm 1.8cm},clip]{2021_05/figures/11.pdf}
	\mycaption{Icosahedral-like ordering versus quench rates in CBMGs}{Variation of ILO in the various CBMGs and MGs, compared to their
		precursor MGs and \gls{mght}s for cluster derived from (a) \qr{10}, (b) \qr{12}, and (c) \qr{14} \cz RQ MGs.}
	\label{f:camg_ilo}
\end{figure}

Unlike in the MG and the \gls{mght}, the ILO packing and stability do not correlate with each other in the CAMGs and NGs. For a CAMG prepared at a given impact energy, the ILO packing increases with quenching rate. The strain energy due to CAMG processing could be dominating the stability gained from the slow quenching rate in the NGs and CAMGs, leading to the observed behavior. In Figure~\ref{f:film_network}d, the strain analysis shows that von-Mises strains for the 3 nm clusters studied here are higher in both NGs and CAMGs, compared to previous reports for 7 nm cluster NGs \cite{Adjaoud2018}. This leads to the conclusion that the size of the clusters plays an important role in the final structures attained by the CAMGs. Further studies with different cluster sizes, including size distributions, and random deposition locations are needed to gain further understanding on the role of the processing parameters in cluster assembled metallic glasses prepared by compaction (NGs) and by energetic impact (CAMGs). \par
\end{selfcite}

\begin{quote}
Figure~\ref{f:camg_pote}a shows the average potential energy per atom for the six glasses, for the three cooling rates. Also, a possible representation of the PEL has been illusterate and overlayed upon the potential energy states, in Figure~\ref{f:camg_pote}b. We observe that while in the low cooling rate cases of \qr{10} and \qr{12} \cz glasses, the \gls{mght}s are at a higher energy states, indicating a rejuvenation process \cite{Wakeda2015,Saida2017}. This is consistent with the ILO behaviour of MG and \gls{mght}. The CAMGs and NGs however, seem to be at higher energy states than that MG \todo[inline]{working out an argument; this is not true for \qr{12} 300 meV/atom case)}. The ILO packing and stability are not correlated in the CAMGs and NGs. While the general increasing trend with deposition energy is seen at all quench rates, the quench rate itself has little influence on the final state of these cluster. This indicates that if \gls{mght} are sitting at one local minima for their respective quench rates in the potential energy landscape, the CAMGs and NGs are displaced to a nearest slightly lower local minima. The answer to why the CAMGs and NGs remain unaffected by quench rates could be the deformation processes in these glasses (competition between strain energy and stability gain from slowing quench rate). We refer back to Figure 4, where the strain analysis shows that von-Mises strains for the 3nm clusters are quite high for both NGs and CAMGs, especially in comparison with previous reports for 7nm cluster NGs \cite{Adjaoud2018}, leading us to believe that size of the clusters may indeed also play a role in those complex phenomena. \par
\end{quote}

\begin{figure}[!htp]
	\begin{subfigure}{\textwidth}
		\includegraphics[width=\textwidth]{collated/pote_coll_3nm_Cu50Zr50_CAMG_vs_MG}
	\end{subfigure}%
	\vfill
	\begin{subfigure}{\textwidth}
	\includegraphics[width=\textwidth]{pe-ldscp.png}
	\end{subfigure}%
	\mycaption{Energy versus quench rates}{(a) Variation of FI in the various CAMGs and MGs, in comparison to their precursor MGs and MG$_{ht}$s is uninfluenced by the quench rate used in the processing. (b) Average potential energy per atom in the metallic glasses, along with an imagined depiction of the \gls{pel}.}
	\label{f:camg_pote}
\end{figure}
\section{Influence of Quench Rates on Final Structure of CAMGs} \label{s:camg_quenchrt}

\begin{changebar}
In this section, we explore the dependence of the final metastable states of the \gls{ng} and \gls{camg}s on the initial processing conditions of the clusters themselves. This could help to understand the \gls{cibd} process and would allow one to traverse the \gls{pel} of metallic glasses. One important processing condition in the simulation is the quenching rate of the \gls{mg} from which the clusters are formed. Therefore a comparison of \gls{camg}s and \gls{ng}s, prepared with different quenching rates, is performed. \par
\end{changebar}

The \gls{mg}, and \gls{mght}s, were prepared at three quench rates of \qr{10}, \qr{12} and \qr{14}. Subsequently, clusters were derived from them for the three quench rates, using which the \gls{camg}s and the \gls{ng}s were prepared. All the glasses were compared against one another in terms of local icosahedral \gls{sro} and potential energy. As mention in Section~\ref{s:vol-mgs}, the current potential does not correctly reproduce the volume behaviour of \gls{rq} \gls{mg}s with quench rates, and hence the comparison glasses' volume behaviour with quench rates is not attempted. \par

Figure~\ref{f:camg_fi} describes the \gls{fi} order in the glasses. The fraction of \gls{fi} in \gls{rq} \gls{mg}s is highest at 7\% for the \qr{10} \gls{mg}, and at 5.5\% and 3.8\% for the \qr{12} and \qr{14} \gls{mg}s respectively. The \gls{mght} \gls{fi} drops at low quench rate of \qr{10}, but increases at \qr{14}, revealing that the heat-treatment causes different effects at different quench rates. Compared to the \gls{mght}, the NGs and \gls{camg}s appear to consistently demonstrate lesser \gls{fi}-packed states for the three quench rates. For the \gls{camg}s and \gls{ng}s, the \gls{fi}-order in the core and interfacial regions are also represented. As seen in Section~\ref{f:voro_camg}, the interfaces possess higher \gls{fi}-packing than the cores for the \gls{camg}s and \gls{ng}s. The order in the entire NGs and \gls{camg}s sample is depicted as 'All' in the Figure~\ref{f:camg_fi}. Amongst the \gls{camg}s and \gls{ng}s, the \gls{ng}s have the lowest \gls{fi}-order. For the three quench rates, a trend of increase in the \gls{fi}-order with increasing deposition energy is hinted; as the \gls{fi}-order for the 600 meV/atom \gls{camg} is greater than that for the 60 meV/atom \gls{camg}. However the \gls{fi}-300 meV/atom \gls{camg} does not follow this trend for the \qr{12} and \qr{14} cases. The \gls{ilike} was considered as an alternative candidate to explore \gls{sro}. \par

\begin{figure}[!h] \centering
\includegraphics[width=\textwidth]{collated/ico_coll_3nm_Cu50Zr50_CAMG_vs_MG}
\mycaption{Full-icosahedral ordering versus quench rates}{The \gls{fi} in the \gls{camg}s and \gls{mg}s, in comparison to their precursor \gls{mg}s and \gls{mght}s is invariant with the quench rate used.}
\label{f:camg_fi}
\end{figure}

\begin{changebar}
It has been discussed previously in literature \cite{Ding2014} that the distinction between \gls{fi} and \gls{ilike} coordinations is ambiguous, owing to the defined threshold of a distortion tolerance set while performing the Voronoi tessellation. In the previous sections, it has also been mentioned that the \gls{ilo} is a well-known indicator of glass stability and of packing \cite{Ding2014,Ding2014a,Yue2018,Cheng2008}. Now, the ILO behavior in CAMGs is observed with respect to the processing conditions of the RQ MGs, from which the clusters are derived. CAMGs made from fast-quenched \qr{12}-MGs and 
\qr{14}-MGs were investigated. For these reasons, \gls{ilo}s of the six glasses are also examined. Figure~\ref{f:camg_ilo} shows the variation of the \gls{ilo} for three different quenching rates of the \gls{mg}s from which the clusters were prepared. Firstly, it can be seen that the \gls{ilo} in the \gls{mg} decreases with increasing quenching rates. Secondly, the heat-treatment for the \gls{mg}s results in different behaviours for the different quenching rates. For the \gls{mg}s prepared with cooling rates \qr{10} (Figure~\ref{f:camg_ilo}a) and \qr{12} (Figure~\ref{f:camg_ilo}b), the heat-treated glasses \gls{mght} exhibit lower \gls{ilo} than the \gls{mg}. At the highest cooling rate of \qr{14}, the same heat-treatment places the \gls{mght} at a state with higher \gls{ilo} value (Figure~\ref{f:camg_ilo}c). Given that the clusters undergo the same heat-treatment as \gls{mght}, the \gls{ng}s and \gls{camg}s can be compared with the \gls{mght}. \par

At all quenching rates, it is noted that the interfaces exhibit higher \gls{ilike} \gls{sro} than the cores, due to the chemical effects discussed in Section~\ref{s:vorocamg}. The \gls{camg}s, however, show an increase in the \gls{ilo} with increasing impact energies, for all the three quench rates used in the present study. Therefore, it is concluded that for a given cooling rate of the as-prepared clusters, the \gls{cibd} process determines the final \gls{ilike} states of the \gls{camg}s. \par

\begin{figure}[!h]
	\includegraphics[width=\textwidth,trim={0.65cm 2.4cm 0.65cm 1.8cm},clip]{camg_3nm/11.pdf}
	\mycaption{Icosahedral-like ordering versus quench rates in CAMGs}{Variation of \gls{ilo} in the various NGs and \gls{camg}s and \gls{mg}s, compared to their precursor \gls{mg}s and \gls{mght}s for cluster derived from (a) \qr{10}, (b) \qr{12}, and (c) \qr{14} \cz \gls{rq} \gls{mg}s.}
	\label{f:camg_ilo}
\end{figure}
\end{changebar}

\begin{changebar}
Figure~\ref{f:camg_pote}a shows the average \gls{pe} per atom for the cores, interfaces and the entire sample for the six glasses, and for the three cooling rates. In Figure~\ref{f:camg_pote}b, the average \gls{pe} of only the entire sample is represented, and a possible representation of the \gls{pel} has also been illustrated and overlayed upon the plots. We observe that while in the low cooling rate cases of \qr{10} and \qr{12} \cz glasses, the \gls{mght}s are at a higher energy states, indicating a rejuvenation process \cite{Wakeda2015,Saida2017}. This is consistent with the \gls{ilo} behaviour of \gls{mg} and \gls{mght}. Furthermore, when examined across the quench rates, the potential energies of the glasses are lower at lower quench rates. \par
\end{changebar}

\begin{figure}[!h] %[!htp]
	\begin{subfigure}{\textwidth}
		\includegraphics[width=\textwidth,trim={0cm 0.45cm 0cm 0cm},clip]{collated/pote_coll_3nm_Cu50Zr50_CAMG_vs_MG}
	\end{subfigure}%
	\vfill
	\begin{subfigure}{\textwidth}
		\includegraphics[width=\textwidth,trim={0cm 0cm 0cm 0.75cm},clip]{pe-ldscp.png}
	\end{subfigure}%
	\mycaption{Average \gls{pe} of \gls{camg}s versus quench rates}{(a) The normalised \gls{pe} per atom in the various \gls{camg}s and \gls{ng}s, in comparison to their precursor \gls{mg}s and \gls{mght}s for three different quench rate cases. (b) The average \gls{pe} per atom in the metallic glasses, along with an imagined depiction of the \gls{pel}.}
	\label{f:camg_pote}
\end{figure}

The \gls{camg}s and \gls{ng}s however, for any given cooling rate, do not show any clear trend in the potential energies. Unlike in the \gls{mg} and the \gls{mght}, the stability and \gls{ilo} packing do not correlate with each other in the \gls{camg}s and \gls{ng}s. For a \gls{camg} prepared at a given impact energy, the \gls{ilo} packing increases with quenching rate. The \gls{ng} is consistently at a higher energy state than the \gls{mght} at all quench rates. In contrast, the potential energies of the \gls{camg}s decrease with impact energy for the lower quench rate case of \qr{10}, but at higher quench rates of \qr{12} and \qr{14}, the difference in the energy states of the \gls{ng}s and \gls{camg}s is seen to diminish. This seemingly random behaviour could be an effect of the local atomic strain accumulated during the deposition process. \par

\begin{changebar}
It may be that the strain energy has a lower effect for glasses quenched at lower rates. It may also be that the strain energy gained due to \gls{camg} processing could be erratic, and dominating the stability gained from the quenching in the \gls{ng}s and \gls{camg}s, leading to the observed behavior. In Figure~\ref{f:film_network}d, the strain analysis shows that von-Mises strains for the 3 nm clusters studied here are higher in both \gls{ng}s and \gls{camg}s, compared to previous reports for 7 nm cluster \gls{ng}s \cite{Adjaoud2018}. This leads to the conclusion that the size of the clusters plays an important role in the final structures attained by the \gls{camg}s. Further studies with different cluster sizes, including size distributions, and random deposition locations are needed to gain further understanding on the role of the processing parameters in cluster assembled metallic glasses prepared by compaction (\gls{ng}s) and by energetic impact (\gls{camg}s).
\end{changebar} \par

%\section{Summary}

\begin{selfcite}
In this study, a molecular dynamics simulation protocol was developed to study the local structure of \cz cluster-assembled metallic glasses (CAMGs). The present model of CAMGs uses chemically segregated amorphous Cu50Zr50 clusters of ~800 atoms, which are 3 nm in diameter, being deposited onto a substrate at different impact energies. These CAMGs are compared with NGs produced by mechanical compaction of the same clusters, and to the conventionally prepared melt-quenched metallic glasses of the same composition. The main results of the study are as follows:

1. In the CAMGs, two chemically distinct amorphous phases were observed: cores and interfaces, which constitute an interconnected network of interfaces in which the cores with their distinctly different local structure are embedded. The formation of Cu-rich interfaces is observed at impact energies up to 600 meV/atom. The interfaces appear to be completely absent at the extreme impact energy of 6000 meV/atom.
2. The FI and ILO short-range order parameters are lower in the NG and CAMG, for both the cores and interfaces. The interfaces exhibit higher FI and ILO compared to the cores, with a higher density than the cores. Due to the chemical heterogeneity between cores and interfaces, the core regions occupy lower energy states, thus stabilizing the CAMG structures.
3. The local ILO as well as the MRO in CAMGs are found to increase with the impact energy. Furthermore, the ILO increases with impact energy, irrespective of the quenching rates used to prepare the clusters. Consequently, at a fixed overall bulk composition of the metallic glass, control of the local structure is possible by modifying the processing conditions. The SRO and MRO in CAMGs recover towards the metallic glass values with increasing impact energies.
\end{selfcite}



\begin{figure}
	\includegraphics[width=\textwidth]{/home/mj0054/Documents/work/writing/articles/2021_05/figures/graphical_abstract2.pdf}
	\label{f:camg-summary}
	\mycaption{CAMG summary}{CAMG summary}
\end{figure}
\section{Summary}
\begin{changebar}
In this chapter, the simulated \cz \gls{camg}s were studied and characterised. The present model of \gls{camg}s uses chemically segregated amorphous \cz clusters of $\sim$800 atoms, which are 3 nm in diameter, being deposited onto a substrate at different impact energies. These \gls{camg}s are compared with \gls{ng}s produced by mechanical compaction of the same precursor clusters, and to the conventionally prepared melt-quenched metallic glasses of the same composition. \par

In the \gls{camg}s, two chemically distinct amorphous phases were observed: cores and interfaces, which constitute an interconnected network of interfaces in which the cores with their distinctly different local structure are embedded. The formation of Cu-rich interfaces is observed at impact energies up to 600 meV/atom. Due to the chemical heterogeneity between cores and interfaces, the core regions occupy lower energy states, thus stabilising the \gls{camg} structures. The interfaces appear to be completely absent at the extreme impact energy of 6000 meV/atom. \par

The \gls{fi} and \gls{ilo} short-range order parameters are lower in the \gls{ng} and \gls{camg}, in the conventional (\qr{10} quench rate) case, for both the cores and interfaces. The interfaces exhibit higher \gls{fi} and \gls{ilo} compared to the cores, with a higher density than the cores. A graphical depiction of these findings is represented in Figure~\ref{f:camg-summary}. \par

The present simulations constitute the first model explaining the mechanisms of local \gls{sro} and \gls{mro} tailoring in \gls{camg}s by increasing the impact energy. Furthermore, the \gls{ilo} increases with impact energy, irrespective of the quenching rates used to prepare the 3 nm sized clusters. Consequently, at a fixed overall bulk composition of the metallic glass, control of the local structure is possible by simply modifying the processing conditions. The \gls{sro} and \gls{mro} in \gls{camg}s recover towards the metallic glass values with increasing impact energies.


\begin{figure}
	\includegraphics[width=\textwidth,trim={0 3cm 0cm 2.1cm},clip]{camg_3nm/graphical_abstract2.pdf}
	\mycaption{\acrlong{sro} and \acrlong{mro} tailoring in CAMGs}{A visual summary of the formation of core-interface structures, and the tailoring of FI-order and FI-string size with deposition energy in \gls{camg}s.}
	\label{f:camg-summary}
\end{figure}
\end{changebar}

\chapter{Cluster size effects on Cluster-based Metallic Glasses} \label{c:cbmg}
\chapter{Cluster-size effects in Cluster-assembled Metallic Glasses} \label{c:cbmg}
In the last two chapters, a protocol was established to study the \gls{camg}, and contrast it with \gls{rq} \gls{mg} and \gls{ng}. We observed that the processing of \cz \gls{camg}s and \gls{ng} led to the formation of a core-interface network which influenced the structural characteristics of the materials. Consequently, the nature of the interfaces in terms of length scales are therefore expected to final states of the glassy systems. \par

For instance, it has recently been shown in various models of the \gls{ng}s that the diminution of the nanoparticle size or the ``grain size" improves the plasticity of the material drastically, characterized by a reduced flow stress \cite{Adibi2013,Adibi2014,Cheng2019a}. Adopting smaller particle sizes and \gls{pvd} have also lately been shown to correlate with thermal ultra-stability in small-chain polymer glasses experiments \cite{Raegen2020} and simulated \gls{lj} model-atoms \cite{Singh2013}. The current chapter discusses the attempts to explore such size effects in \gls{camg}s and \gls{ng}s. \par

\begin{figure}[!h] \centering
	\includegraphics[width=0.94\linewidth,trim={0 2cm 0 1.5cm},clip]{NG_1-7nm/9.pdf}
	\mycaption{Cluster-sizes and associated length scales}{\cz CAMG building blocks with diameters in the range of 1-7 nm. The 7 nm nanoparticle has $\sim$300 times as many atoms as the 1 nm cluster.}
	\label{f:clussizes}
\end{figure}

%\cz CAMG building blocks with diameters in the range of 1-7 nm.
%\cz CAMG building blocks ranging from 7 nm to 1 nm in diameter. 
%{\cz CAMG building blocks ranging from 7 nm nanoparticles to 1 nm clusters; the 7 nm nan in population range by $\sim$300 times.}

\newpage
\section{Size-effects in CAMGs}
Chapters~\ref{c:dev} and ~\ref{c:camg} describe in detail the \gls{camg}s and \gls{ng}s made from 3 nm sized clusters. In this section, CAMGs made from monodisperse nanoparticles of two sizes (\mbox{1 nm} and 7 nm in diameter) are studied as a first step towards understanding cluster-size influence. 

\subsection{7 nm \cz Single Nanoparticle Synthesis and Deposition} 
In previous simulation-based investigations of \gls{ng}s, nanoparticles of sizes 7-15 nm were chosen  \cite{Adjaoud2016,Adjaoud2018,Kalcher2017,Cheng2019}. Therefore, CAMGs were explored with this nanoparticle\footnote{It is worth noting that in this chapter, the clusters and nanoparticles, although the same in constitution and composition, refer to two different entities. The clusters refer to small aggregations of atoms---usually in the size range of 10-2000 atoms \cite{Kartouzian2013,Kartouzian2014,Benel2018,Benel2019,Gack2020}, whereas the nanoparticles are the large masses of atoms typically used in the context of NGs \cite{Jing1989,Nandam2017,Adjaoud2016,Adjaoud2018}.} size range. A large spherical nanoparticle, 7 nm in diameter, was cut from a \cz \qr{10} \gls{mg}\footnote{All simulated MGs, NGs and CAMGs discussed in this chapter are made from \cz \qr{10} MGs.}. The resulting nanoparticle contained $\sim$10,000 atoms, and had a \cz composition with 0.3\% deviation. After cutting the sphere, the nanoparticle was subject to a short heating above \glsdesc{tg} (\gls{tg}), and cooled to 50 K, similar to the 3 nm cluster (as described in Section \ref{s:clus}). \par

\begin{figure}[!h] \centering
	\includegraphics[width=0.9\textwidth,trim={0 1.5cm 0 1.2cm},clip]{NG_1-7nm/1.pdf}
	\mycaption{Compositional variation in the 7 nm \cz nanoparticle}{The evolution of chemical segregation of (a) Cu-atoms and (b) Zr-atoms, in the core and the shell regions of the \cz amorphous nanoparticle.}
	\label{f:7nm-clus}
\end{figure}

\clearpage

A core-shell structure was found to evolve in the 7 nm nanoparticle (See Figure~\ref{f:clus_rad-3nm} in \nameref{c:supple}), with a shell thickness of 3 nm evaluated using a radial composition analysis like in Figure~\ref{f:clus_rad-3nm}, Chapter~\ref{c:dev}. Figure~\ref{f:7nm-clus} describes the variation of Cu and Zr concentrations in the core and shell regions. Like in the \mbox{3 nm} cluster (Figure~\ref{f:clus_comp-3nm}), a sharp change in the compositions in the core and shell is observed with the heat treatment. The high temperature mobilizes the atoms to move to their preferred positions. This configuration is by no means the equilibrium concentration of the nanoparticle, as the atoms need a longer time above the \gls{tg} to diffuse across the sphere completely \cite{Adjaoud2016}. However, long duration of heat-treatment above \gls{tg} is known to produce crystalline nuclei in the \gls{rq} \gls{mg} experiments. Moreover, it was necessary to set the simulation parameters in the \mbox{7 nm} nanoparticle same as the 3 nm cluster for the sake of consistency. \par

\begin{figure}[!h]
	\begin{subfigure}{0.55\textwidth} \centering
		\begin{subfigure}{\textwidth} \centering
			\includegraphics[width=\textwidth,trim={2.8cm 1cm 2.8cm 1cm},clip]{NG_1-7nm/2.pdf}
		\end{subfigure}%
		%	\hspace{-0.5cm}
		\vfill
		\begin{subfigure}{\textwidth} \centering
			\includegraphics[width=\textwidth,trim={2.8cm 2cm 2.8cm 1cm},clip]{NG_1-7nm/3.pdf}
		\end{subfigure}
	\end{subfigure}%
	\hspace{0.3cm}
	\begin{subfigure}{0.45\textwidth} \centering
		\includegraphics[width=\textwidth,trim={8.8cm 1cm 0.5cm 1cm},clip]{NG_1-7nm/4.pdf}
	\end{subfigure}%
	\mycaption{7 nm \cz single nanoparticle deposition}{The cross sections of the as-deposited states of the nanoparticle being deposited at varying energies (per atom) colour coded by (a) the core and shell regions of undeposited nanoparticle, like in Figure \ref{f:clus_single3}. (b) The local atomic shear strains of the nanoparticle and substrate, as compared to their pre-deposition states. In (c), the as deposited states are characterized by \gls{rc} and \gls{rmsds} (See text for more details).}
	\label{f:7nm-cibds}
\end{figure}

Nevertheless, the Cu-concentration in the 7 nm nanoparticle is seen to increase in the nanoparticle shell, with a corresponding decrease in the core region. The opposite is observed with the Zr-concentration, which is seen to preferentially segregate to the core. A brief discussion on the driving force behind this segregation is mentioned in Section \ref{s:clus} and Reference \cite{Adjaoud2016}. \par

After the preparation of the nanoparticle, its deposition on a substrate is studied (similar to Chapter~\ref{c:dev}). A layered substrate thermal model is once again implemented for reasons mentioned in Chapters~\ref{c:dev}~and~\ref{c:camg}. The substrate chosen is 15 nm $\times$ 15 nm in the XY-plane, larger than the one used for the single 3 nm cluster deposition. The substrate thermal layers were flat, unlike for the 3 nm single deposition case. In Figure \ref{f:7nm-cibds}a, the cross sections of a single 7 nm nanoparticle deposition are depicted, 2 ns after impact\footnote{The singly deposited nanoparticles at various energies were found to attain a stable configuration after \mbox{2 ns} of deposition, see Figure~\ref{f:cibds_eval} in \nameref{c:supple} for more details.}. The impact energy of the deposition is varied from 6 meV/atom to 3000 meV/atom. The 7 nm nanoparticle, like the 3 nm cluster (see Figure~\ref{f:cibdsmod}) is seen to deform with increasing impact energy. Furthermore, the nanoparticle is also seen to be embedded more into the substrate at higher energies. \par

At 3000 meV/atom impact energy, some nanoparticle and substrate atoms can be seen to ejected from the substrate, not just from the sides of the impacted nanoparticle, but also along the impact-center, indicating a reflection of impact-shockwaves by the substrate. Such an explosive behaviour would not only be a consequence of the high momentum of deposited nanoparticle, but also a result of the inability of the modelled substrate to dispel sufficient heat from the system (See Section \ref{s:camgdev} for more details). \par

Although the energy per atom is half the maximum deposition energy of 6000 meV/atom in the 3 nm clusters case (See Sections \ref{s:camgdev} and \ref{c:cibd_single}), the deposition of the bigger 7 nm size nanoparticle introduces a larger momentum into the system. Since the high impact is not desirable for formation of interfaces, the deposition at 3000 meV/atom case will not be studied in detail. However, to simulate high impact and high momentum depositions, hybrid substrate models using \gls{md} and continuum mechanics would serve as better physical models, for instance, by reducing the shockwave reflections upon impact \cite{Insepov1997,Allen2002}. \par

When the atoms in the simulated cross-sections are colour-coded by their local von Mises strain as in Figure \ref{f:7nm-cibds}b, the deformation of the 7 nm nanoparticle is clearly visible. The deformed zones in the nanoparticle-substrate system are much wider than in the 3 nm cluster case, hinting that the cluster-cluster interfaces in CAMGs would increase in width with the size of the chosen clusters. \par

In Chapter~\ref{c:dev}, two quantities \gls{rc} and \gls{rmsds} were defined to quantify the deposited states of a single 3 nm cluster, 2 ns after deposition. Figure~\ref{f:7nm-cibds}c shows the \gls{rc} and \gls{rmsds} corresponding the depositions of the 7 nm nanoparticle. Unsurprisingly, the \gls{rmsds} is seen to increase drastically with impact energy. At an impact energy of 600 meV/atom, a $\sim$6\% increase in \gls{rmsds} relative to the 6 meV/atom soft-landed state. In contrast, for a 3 nm cluster at the same impact energy, the relative increase in \gls{rmsds} from the 6 meV/atom soft-landed state is only $\sim$1.9\%. Correspondingly, the convexity of the nanoparticle, characterized by \gls{rc}, is seen to change with deposition energy. In the energy ranges of 0-600 meV/atom, the nanoparticle remains largely convex in nature on the substrate. In comparison, the 3 nm cluster loses its convexity already at 600 meV/atom impact energy (Figure~\ref{f:clus_single3}). At 1000 meV/atom, the 7 nm nanoparticle is more concave, indicating that the nanoparticle begins to embed into the substrate beyond this energy. The CAMG of the 3000 meV/atom case---as mentioned before---was observed to be losing atoms likely due to shockwave reflections. If the snapshots were made later in time, more atoms would have been lost from the deposited nanoparticle at the 3000 meV/atom energy. The \gls{rc} and \gls{rmsds} values in this case are hence merely representative, and will not describe the end state of the deposited nanoparticle. To facilitate a comparison with the \mbox{3 nm} CAMGs studied in Chapter~\ref{c:camg}, CAMGs made from \mbox{7 nm} nanoparticles are simulated with 60, 300 and 600 meV/atom impact energies in the next section. \par

\subsection{7 nm \cz Multiple Nanoparticle Deposition}
The deposition of the CAMG film was made in an \gls{hcp} pattern (see Section~\ref{c:dev} for more details regarding this choice). To fit the deposited CAMG exactly onto the substrate, the substrate dimensions were chosen to be 14 nm $\times$ 13 nm in the XY-plane. This helped save simulation costs. In Figure~\ref{f:7nm-cibdpores}, the deposited film for the low energy case of 60 meV/atom is shown. Like the 3 nm clusters, all the nanoparticles retain most of their sphericity. In Figure~\ref{f:7nm-cibdpores}a, the film atoms are once again colour-coded with the scheme used in Chapters~\ref{c:dev}~and~\ref{c:camg}. While the substrate atoms are coloured-black, the atoms in the shell region of the undeposited nanoparticle are coloured yellow, the core atoms are coloured magneta. These core and shell atoms defined from the undeposited nanoparticle are once again used to define the core and interface atoms in the CAMGs films. \par

\begin{figure}[!h] \centering
	\includegraphics[width=\linewidth,trim={2cm 2.8cm 2cm 2cm},clip]{NG_1-7nm/5.pdf}
	\mycaption{A perspective view of the 60 meV/atom 7 nm \cz CAMG}{(a) The deposited film atoms are coloured by a core-shell colour scheme also used in Figures~\ref{f:clus_rad-3nm}~and~\ref{f:film_network}. (b) A sliced view of a constructed surface mesh in the same perspective view as Figure~\ref{f:7nm-cibdpores}a shows the presence of pores in the CAMG film. In (c) the atoms shown in Figure\ref{f:7nm-cibdpores}a are represented by their local atomic strains, corroborating the presence of interface regions (similar to Figure~\ref{f:film_network}).}
	\label{f:7nm-cibdpores}
\end{figure}

A surface mesh construction \cite{Stukowski2010a,Stukowski2014} with a probe sphere radius of 3 \r{A} revealed the formation of pores in the films (Figure~\ref{f:7nm-cibdpores}b). Even when the deposition of the nanoparticles were adjusted (deviating from the HCP pattern), the pores were seen to not fully close up. This can be attributed to the large size of the nanoparticles. In atomic clusters of smaller sizes, voids formed in cluster-cluster interfaces can easily be closed up in a \gls{hcp} deposition. \par

The local shear strain or von Mises strain of the CAMG atoms and the substrate are depicted by a colour coding scheme in Figure~\ref{f:7nm-cibdpores}c. In this manner, the deformed interfaces can be clearly visible, as observed in Figures~\ref{f:film_network}~and~\ref{f:7nm-cibds}. \par

\subsection{Effect of Cluster Size on CAMG SRO and Energetic States}
After setting up a deposition protocol for the 7 nm nanoparticle, it is now possible to proceed to the evaluation of size effects in the 7 nm CAMGs, which were prepared at three energies of 60, 300 and 600 meV/atom. With the exception of the 60 meV/atom CAMG, the other two simulated samples were observed to not show any pores. To avoid the inclusion of any surface artefacts in the non-porous films (See Section~\ref{s:corint} for more details), representative slabs were constructed for the CAMGs. \par

\begin{figure}[!ht] \centering
	\begin{subfigure}{\textwidth}
		\includegraphics[width=\textwidth,trim={0.65cm 3.65cm 1cm 2cm},clip]{NG_1-7nm/6.pdf}
	\end{subfigure}%
	\vfill
	\begin{subfigure}{\textwidth}
		\includegraphics[width=\textwidth,trim={0.65cm 3.65cm 1cm 2cm},clip]{NG_1-7nm/7.pdf}
	\end{subfigure}%
	\vfill
	\begin{subfigure}{\textwidth}
		\includegraphics[width=\textwidth,trim={0.65cm 2.8cm 1cm 2cm},clip]{NG_1-7nm/8.pdf}
	\end{subfigure}%
	\mycaption{Comparison of SRO and energetic states of 3 nm and 7 nm CAMGs}{(a)-(c) Full-icosahedral (FI) packing, (d)-(f) icosahedral-like ordering (ILO), and (g)-(i) average P.E/atom of the CAMGs.}
	\label{f:7nm-cibdeval}
\end{figure}

The 7 nm CAMGs are compared to the 3 nm CAMGs based on their energetic states and local \gls{sro} in Figure~\ref{f:7nm-cibdeval}. The \gls{rq} \gls{mg} and \gls{mght}, like in Sections~\ref{s:vorocamg}~and~\ref{s:camg-pe} are also represented as a reference. For reasons elucidated in Section~\ref{s:corint}, the CAMGs are compared with 7 nm NGs (compacted in an HCP layout), \gls{rq} \gls{mg}, and \gls{mght}. The porous 7 nm 60 meV/atom CAMG, which is present with surface artefacts is marked in all the sub-figures with a red cross (`\textcolor{red}{\textbf{\faClose}}'). The \gls{fi} order and \gls{ilo} are used to describe the SRO. Figures~\ref{f:7nm-cibdeval}(a)-(c) show the \gls{fi} variation in the entire CAMG representative slab, and also in the core and the interface regions. Likewise, the \gls{ilo} is depicted in Figures~\ref{f:7nm-cibdeval}(d)-(f). It is clearly observed for both the 3 nm and 7 nm CAMGs that the \gls{fi} order and \gls{ilo} both increase in the core and interfaces with increasing deposition energy. \par

In the core regions, a clear difference is seen in the \gls{sro} with cluster size. For the case of the CAMG with the larger nanoparticle, the \gls{fi} order and \gls{ilo} are both higher than for the 3 nm CAMG of the corresponding deposition energies. This same trend is also observed in the NGs. The average P.E/atom of the CAMGs of different cluster sizes show the opposite behaviour in Figure~\ref{f:7nm-cibdeval}(g)-(i). The change in the \gls{fi} order and \gls{ilo} in both the 3 nm and 7 nm CAMGs is a clear evidence of tailoring of local structure with the deposition energy. The variation in \gls{sro} and P.E/atom between the 3 nm and 7 nm CAMGs can be correlated to the difference in local compositions, as indicated in Table~\ref{t:chem_heter}. In the entire CAMG samples overall, the SRO and P.E/atom are seen to be invariant with cluster size. The observed change in the formation of interfaces and previous reports of scaling of elastic properties with the size of the cluster in nanoglasses \cite{Cheng2019a} motivates the exploration of CAMGs with atomic clusters of 10-30 atoms in size, wherein the scope of interface formation improves drastically. \par 

\begin{table} \centering
	\begin{tabular}{c c c c} 
		\hline \hline	
	    \multirow{2}{2cm}{\centering Cluster diameter} & \multicolumn{3}{c}{Composition}\\
%	    \cline{2-4}
		 & Total sample & Core & Interface \\
		\hline	
		3 nm	& Cu$_{50}$Zr$_{50}$ & Cu$_{46}$Zr$_{54}$ & Cu$_{55}$Zr$_{45}$ \\
		7 nm	& Cu$_{50}$Zr$_{50}$ & Cu$_{49}$Zr$_{51}$ & Cu$_{52}$Zr$_{48}$ \\
		\hline \hline	
	\end{tabular}
	\mycaption{Chemical heterogeneity in the 3 nm and 7 nm CAMGs}{The cores and interfaces are in the two CAMGs are found to exhibit distinct chemical concentrations, although the macroscopic chemical composition remains constant at \cz.}
	\label{t:chem_heter}
\end{table}

\subsection{Small-cluster CAMGs}
One of the objectives of this work was to explore the effects of CAMGs made from clusters of 1 nm diameter. Based on previous works on \gls{pvd} glasses \cite{Singh2013,Raegen2020}, these CAMGs can be expected to show thermodynamic ultrastability. Since the 1 nm clusters contain only $\sim$30 atoms, the stability of the cluster on the surface after deposition is of concern. Theoretically, after deposition, it is expected that a certain amount of the impact energy dissipated from the cluster is transferred back to it. However, the cluster will be stable on the substrate as long as the energy transferred back to it is lesser than the \glsdesc{ecoh} (\gls{ecoh}) of the material. Even in an extreme case where 100\% of the deposition energy is transferred back to the cluster, the depositions in the energy range of 60-600 meV/atom should result in stable clusters, as the energy is much lower than $\sim$-4.88 eV/atom---the \gls{ecoh} of a \cz \gls{rq} \gls{mg} \cite{Jekal2019}. The following paragraph briefly describes the initial simulations of CAMGs made from small clusters performed with the assistance of Ms. Veronika Stangier, who briefly worked as a student researcher in the group of Prof. Wolfgang Wenzel at the \gls{kit}.  \par

In conjunction with the work discussed in this dissertation, attempts to simulate \cz CAMGs from clusters of 20-30 atoms were also made to explore local chemical heterogeneity at the size scale of $\leq$ 1 nm, as envisioned by \textcite{Kartouzian2014}. In contrary to the expectations elucidated above, the clusters were found to spontaneously melt onto the surface, even with soft landing of 6 meV/atom energy. Alternatively, clusters which were first arranged on the substrate in a `frozen' state (with their thermal models and net computed forces set to zero)--were also found to dissolve onto the surface once they were unfrozen. This led to conclusion that the dissolution of clusters is likely due to their cohesion with the \cz substrate. \par

As a workaround to these modeling limitations with small clusters in \gls{camg}s, the grain size effects were explored in \gls{ng}s, as discussed in the following section.

\section{Size-effects in Nanoglasses}
In this section, the influence of grain size or cluster size on the \gls{ng}s is studied. Firstly, the clusters of diameters 1, 2, 3, 5, and 7 nm were once again prepared as discussed in Chapter~\ref{c:dev} and in the above sectin. It was also ensured that the clusters were made with an exact \cz composition, with no deviation in composition. The NGs were then made from monodisperse cluster distributions, inserted into a simulation box in a random fashion\footnote{In this section, all NGs are made from a random cluster-insertion model unlike the HCP packing used in previous sections and chapters. Without the constraint of a patterned arrangement, the total number of clusters in the box can be controlled to ensure that the simulated NG samples made from varying cluster sizes have approximately the same number of atoms, thereby ruling out any possible simulation-box size effects.} and compacted at 5 GPa pressure before unloading (See Figure~\ref{f:NGintersiz}a).   \par

\subsection{Variation of Interfaces with Cluster Size}

\begin{figure}[!h]
	\includegraphics[width=\linewidth,trim={1.2cm 0cm 0.6cm 0cm},clip]{NG_1-7nm/10.pdf}
	\mycaption{NG interfaces vary with cluster size}{(a) The cluster compaction model is illustrated. (b)-(f) A frontal sliced view of the NGs, with the atoms coloured by their von Mises shear strain made from clusters of sizes 1-7 nm shows the homogenization of interface regions with reducing cluster size.}
	\label{f:NGintersiz}
\end{figure}

Figures~\ref{f:NGintersiz}(b)-(f) show the cross-sectional views of the NGs. Similar to Figure~\ref{f:film_network}(d), here the atoms are colour-coded with their local \glsdesc{vmstr} (\gls{vmstr}) or atomic strain values evaluated using \gls{ovito} \cite{Stukowski2014} to distinguish the interfaces from the core by means of atomic-level deformation. As mentioned in Section~\ref{s:corint}, this is done because the interfacial atoms are expected to deviate more from their original positions near the surface of the uncompressed cluster/nanoparticle, than the core atoms \cite{Gleiter1991}. With the decreasing cluster size, the interfacial width is not only found to drop, but also simultaneously one can visually observe how the interfacial atoms begin to dominate the bulk of the \gls{ng}s with smaller clusters. This is in agreement with previous NG simulations \cite{Cheng2019a} and also the expectation that smaller clusters have a higher surface to volume ratio, thereby enabling the possibility for more interfaces to occur. In other words, the net number of atoms participating in interface formation increase with decreasing cluster size. \par

\subsection{Atomic Readjustment upon Compaction}
In the previous subsection, the formation of interfaces in NGs was observed using the local atomic strain or \glsdesc{vmstr} (\gls{vmstr}). The change in interfacial characteristics was visually noticeable when the NG building blocks were varied from large nanoparticles (5-7 nm) to small clusters (1-3 nm). The \glsdesc{vmstr}-information of every atom in the \glspl{ng} are now used to further explore the cluster-size effects on the atomic displacement which occurs during formation of the cluster-cluster interfaces. \par

\begin{figure}[!h]
	\includegraphics[width=\linewidth,trim={2cm 0.3cm 2cm 0.475cm},clip]{NG_1-7nm/11.pdf}
	\mycaption{NG atomic readjustment after compaction}{(a) The fraction of atoms with a local strain higher than a threshold of $\eta^{thres}$ (indicated in the legend). (b) Atoms with $\eta^{thres}=1$ are seen to readjust further with annealing, and with decrease in cluster size. (c) The average strain of NG atoms increases with annealing and with lowering cluster size.}
	\label{f:NGvonMises}
\end{figure}

In Figure~\ref{f:NGvonMises}a, the `Atomic readjustment' in the compacted NGs---defined as the percentage of atoms having \gls{vmstr} above a threshold of $\eta^{thres}$ is illustrated. The atomic readjustment is named so, as the von Mises strain essentially captures the change of the local environment of around an atom in a system of atoms, with respect to a reference configuration \cite{Adjaoud2018,Cheng2019,Zheng2021,Cheng2009}. In this manner, one can quantify the change in the environment of each atom in a cluster before and after compaction into a \gls{ng}. At low thresholds of $\eta^{thres} \leq 1$, the atomic readjustment in the NGs is seen to increase with decreasing cluster size. It is also noticed that above $\eta^{thres} = 1$, the atomic readjustment does not vary with cluster size. \par

Both these observations together indicate that the number of highly readjusting interfacial atoms in the \glspl{ng} do not vary significantly with cluster size. More number of atoms participate in the atomic-level deformation in the smaller cluster NGs, overall. This is also confirmed by Figures \mbox{\ref{f:NGvonMises}(b)-(c)}. Additionally, the NG samples were annealed below \gls{tg} at 600 K for 2 ns. The heating from 50 K to 600 K was done at a rate of \qr{12}. After equilibration at 600 K for 2 ns, the samples were cooled down back to 50 K at a rate of \qr{12} and equilibrated again for 2 ns. Upon annealing, the increase in atomic readjustment (at a constant threshold of $\eta^{thres} = 1$) is greater with the decrease of NG cluster size. Additionally, the average von Mises strain in the as-prepared NG system is higher with reduction of cluster size, and the increase in this value upon annealing is also higher at smaller cluster sizes. \par

\subsection{Short-range Order and Thermal Behaviour of the NGs}
The as-prepared and annealed NGs are next characterized by their local \gls{sro} and energetic states. In Figure~\ref{f:NG_str-PE}, the characteristics of the entire NG samples are depicted. The core and interface atoms are not individually investigated. The FI order, \gls{ilo}, and P.E/atom in the as-prepared samples are noted to vary slightly with the cluster size. However, in proportion to the effect of quench rates or energetic deposition as seen in Figures~\ref{f:camg_pote}~and~\ref{f:7nm-cibdeval}, respectively, the \gls{sro} and energetic states are fail to show the expected cluster size dependency. \par

\begin{figure}[!h] \centering
	\includegraphics[width=0.98\linewidth,trim={2cm 0.3cm 2cm 0.475cm},clip]{NG_1-7nm/12.pdf}
	\mycaption{Influence of cluster size on structural and energetic states of NGs}{(a) FI-ordering (b) ILO (c) P.E/atom states of the various as-prepared and annealed NGs.}
	\label{f:NG_str-PE}
\end{figure}

Upon thermal relaxation after annealing, the modelled NG samples are observed to have an increased SRO and a lowered energetic state in comparison to their as-prepared states. This appears to be a consequence of the atomic readjustment that was noted to be promoted by the annealing process in the previous subsection. The annealing possibly assists the simulated NGs to attain their final states, which may not be accessible by the as-prepared samples due the short \gls{md} simulation time scales. However, even after the thermal treatment, no discernible variation of the \gls{sro} and energetic features with cluster size can be noted. This study on \cz NGs is in agreement with previous works on \czsix NGs \cite{Cheng2019a}. \par

The preceding research activity on cluster-size influence in the NGs, although demonstrating an invariance of local order with cluster-size, reported some interesting mechanical properties in the Cu$_{36}$Zr$_{64}$, \cz and \czsix Voronoi-tesselated NGs \cite{Adibi2013,Adibi2014} and cluster-compacted NGs \cite{Cheng2019a}. Additionally, the mechanical and thermal properties of \gls{rq} \gls{mg}s have been linked in earlier works \cite{Su2016,Battezzati2009,Bian2021}. This motivates the need to explore the thermal properties of the \gls{ng}s, in which the reduced grain/cluster-size has notably demonstrated a reduction of flow stress and increased plasticity. \par

\begin{figure}[!h]
	\includegraphics[width=\linewidth,trim={0cm 1cm 0cm 1cm},clip]{NG_1-7nm/13.pdf}
	\mycaption{Thermal behaviour of MGs and NGs}{The average enthalpy per atom with increasing temperature in (a) the simulated reference samples of 8000-atom sized \cz RQ MGs of varying quench rates and (b) the NGs of varying cluster sizes. In the insets of the respective figures, the system temperature with time is indicated.}
	\label{f:NG_enth}
\end{figure}

The simulated \gls{ng}s of various cluster sizes were subjected to a heat-treatment at zero pressure from a low temperature to well above the \gls{tg} (~800 K for simulated \cz \gls{rq} \gls{mg}s), and compared with a reference system of \gls{rq} \gls{mg}s. The average atomic enthalpy\footnote{The enthalpy H = U + p$\cdot$V, where U, p, and V are the internal energy, pressure and volume of the system, respectively. In the heating simulations, as p=0, the enthalpy can be expressed as H = U.} as a function of temperature is represented for both the NGs and the reference \cz \gls{rq} \gls{mg}s\footnote{Further studies to establish the protocol for heating and cooling of \gls{rq} \gls{mg}s are discussed in Section \ref{s:mgsquench} of the \nameref{c:supple} chapter.} in Figure~\ref{f:NG_enth}. The NG samples, in their as-prepared states at 50 K are first equilibrated for 2 ns and then heated to 1200 K at a rate of 0.25 K/ps. For the reference \gls{rq} \gls{mg}s prepared at various quench rates, the initial average enthalpies at 50 K are found to be ordered according to quench rates associated with their formation: the lower the quench rate, the lower the enthalpy. This trend continues in the entire temperature regime below \gls{tg}, after which the enthalpies of the glasses of various quench rates are equal. These observations are in agreement with known knowledge from literature \cite{Berthier2016,Ediger1996}. In contrast, the thermal behaviour of the \gls{ng}s is found to be independent of cluster-size. The curves depicting average enthalpy versus temperature for various cluster-sizes overlap with each other. \par

\section{Discussion and Summary}
The ability to design a network of distinct amorphous cluster-core and cluster-cluster interface phases into the CAMGs and NGs begets the emergence of cluster-size/grain size effects in these novel amorphous materials. In the current chapter, simulated \cz CAMGs and NGs prepared from monodisperse clusters of varying sizes were studied. \par

First, the CAMGs made from a 3 nm cluster in Chapter~\ref{c:camg} was compared to those prepared from a 7 nm nanoparticle at corresponding per-atom deposition energies. The \gls{sro} of the 7 nm CAMGs were also observed to be tailorable with deposition energy in the samples. However, the \gls{sro} and the average \gls{pe}/atom of CAMGs (for a given deposition energy) were found to be invariant with the cluster-size. Upon further inspection, it was revealed that within the core and interfacial regions, the \gls{sro} and the average energetic states in the CAMGs varied considerably with cluster size. These effects were attributed to the difference in the local chemical heterogeneity introduced into the simulated CAMGs from the precursor clusters. An attempt to prepare CAMGs from clusters of 1 nm diameter ($\sim$30 atoms) proved to be non-viable, owing to spontaneous melting of deposited clusters onto the substrate even at per-atom deposition energies below the cohesive energy of \cz . Replacing the \cz substrate with a Si substrate may reduce cluster-substrate cohesion, promoting retention of the structure of the smaller clusters after deposition. \par

The influence of grain size studies in smaller clusters (diameter $\leq$ 2 nm) were explored in the \cz NGs made using a monodisperse clusters inserted randomly before compaction. This approach offers the advantage of exploring size effects without considering additional substrate interactions unlike in the CAMGs. With the reduction of the cluster size in the NGs, the interfacial width was seen to correspondingly reduce, while the cluster atoms participating in readjustment upon compaction and annealing increase drastically. Despite the readjustment of the interfacial atoms, and the opportunity for interfaces to relax better, no significant change is seen in the NGs with the cluster size. Additionally, the smaller 1-2 nm clusters present a challenge of defining interfaces in the NGs, in terms of distinguishing the shell atoms from the core. In the future, using the von Mises strain or the \gls{qna} approach \cite{Feng2020} may serve as better parameters to identify glass-glass interfaces by making no \textit{a priori} inferences about the clusters' core and shell regions. Further studies on the evolution of density and free volume with the composition and size of the precursor clusters are in progress. \par

\chapter{Conclusions}
\chapter{Conclusions} \label{c:conclusions}

\section{Summary}
In this dissertation the novel \cz \gls{camg}, prepared by the energetic deposition of amorphous nanoclusters, has been studied by means of \gls{md} simulations. Such a careful cluster assembly experiment is expected to give rise to new amorphous structures different from the conventional \gls{rq} \gls{mg} of the same composition, and also the \gls{ng} produced by mechanical compaction of the same amorphous clusters. Understanding the formation mechanisms and final structures of the so-obtained \gls{camg} opens up possibilities to design and control amorphous structures. The key results obtained in this doctoral work are summarized below. \par

\begin{enumerate}[leftmargin=*]
\item \textbf{First virtual insights into \gls{camg}s}
%\item \textbf{Development of a simulation protocol for \gls{camg}s}
\begin{enumerate}[leftmargin=*]
%\item \textbf{Synthesis of a 3 nm \cz cluster}\\
%%\item \textbf{Deposition of single 3 nm clusters hint at morphologies adopted in CAMGs} \\
%Amorphous \cz clusters (3 nm in diameter) are first derived from the bulk of an \gls{rq} \gls{mg}, followed by a heat-treatment (by the method of \textcite{Adjaoud2016}) to induce a structure that is expected from experimental \gls{igc} preparation. The heat-treated 3 nm cluster is found to have a radial compositional variation, with a core-shell structure. The Cu-concentration is \mbox{56 \%} in a 0.2 nm thick shell region, and \mbox{44 \%} in the core. An \gls{igc} model to prepare clusters is also briefly discussed, however, the bulk-derived clusters are utilized in this dissertation.

\item \textbf{Development of a simulation protocol for cluster deposition} \\
First, a single 3 nm cluster, derived from the bulk of a \gls{rq} \gls{mg} followed by a heat-treatment to induce a structure that is expected from experimental \gls{igc} preparation. This cluster was deposited at various impact energies ranging from 6-6000 meV/atom onto a \cz substrate. Its shape and deviation from its undeposited state were noted to change with impact energy. At low energies (6-60 meV/atom) the cluster adopted a nearly spherical morphology, while at medium impact energy of 300 meV/atom the cluster is disorted further although still adopting a convex drop-like shape. At deposition energies higher than 600 meV/atom the cluster begins to turn concave-like, embedding itself more into the substrate. The cluster core-shell structure disintegrates in the extreme landing (6000 meV/atom energy) case. \par

\item \textbf{Simulating \cz \gls{camg}s} \\
The protocol developed for cluster deposition was extended to simulate \gls{camg}s. It was chosen to deposit 3 nm size \cz clusters in a \gls{hcp} pattern to remove surface artifacts, maximize cluster-cluster interface formation, and improve simulation efficiency. Upon assembling the \gls{camg}s from the 3 nm sized clusters, it was found that the chemically segregated core and shell structure of the undeposited clusters gave rise to two chemically distinct Zr-rich cores and Cu-rich interfaces phases; the core regions being interspersed within an interconnected network of the cluster-cluster interfaces. The interfaces were found to be stable in \gls{md} timescales for impact energies up to 600 meV/atom, however they disappear at extreme landing energy of 6000 meV/atom energy. \par

\item \textbf{Short-to-medium range order tailoring in \cz \gls{camg}s}\\
The \gls{fi} and \gls{ilo} evaluated by Voronoi tesselation were used to describe the \gls{sro} of the glasses. Strings of \gls{fi}-atoms are used to indicate \gls{mro}. The \gls{sro} of the \gls{ng} and \gls{camg} are found to differ from the \gls{rq} \gls{mg}s, irrespective of the quench rate associated with the bulk-derived clusters. In the conventional (\qr{10} quench rate) case, \gls{sro} is lower in the \gls{ng} and \gls{camg} as compared to \gls{rq} \gls{mg}s, for both the cores and interfaces. However, irrespective of the quench rate used both, the \gls{sro} and \gls{mro} in \gls{camg}s recover towards the metallic glass values with increasing impact energies. As a result, adjusting the processing conditions in \gls{camg}s makes it possible to control the local structure of metallic glasses.
\end{enumerate}

\item \textbf{Cluster-size effects in \gls{camg}s and \gls{ng}s}\\
An influence of cluster size on the interplay between interface and cores regions was expected, motivating an investigation of further changes within the \gls{camg} structure.

\begin{enumerate}[leftmargin=*]
\item \textbf{The 3 nm cluster vs a 7 nm \cz nanoparticle}\\
A large 7 nm sized \cz nanoparticles was derived from the bulk, like the 3 nm clusters. The 7 nm nanoparticles were also present with a core-shell structure, with a shell of 0.3 nm thickness. The Cu-enrichment in the nanoparticle shell was 52 \%, which is 4\% lesser than in the 3 nm cluster. Upon testing the single nanoparticle deposition, the 7 nm nanoparticle was found to retain its morphology better at any given impact energy as compared to the 3 nm cluster. Like in the 3 nm cluster, the shell atoms of the 7 nm nanoparticle were also found to deviate from their original positions with a local strain. The amount of deviation was higher at any given energy for the nanoparticle. At the deposition site, the atoms were found to increasingly shear (with respect to their un-deposited states) with deposition energy.
%The starkness ??? of the cluster reduces with impact energy, but the amount of reduction is lower for larger clusters. For 7 nm clusters, until 600 meV/atom energy. At the deposition site, the atoms were found to shear, and progressively with deposition energy.
At a given energy, the larger 7 nm nanoparticle was found to larger volume of deformation at the site of deposition, hinting at formation of larger interfaces. \par 

\item \textbf{Local order in 7 nm \cz CAMGs}\\
Like the 3 nm CAMGs, the CAMGs from a 7 nm nanoparticles were also deposited in the \gls{hcp} pattern. As expected from the single deposition, the interfacial width was larger, and at 60 meV/atom energy, the nanoparticles could not deform enough, leaving large pores in the sample. However, at higher energies, the pores closed up and the \gls{sro} of the 7 nm CAMGs increased with deposition energy. At a fixed impact energy, the \gls{sro} and the average energetic states of CAMGs were found to be invariant with the cluster-size. The difference in the local chemical heterogeneity introduced from the precusor clusters into the simulated CAMGs, brought about a significant variation in the \gls{sro} and the average P.E./atom within the core and interfacial regions with cluster size. 

\item \textbf{Building \gls{ng}s with blocks ranging from atomic clusters to large nanoparticles} \\
Monodisperse clusters of diameters ranging between 1-7 nm were arranged randomly before compaction to prepare \cz NGs. Lowering the cluster size was observed to lower the interfacial width as in the CAMGs. The number of cluster atoms participating in readjustment at the interfaces upon compaction and annealing was also found to increase with decreasing cluster sizes. The change in cluster size appear to not have a significant influence on the \gls{sro}, average energetic states, and the thermal behaviour in the simulated NGs. Regardless, the pronounced increase in readjustment of the interfacial atoms with reduced cluster size offers scope for interesting local structural changes not observed in the present simulations. A deeper investigation is required to better understand these novel glasses. 
\end{enumerate}
\end{enumerate}

The establishment of an \gls{md} simulation protocol for \gls{camg}s and the study of their local characteristics now lends support to the idea of controllable local order and microstructures in metallic glasses. There is an immense need for future work on this class of novel materials to better understand them and harness their properties.

\clearpage
\section{Outlook}
The primary contributions of this work are the initial \textit{in silico} investigations on the cluster-assembly of metallic glasses, as discussed above. While some preliminary questions have been answered, the quest to fully understand \gls{camg}s and the consequences of their tailorability of their local order remains to be explored.

\begin{enumerate}[leftmargin=*]
\item \textbf{Thermal stability of CAMGs made from vapour-condensed clusters}\\
In a previous study, \textcite{Danilov2016} reported the thermal ultrastability of \gls{ng}s from \gls{igc}-derived nanoparticles, as compared to \gls{ng}s from bulk-derived nanoparticles, also also conventional \gls{rq} \gls{camg}s. This remains to be explored in \gls{camg}s. The \gls{igc}-like vapour condensed cluster growth method described in Chapter~\ref{c:dev} can be combined with gas-phase condensation methods \cite{Krasnochtchekov2003,Zheng2020} to prepare spherical \cz clusters. It was also determined in the current work, that the cluster assembly of amorphous nanoclusters leads to the increase of local \gls{sro} with impact energy. The stability gained from using \gls{igc}-derived nanoparticles as building blocks, compounded with the increase in \gls{sro} with the deposition energy can result in a synergetic effect to form thermally ultrastable \gls{camg}s.

\item \textbf{Isolating the role of cluster-deposition processing in CAMGs}\\
In the present thesis, a heat-treatment was given to the clusters to allow the cluster-atoms to diffuse, enabling a chemical segregation. The phenomena observed in \gls{camg}s, discussed in Chapters~\ref{c:dev}, \ref{c:camg} and \ref{c:cbmg} are a result of the deposition process and also the chemical segregation. A comparison of \gls{camg}s made from unsegregated clusters can help isolate the role of the deposition processing. Moreover, the difference in local packing and density in comparison to the segregated interfaces and unsegregated interfaces can lead to interesting changes in properties of CAMGs.

\item \textbf{Novel mechanical properties of CAMGs} \\
The tailorable local structure of the CAMGs may result in exciting properties. Previously, reducing the cluster size of \gls{ng}s, has been shown to reduce flow stress \cite{Adibi2014,Cheng2019a} of the material. This change in property was driven by the reduction of interface width and simultaneous increase of interface volume---the nature of the interface plays an important influence on the mechanical properties of bottom-up glasses. Consequently, comparing the mechanical properties of \gls{camg}s made from unsegregated and vapour condensed cluster via nanoindentation studies, specifically observing the stress-strain relations, and strain localization and propagation in the two kinds of CAMG interfaces would be interesting.

\item \textbf{CAMG simulations at mesoscopic timescales}\\
A challenge encountered in this dissertation was the formation of large pores in the \gls{camg}s when the clusters were deposited randomly (discussed in Chapter~\ref{c:dev}). It is very possible that the clusters at long experimental times scales, may reach their preferred states which result in the pores being closed up. The short simulations achieved by \gls{md} thereby hinder the exact replication of \gls{camg} experiments. A recent technique called the \gls{abc} has been developed to traverse the \gls{pel} to general atomic trajectories in the timescale of seconds \cite{Cao2012,Fan2018}. This basin-hopping method finds immense value in \gls{camg} simulations to access mesoscopic timescales and better model the experiments.

\item \textbf{Investigating crystalline \gls{cam}}\\
Recently, polycrystalline materials of small nm-sized grains have been shown to exhibit minimal-interface configurations, exhibiting higher strengths and thermal stability \cite{Li2020,Hu2022} in contrast to the expected \gls{ihp} behavior. The simulation protocol developed within the framework of this thesis are already being applied to simulate nm-sized crystalline clusters \gls{cam} in the group of Prof. Penghui Cao, University of California-Irvine, to test the \gls{ihp} relationship with cluster-assembly.

\item \textbf{Controlling chemical heterogeneity in \gls{camg}s}\\
As mentioned in the \nameref{c:theory} chapter, one of the first works on \gls{camg}s reported a new amorphous class of materials built from chemically distinct clusters 10-16 atoms in size, in an attempt to engineer a locally heterogenous structure in \gls{mg}s \cite{Kartouzian2013,Kartouzian2014}. Such exotic \gls{camg}s have not been investigated beyond synchrotron surface \gls{xrd} measurements. In Chapter~\ref{c:cbmg}, the possibility of simulating locally heterogenous clusters is briefly discussed. In the present thesis, the simulation of CuZr clusters of varying compositions with $\sim$30 atoms each were not stable upon deposition despite the impact energies being much lower than the cohesive energy of the glassy solid. Presently, the dissolution of the small clusters is attributed to the cohesive forces from the \cz substrate used, and hence using a Si substrate may offer more success. The \gls{camg}s of a given macroscopic composition may then be simulated at soft-landing energies with varying local heterogeneity. The resulting systems could be investigated by their local \gls{sro} and \gls{mro}, but also by the phonon density of states, which is sensitive to the local atomic structure.



%\item \textbf{Soft spots and vibrational modes}\\
%$\alpha$ relaxation, $\beta$ relaxation, PDOS (out of scope of thesis)
%Thermal stability, potential energy states, further expanding our knowledge of metallic glasses in general
%\gls{gum} in \gls{camg}s could be discussed \cite{Ding2014}
\end{enumerate}




%\addcontentsline{toc}{chapter}{Appendix}

\addcontentsline{toc}{chapter}{A1 Disclaimer}

\addcontentsline{toc}{chapter}{A2 Acknowledgements}

\addcontentsline{toc}{chapter}{A3 Simulation Repository}

\addcontentsline{toc}{chapter}{A4 Curriculum Vitae}

\bibliographystyle{natnourl}

%Weitere Verzeichnisse wie \listoffigures oder ein  Abkürzungsverzeichnis


\printbibliography

\end{document}
